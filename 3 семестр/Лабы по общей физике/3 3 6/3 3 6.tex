\documentclass[a4paper,12pt]{article} % тип документа
\usepackage{mathtext} 				% русские буквы в формулах
% Поля страниц
\usepackage[left=2.5cm,right=2.5cm,
    top=2cm,bottom=2cm,bindingoffset=0cm]{geometry}
    
%Пакет дял таблиц   
\usepackage{multirow} 
    
%Отступ после заголовка    
\usepackage{indentfirst}


% Рисунки
\usepackage{floatrow,graphicx,calc}
\usepackage{wrapfig}

% Создаёем новый разделитель
\DeclareFloatSeparators{mysep}{\hspace{1cm}}

% Ссылки?
\usepackage{hyperref}
\usepackage[rgb]{xcolor}
\hypersetup{				% Гиперссылки
    colorlinks=true,       	% false: ссылки в рамках
	urlcolor=blue          % на URL
}


%  Русский язык
\usepackage[T2A]{fontenc}			% кодировка
\usepackage[utf8]{inputenc}			% кодировка исходного текста
\usepackage[english,russian]{babel}	% локализация и переносы


% Математика
\usepackage{amsmath,amsfonts,amssymb,amsthm,mathtools, mathrsfs}


% Что-то 
\usepackage{wasysym}


\begin{document}
\title{\begin{center}Лабораторная работа № 3.3.6\end{center}
Влияние магнитного поля на проводимость полупроводников}
\author{Струков О. И. \\ Б04-404}
\date{}

\thispagestyle{empty}
\pagenumbering{gobble}
\maketitle
\newpage
\pagenumbering{arabic}
\textbf{Цель работы:} измерение зависимости сопротивления полупроводниковых образцов различной формы от индукции магнитного поля.

\textbf{В работе используются:} электромагнит, миллитесламетр (на основе датчика Холла), вольтметр, амперметр, миллиамперметр, реостат, образцы монокристаллического антимонида индия (InSb) n-типа.



\section*{Теоретическая часть}

В работе исследуется эффект зависимости сопротивления проводника от магнитного поля на примере диска Корбино.



При отстутствии магнитного поля, направленного перпендикулярно плоскости диска, по диску течёт ток, определяемый по закону 

\begin{equation*}
    I = \frac{U}{R_0}, \; R_0 = \frac{\ln{\frac{r_2}{r_1}}}{\sigma_0 2 \pi r h}
\end{equation*}

Однако при включении магнитного поля индукции $B$ на частицы-переносчики тока начинает действовать сила Лоренца, из-за чего траектория частиц увеличивается в расстоянии, проходимом между двумя точками с фиксированной разницей потенциалов $U$.

В этом случае проводимость  равна 

\begin{equation*}
    \sigma_r = \frac{\sigma_0}{1 + (\mu B)^2}
\end{equation*}

Закон Ома преобразовывается в следующий вид:

\begin{equation*}
    I = \frac{U}{R}, \; R = R_0 (1 + (\mu B)^2)
\end{equation*}

Таким образом, зависимость $I(U)$ поменялась из-за геометрических особенностей диска Корбино. Такой эффект называют геометрическим магнетосопротивлением. 

Отсюда можно выразить подвижность носителей заряда:

\[
k = \dfrac{R}{B^2} = R_0\mu^2 \quad \Rightarrow \quad \mu = \sqrt{\frac{k}{R_0}}
\]
где $k$ -- угловой коэффициент зависимости $R(B^2)$, $R_0$ -- сопротивление при $B = 0$.

\section*{Экспериментальная установка}

Зависимость $R(B)$ исследуется следующим образом:

\begin{enumerate}
    \item Производится калибровка электромагнита: находится зависимость индукции создаваемого магнитного поля от тока в контуре электродвигателя $B(I_m)$, который регистрируется амперметром $A_1$, чтобы в дальнейшем считать величину магнитного поля с помощью тока в контуре $I_m$.
    \item При постоянной силе тока $I_0$, которая настривается с помощью сопротивления реостата в контуре с источником питания, меняется величина индукции магнитного поля, тем самым меняется напряжение $U$, подаваемое на диск Корбино. Исследуется зависимость $R(B)$ через калибровочную кривую и зависимость $U(I_m)$.
    \item Проводится тот же самый опыт с прямоугольной пластинкой с исследованием зависимости её сопротивления $R(B)$.
\end{enumerate}
\begin{figure}[h]
    \centering
    \includegraphics[width = 5 cm]{1.png}
    \caption{Диск Корбино}
    \label{karb}
\end{figure}
\begin{figure}[h!]
    \centering
    \includegraphics[width = 13 cm]{2.png}
    \caption{Схемы экспериментальных установок}
    \label{scheme}
\end{figure}


\section*{Ход работы}
\begin{enumerate}
    \item Были проведена калибровка электромагнита и получена зависимость B(I), которую в рассматриваемом диапазоне значений силы тока можно аппроксимировать квадратичной функцией $B = 11,5 + 1550I - 1560I^2$ (размерность I -- А, B -- мТл). Экспериментальные представлены в таблице, зависимость изображена на графике.
    \begin{table}[H]
\begin{tabular}{|c|c|}\hline
I, A & B, mT \\ \hline
(0,00 $\pm$ 0,01)  & (11,5 $\pm$ 0,1)  \\\hline
(0,08 $\pm$ 0,01) & (120,3 $\pm$ 0,1) \\\hline
(0,11 $\pm$ 0,01) & (151,8 $\pm$ 0,1) \\\hline
(0,16 $\pm$ 0,01) & (222,4 $\pm$ 0,1) \\\hline
(0,23 $\pm$ 0,01) & (288 $\pm$ 1)   \\\hline
(0,29 $\pm$ 0,01) & (340 $\pm$ 1)   \\\hline
(0,38 $\pm$ 0,01) & (368 $\pm$ 1)  \\\hline
\end{tabular}
\end{table}
        \begin{figure}[H]
		\center{\includegraphics[scale=0.65]{B(I).pdf}}
		\caption{Зависимость $B = f(I)$.}
		% \label{Рис. 3}
	\end{figure}
\item Проведено исследование магнетосопротивления образцов. 
Для этого сначала была измерена зависимость напряжения на образце от силы тока в электромагните, откуда затем была выражена зависимость $R = f(B^2)$. Результаты исследования с диском Карбино и пластиной, расположенной в первом случае вдоль, а во втором поперёк поля представлены на графике.
        \begin{figure}[H]
		\center{\includegraphics[scale=0.8]{R(B).pdf}}
		\caption{Зависимость $R = f(B^2)$.}
		% \label{Рис. 3}
	\end{figure}
\item Определены угловые коэффициенты наклона прямых $k_1$ -- для диска Карбино, $k_2$ -- для образца вдоль поля, $k_3$ -- для образца поперёк поля, и из них определена подвижность носителей заряда:

\begin{table}[!ht]
    \centering
    \begin{tabular}{|c|c|c|}
    \hline
	$k_1$ & $k_2$ & $k_3$ \\ \hline
	$(0,99 \pm 0,09)~Ом/Тл^2$& $(0,29 \pm 0,04)~Ом/Тл^2$ & $(0,46 \pm 0,06)~Ом/Тл^2 $\\ \hline
    \end{tabular}
\end{table}

\begin{table}[!ht]
    \centering
    \begin{tabular}{|c|c|c|}
    \hline
	$\mu_1$ & $\mu_2$ & $\mu_3$ \\ \hline
	$(5,35 \pm 0,4)~м^2/(В\cdotс)$& $(1,28 \pm 0,2)~м^2/(В\cdotс)$ & $(2,6 \pm 0,4)~м^2/(В\cdotс)$\\ \hline
    \end{tabular}
\end{table}

\item По известным параметрам диска Карбино вычислено его удельное сопротивление:
\[
\rho = R_0 \cdot \frac{2\pi h}{\ln\left(\dfrac{D}{d}\right)}
\]
	
\begin{table}[!ht]
	% \caption{Параметры диска Карбино}
	% \label{table:const}
	\begin{tabular}{|c|c|c|c|c|}
		\hline
		Больший диам. D& Меньший диам. d& Высота h & Сопр. $R_0$ & Удельное сопр. $\rho$\\ \hline
		18 мм & 3 мм & 1,8 мм & $(0,034 \pm 0,02)$ Ом & $(2,2 \pm 0,1) \cdot10^{-4}~Ом \cdotм$ \\ \hline                                                                           
		\end{tabular}
\end{table}

\item Через полученное удельное сопротивление можно выразить концентрацию носителей заряда:
\[n = \dfrac{1}{\rho e \mu} = (5,4 \pm 0,4) ~м^{-3}\]








\end{enumerate}

\newpage
\section*{Вывод}
В результате выполнения работы была проведена калибровка электромагнита, исследована зависимость сопротивления образцов от магнитного поля и найдена концентрация носителей заряда в диске Карбино.
Табличное значение подвижности носителей заряда в антимониде индия составляет 7,7 $м^2/(В\cdotс)$, что близко к полученному значению, однако не входит в рамки оценённой погрешности. 
Концентрация носителей заряда равна $1,6\cdot10^{22}~ м^{-3}$, это примерно в три раза больше полученного значения.








\end{document}