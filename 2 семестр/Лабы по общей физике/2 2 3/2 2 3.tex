\documentclass[a4paper, 12pt]{article}
\usepackage[a4paper,top=1.5cm, bottom=1.5cm, left=1cm, right=1cm]{geometry}
\usepackage{cmap}					% поиск в PDF
\usepackage{mathtext} 				% русские буквы в формулах
\usepackage[T2A]{fontenc}			% кодировка
\usepackage[utf8]{inputenc}			% кодировка исходного текста
\usepackage[english,russian]{babel}	% локализация и переносы

\usepackage{amsmath}
\usepackage{indentfirst}
\usepackage{longtable}
\usepackage{graphicx}
\usepackage{array}

\usepackage{wrapfig}
\usepackage{siunitx} % Required for alignment
\usepackage{subfigure}
\usepackage{multirow}
\usepackage{rotating}
\usepackage{caption}

\graphicspath{{.}}


\title{\begin{center}Лабораторная работа №2.2.3\end{center}
Измерение теплопроводности воздуха при атмосферном давлении}
\author{Струков О. И. \\ Б04-404}
\date{}

\begin{document}
    \pagenumbering{gobble}
    \maketitle
    \newpage
    \pagenumbering{arabic}

    \textbf{Цель работы:} измерить коэффициент теплопроводности воздуха при атмосферном давлении в зависимости от температуры.

    \textbf{Оборудование:} цилиндрическая колба с натянутой по оси нитью; термостат; вольтметр и амперметр; источник постоянного напряжения; магазин сопротивлений.

    \section*{Теоретические сведения}

        \textit{Теплопроводность} — это процесс передачи тепловой энергии от нагретых частей системы к холодным за счёт хаотического движения частиц среды (молекул, атомов и т.п.). В газах теплопроводность осуществляется за счёт  непосредственной передачи кинетической энергии от быстрых молекул к медленным при их столкновениях. Перенос тепла описывается законом Фурье, утверждающим, что плотность потока энергии $\overline{q} = -k \nabla T$, где $k \left[ \dfrac{\text{Вт}}{\text{м} \cdot \text{К}} \right]$ - \textit{коэффициент теплопроводности}.

        Молекулярно-кинетическая теория дает следующую оценку для коэффициента теплопроводности газов:

        \begin{align}
            k &\sim \lambda \overline{\nu} \cdot n c_V \label{k}
        \end{align}

        С помощью некоторых преобразований мы получаем, что

        \begin{align}
            Q &= \dfrac{2 \pi L}{\ln \dfrac{r_0}{r_1}} k  \cdot \Delta T \label{Q}
        \end{align}


        \section*{Экспериментальная установка}
        \begin{wrapfigure}{r}{0.4\textwidth}
        \begin{center}
            \includegraphics[width = 0.3\textwidth]{ustanovka.png}
        \end{center}
        \textbf{\caption{Схема установки}}
        \end{wrapfigure}
        Схема установки приведена на рис. 1. На оси полой цилиндрической трубки с внутренним диаметром $2r_0 \sim 1$ см размещена металлическая нить диаметром $2r_1 \sim 0,05$ мм и длиной $L \sim 40$ см (материал нити и точные геометрические размеры указаны в техническом описании установки). Полость трубки заполнена воздухом (полость через небольшое отверстие сообщается с атмосферой). Стенки трубки помещены в кожух, через которых пропускается вода из термостата, так что их температура $t_0$ поддерживается постоянной. Для предотвращения конвекции трубка расположена вертикально.

        Металлическая нить служит как источником тепла, так и датчиком температуры (термометром сопротивления). По пропускаемому через нить постоянному току $I$ и напряжению $U$ на ней вычисляется мощность нагрева по закону Джоуля–Ленца: $Q = UI$, и сопротивление нити по закону Ома: $R = \dfrac{U}{I}$.

        Сопротивление нити является однозначной функцией её температуры $R (t)$.
        Эта зависимость может быть измерена с помощью термостата по экстраполяции мощности нагрева к нулю $Q \rightarrow 0$, когда температура нити и стенок совпадают $t_1 \approx t_0$. Альтернативно, если материал нити известен, зависимость его удельного сопротивления от температуры может найдена по справочным данным.

        На рис. 2 представлена схема электрической установки:
        \begin{figure}
            \begin{center}
                \includegraphics[width = 0.5\textwidth]{Схема.jpg}
            \end{center}
            \textbf{\caption{Электрическая схема измерения сопротивления нити и мощности нагрева}}
        \end{figure}

        Схема рис. 2 предусматривает использование одного вольтметра и эталонного сопротивления $R_{\text{э}} \sim 10$ Ом (точное значение $R_{\text{э}}$ и его класс точности указаны в техническом описании установки), включённого последовательно с нитью. В положении переключателя 2 вольтметр измеряет напряжение на нити, а в положении 1 — напряжение на $R_{\text{э}}$, пропорциональное току через нить. Для исключения влияния контактов и подводящих проводов эталонное сопротивление $R_{\text{э}}$ также необходимо подключать в цепь по четырёхпроводной схеме. Ток в цепи в обеих схемах регулируется с помощью реостата или магазина сопротивлений $R_{\text{м}}$, включённого последовательно с источником напряжения.

    \section*{Методика измерений}

        Принципиально неустранимая систематическая ошибка измерения температуры с помощью термометра сопротивления возникает из-за необходимости пропускать через резистор (нить) измерительный ток. Чем этот ток выше, тем с большей точностью будет измерен как он сам, так и напряжение. Однако при этом квадратично возрастает выделяющаяся на  резисторе мощность $Q = UI = I^2R$. Следовательно, температура резистора становится выше, чем у объекта, температуру которого надо измерить. Измерения же при малых токах не дают достаточной точности (в частности, из-за существенного вклада термоэлектрических явлений в проводниках и контактах). Эта проблема решается построением нагрузочной кривой - зависимости измеряемого сопротивления $R$ от выделяющейся в нём мощности $R(Q)$, с последующей экстраполяцией к нулевой мощности $Q \to 0$ для определения сопротивления $R_0 = R(0)$, при котором его температура равна температуре измеряемого объекта. Кроме того, в данной работе измерение нагрузочных кривых позволяет в ходе эксперимента получить температурную зависимость сопротивления нити, так как при $Q \to 0$ температура нити равна температуре термостата ($T \approx T_0$). В исследуемом интервале температур (20-80 $^0C$) зависимость сопротивления от температуры можно с хорошей точностью аппроксимировать линейной функцией:

        \begin{align}
            R(t) = R_{273} \cdot (1 + \alpha t) \label{RT}
        \end{align}

        где $\alpha = \dfrac{1}{R_{273}} \dfrac{dR}{dT}$ - температурный коэффициент сопротивления материала.

\newpage{}
    \section*{Ход работы}
\begin{enumerate}
    \item В первую очередь были проведены предварительные расчёты параметров опыта: максимально допустимый перегрев нити относительно термостата был принят равным $\Delta t_{max} = 30~^oC$. По формуле (\ref{Q}) оценена максимальная мощность нагрева $Q_{max} = 0,377 ~Вт$. Отсюда найдена максимальная сила тока $I_{max} = 0,137~А$, которая не должна быть превышена в ходе эксперимента.
    \item Экспериментальная установка была подготовлена к работе: на магазине было выставлено максимальное сопротивление (99 кОм), чтобы ток в цепи при её замыкании был минимален. Были включены и настроены вольтметр и амперметр, включены источник питания и термостат.
    \item Далее были проведены измерения силы тока и напряжения в зависимости от выставленного сопротивления при пяти различных температурах. В таблице приведены значения сопротивления нити R в зависимости от мощности нагрева Q.
    \begin{table}[!ht]
        \centering
        \begin{tabular}{|c|c|c|c|c|c|c|c|c|c|}
        \hline
        \multicolumn{2}{|c|}{23 $^oC$} & \multicolumn{2}{c|}{38 $^oC$} & \multicolumn{2}{c|}{51 $^oC$} & \multicolumn{2}{c|}{64 $^oC$} & \multicolumn{2}{c|}{78 $^oC$} \\ \hline
            R, Ом & Q, мВт & R, Ом & Q, мВт & R, Ом & Q, мВт & R, Ом & Q, мВт & R, Ом & Q, мВт \\ \hline
            19,7 & 0,661 & 20,8 & 3,74 & 21,8 & 3,86 & 22,6 & 0,75 & 23,6 & 0,3 \\ \hline
            19,9 & 32,16 & 20,9 & 32,99 & 21,9 & 33,6 & 22,8 & 34,2 & 23,8 & 34,8 \\ \hline
            20,2 & 89,04 & 21,2 & 89,5 & 22,1 & 89,8 & 23,1 & 89,9 & 24,1 & 90,1 \\ \hline
            20,8 & 195,6 & 21,5 & 147,8 & 22,4 & 146,3 & 23,3 & 145,4 & 24,3 & 152,3 \\ \hline
            21,03 & 238,3 & 21,8 & 192,4 & 22,7 & 189,8 & 23,5 & 187,2 & 24,5 & 184,1 \\ \hline
            21,12 & 258,8 & 21,99 & 232,9 & 22,9 & 228,6 & 23,7 & 224,4 & 24,7 & 219,8 \\ \hline
            21,16 & 268,3 & 22,1 & 257,6 & 22,97 & 252,3 & 23,8 & 246,9 & 24,8 & 241,1 \\ \hline
        \end{tabular}
    \end{table}
    \item По результатам измерений был построен график зависимости R(Q) для пяти различных температур:
    \begin{figure}[ht]
        \center{\includegraphics[scale=0.78]{Сопротивления 2.pdf}}
        \caption{График зависимости R(Q)}
    \end{figure}
    \item Для каждой прямой был найден угловой коэффициент наклона $\dfrac{dR}{dQ}$ и точка пересечения с осью ординат $R_0$, в которой температура нити соответствовала бы температуре термостата. Погрешности оценены с помощью МНК. Результаты внесены в таблицу: 
    \begin{table}[!ht]
        \centering
        \begin{tabular}{|c|c|c|}
            \hline

            $t, ^oC$ & $dR/dQ$, Ом/Вт & $R_0$, Ом\\ \hline
            $23$ & $0,603 \pm 0,02$ & $19,58 \pm 0,03$\\ \hline
            $38$ & $0,516 \pm 0,04$ & $20,78 \pm 0,03$\\ \hline
            $51$ & $0,507 \pm 0,05$ & $21,71 \pm 0,04$\\ \hline
            $64$ & $0,486 \pm 0,05$ & $22,60 \pm 0,04$\\ \hline
            $78$ & $0,467 \pm 0,05$ & $23,65 \pm 0,04$\\ \hline

        \end{tabular}
    \end{table}
    \item По полученным данным был построен график зависимости сопротивления проволоки при отсутствии нагрева от температуры. Видно, что все точки примерно ложатся на одну прямую.
    \begin{figure}[ht]
        \center{\includegraphics[scale=0.8]{R(T) 2.pdf}}
        \caption{График зависимости R(t)}
    \end{figure}
    \item С помощью МНК был определён наклон полученной прямой: $\dfrac{dR}{dT} = 0,51 \pm 0,04$ Ом/К.
    \item С использованием полученных данных была найдена зависимость выделяющейся на нити мощности $Q$ от её перегрева $\Delta T$ относительно стенок. Затем с помощью формулы (\ref{Q}) были определены коэффициенты теплопроводности воздуха $k$ при каждой из исследуемых температур:

    \begin{align*}
        \frac{dQ}{d(\Delta T)} &= \frac{dR_0}{dT} / \frac{dR}{dQ}, \qquad k = \frac{dQ}{d(\Delta T)} / \frac{2 \pi L}{ln \frac{r_0}{r_1}} = \frac{dR_0}{dT} \frac{ln \frac{r_0}{r_1}}{2 \pi L} / \frac{dR}{dQ}\\
    \end{align*}

    \begin{table}[!ht]
        \centering
        \begin{tabular}{|c|c|}
            \hline

            $t, ^oC$ & $k, 10^{-3}\frac{Вт}{м \cdot с}$\\ \hline
            $23$ & $23,8$\\ \hline
            $38$ & $27,9$\\ \hline
            $51$ & $28,4$\\ \hline
            $64$ & $29,6$\\ \hline
            $78$ & $30,9$\\ \hline
        \end{tabular}
    \end{table}
    \item Построен график зависимости теплопроводности воздуха от его температуры $k(t)$
    \begin{figure}[ht]
        \center{\includegraphics[scale=0.7]{Теплопроводность.pdf}}
        \caption{График зависимости $k(t)$}
    \end{figure}
    \item  Исходя из предположения, что коэффициент теплопроводности газа степенным образом зависит от абсолютной температуры, был построен график $\ln k (\ln T)$ и из него определён показатель степени $\beta$ для зависимости $k \sim T^{\beta}$.
    \begin{figure}[ht]
        \center{\includegraphics[scale=0.7]{ln.pdf}}
        \caption{График зависимости $\ln k(\ln T)$}
    \end{figure}

    \newpage{}
    Таким образом, $\beta \approx 0,58$

    Поскольку коэффициент теплопроводности газа пропорционален $\sqrt{T}$, искомый показатель степени должен быть $ \beta = 0,5$.



\end{enumerate}
\section*{Вывод}
В результате выполнения работы была исследована зависимость сопротивления платиновой проволоки от температуры и мощности проходящего через неё тока, после чего был определён коэффициент теплопроводности воздуха при пяти различных температурах, а также найден показатель степени $ \beta $ зависимости $k \sim T^{\beta}$. Полученные значения близки к теоретическим предположениям.


\end{document}