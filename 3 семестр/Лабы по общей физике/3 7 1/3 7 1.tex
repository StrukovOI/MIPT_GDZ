\documentclass[12pt, a4paper]{article}
\usepackage{geometry}
\usepackage{amsmath, amsfonts, amssymb, amsthm} % стандартный набор AMS-пакетов для математ. текстов
\usepackage{mathtext}
\usepackage[utf8]{inputenc} % кодировка utf8
\usepackage[russian]{babel} % русский язык
\usepackage{booktabs}
\usepackage[pdftex,dvipsnames]{xcolor} % работа с цветами
\usepackage[pdftex]{graphicx} % графика (картинки)
\usepackage{tikz,pgfplots} % рисунки
\pgfplotsset{compat=1.18}
\geometry{left=15mm, right=15mm, top=20mm, bottom=20mm}
\usepackage{indentfirst}
%\usepackage[labelfont=bf,labelsep=endash,skip=3pt]{caption} % подпись картинок
% \usepackage{fancyhdr,pageslts} % настройка колонтитулов
\usepackage{enumitem} % работа со списками
\usepackage{floatrow,multicol,multirow,longtable,hhline} % работа с таблицами
\usepackage{float,wrapfig} % плавающие объекты
\usepackage{tcolorbox} % рамка вокруг текста
%\usepackage[calc]{datetime2} % дата
\usepackage{bm} % жирное начертание в формулах
\usepackage{physics} % физический пакет
\DeclareMathAlphabet\mathbfcal{OMS}{cmsy}{b}{n}
\usepackage{pgfornament} % красивые рюшечки и вензеля
\usepackage{mdframed}
\usepackage{derivative}
\usepackage{mathrsfs} %EDS
\usepackage{soul} % strikethorugh
\usepackage{siunitx}


\newcommand{\V}{~\mathrm{V}}
\newcommand{\mV}{~\mathrm{mV}}
\newcommand{\A}{~\mathrm{A}}
\newcommand{\mA}{~\mathrm{mA}}
\newcommand{\uT}{~\mathrm{\mu T}}
\newcommand{\mT}{~\mathrm{mT}}
\newcommand{\kHz}{~\mathrm{kHz}}
\newcommand{\Hz}{~\mathrm{Hz}}
\newcommand{\mm}{~\mathrm{mm}}


\begin{document}
\title{\begin{center}Лабораторная работа № 3.7.1\end{center}
Скин-эффект}
\author{Струков О. И. \\ Б04-404}
\date{}

\thispagestyle{empty}
\pagenumbering{gobble}
\maketitle
\newpage
\pagenumbering{arabic}

% \section{Цель работы}
% Исследовать явление проникновение переменного магнитного поля в медный полый цилиндр. 
\textbf{Цель работы:} Исследовать явление проникновения переменного магнитного поля в медный полый цилиндр.
\textbf{В работе используются:} генератор сигналов АКИП–3420, соленоид, намотанный на полый цилиндрический каркас, медный экран в виде полого цилиндра, измерительная катушка, амперметр, вольтметр, двухканальный осциллограф GOS–620, RLC-метр.

\section*{Теоретическая часть}
\subsection*{Скин-эффект для полупространства}
\vspace{1cm}
\begin{wrapfigure}{l}{0.3\textwidth}
  \begin{center}
    \includegraphics[width=0.8\textwidth]{poluprostranstvo}
  \end{center}
%   \caption{Скин-эффект в полупространстве}\label{fig:poluprostranstvo}
\end{wrapfigure}

Рассмотрим квазистационарное поле внутри проводящей среды в простейшем плоском случае.
Пусть вектор $\vb*{E}$ направлен всюду вдоль оси $y$ и зависит только от координаты $x$, т. е. ${E_x} = {E_z} \equiv 0$, $E_y=E_y(x,t)$.
В квазистационарном приближении
\begin{equation*}
  \grad \times \vb*{H} = \sigma \vb*{E}
\end{equation*}

Преобразуя это уравнение, можно получить уравнение, схожее с уравнением диффузии:
\begin{equation}
  \grad^2\vb*{H}=\sigma\mu\mu_0 \frac{\partial \vb*{H}}{\partial t}\label{eq:laplacian_H}
\end{equation}
Точно такое же уравнение имеет место и для вектора $E:$
\begin{equation}
  \grad^2\vb*{E}=\sigma\mu\mu_0 \frac{\partial \vb*{E}}{\partial t}\label{eq:diffusion}
\end{equation}

Подставляем в (\ref{eq:diffusion}) наше электрическое поле $E_y=E_y(x,t)$
\begin{equation}
  \frac{\partial^2 E_y}{\partial x^2} = \sigma\mu\mu_0\frac{\partial E_y}{\partial t}
  \label{eq:diffusion_chastni}
\end{equation}
Если $E_y(0,t)=E_0 e^{i\omega t}$ то решением (\ref{eq:diffusion_chastni}) будет функция вида
\begin{equation}
  E_y(x,t)=E_0 e^{-x/\delta} e^{i(\omega t - x/\delta)}
  \label{eq:skin_effect_poluprostranstvo}
\end{equation}
где
\begin{equation}
  \delta = \sqrt{\frac{2}{\omega\sigma\mu\mu_0}}
  \label{eq:delta}
\end{equation}

\newpage
\subsection*{Скин-эффект в тонком полом цилиндре}
\vspace{0cm}
\begin{wrapfigure}[30]{l}{0.3\textwidth}
  \vspace{-7mm}
  \begin{center}
    \includegraphics[width=0.8\textwidth]{cilindr}
  \end{center}
%   \caption{Эл-магнитные поля в цилиндре и у стенки}\label{fig:cilindr}

  \begin{center}
    \includegraphics[width=0.8\textwidth]{stenka}
  \end{center}
  \label{fig:stenka}
\end{wrapfigure}

Перейдем теперь к описанию теории в нашей работе. Из соображении симметрии и
непрерывности соответствующих компонет векторов $\vb*{E}$ и $\vb*{H}$ можем сказать что
\begin{equation*}
  H_z = H(r)e^{i\omega t} \text{, } E_\varphi = E(r)e^{i\omega t}
\end{equation*}
и при этом функции $H(r)$ и $E(r)$ непрерывны.

Внутри цилиндра токов нет, следовательно $H(r)=H_1=\text{const}$ внутри цилиндра.
По теореме об электромагнитной индукции
\begin{equation*}
  E(r) = -\frac{1}{2}\mu_0 r \cdot i \omega H_1
\end{equation*}
откуда мы получаем граничное условие
\begin{equation}
  E_1=E(a)= -\frac{1}{2}\mu_0 a \cdot i \omega H_1
  \label{eq:granichnoe_uslovie_E}
\end{equation}

В прближении $h \ll a$ можем пренебречь кривизной стенки и смоделировать
его бесконечной полосой. Тогда, надо решить уравнение (\ref{eq:laplacian_H})
с граничными условиями. Решая уравнение получим связь полей $H_1$
(поле внутри цилиндра которое мы будем измерять) и $H_0$, которое колебается с частотой
$\omega$

\begin{equation}
  H_1 = \frac{H_0}{\ch(\alpha h) + \frac{1}{2} \alpha a \sh(\alpha h)}
  \text{\ \ \ }
  \alpha = \sqrt{i\omega \sigma \mu_0} = \frac{\sqrt{2}}{\delta}e^{i\pi/4}
  \label{eq:svyaz_poley}
\end{equation}

из этой формулы получим сколько по фазе отстает поле $H_1$ от $H_0$. При $\delta \ll h$
(высокачастотная область)

\begin{equation}
  \psi \approx \frac{\pi}{4} + \frac{h}{\delta} =
  \frac{\pi}{4} + h \sqrt{\frac{\omega \sigma \mu_0}{2}}
  \label{eq:faza_high_freq}
\end{equation}

При $\delta \gg h$ (низкочастотная область)

\begin{equation}
  \tg \psi \approx \frac{ah}{\delta^2} = \pi a h \sigma \mu \mu_0 f
  \label{eq:phase_low_freq}
\end{equation}
и кроме того
\begin{equation}
  \left(\frac{|H_1|}{|H_0|}\right)^2 = \dfrac{1}{1 + \frac{1}{4}(ah\sigma \mu_0 \omega)^2}
  \label{eq:low_freq_ampl}
\end{equation}

\subsection*{Влияние скин-эффекта на индуктивность катушки}
Из-за скин эффекта индуктивность соленоида с медным цилиндрическим экраном
внутри будет зависеть от частоты тока. На высоких частотах магнитное поле не проника
ет внутрь соленоида (за экран), поэтому суммарный магнитный поток, пронизывающий
катушку, уменьшается, и, соответственно, уменьшается и индуктивность катушки. Можно показать,
что индуктивность катушки зависит от частоты как:
\begin{equation}
  \frac{L_\text{max} - L}{L - L_\text{min}} = (\pi ah \mu_0 \sigma f)^2
  \label{eq:ind_of_freq}
\end{equation}
\newpage
\section*{Лабораторная установка}
% \textbf{В работе используются:} генератор сигналов АКИП–3420, соленоид,
% намотанный на полый цилиндрический каркас, медный экран в виде полого цилиндра,
% измерительная катушка, ам­перметр, вольтметр, двухканальный осциллограф GOS–620, RLC-метр.

\begin{wrapfigure}[13]{l}{0.4\textwidth}
  \centering
  \vspace{-8mm}
  \begin{center}
    \includegraphics[width=0.8\textwidth]{setup}
  \end{center}
  \caption{Установка}\label{fig:ustanovka}
\end{wrapfigure}

Переменное магнитное поле создается соленоидом 1, на который подается переменный ток со звукового генератора ЗГ. Внутри соленоида расположен медный экран 2. Магнитное поле внутри цилиндра измеряется катушкой 3. Напряжение на катушке пропорциональна производной $\dot{B_1}(t)$
\begin{equation*}
  U(t) \propto \dot{B_1}(t) = -i\omega \mu \mu_0 H_1 e^{i\omega t}
\end{equation*}
Поле внутри цилиндра пропорциональна току через соленоид
\begin{equation*}
  H_0(t) \propto I(t)
\end{equation*}
Отсюда несложно увидеть, что
\begin{equation}
  \frac{\abs{H_1}}{\abs{H_0}} = c \cdot \frac{U}{f I} = \xi / \xi_0
  \label{eq:ampl_share}
\end{equation}
где константу  $\xi_0$ можно определить из условия $\abs{H_1}/\abs{H_0} \rightarrow 1$ при
$\nu \rightarrow 0$.\\

При измерениях разности фаз нужно учесть, что первый сигнал на осциллографе
пропорционален магнитному полю снаружи, а второй пропорционален производному
поля внутри цилиндра по времени, поэтому измеренная на осциллографе разность фаз $\varphi$ будет на $\frac{\pi}{2}$ больше реальной $\psi$:
\[\varphi = \psi + \frac{\pi}{2}\]


\newpage


\section*{Ход работы}
По параметрам установки $h = \delta = 1,5$ мм, приняв проводимость меди для оценки равной $\sigma \sim 5\cdot10^{7}$ Сименс/м, а $\mu \approx 1$, из формулы (5) была найдена частота $\nu_h \approx 2250$ Гц, при которой толщина стенок экрана равна скиновой длине.

В области низких частот 20 -- 160 Гц была получена зависимость отношения $\xi = U/(\nu I)$ от частоты $\nu$ и построен график зависимости $1/\xi^2 = f(\nu^2)$. Хорошо видно, что зависимость является линейной.

\begin{center}
\begin{tikzpicture}
\begin{axis}[
    width=14cm,
    height=8cm,
    xlabel={$\nu^2$, Гц$^2$},
    ylabel={$1/\xi^2$},
    grid=major,
    legend pos=north west,
    title={Зависимость $1/\xi^2$ от $\nu^2$ в области низких частот},
    xmin=0,
    xmax=27000,
    ymin=4500,
    ymax=11000
]

% Данные измерений с погрешностями
\addplot+[only marks, error bars/.cd, y dir=both, y explicit] coordinates {
    (400, 5043.113317)  +- (0, 101.366)
    (900, 5170.188778)  +- (0, 103.921)
    (1600, 5336.809723) +- (0, 107.270)
    (2500, 5543.229479) +- (0, 111.419)
    (3600, 5763.735909) +- (0, 115.851)
    (4900, 6016.564237) +- (0, 120.933)
    (6400, 6301.962143) +- (0, 126.669)
    (8100, 6626.473969) +- (0, 133.192)
    (10000, 6983.896687) +- (0, 140.346)
    (12100, 7375.005787) +- (0, 148.238)
    (14400, 7783.832351) +- (0, 156.455)
    (19600, 8747.649658) +- (0, 175.828)
    (25600, 9857.533446) +- (0, 198.136)
};

% Линейная аппроксимация по МНК
\addplot[domain=0:27000, red, thick] {5005.93 + 0.1896*x};
\legend{Экспериментальные точки, Линейная аппроксимация}
\end{axis}
\end{tikzpicture}
\end{center}


\subsection*{Определение $\xi_0$}
Методом наименьших квадратов получена линейная зависимость:
\[
\frac{1}{\xi^2} = (5005,93 \pm 15,76) + (0,1896 \pm 0,0013) \cdot \nu^2
\]
Экстраполяцией её к точке $\nu = 0$ определено:
\[
\frac{1}{\xi_0^2} = 5005,93 \pm 15,76 \quad \Rightarrow \quad \xi_0 = \frac{1}{\sqrt{5005,93}} = 0,01414
\]
С учётом погрешности $\xi_0 = 0,01414 \pm 0,00002$

\section*{Определение проводимости меди различными методами}

% В данной работе проводимость меди определялась четырьмя различными методами, основанными на особенностях проявления скин-эффекта. Каждый метод использует свои экспериментальные данные и теоретические соотношения, что позволяет провести взаимную проверку результатов.

\subsection*{Зависимость $1/\xi^2 = f(\nu^2)$ в области низких частот}

Данный метод основан на использовании теоретической зависимости амплитуды магнитного поля внутри цилиндра от частоты в низкочастотной области ($\delta \gg h$). Согласно формуле (\ref{eq:low_freq_ampl}), квадрат отношения амплитуд полей связан с частотой следующим образом:
\[
\left(\frac{|H_1|}{|H_0|}\right)^2 = \dfrac{1}{1 + \frac{1}{4}(ah\sigma \mu_0 \omega)^2}
\]

С учетом соотношения (\ref{eq:ampl_share}), связывающего измеряемую величину $\xi$ с отношением амплитуд полей, получаем линейную зависимость $1/\xi^2$ от $\nu^2$, угловой коэффициент которой содержит информацию о проводимости материала.

Используя найденный угловой коэффициент линейной зависимости $k = 0,1896 \pm 0,0013$ и значение $\xi_0$, проводимость меди рассчитывается по формуле:
\[
\sigma = \frac{\xi_0}{\pi a h \mu_0} \sqrt{k}
\]

Подставляя геометрические параметры установки ($2a = 0,045$ м, $h = 0,0015$ м) и $\mu_0 = 4\pi \cdot 10^{-7}$ Гн/м, получаем:
\[
\sigma = (4,62 \pm 0,06) \cdot 10^7 \, \text{См/м}
\]

\subsection*{Зависимость $\tg\psi = f(\nu)$ в области низких частот}

Этот метод основан на измерении фазового сдвига между магнитными полями внутри и вне цилиндра в низкочастотной области. Согласно теории, при $\delta \gg h$ фазовый сдвиг определяется формулой (\ref{eq:phase_low_freq}):
\[
\tg \psi \approx \pi a h \sigma \mu \mu_0 f
\]

\begin{center}
\begin{tikzpicture}
\begin{axis}[
    width=14cm,
    height=8cm,
    xlabel={$\nu$, Гц},
    ylabel={$\tg\psi$},
    grid=major,
    legend pos=north west,
    title={Зависимость $\tg\psi$ от частоты $\nu$ в области низких частот},
    xmin=0,
    xmax=550,
    ymin=0,
    ymax=7
]

% Экспериментальные точки
\addplot+[only marks] coordinates {
    (100,0.4700)
    (120,0.7265)
    (140,0.8107)
    (160,1.0000)
    (180,1.2602)
    (200,1.3764)
    (300,2.5232)
    (400,5.7979)
    (500,6.3138)
};

% Аппроксимирующая прямая для точек 100-200 Гц
\addplot[domain=100:200, red, thick] {0.009032*x - 0.41417};
\legend{Экспериментальные точки, Линейная аппроксимация}
\end{axis}
\end{tikzpicture}
\end{center}
Методом наименьших квадратов по точкам в диапазоне 100-200 Гц определён коэффициент наклона
\[
k = 0,00903 \pm 0,00056 \, \text{Гц}^{-1}
\]

Проводимость меди рассчитывается по формуле:
\[
\sigma = \frac{k}{\pi ah\mu_0}
\]

Подставляя значения параметров, получаем:
\[
\sigma = (6,78 \pm 0,56) \cdot 10^7 \, \text{См/м}
\]

\subsection*{Зависимость $\psi - \pi/4 = f(\sqrt{\nu})$ в области высоких частот}

Данный метод использует поведение фазового сдвига в высокочастотной области ($\delta \ll h$), описываемое формулой (\ref{eq:faza_high_freq}):
\[
\psi \approx \frac{\pi}{4} + h \sqrt{\frac{\omega \sigma \mu_0}{2}} = \frac{\pi}{4} + h \sqrt{\pi \sigma \mu_0} \cdot \sqrt{\nu}
\]

\begin{center}
\begin{tikzpicture}
\begin{axis}[
    width=14cm,
    height=8cm,
    xlabel={$\sqrt{\nu}$, Гц$^{1/2}$},
    ylabel={$\psi - \pi/4$, рад},
    grid=major,
    legend pos=north west,
    title={Зависимость $\psi - \pi/4$ от $\sqrt{\nu}$},
    xmin=0,
    xmax=180,
    ymin=-0.5,
    ymax=4
]

% Данные низких частот (п.4)
\addplot+[only marks, blue] coordinates {
    (10.00, -0.3456)
    (10.95, -0.1571)
    (11.83, -0.0943)
    (12.65, 0.0000)
    (13.42, 0.1255)
    (14.14, 0.1571)
    (17.32, 0.4084)
    (20.00, 0.6597)
    (22.36, 0.7226)
};

% Данные высоких частот (п.5)
\addplot+[only marks, red] coordinates {
    (31.62, 0.7854)
    (36.06, 0.8482)
    (40.00, 0.9111)
    (44.72, 1.0367)
    (50.99, 1.1936)
    (57.45, 1.3192)
    (65.57, 1.4765)
    (74.16, 1.7279)
    (83.67, 1.9477)
    (94.87, 2.4184)
    (106.30, 2.7010)
    (120.42, 3.1416)
    (136.01, 3.3917)
    (153.30, 3.3298)
    (173.21, 3.5182)
};

% Аппроксимирующая прямая для высоких частот
\addplot[domain=0:180, green, thick] {0.02310*x};
\legend{Низкие частоты, Высокие частоты, Аппроксимация}
\end{axis}
\end{tikzpicture}
\end{center}
Методом наименьших квадратов по точкам в диапазоне высоких частот ($\nu > 2000$ Гц) получено
\[
k = 0,02310 \pm 0,00067 \, \text{рад/Гц}^{1/2}
\]

Проводимость меди определяется по формуле:
\[
\sigma = \frac{k^2}{h^2 \pi \mu_0}
\]

Результат расчета:
\[
\sigma = (6.01 \pm 0.47) \cdot 10^7 \, \text{См/м}
\]

\subsection*{Зависимость индуктивности катушки от частоты}

Данный метод основан на влиянии скин-эффекта на индуктивность соленоида с цилиндрическим экраном внутри. Согласно формуле (\ref{eq:ind_of_freq}), зависимость индуктивности от частоты описывается соотношением:
\[
\frac{L_\text{max} - L}{L - L_\text{min}} = (\pi ah \mu_0 \sigma f)^2
\]

\begin{center}
\begin{tikzpicture}
\begin{axis}[
    width=14cm,
    height=8cm,
    xlabel={$\nu$, Гц},
    ylabel={$L$, мГн},
    grid=major,
    legend pos=north east,
    title={Зависимость индуктивности катушки от частоты},
    xmin=0,
    xmax=8000,
    ymin=4,
    ymax=20
]

% Плавная красная кривая через точки измерений
\addplot[red, smooth, thick, no marks] coordinates {
    (40,19.2)
    (100,15.3)
    (200,11.27)
    (300,8.9)
    (400,7.8)
    (500,7.1)
    (600,6.6)
    (800,5.6)
    (1500,5.67)
    (2000,5.69)
    (2500,5.46)
    (4000,5.05)
    (6000,5.48)
    (7500,5.19)
};

% Синие точки измерений
\addplot[blue, only marks, mark=*, mark size=2pt] coordinates {
    (40,19.2)
    (100,15.3)
    (200,11.27)
    (300,8.9)
    (400,7.8)
    (500,7.1)
    (600,6.6)
    (800,5.6)
    (1500,5.67)
    (2000,5.69)
    (2500,5.46)
    (4000,5.05)
    (6000,5.48)
    (7500,5.19)
};

\end{axis}
\end{tikzpicture}
\end{center}

Максимальное значение индуктивности: $L_{\text{max}} = 19,2$ мГн

Минимальное значение индуктивности: $L_{\text{min}} = 5,05$ мГн

\subsection*{График зависимости $\frac{L_{\text{max}} - L}{L - L_{\text{min}}} = f(\nu^2)$}

\begin{center}
\begin{tikzpicture}
\begin{axis}[
    width=14cm,
    height=8cm,
    xlabel={$\nu^2$, Гц$^2$},
    ylabel={$\frac{L_{\text{max}} - L}{L - L_{\text{min}}}$},
    grid=major,
    legend pos=north west,
    title={Зависимость $\frac{L_{\text{max}} - L}{L - L_{\text{min}}}$ от $\nu^2$},
    xmin=0,
    xmax=700000,
    ymin=0,
    ymax=30
]

% Данные для первых 8 точек
\addplot+[blue, only marks] coordinates {
    (1600,0)
    (10000,0.38049)
    (40000,1.27492)
    (90000,2.67532)
    (160000,4.14545)
    (250000,6.90244)
    (360000,10.12903)
    (640000,24.72727)
};

% Линейная аппроксимация
\addplot[domain=0:700000, red, thick] {3.495e-5*x};
\legend{Экспериментальные точки, Линейная аппроксимация}
\end{axis}
\end{tikzpicture}
\end{center}
Методом наименьших квадратов по первым 8 точкам, лежащим примерно на одной прямой, получен
\[
k = (3,50 \pm 0,29) \cdot 10^{-5} \, \text{Гц}^{-2}
\]

Проводимость меди рассчитывается по формуле:
\[
\sigma = \frac{\sqrt{k}}{\pi a h \mu_0}
\]

Результат:
\[
\sigma = (4.44 \pm 0.27) \cdot 10^7 \, \text{См/м}
\]

\section*{Сравнение теоретической и экспериментальной зависимостей коэффициента ослабления поля}

Экспериментальные значения коэффициента ослабления поля были определены по формуле:
\[
\frac{|H_1|}{|H_0|}_{\text{эксп}} = \frac{\xi}{\xi_0} = \frac{U/(\nu I)}{\xi_0}
\]
где $\xi_0 = 0,01414$.

Для расчёта теоретической зависимости использовалась формула (7):
\[
\frac{H_1}{H_0} = \frac{1}{\ch(\alpha h) + \frac{1}{2} \alpha a \sh(\alpha h)},
\quad \alpha = \sqrt{i\omega\sigma\mu_0},
\]
% где $\omega = 2\pi\nu$, $a = 0,0225$~м --- радиус цилиндра, $h = 0,0015$~м --- толщина стенки, $\mu_0 = 4\pi \cdot 10^{-7}$~Гн/м.

% Для вычисления гиперболических функций комплексного аргумента использовались соотношения:
% \[
% \ch(x+iy) = \ch x \cos y + i\sh x \sin y,
% \]
% \[
% \sh(x+iy) = \sh x \cos y + i\ch x \sin y.
% \]

В качестве значений проводимости $\sigma$ были взяты минимальное, максимальное значения, полученные в работе, и табличное:
\[
\sigma_{\text{min}} = 4,44 \cdot 10^7~\text{См/м},
\quad
\sigma_{\text{max}} = 6,78 \cdot 10^7~\text{См/м},
\quad
\sigma_{\text{табл}} = 5,60 \cdot 10^7~\text{См/м}.
\]

Результаты представлены на графике:

    \begin{figure}[H]
    \centering
    \includegraphics[width=1\linewidth]{12.pdf}
\end{figure}



Наиболее близкое совпадение с экспериментальными данными демонстрирует теоретическая кривая, построенная для проводимости $\sigma = 4,44 \cdot 10^7$~См/м, определённой по зависимости индуктивности катушки от частоты. Это говорит о наиболее высокой точности данного метода определения проводимости.
\newpage
\section*{Выводы}
В ходе работы была исследована проводимость меди экрана с использованием четырёх различных методов, основанных на явлении скин-эффекта.

\begin{table}[h]
\centering
\begin{tabular}{|c|c|}
\hline
\textbf{Метод} & \textbf{$\sigma$, См/м} \\
\hline
По зависимости $1/\xi^2 = f(\nu^2)$ & $(4,62 \pm 0,06) \cdot 10^7$ \\
\hline
По зависимости $\tg\psi = f(\nu)$ & $(6,78 \pm 0,56) \cdot 10^7$ \\
\hline
По зависимости $\psi - \pi/4 = f(\sqrt{\nu})$ & $(6,01 \pm 0,47) \cdot 10^7$ \\
\hline
По зависимости $\frac{L_{\text{max}} - L}{L - L_{\text{min}}} = f(\nu^2)$ & $(4,44 \pm 0,27) \cdot 10^7$ \\
\hline
Табличное значение & $5,60 \cdot 10^7$\\
\hline
\end{tabular}
\end{table}

Значения проводимости, полученные разными методами, лежат в диапазоне от $4,44\cdot 10^7$ до $6,78\cdot 10^7$ См/м. Наиболее близкое к табличному значению $(5,6\cdot 10^7$ См/м) дал метод, основанный на измерении фазового сдвига в высокочастотной области -- отклонение составляет всего +4\%. Однако судя по графику сравнения экспериментального и теоретического коэффициента ослабления магнитного поля от частоты, наиболее точным оказался метод определения проводимости по зависимости индуктивности катушки от частоты.

Наиболее неточные результаты дали методы, основанные на определении разности фаз, что вызвано неудобством и неточностью определения их разности на экране осциллографа.
% Разброс значений между методами может объясняться различной чувствительностью к точности определения геометрических параметров экрана, а также особенностями обработки экспериментальных данных. В целом, все методы подтвердили правильный порядок величины проводимости меди и продемонстрировали работоспособность теоретических моделей скин-эффекта.






























% \section*{Ход работы}
% По параметрам установки $h = \delta = 1,5$ мм, приняв проводимость меди для оценки равной $\sigma \sim 5\cdot10^{7}$ Сименс/м, а $\mu \approx 1$, из формулы (5) была найдена частота $\nu_h \approx 2250$ Гц.


% В области низких частот 20 -- 160 Гц была получена зависимость отношения $\xi = U/(\nu I)$ от частоты $\nu$ и построен график зависимости $1/\xi^2 = f(\nu^2)$. Хорошо видно, что зависимость является линейной.

% \begin{center}
% \begin{tikzpicture}
% \begin{axis}[
%     width=14cm,
%     height=8cm,
%     xlabel={$\nu^2$, Гц$^2$},
%     ylabel={$1/\xi^2$},
%     grid=major,
%     legend pos=north west,
%     title={Зависимость $1/\xi^2$ от $\nu^2$ в области низких частот},
%     xmin=0,
%     xmax=27000,
%     ymin=4500,
%     ymax=11000
% ]

% % Данные измерений с погрешностями
% \addplot+[only marks, error bars/.cd, y dir=both, y explicit] coordinates {
%     (400, 5043.113317)  +- (0, 101.366)
%     (900, 5170.188778)  +- (0, 103.921)
%     (1600, 5336.809723) +- (0, 107.270)
%     (2500, 5543.229479) +- (0, 111.419)
%     (3600, 5763.735909) +- (0, 115.851)
%     (4900, 6016.564237) +- (0, 120.933)
%     (6400, 6301.962143) +- (0, 126.669)
%     (8100, 6626.473969) +- (0, 133.192)
%     (10000, 6983.896687) +- (0, 140.346)
%     (12100, 7375.005787) +- (0, 148.238)
%     (14400, 7783.832351) +- (0, 156.455)
%     (19600, 8747.649658) +- (0, 175.828)
%     (25600, 9857.533446) +- (0, 198.136)
% };

% % Линейная аппроксимация по МНК
% \addplot[domain=0:27000, red, thick] {5005.93 + 0.1896*x};
% \legend{Экспериментальные точки, Линейная аппроксимация}
% \end{axis}
% \end{tikzpicture}
% \end{center}

% \subsection*{Определение $\xi_0$}
% Методом наименьших квадратов получена линейная зависимость:
% \[
% \frac{1}{\xi^2} = (5005,93 \pm 15,76) + (0,1896 \pm 0,0013) \cdot \nu^2
% \]
% Экстраполяцией её к точке $\nu = 0$ определено:
% \[
% \frac{1}{\xi_0^2} = 5005,93 \pm 15,76 \quad \Rightarrow \quad \xi_0 = \frac{1}{\sqrt{5005,93}} = 0,01414
% \]
% С учётом погрешности $\xi_0 = 0,01414 \pm 0,00002$
% % \[
% % \delta\xi_0 = \frac{1}{2} \cdot \xi_0^3 \cdot \delta\left(\frac{1}{\xi_0^2}\right) = \frac{1}{2} \cdot (0.01414)^3 \cdot 15.76 \approx 0.00002
% % \]



% \subsection*{Расчёт проводимости меди $\sigma$}

% % Используя (\ref{eq:low_freq_ampl}), найдем проводимость меди из углового коэффициента графика:
% % \[\tilde{k} = k \cdot \xi_0^2 = 0.188/5134 \Hz^{-2}= 36.6\cdot10^{-6} \Hz^{-2}\]
% % \[\tilde{k} = (ah\sigma \mu_0 \pi)^2\]
% % Учтём параметры установки $2a = 45 \mm$, $h = 1.5 \mm$, и рассчитаем $\sigma$:
% % \[\sigma = 45.4~\mathrm{MSm/m}\]
% % что хорошо совпадает с оценочным значением в $5\cdot 10^7~\mathrm{Sm/m}$, которое было взято для вычисления $f_0$.

% % Используем формулу:
% % \[
% % \left(\frac{|H_1|}{|H_0|}\right)^2 = \dfrac{1}{1 + \frac{1}{4}(ah\sigma \mu_0 \omega)^2}
% % \]
% Используя (\ref{eq:phase_low_freq}), а также найденный угловой коэффициент линейной зависимости $k = 0,1896 \pm 0,0013$, проводимость меди можно найти по формуле:
% \[
% \sigma = \frac{\xi_0}{\pi a h \mu_0} \sqrt{k}
% \]
% Геометрические параметры установки: $2a = 0,045$ м, $h = 0,0015$ м; $\mu_0 = 4\pi \cdot 10^{-7}$ Гн/м. Таким образом,
% \[
% \sigma = (4,62 \pm 0,06) \cdot 10^7 \, \text{См/м}
% \]
% % \subsection*{Расчёт погрешностей}

% % Погрешность измерения $\xi$:
% % \[
% % \delta_\xi = \xi \cdot \sqrt{(0.001)^2 + (0.0001)^2} = \xi \cdot 0.001005
% % \]

% % Погрешность $1/\xi^2$:
% % \[
% % \delta\left(\frac{1}{\xi^2}\right) = 2 \cdot \frac{1}{\xi^2} \cdot \frac{\delta_\xi}{\xi} = 2 \cdot \frac{1}{\xi^2} \cdot 0.001005
% % \]

% % Относительная погрешность проводимости:
% % \[
% % \frac{\delta\sigma}{\sigma} = \frac{\delta\xi_0}{\xi_0} + \frac{\delta a}{a} + \frac{\delta h}{h} + \frac{1}{2} \cdot \frac{\delta k}{k}
% % \]

% % При условии $\delta\xi_0/\xi_0 \approx 0.14\%$, $\delta k/k \approx 0.69\%$, $\delta a/a \approx 1\%$, $\delta h/h \approx 1\%$:
% % \[
% % \frac{\delta\sigma}{\sigma} \approx 0.0014 + 0.01 + 0.01 + 0.5 \cdot 0.0069 = 0.0249 = 2.49\%
% % \]

% % Абсолютная погрешность:
% % \[
% % \delta\sigma = 3.12 \times 10^7 \cdot 0.0249 \approx 0.78 \times 10^6 \, \text{См/м}
% % \]
% % Результат с учётом погрешности:
% % \[
% % \sigma = (3.12 \pm 0.08) \times 10^7 \, \text{См/м}
% % \]


% % \begin{table}[h]
% % \centering
% % \begin{tabular}{ccc}
% % \toprule
% % $\nu^2$, Гц$^2$ & $1/\xi^2$ & $\delta(1/\xi^2)$ \\
% % \midrule
% % 400 & 5043.11 & 101.37 \\
% % 900 & 5170.19 & 103.92 \\
% % 1600 & 5336.81 & 107.27 \\
% % 2500 & 5543.23 & 111.42 \\
% % 3600 & 5763.74 & 115.85 \\
% % 4900 & 6016.56 & 120.93 \\
% % 6400 & 6301.96 & 126.67 \\
% % 8100 & 6626.47 & 133.19 \\
% % 10000 & 6983.90 & 140.35 \\
% % 12100 & 7375.01 & 148.24 \\
% % 14400 & 7783.83 & 156.46 \\
% % 19600 & 8747.65 & 175.83 \\
% % 25600 & 9857.53 & 198.14 \\
% % \bottomrule
% % \end{tabular}
% % \caption{Результаты измерений с погрешностями}
% % \end{table}


% \subsection*{График зависимости $\tg\psi = f(\nu)$}

% \begin{center}
% \begin{tikzpicture}
% \begin{axis}[
%     width=14cm,
%     height=8cm,
%     xlabel={$\nu$, Гц},
%     ylabel={$\tg\psi$},
%     grid=major,
%     legend pos=north west,
%     title={Зависимость $\tg\psi$ от частоты $\nu$ в области низких частот},
%     xmin=0,
%     xmax=550,
%     ymin=0,
%     ymax=7
% ]

% % Экспериментальные точки
% \addplot+[only marks] coordinates {
%     (100,0.4700)
%     (120,0.7265)
%     (140,0.8107)
%     (160,1.0000)
%     (180,1.2602)
%     (200,1.3764)
%     (300,2.5232)
%     (400,5.7979)
%     (500,6.3138)
% };

% % Аппроксимирующая прямая для точек 100-200 Гц
% \addplot[domain=100:200, red, thick] {0.009032*x - 0.41417};
% \legend{Экспериментальные точки, Линейная аппроксимация}
% \end{axis}
% \end{tikzpicture}
% \end{center}

% \subsection*{Определение проводимости меди $\sigma$}
% % Из теории скин-эффекта для низких частот следует:
% % \[
% % \tg\psi = \frac{ah}{\delta^2} = \frac{ah\sigma\mu_0\omega}{2} = \pi ah\sigma\mu_0\nu
% % \]
% Из формулы (9) следует, что зависимость $\tg\psi$ от $\nu$ линейная с угловым коэффициентом
% \[
% k = \pi ah\sigma\mu_0
% \]
% Методом наименьших квадратов по точкам в диапазоне 100-200 Гц определён коэффициент наклона
% \[
% k = 0,00903 \pm 0,00056 \, \text{Гц}^{-1}
% \]
% % Геометрические параметры экрана:
% % \[
% % 2a = 0.045 \, \text{м} \Rightarrow a = 0.0225 \, \text{м}, \quad h = 0.0015 \, \text{м}
% % \]
% % \[
% % \mu_0 = 4\pi \times 10^{-7} \, \text{Гн/м}
% % \]
% Таким образом, проводимость меди:
% \[
% \sigma = \frac{k}{\pi ah\mu_0} = (6,78 \pm 0,56) \cdot 10^7 \, \text{См/м}
% \]

% % \subsection*{Расчёт погрешностей}

% % Погрешность углового коэффициента:
% % \[
% % \delta k = 0.000558 \, \text{Гц}^{-1}, \quad \frac{\delta k}{k} = 6.18\%
% % \]

% % Погрешности геометрических параметров:
% % \[
% % \frac{\delta a}{a} = 1\%, \quad \frac{\delta h}{h} = 1\%
% % \]

% % Относительная погрешность проводимости:
% % \[
% % \frac{\delta\sigma}{\sigma} = \sqrt{\left(\frac{\delta k}{k}\right)^2 + \left(\frac{\delta a}{a}\right)^2 + \left(\frac{\delta h}{h}\right)^2}
% % = \sqrt{(0.0618)^2 + (0.01)^2 + (0.01)^2} = 0.0818 = 8.18\%
% % \]

% % Абсолютная погрешность:
% % \[
% % \delta\sigma = 6.78 \times 10^7 \cdot 0.0818 = 0.56 \times 10^7 \, \text{См/м}
% % \]

% % Окончательный результат:
% % \[
% % \sigma = (6.78 \pm 0.56) \times 10^7 \, \text{См/м}
% % \]

% % Табличное значение проводимости меди: $\sigma_{\text{табл}} = 5.8 \times 10^7$ См/м. 
% % Полученное экспериментальное значение хорошо согласуется с табличным в пределах погрешности измерений.

% % \begin{table}[h]
% % \centering
% % \begin{tabular}{cccccc}
% % \toprule
% % $\nu$, Гц & $\varphi/\pi$ & $\varphi$, рад & $\psi$, рад & $\tg\psi$ \\
% % \midrule
% % 100 & 0.64 & 2.0106 & 0.4398 & 0.4700 \\
% % 120 & 0.70 & 2.1991 & 0.6283 & 0.7265 \\
% % 140 & 0.72 & 2.2619 & 0.6911 & 0.8107 \\
% % 160 & 0.75 & 2.3562 & 0.7854 & 1.0000 \\
% % 180 & 0.79 & 2.4817 & 0.9109 & 1.2602 \\
% % 200 & 0.80 & 2.5133 & 0.9425 & 1.3764 \\
% % 300 & 0.88 & 2.7646 & 1.1938 & 2.5232 \\
% % 400 & 0.96 & 3.0159 & 1.4451 & 5.7979 \\
% % 500 & 0.98 & 3.0788 & 1.5080 & 6.3138 \\
% % \bottomrule
% % \end{tabular}
% % \caption{Результаты измерений фазового сдвига}
% % \end{table}




% \subsection*{График зависимости $\psi - \pi/4 = f(\sqrt{\nu})$}

% \begin{center}
% \begin{tikzpicture}
% \begin{axis}[
%     width=14cm,
%     height=8cm,
%     xlabel={$\sqrt{\nu}$, Гц$^{1/2}$},
%     ylabel={$\psi - \pi/4$, рад},
%     grid=major,
%     legend pos=north west,
%     title={Зависимость $\psi - \pi/4$ от $\sqrt{\nu}$},
%     xmin=0,
%     xmax=180,
%     ymin=-0.5,
%     ymax=4
% ]

% % Данные низких частот (п.4)
% \addplot+[only marks, blue] coordinates {
%     (10.00, -0.3456)
%     (10.95, -0.1571)
%     (11.83, -0.0943)
%     (12.65, 0.0000)
%     (13.42, 0.1255)
%     (14.14, 0.1571)
%     (17.32, 0.4084)
%     (20.00, 0.6597)
%     (22.36, 0.7226)
% };

% % Данные высоких частот (п.5)
% \addplot+[only marks, red] coordinates {
%     (31.62, 0.7854)
%     (36.06, 0.8482)
%     (40.00, 0.9111)
%     (44.72, 1.0367)
%     (50.99, 1.1936)
%     (57.45, 1.3192)
%     (65.57, 1.4765)
%     (74.16, 1.7279)
%     (83.67, 1.9477)
%     (94.87, 2.4184)
%     (106.30, 2.7010)
%     (120.42, 3.1416)
%     (136.01, 3.3917)
%     (153.30, 3.3298)
%     (173.21, 3.5182)
% };

% % Аппроксимирующая прямая для высоких частот
% \addplot[domain=0:180, green, thick] {0.02310*x};
% \legend{Низкие частоты, Высокие частоты, Аппроксимация}
% \end{axis}
% \end{tikzpicture}
% \end{center}

% \subsection*{Определение проводимости меди $\sigma$}

% Для высоких частот ($\nu >> \nu_h$) выполняется равенство:
% \[
% \psi - \frac{\pi}{4} = h \sqrt{\frac{\omega \sigma \mu_0}{2}} = h \sqrt{\pi \sigma \mu_0} \cdot \sqrt{\nu}
% \]

% Таким образом, зависимость $\psi - \pi/4$ от $\sqrt{\nu}$ линейная с угловым коэффициентом $k = h \sqrt{\pi \sigma \mu_0}$.

% Методом наименьших квадратов по точкам в диапазоне высоких частот ($\nu > 2000$ Гц) получено
% \[
% k = 0,02310 \pm 0,00067 \, \text{рад/Гц}^{1/2}
% \]
% Проводимость меди:
% \[
% \sigma = \frac{k^2}{h^2 \pi \mu_0} = (6.01 \pm 0.47) \cdot 10^7 \, \text{См/м}
% \]

% % \subsection*{Расчёт погрешностей}

% % Погрешность углового коэффициента:
% % \[
% % \delta k = 0.00067 \, \text{рад/Гц}^{1/2}, \quad \frac{\delta k}{k} = 2.9\%
% % \]

% % Погрешности геометрических параметров:
% % \[
% % \frac{\delta h}{h} = 1\%
% % \]

% % Относительная погрешность проводимости:
% % \[
% % \frac{\delta\sigma}{\sigma} = 2 \cdot \frac{\delta k}{k} + 2 \cdot \frac{\delta h}{h} = 2 \cdot 0.029 + 2 \cdot 0.01 = 0.078 = 7.8\%
% % \]

% % Абсолютная погрешность:
% % \[
% % \delta\sigma = 6.01 \times 10^7 \cdot 0.078 = 0.47 \times 10^7 \, \text{См/м}
% % \]

% % Окончательный результат:
% % \[
% % \sigma = (6.01 \pm 0.47) \times 10^7 \, \text{См/м}
% % \]

% % Табличное значение проводимости меди: $\sigma_{\text{табл}} = 5.8 \times 10^7$ См/м. 
% % Полученное экспериментальное значение хорошо согласуется с табличным в пределах погрешности измерений.

% % \begin{table}[h]
% % \centering
% % \begin{tabular}{cccccc}
% % \toprule
% % $\nu$, Гц & $\sqrt{\nu}$, Гц$^{1/2}$ & $\varphi/\pi$ & $\varphi$, рад & $\psi$, рад & $\psi-\pi/4$, рад \\
% % \midrule
% % 100 & 10.00 & 0.64 & 2.0106 & 0.4398 & -0.3456 \\
% % 120 & 10.95 & 0.70 & 2.1991 & 0.6283 & -0.1571 \\
% % 140 & 11.83 & 0.72 & 2.2619 & 0.6911 & -0.0943 \\
% % 160 & 12.65 & 0.75 & 2.3562 & 0.7854 & 0.0000 \\
% % 180 & 13.42 & 0.79 & 2.4817 & 0.9109 & 0.1255 \\
% % 200 & 14.14 & 0.80 & 2.5133 & 0.9425 & 0.1571 \\
% % 300 & 17.32 & 0.88 & 2.7646 & 1.1938 & 0.4084 \\
% % 400 & 20.00 & 0.96 & 3.0159 & 1.4451 & 0.6597 \\
% % 500 & 22.36 & 0.98 & 3.0788 & 1.5080 & 0.7226 \\
% % 1000 & 31.62 & 1.00 & 3.1416 & 1.5708 & 0.7854 \\
% % 1300 & 36.06 & 1.02 & 3.2044 & 1.6336 & 0.8482 \\
% % 1600 & 40.00 & 1.04 & 3.2673 & 1.6965 & 0.9111 \\
% % 2000 & 44.72 & 1.08 & 3.3929 & 1.8221 & 1.0367 \\
% % 2600 & 50.99 & 1.13 & 3.5498 & 1.9790 & 1.1936 \\
% % 3300 & 57.45 & 1.17 & 3.6754 & 2.1046 & 1.3192 \\
% % 4300 & 65.57 & 1.22 & 3.8327 & 2.2619 & 1.4765 \\
% % 5500 & 74.16 & 1.30 & 4.0841 & 2.5133 & 1.7279 \\
% % 7000 & 83.67 & 1.37 & 4.3039 & 2.7331 & 1.9477 \\
% % 9000 & 94.87 & 1.52 & 4.7746 & 3.2038 & 2.4184 \\
% % 11300 & 106.30 & 1.61 & 5.0572 & 3.4864 & 2.7010 \\
% % 14500 & 120.42 & 1.75 & 5.4978 & 3.9270 & 3.1416 \\
% % 18500 & 136.01 & 1.83 & 5.7479 & 4.1771 & 3.3917 \\
% % 23500 & 153.30 & 1.81 & 5.6860 & 4.1152 & 3.3298 \\
% % 30000 & 173.21 & 1.87 & 5.8744 & 4.3036 & 3.5182 \\
% % \bottomrule
% % \end{tabular}
% % \caption{Результаты измерений фазового сдвига для всех частот}
% % \end{table}



% \newpage
% \subsection*{График зависимости индуктивности катушки от частоты $L(\nu)$}

% \begin{center}
% \begin{tikzpicture}
% \begin{axis}[
%     width=14cm,
%     height=8cm,
%     xlabel={$\nu$, Гц},
%     ylabel={$L$, мГн},
%     grid=major,
%     legend pos=north east,
%     title={Зависимость индуктивности катушки от частоты},
%     xmin=0,
%     xmax=8000,
%     ymin=4,
%     ymax=20
% ]

% % Плавная красная кривая через точки измерений
% \addplot[red, smooth, thick, no marks] coordinates {
%     (40,19.2)
%     (100,15.3)
%     (200,11.27)
%     (300,8.9)
%     (400,7.8)
%     (500,7.1)
%     (600,6.6)
%     (800,5.6)
%     (1500,5.67)
%     (2000,5.69)
%     (2500,5.46)
%     (4000,5.05)
%     (6000,5.48)
%     (7500,5.19)
% };

% % Синие точки измерений
% \addplot[blue, only marks, mark=*, mark size=2pt] coordinates {
%     (40,19.2)
%     (100,15.3)
%     (200,11.27)
%     (300,8.9)
%     (400,7.8)
%     (500,7.1)
%     (600,6.6)
%     (800,5.6)
%     (1500,5.67)
%     (2000,5.69)
%     (2500,5.46)
%     (4000,5.05)
%     (6000,5.48)
%     (7500,5.19)
% };

% \end{axis}
% \end{tikzpicture}
% \end{center}

% Максимальное значение индуктивности: $L_{\text{max}} = 19,2$ мГн

% Минимальное значение индуктивности: $L_{\text{min}} = 5,05$ мГн

% \subsection*{График зависимости $\frac{L_{\text{max}} - L}{L - L_{\text{min}}} = f(\nu^2)$}

% \begin{center}
% \begin{tikzpicture}
% \begin{axis}[
%     width=14cm,
%     height=8cm,
%     xlabel={$\nu^2$, Гц$^2$},
%     ylabel={$\frac{L_{\text{max}} - L}{L - L_{\text{min}}}$},
%     grid=major,
%     legend pos=north west,
%     title={Зависимость $\frac{L_{\text{max}} - L}{L - L_{\text{min}}}$ от $\nu^2$},
%     xmin=0,
%     xmax=700000,
%     ymin=0,
%     ymax=30
% ]

% % Данные для первых 8 точек
% \addplot+[blue, only marks] coordinates {
%     (1600,0)
%     (10000,0.38049)
%     (40000,1.27492)
%     (90000,2.67532)
%     (160000,4.14545)
%     (250000,6.90244)
%     (360000,10.12903)
%     (640000,24.72727)
% };

% % Линейная аппроксимация
% \addplot[domain=0:700000, red, thick] {3.495e-5*x};
% \legend{Экспериментальные точки, Линейная аппроксимация}
% \end{axis}
% \end{tikzpicture}
% \end{center}

% \subsection*{Определение проводимости меди $\sigma$}

% % Из теории следует:
% % \[
% % \frac{L_{\text{max}} - L}{L - L_{\text{min}}} = (\pi a h \mu_0 \sigma \nu)^2
% % \]

% Поскольку максимальное значение индуктивности достигается при минмальной частоте, рассматриваемая зависимость $\frac{L_{\text{max}} - L}{L - L_{\text{min}}}$ от $\nu^2$ проходит через (0; 0) и является линейной (формула 11) с угловым коэффициентом $k = (\pi a h \mu_0 \sigma)^2$.

% Методом наименьших квадратов по первым 8 точкам, лежащим примерно на одной прямой, получен
% \[
% k = (3,50 \pm 0,29) \cdot 10^{-5} \, \text{Гц}^{-2}
% \]

% Отсюда проводимость меди:
% \[
% \sigma = \frac{\sqrt{k}}{\pi a h \mu_0} = (4.44 \pm 0.27) \cdot 10^7 \, \text{См/м}
% \]



% % Табличное значение проводимости меди: $\sigma_{\text{табл}} = 5.8 \times 10^7$ См/м. 
% % Полученное экспериментальное значение хорошо согласуется с табличным в пределах погрешности измерений.

% % \begin{table}[h]
% % \centering
% % \begin{tabular}{cccc}
% % \toprule
% % $\nu$, Гц & $\nu^2$, Гц$^2$ & $L$, мГн & $\frac{L_{\text{max}} - L}{L - L_{\text{min}}}$ \\
% % \midrule
% % 40 & 1600 & 19.2 & 0 \\
% % 100 & 10000 & 15.3 & 0.38049 \\
% % 200 & 40000 & 11.27 & 1.27492 \\
% % 300 & 90000 & 8.9 & 2.67532 \\
% % 400 & 160000 & 7.8 & 4.14545 \\
% % 500 & 250000 & 7.1 & 5.90244 \\
% % 600 & 360000 & 6.6 & 8.12903 \\
% % 800 & 640000 & 5.6 & 24.72727 \\
% % 1500 & 2250000 & 5.67 & 21.82258 \\
% % 2000 & 4000000 & 5.69 & 21.10938 \\
% % 2500 & 6250000 & 5.46 & 33.51220 \\
% % 4000 & 16000000 & 5.05 & - \\
% % 6000 & 36000000 & 5.48 & 31.90698 \\
% % 7500 & 56250000 & 5.19 & 100.07143 \\
% % \bottomrule
% % \end{tabular}
% % \caption{Результаты измерений индуктивности}
% % \end{table}












% \section*{Выводы}
% В ходе работы была исследована проводимость меди экрана с использованием четырёх различных методов, основанных на явлении скин-эффекта.

% \begin{table}[h]
% \centering
% \begin{tabular}{|c|c|}
% \hline
% \textbf{Метод} & \textbf{$\sigma$, См/м} \\
% \hline
% По зависимости $1/\xi^2 = f(\nu^2)$ & $(4,62 \pm 0,06) \cdot 10^7$ \\
% \hline
% По зависимости $\tg\psi = f(\nu)$ & $(6,78 \pm 0,56) \cdot 10^7$ \\
% \hline
% По зависимости $\psi - \pi/4 = f(\sqrt{\nu})$ & $(6,01 \pm 0,47) \cdot 10^7$ \\
% \hline
% По зависимости $\frac{L_{\text{max}} - L}{L - L_{\text{min}}} = f(\nu^2)$ & $(4,44 \pm 0,27) \cdot 10^7$ \\
% \hline
% Табличное значение & $5,80 \cdot 10^7$\\
% \hline
% \end{tabular}
% \end{table}

% Значения проводимости, полученные разными методами, лежат в диапазоне от $4,44\cdot 10^7$ до $6,78\cdot 10^7$ См/м. Наиболее близкое к табличному значению $(5,80\cdot 10^7$ См/м) дал метод, основанный на измерении фазового сдвига в высокочастотной области -- отклонение составляет всего +4\%. 

% Разброс значений между методами может объясняться различной чувствительностью к точности определения геометрических параметров экрана, а также особенностями обработки экспериментальных данных. В целом, все методы подтвердили правильный порядок величины проводимости меди и продемонстрировали работоспособность теоретических моделей скин-эффекта.








\end{document}