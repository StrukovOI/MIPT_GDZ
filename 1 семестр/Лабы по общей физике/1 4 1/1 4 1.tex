\documentclass[a4paper,12pt]{article}
\usepackage[a4paper,top=1.3cm,bottom=2cm,left=1.5cm,right=1.5cm,marginparwidth=0.75cm]{geometry}
\usepackage{setspace}
\usepackage{cmap}					
\usepackage{mathtext} 				
\usepackage[T2A]{fontenc}			
\usepackage[utf8]{inputenc}			
\usepackage[english,russian]{babel}
\usepackage{multirow}
\usepackage{graphicx}
\usepackage{wrapfig}
\usepackage{tabularx}
\usepackage{float}
\usepackage{longtable}
\usepackage{hyperref}
\hypersetup{colorlinks=true,urlcolor=blue}
\usepackage[rgb]{xcolor}
\usepackage{amsmath,amsfonts,amssymb,amsthm,mathtools} 
\usepackage{icomma} 
\mathtoolsset{showonlyrefs=true}
\usepackage{euscript}
\usepackage{mathrsfs}

\DeclareMathOperator{\sgn}{\mathop{sgn}}
\newcommand*{\hm}[1]{#1\nobreak\discretionary{}
	{\hbox{$\mathsurround=0pt #1$}}{}}
 
\title{\begin{center}Лабораторная работа №1.4.1\end{center}
ыявчаспмгрщшозщлхздхъж}
\author{Струков О. И. \\
Б04-404}
\date{}
 
 \begin{document}
 	
 	\maketitle
 	\newpage
 	\section*{Введение}
 	
 	\textbf{Цели работы:} 1) на примере измерения периода свободных колебаний физического
 	маятника познакомиться с систематическими и случайными погрешностями, прямыми и косвенными измерениями; 2) проверить справедливость формулы для периода колебаний физического маятника и определить значение ускорения свободного падения; 3) убедиться в справедливости теоремы Гюйгенса об обратимости
 	точек опоры и центра качания маятника; 4) оценить погрешность прямых и косвенных измерений и конечного результата.\\
 	\textbf{Оборудование:} металлический стержень; опорная призма; торцевые
 	ключи; закреплённая на стене консоль; подставка с острой гранью для определения
 	цента масс маятника; секундомер; металлическая линейка;
	электронные весы; математический маятник (небольшой груз,
 	подвешенный на нитях).
 	
 	\section*{Теоретические сведения}
 	
 	\begin{wrapfigure}{l}{6cm}
 		\includegraphics[width=1\linewidth]{ustanovka}
 		\caption{Физический маятник}\label{risunok}
 	\end{wrapfigure}
 	
 	\par В работе изучается динамика движения физического маятника.
 	Физический маятник, используемый в работе, представляет собой однородный стальной стержень массы $m$, длина которого $l$ много больше ее диаметра. На стержне закрепляется опорная призма, острое ребро которой является осью качания маятника.
 	
 	Второй закон Ньютона определяет динамику движения тела точечной массы m. Импульс тела $P=mv$ изменяется во времени $t$ под действием силы $F$:
 	\begin{equation}
 		F = \frac{dP}{dt}
 	\end{equation}
 	Если рассмотреть точечную массу, которая движется по окружности радиуса $r$ с угловой скоростью $\omega$, тогда линейная скорость $v = \omega r$, то формулу для силы можно преобразовать:
 	\begin{equation}
 		Fr = \frac{dP}{dt}r
 	\end{equation}
 	\begin{equation}
 		M=\frac{dP}{dt}r=\frac{dL}{dt}
 	\end{equation}
 	\noindent где $L = J\omega$, и $J = mr^2$. Величину $J$ называют \textit{моментом инерции}.
 	\begin{equation}
 		J = \sum_{i=1} m_i r_i^2
 	\end{equation}
 	\par Посчитаем момент инерции для данного нам стержня, при вращении вокруг препендикулярной стержню оси. Для этого разобьем стержень на отрезки $dr$ и $dm = m\cdot\frac{dr}{l}$ и возьмем интеграл:
 	\begin{equation}
 		J_c = \int_{-\frac{l}{2}}^{\frac{l}{2}}r^2dm = \int_{-\frac{l}{2}}^{\frac{l}{2}}\frac{mr^2}{l}dm = \frac{ml^2}{12} 
 	\end{equation}
 	
 	
 	Призму можно перемещать вдоль стержня, меняя таким образом расстояние $ OC $ от точки опоры маятника до его центра масс. Пусть это расстояние равно $ a $. Тогда по теореме Гюйгенса-Штейнера момент инерции маятника
 	
 	\begin{equation}
 		J=\frac{ml^2}{12}+ma^2,
 	\end{equation}
 	
 	\noindent где $ m $ -- масса маятника.
 	
 	
 	Период колебаний получим через аналогию с пружинным маятником, как известно: \begin{equation}
 		T_\text{п}=2\pi\sqrt{\frac{m}{k}}
 	\end{equation}
 	\noindent В нашем случае, роль массы играет \textit{момент инерции тела $J$}, а \textit{жесткость пружины $k$} - коэффицент пропорциональности $mga$. Таким образом приходим к следующей формуле колебаний произвольного физического маятника:
 	\begin{equation}
 		T=2\pi\sqrt{\frac{J}{mga}}
 	\end{equation}
 	\noindent После подстановки период колебаний, для стержня длиной $l$ подвешенного на расстоянии $a$ от центра, равен:
 	
 	\begin{equation}\label{time_a}
 		T=2\pi\sqrt{\frac{a^2+\frac{l^2}{12}}{ag}}
 	\end{equation}
 	
 	Таким образом, период малых колебаний не зависит ни от начальной фазы, ни от амплитуды колебаний.
 	\medspace
 	
 	Период колебаний математического маятника определяется формулой
 	\begin{equation}
 		T_\text{м}=2\pi\sqrt{\frac{l}{g}},
 	\end{equation}
 	где $ l $ -- длина математического маятника. Поэтому величину
 	\begin{equation}\label{prived}
 		l_\text{пр}=a+\frac{l^2}{12a}
 	\end{equation}
 	называют приведённой длиной математического маятника. Поэтому точку $ O' $ (см. рис. \ref{risunok}), отстоящую от точки опоры на расстояние $ l_\text{пр} $, называют центром качания физического маятника. Точка опоры и центр качания маятника обратимы, т.е. при качании маятника вокруг  точки $ O' $ период будет таким же, как и при качании вокруг точки $ O $.
 	
 	\section* {Экспериментальная установка}
 	Тонкий стальной стержень, подвешенный на прикрепленной к стене консоли с помощью небольшой призмы, которая опирается на поверхность консоли острым основанием. Призму можно перемещать вдоль стержня, изменяя положение точки подвеса. Период колебаний измеряется с помощью секундомера, расстояния измеряются линейкой и штангенциркулем. Положение центра масс можно определить с помощью балансирования маятника на вспомогательной подставке.
 	
 	
 	
 	\subsection*{Расчет поправок}
 	Чтобы получить более точные результаты, для вычисления периода следует использовать формулу, учитывающую оба тела (и стержень, и призму):
 	\begin{equation}
 		T = 2\pi\sqrt{\frac{J_\text{ст} + J_\text{пр}}{m_\text{ст}ga_\text{ст}-m_\text{пр}ga_\text{пр}}}
 	\end{equation}
 	Однако призма мала по размеру и массе, поэтому поправка на момент инерции призмы в условиях опыта составляет не более 0,1\% $\Rightarrow$ ей можно пренебречь.
 	
 	Сравним теперь моменты сил, действующие на призму и стержень при $a = 10$ см:
 	\begin{equation}
 		\frac{M_\text{пр}}{M_\text{ст}} = \frac{m_\text{пр}ga_\text{пр}}{m_\text{ст}ga_\text{ст}} \approx 10^-2
 	\end{equation}
 	
 	В данном случае поправка достигает 1\% $\Rightarrow$ ей пренебречь нельзя. Учесть влияние призмы можно -- исключив $a_\text{пр}$, изменяя положение центра сиситемы. Пусть $x_\text{ц}$ -- расстояние от центра масс системы до точки подвеса, тогда:
 	\begin{equation}
        x_\text{ц}=\frac{m_\text{ст}a_\text{ст}-m_\text{пр}a_\text{пр}}{m_\text{ст} + m_\text{пр}}
 	\end{equation}
 	
 	
 	Исключая из двух уравнений $a_\text{пр}$, получаем:
 	\begin{equation}\label{period}
 		T = 2\pi \sqrt{\frac{\frac{l^2}{12}+a^2}{g(1+\frac{m_\text{пр}}{m_\text{ст}}) x_\text{ц}}}
 	\end{equation}	
 	\section*{Ход работы}
\begin{enumerate}
	\item Произведена оценка погрешностей измерительных приборов, используемых в работе.
	\begin{center}
		\begin{tabular}{|c|c|c|}
		\hline 
		Прибор & Измеряемые величины & Погрешность измерений\\ \hline
		Линейка & Длины участков стержня & $\sigma_l = 1$ мм \\ \hline
		Электронные весы & Массы стержня и призмы & $\sigma_m = 0,1$ г\\ \hline
		Секундомер & Время колебаний & $\sigma_t = 0,067$ с\\ \hline
		\end{tabular}
	\end{center} 
Было потребовано измерить период колебаний маятника с относительной погрешностью не более $\epsilon_{max} = 0,5 \%$.
\item Измерена длина и масса стержня и масса призмы, определены погрешности
\begin{center}
	\begin{tabular}{|c|c|c|c|}
	\hline 
	Величина & Значение & Абсолютная погрешность & Относительная погрешность\\ \hline
	Длина стержня & $l = 100,2$ см & $\sigma_l = 0,1$ см & $\epsilon_l = 0,010$ \% \\ \hline
	Масса стержня & $m_\text{ст} = 870,7$ г & $\sigma_{m_\text{ст}} = 0,1$ г & $\epsilon_{m_\text{ст}} = 0,011$ \% \\ \hline
	Масса призмы & $m_\text{пр} = 72,3$ г & $\sigma_{m_\text{пр}} = 0,1$ г & $\epsilon_{m_\text{пр}} = 0,138$ \% \\ \hline
	\end{tabular}
\end{center}
\item С помощью подставки было определено положение центра масс стержня без призмы. Как и ожидалось, оно находится ровно посередине стержня.
\item Призма была установлена на расстоянии $a$ от центра стержня. С помощью подставки было определено положение центра масс конструкции $x_\text{ц}$ относительно точки подвеса. Поскольку длины определялись с помощью линейки, их абсолютная погрешность составила 0,1 см.
\begin{center}
	\begin{tabular}{|c|c|c|c|}
	\hline 
	Величина & Значение & Абсолютная погрешность & Относительная погрешность\\ \hline
	$a$ & $a = 47,0$ см & $\sigma_a = 0,1$ см & $\epsilon_a = 0,21$ \% \\ \hline
	$x_\text{ц}$ & $x_\text{ц} = 43,3$ см & $\sigma_{x_\text{ц}} = 0,1$ см & $\epsilon_{x_\text{ц}} = 0,23$ \% \\ \hline
	\end{tabular}
\end{center}
\item Был проведён предварительный опыт по измерению периода колебаний. Для этого 5 раз было измерено время, за которое маятник совершает 10 колебаний, определён их период, его среднее значение $T_{\text{ср}}$ и погрешности.
\[\sigma_{\text{сист}} = \frac{\sigma_{t}}{10} 	\qquad \sigma_{\text{случ}} = \sqrt{\frac{1}{5-1}\sum_{i = 1}^{5}(T_i-T_{\text{ср}})^2} \qquad \sigma_T = \sqrt{\sigma_{\text{сист}}^2+\sigma_{\text{случ}}^2} \qquad \epsilon_T = \frac{\sigma_T}{T}\cdot100\%\]
\begin{center}
	\begin{tabular}{|c|c|c|}
	\hline 
	Номер опыта & Общее время колебаний, с & Период колебаний, с\\ \hline
	1 & 16,111 & 1,6111\\ \hline
	2 & 16,200 & 1,6200\\ \hline
	3 & 16,133 & 1,6133\\ \hline
	4 & 16,167 & 1,6167\\ \hline
	5 & 16,200 & 1,6200\\ \hline
	\end{tabular}
\end{center}

\begin{center}
	\begin{tabular}{|c|c|c|c|c|}
	\hline 
	$T_{\text{ср}}$ & $\sigma_{\text{сист}}$ & $\sigma_{\text{случ}}$ & $\sigma_T$ & $\epsilon_T$\\ \hline
	1,6162 с & 0,0067 с & 0,0040 с & 0,0078 с & 0,48\% \\ \hline
	\end{tabular}
\end{center}
Далее было рассчитано предварительное значение $g_{\text{пред}}$ c учётом поправок, учитывающих существование призмы.
\[ g = \frac{4\pi^2 (\frac{l^2}{12} + a^2)}{T^2 (1+\frac{m_\text{пр}}{m_\text{ст}}) x_\text{ц}} \]
Таким образом, $g_{\text{пред}} = 9,810 \text{ м/с}^2$, что отличается от табличного $g = 9,807 \text{ м/с}^2$ на 0,003 м/с$^2$ , то есть примерно на 0,03\%, что не превышает требуемое максимальное отклонение равное 10\%.
\item Поскольку $\epsilon_T = 0,48\% < \epsilon_{max} = 0,5\% $, как было потребовано в пункте 1, очевидно, что для определения периода колебаний достаточно измерения времени десяти колебаний маятника.
\item Было проведено измерение периода колебаний по десяти полным колебаниям для одиннадцати различных положений призмы на стержне. Для каждого из них было рассчитано значение $g$ по формуле, приведённой в пятом пункте. Результаты занесены в таблицу.
\begin{center}
	\begin{tabular}{|c|c|c|c|c|c|c|}
	\hline
	Номер опыта & $a$, мм & $x_\text{ц}$, мм & n & $t_n$, с & $T_n$, с & $g$, м/с$^2$ \\ \hline
	1 & 470 & 433 & 10 & 16,200 & 1,6200 & 9,77 \\ \hline
	2 & 440 & 406 & 10 & 15,917 & 1,5917 & 9,83 \\ \hline
	3 & 400 & 369 & 10 & 15,667 & 1,5667 & 9,81 \\ \hline
	4 & 360 & 331 & 10 & 15,467 & 1,5467 & 9,82 \\ \hline
	5 & 330 & 304 & 10 & 15,333 & 1,5333 & 9,82 \\ \hline
	6 & 260 & 239 & 10 & 15,300 & 1,5300 & 9,85 \\ \hline
	7 & 220 & 202 & 10 & 15,567 & 1,5567 & 9,83 \\ \hline
	8 & 190 & 175 & 10 & 15,967 & 1,5967 & 9,78 \\ \hline
	9 & 120 & 110 & 10 & 18,233 & 1,8233 & 9,77 \\ \hline
	10 & 100 & 92 & 10 & 19,500 & 1,9500 & 9,76 \\ \hline
	11 & 50 & 46 & 10 & 26,333 & 2,6333 & 9,85 \\ \hline
	\end{tabular}
\end{center}
\item Было определено усреднённое значение $g = 9,81 \text{ м/с}^2$ и оценена его погрешность с помощью следующей формулы:
\[ \sigma_g = \sqrt{\frac{1}{11-1}\sum_{i = 1}^{11}(g_i-g_{\text{ср}})^2} \]
Таким образом, $\sigma_g = 0,03\text{ м/с}^2$.
\item Построен график зависимости периода колебаний $T$ от расстояния подвеса $a$ (рис. 2), на котором явно видно, что зависимость имеет минимум в окрестности точки $a = 26$ см.
\item Построен график зависимости $u(v)$, где $u = T^2x_\text{ц}, v = a^2$ (рис. 3). Данные для графика приведены в таблице ниже. Заметно, что экспериментальные точки графика хорошо ложатся на одну прямую.
\begin{center}
\small
	\begin{tabular}{|c|c|c|c|c|c|c|c|c|c|c|c|}
	\hline 
	$u$, $\text{м}\cdot\text{с}^2$ & 1,1364 & 1,0286 & 0,9057 & 0,7918 & 0,7147 & 0,5595 & 0,4895 & 0,4461 & 0,3657 & 0,3498 & 0,3190\\ \hline
	$v$, м$^2$ & 0,2209 & 0,1936 & 0,16 & 0,1296 & 0,1089 & 0,0676 & 0,0484 & 0,0361 & 0,0144 & 0,01 & 0,0025 \\ \hline
	\end{tabular}
\end{center}
\item С помощью МНК были определены параметры $k$ и $b$ наилучшей прямой $u = kv + b$ и их погрешности $\sigma_k$ и $\sigma_b$. Затем было рассчитано значение $g$ и оценена его погрешность $\sigma_g$ по следующим формулам:
\[g = \frac{4\pi^2k}{1+\frac{m_\text{пр}}{m_\text{ст}}} \qquad \sigma_g = g\sqrt{\left(\frac{\sigma_k}{k}\right)^2+\left(\frac{\sigma_{m_\text{пр}}}{m_\text{пр}}\right)^2+\left(\frac{\sigma_{m_\text{ст}}}{m_\text{ст}}\right)^2}\]
\begin{center}
	\begin{tabular}{|c|c|c|c|c|c|}
	\hline 
	$k$, м/с$^2$ & $b$, м/с$^2$ & $\sigma_k$, м/с$^2$ & $\sigma_b$, м/с$^2$ & $g$, м/с$^2$ & $\sigma_g$, м/с$^2$\\ \hline
	0,2687 & -0,0834 & 0,0077 & 0,0021 & 9,79 & 0,28 \\ \hline
	\end{tabular}
\end{center}
\item Из сравнения результатов расчёта $g$ в пунктах 8 и 11 очевидно, что ближе к истинному значению оказался первый результат. Погрешность в этом случае также оказалась примерно в десять раз меньше. Поэтому можно сделать вывод, что наиболее точным методом измерения $g$ является усреднение отдельных значений, полученных в результате анализа колебаний физического маятника с различными точками закрепления.

\item Был проведён опыт по определению приведённой длины $l_\text{пр}$ маятника для одного значения $a = 22$ см. Сначала приведённая длина была рассчитана теоретически с помощью формулы $l_\text{пр} = a + \frac{l^2}{12a}$. Затем на математическом маятнике была установлена рассчитанная длина и измерен его период колебаний. Все результаты занесены в таблицу, где $T_\text{ст}$ - период колебаний стержня, $T_\text{м}$ - период колебаний математического маятника, $\sigma_{T_\text{ст}}$ и $\sigma_{T_\text{м}}$ - их абсолютные погрешности соответственно.
\begin{center}
	\begin{tabular}{|c|c|c|c|c|c|}
	\hline 
	$a$, см & $l_\text{пр}$, см & $T_\text{ст}$, с & $T_\text{м}$, с & $\sigma_{T_\text{ст}}$, с & $\sigma_{T_\text{м}}$, с \\ \hline
	22 & 60 & 1,5567 & 1,5533 с & 0,0067  & 0,0067 \\ \hline
	\end{tabular}
\end{center}
Было установлено, что периоды колебаний стержня и математического маятника совпадают в рамках погрешности, поскольку $|T_\text{ст} - T_\text{м}| = 0,0034 \text{ с}< \sigma_{T_\text{ст}} + \sigma_{T_\text{м}} = 0,0134$ с.






\end{enumerate}

\begin{figure}[h!]
	\includegraphics[scale=1]{График 1.pdf}
	\caption{Зависимость $ T $ от $ a $}
	\label{graph_file}
\end{figure}

\begin{figure}[h!]
	\includegraphics[scale=0.95]{График 2.pdf}
	\caption{Зависимость $ T^2x_\text{ц} $ от $ a^2 $}
	\label{graph_file}
\end{figure}

\end{document}