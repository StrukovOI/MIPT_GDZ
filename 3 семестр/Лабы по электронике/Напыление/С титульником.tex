\documentclass[a4paper,12pt]{article} % тип документа
\usepackage{tikz}
% Поля страниц
\usepackage[left=2.5cm,right=2.5cm,
    top=2cm,bottom=2cm,bindingoffset=0cm]{geometry}
%%% Работа с русским языком
\usepackage{cmap}                           % поиск в PDF
\usepackage{mathtext} 			 	       % русские буквы в формулах

%Матеша
\usepackage{amsmath,amsfonts,amssymb,amsthm,mathtools} % AMS
\usepackage{icomma} % "Умная" запятая

%\mathtoolsset{showonlyrefs=true} % Показывать номера только у тех формул, на которые есть \eqref{} в тексте.

%% Шрифты
\usepackage{euscript}	 % Шрифт Евклид
\usepackage{mathrsfs} % Красивый матшрифт
    
%Отступ после заголовка    
\usepackage{indentfirst}


% Рисунки
\usepackage{floatrow,graphicx,calc}
\usepackage{wrapfig}
\usepackage{xcolor}

% Создаёем новый разделитель
\DeclareFloatSeparators{mysep}{\hspace{1cm}}

% Ссылки?
\usepackage{hyperref}
\usepackage[rgb]{xcolor}
\hypersetup{				% Гиперссылки
    colorlinks=true,       	% false: ссылки в рамках
	urlcolor=blue          % на URL
}


%  Русский язык
\usepackage[T2A]{fontenc}			% кодировка
\usepackage[utf8]{inputenc}			% кодировка исходного текста
\usepackage[english,russian]{babel}	% локализация и переносы


% Математика
\usepackage{amsmath,amsfonts,amssymb,amsthm,mathtools}


% Что-то 
\usepackage{wasysym}


%Заговолок
% \author{Серебренников Даниил Б02-826}
% \title{Лабораторная работа \No 2.2.3}


\begin{document}
\include{titlepage}
\begin{titlepage}
\thispagestyle{empty}

\begin{center}
\small
Министерство науки и высшего образования Российской Федерации\\[2mm]
\textbf{ФЕДЕРАЛЬНОЕ ГОСУДАРСТВЕННОЕ АВТОНОМНОЕ
ОБРАЗОВАТЕЛЬНОЕ УЧРЕЖДЕНИЕ ВЫСШЕГО ОБРАЗОВАНИЯ}\\[1mm]
«МОСКОВСКИЙ ФИЗИКО-ТЕХНИЧЕСКИЙ ИНСТИТУТ\\
(НАЦИОНАЛЬНЫЙ ИССЛЕДОВАТЕЛЬСКИЙ УНИВЕРСИТЕТ)»\\
(МФТИ, Физтех)
\end{center}

\vspace{18mm}

\begin{center}
\large \textbf{КАФЕДРА ЭЛЕКТРОНИКИ}\\[10mm]
\Large \textbf{ОТЧЁТ}\\[2mm]
\large \textbf{ПО ЛАБОРАТОРНОЙ РАБОТЕ}\\[14mm]
\Large \textbf{НАПЫЛЕНИЕ ТОНКИХ ПЛЕНОК}
\end{center}

\vfill

% Блоки с подписями слева, линии и пояснения справа
\noindent
\begin{tabular}{@{}p{0.44\linewidth}p{0.52\linewidth}@{}}
\textit{Работу выполнил:} & Топольский Михаил. Б04-404 \\[8mm]

\textit{Работу принял, оценка:} & \\[3mm]
 \\
\end{tabular}


\vspace{12mm}

\begin{center}
\small Долгопрудный, \the\year{}~г.
\end{center}

\end{titlepage}
\newpage

\section*{Цель работы}
Ознакомление с методами термовакуумного напыления тонких плёнок, практическое получение алюминиевой плёнки заданной толщины термовакуумным испарением с измерением её параметров.

\section*{Теоретическая часть}

Термовакуумное напыление -- это метод получения тонких плёнок путём испарения материала в условиях высокого вакуума с последующей конденсацией паров на подложке. Основное требование к процессу -- обеспечение свободного пробега атомов от источника до подложки без столкновений с молекулами остаточного газа. Для этого необходимо, чтобы длина свободного пробега значительно превышала расстояние между испарителем и подложкой.

Длина свободного пробега при комнатной температуре ($T \approx 300\text{ К}$) оценивается по формуле
\begin{equation}
    \lambda = \frac{k_B T}{\sqrt{2} \pi d^2 P},
\end{equation}
где $k_B = 1,38 \cdot 10^{-23}\text{ Дж/К}$ — постоянная Больцмана, $d \approx 3,7 \cdot 10^{-10}\text{ м}$ — эффективный диаметр молекулы воздуха, $P$ — давление остаточного газа. %При $P = 10^{-5}\,\text{мм\,рт.\,ст.} \approx 1,33 \cdot 10^{-3}\,\text{Па}$ получаем $\lambda \gtrsim 50\text{ см}$, что удовлетворяет условию $\lambda \gg R$.

Испарение алюминия осуществляется с резистивного испарителя — вольфрамовой проволоки, изогнутой в виде шпильки (рис. 1а). Алюминий имеет температуру плавления $T_{\text{пл}} = 660^\circ\text{C}$ и температуру интенсивного испарения $T_{\text{исп}} \approx 1150^\circ\text{C}$. Вольфрам, в свою очередь, плавится при $T_{\text{пл}} \approx 3400^\circ\text{C}$ и обладает пренебрежимо малым давлением насыщенных паров при температурах испарения Al. Это позволяет использовать его в качестве стабильного подогревателя. Важно, что алюминий хорошо смачивает вольфрам, что обеспечивает стабильность капли и равномерность испарения.

    \begin{figure}[H]
        \center{\includegraphics[width=0.8\textwidth]{Испарители.png}}
        \caption{Испарители: 
а) из проволоки, б) из металлической фольги,
в) для сублимируемых материалов}
    \end{figure}

Давление насыщенных паров испаряемого вещества $P_{\text{нас}}$ определяется уравнением Клаузиуса–Клапейрона:
\begin{equation}
    \ln P_{\text{нас}} = -\frac{L}{R_g T} + \text{const},
\end{equation}
где $L$ — удельная теплота испарения, $R_g = 8,31 \text{ Дж/(моль·К)}$ — универсальная газовая постоянная. Для устойчивого испарения обычно выбирают такие температуры, при которых $P_{\text{нас}} \sim 10^{-2}\,\text{мм\,рт.\,ст.}$, чтобы избежать брызг расплава и обеспечить достаточную скорость напыления.

Атомы алюминия, покидающие поверхность расплава, обладают средней кинетической энергией:
\begin{equation}
    E_k = \frac{3}{2} k_B T_{\text{исп}} \approx 0,1 \text{ эВ},
\end{equation}
что значительно меньше энергии связи атомов в конденсированной фазе ($\sim 3 \text{ эВ}$) и энергии активации поверхностной миграции ($\sim 0,3\,\text{эВ}$). В результате осаждённые атомы слабо взаимодействуют с подложкой и формируют островковую структуру (режим Вольмера–Вебера). При толщине плёнки $d \gtrsim 50\,\text{нм}$ островки смыкаются, образуя сплошной слой. Оптимальная толщина для зеркальных покрытий — около $100\,\text{нм}$, при которой коэффициент пропускания $T < 10^{-3}$, а коэффициент отражения достигает $R \approx 95\%$.


\subsection*{Геометрическое распределение толщины плёнки}

При термовакуумном напылении из точечного источника толщина алюминиевой плёнки в центре подложки определяется массой испарённого вещества $m$, расстоянием от источника до подложки $R_0$ и плотностью материала $\rho$. \textcolor{gray}{Предположим, что мы ограничили бы пространство сферой радиуса $(R_0+t_0)$, то получили бы ситуацию, когда напыляется только "1 точка" — центральная, из всех точек отрезка, принадлежащего пластине}:
\begin{equation}
    t_0 = \frac{m}{4\pi \rho R_0^2 }, \qquad \rho_{\text{Al}} = 2,7~\text{г/см}^3.
\end{equation}

\textcolor{gray}{Ограничим теперь пространство сферой радиуса $(R+t_1)$, получим ситуацию, когда напыляются все точки на расстоянии L от центра пластины. Толщина плёнки в точке, находящейся на расстоянии $(L+t_1 sin \theta)$ от центра пластины}:
\begin{equation}
   \textcolor{gray}{t(\theta) = t_1 \cos\theta,}
\end{equation}
\textcolor{gray}{где $t_1=\frac{m}{4 \pi \rho R^2}$ — толщина сферы, $R=\sqrt{l^2+R_0^2}$.} Угол $\theta$ связан с геометрией соотношением $\cos\theta = \frac{R_0}{R}=R_0 / \sqrt{R_0^2 + l^2}$
\begin{equation}
   \textcolor{gray}{t(\theta) = \frac{m}{4 \pi \rho R^2} \cos\theta=\frac{mR_0}{4 \pi \rho R^3}}
\end{equation}
относительное изменение толщины можно оценить как:
\begin{equation}
    \frac{t_0}{t(l)} = \frac{R^3}{R_0^3}= \left[1 + \left(\frac{l}{R_0}\right)^2\right]^{3/2}.
\end{equation}

\textcolor{gray}{Такие рассуждения справедливы для любого сечения сферы плоскостью, перпендикулярной пластине.} Эти соотношения позволяют рассчитать ожидаемую толщину плёнки при заданной навеске и геометрии напыления, а также объяснить наблюдаемую неоднородность покрытия.

\begin{figure}[H]
        \center{\includegraphics[width=0.7\textwidth]{photo_5217942325122764094_w.jpg}}
        \caption{Распределение толщины плёнки}
    \end{figure}
Для контроля толщины полупрозрачных для света плёнок ($d \lesssim 70 \text{ нм}$) используется оптический метод, основанный на измерении коэффициента пропускания $T = I/I_0$, где $I_0$ и $I$ — интенсивности падающего и прошедшего света. Связь $T(d)$ нелинейна и определяется экспериментально (рис. 3). При малых толщинах ($d < 20 \text{ нм}$) плёнка состоит из разрозненных островков, и пропускание близко к 100\%. При $d \approx 50\,\text{нм}$ наблюдается резкое падение $T$, связанное с перколяцией и формированием сплошного проводящего слоя. \textcolor{gray}{Перколяция — явление протекания (просачивания) одной фазы сквозь другую.}
    \begin{figure}[H]
        \center{\includegraphics[width=0.6\textwidth]{Толщина.png}}
       Типовая \caption{зависимость напряжения на фотоприёмнике от толщины}
    \end{figure}
Химическая \textcolor{gray}{пассивность (инертность)} алюминия требует высокого вакуума: при давлениях выше $5 \cdot 10^{-4}\,\text{мм\,рт.\,ст.}$ происходит интенсивное окисление с образованием $\text{Al}_2\text{O}_3$, что ухудшает адгезию и оптические свойства плёнки. Уже при комнатной температуре на поверхности алюминия самопроизвольно формируется оксидный слой толщиной $1$–$2 \text{ нм}$, однако в условиях вакуумного напыления этот процесс минимизируется.


\newpage
\section*{Ход работы}

\subsection*{Минимальное допустимое расстояние от капли до плёнки}

Для выполнения задания необходимо напылить алюминиевую плёнку переменной толщины на стеклянную подложку длиной $2l = 7,5~\text{см}$, при этом капля испаряемого алюминия расположена точно над центром подложки. Требуется определить минимальное расстояние $R_0$ от источника до подложки, при котором разница толщины плёнки между центром и краем не превышает 10\%.
Согласно геометрической модели точечного источника, толщина плёнки в точке, удалённой на расстояние $l$ от центра, определяется формулой:
\begin{equation}
    \textcolor{gray}{t(l) = t_1 \cdot cos \theta = \frac{mR_0}{4 \pi \rho R^3}}
\end{equation}
Относительная разница толщины между центром ($t_0$) и краем ($t(l)$) выражается как
\begin{equation}
   \textcolor{gray}{ \frac{t_0 - t(l)}{t_0} = 1 - \frac{R_0^3}{R^3}.}
\end{equation}
Было поставлено условие: эта разница должна быть $\leq 10\% = 0,1$. Тогда:
\[
    \textcolor{gray}{1 - \frac{R_0^3}{R^3} \leq 0,1,}
\]
\[
  \textcolor{gray}{\frac{R_0^3}{R^3} \geq 0,9.}
\]
\[
 \frac{R_0}{0,9^{\frac{1}{3}}}\geq (R_0^2+l^2)^{1/2}, 
\]
\[
 \frac{R_0^2}{0,9^{\frac{2}{3}}}\geq R_0^2+l^2, 
\]
\[
 0,073 \cdot R_0^2\geq l^2, 
\]
\[
R_0\geq \frac{l}{\sqrt{0,073}}=\frac{3,75}{\sqrt{0,073}}=13,9 \text{ см}, 
\]
Таким образом, для обеспечения однородности толщины плёнки в пределах 10\% расстояние от источника до стекла должно составлять не менее 13,9 см.

\subsection*{Максимальное допустимое расстояние от капли до плёнки}

Для обеспечения свободного перелёта атомов алюминия от источника к стеклу необходимо, чтобы их длина свободного пробега в остаточном газе превышала расстояние между испарителем и стеклом. Длина свободного пробега $\lambda$ определяется по формуле
\begin{equation}
    \lambda = \frac{kT}{\sqrt{2}\,\pi d^2 P},
\end{equation}
где $k = 1,38 \cdot 10^{-23} \text{ Дж/К}$ — постоянная Больцмана, $T \approx 293 \text{ К}$ — комнатная температура, $d \approx 3,7 \cdot 10^{-10} \text{ м}$ — эффективный диаметр молекулы воздуха, $P$ — давление остаточного газа.

При давлении $P \sim 10^{-4} \text{ Торр} = 1,333 \cdot 10^{-2} \text{ Па}$ получаем:
\begin{align*}
    \lambda &= \frac{1,38 \cdot 10^{-23} \cdot 293}
    {\sqrt{2} \cdot \pi \cdot (3,7 \cdot 10^{-10})^2 \cdot 1,333 \cdot 10^{-2}} \approx 0,49 \text{ м} = 49 \text{ см}.
\end{align*}

На основе вычисленных значений минимального и максимального расстояния от капли до стекла было решено расположить стекло на высоте около $15$ см от вольфрамовой проволоки. 


\subsection*{Расчёт массы алюминия}
Для получения алюминиевой плёнки заданной толщины $d$ при термовакуумном напылении из точечного источника масса испаряемого алюминия определяется из условия равномерного распределения вещества по поверхности сферы радиуса $R_0$:

\begin{equation}
    d = \frac{m}{4\pi R_0^2 \rho},
\end{equation}
где $\rho = 2,7 \cdot 10^3~\text{кг/м}^3$ — плотность алюминия. Отсюда масса:
\begin{equation}
    m = 4\pi R_0^2 \rho d.
\end{equation}

В таком случае, для получения плёнки толщиной $d = 50~\text{нм} = 5 \cdot 10^{-8}~\text{м}$ при расстоянии $R_0 = 15~\text{см} = 0,15~\text{м}$:

\begin{align*}
    m &= 4\pi \cdot (0,15)^2 \cdot (2,7 \cdot 10^3) \cdot (5 \cdot 10^{-8}) \approx 3,8 \cdot 10^{-5}~\text{кг} = 38~\text{мг}.
\end{align*}
Таким образом, для напыления алюминиевой плёнки требуется кусочек алюминиевой проволоки массой около $38~\text{мг}$.

\subsection*{Подготовка и нанесение напыления}

Для проведения термовакуумного напыления алюминия был подготовлен резистивный испаритель на основе вольфрамовой проволоки. Отрезок вольфрамовой проволоки был тщательно протёрт салфеткой со спиртом для удаления поверхностных загрязнений и оксидов. Затем проволока была изогнута в форме галочки (шпильки) для закрепления алюминиевой проволоки.

Испаритель был установлен в вакуумной камере под стеклянным колпаком и подключён к электродам. Перед загрузкой алюминия проволока была прокалена: вакуумная камера откачивалась до давления $\sim 10^{-4}\text{ Торр}$, после чего через проволоку пропускался ток, вызывающий её нагрев до красно-оранжевого свечения ($\sim 1000^\circ\text{C}$). Это позволило удалить остатки адсорбированных газов и графитовой смазки.

После остывания проволоки на нижний конец галочки был намотан кусочек алюминиевой проволоки рассчитанной ранее массой. Подложка из стекла была закреплена на держателе на расстоянии $R_0 \approx 15\text{ см}$ от источника. Камера вновь была герметизирована и откачана до рабочего давления $\sim 10^{-4} \text{ Торр}$.

Напряжение на испаритель подавалось постепенно. Сначала наблюдалось тёмно-красное свечение, при котором алюминий начал плавиться и образовывать каплю, смачивавшую поверхность вольфрама. При дальнейшем увеличении тока проволока начала излучать очень яркий свет, и началось интенсивное испарение алюминия. Однако в процессе нагрева произошёл разрыв вольфрамовой проволоки. В результате не весь алюминий успел испариться, однако на стеклянной подложке образовалась полупрозрачная алюминиевая плёнка, пригодная для оптического контроля толщины.


\subsection*{Определение толщины напыления}
После получения напыления была проведена оценка толщины алюминиевой плёнки оптическим методом. Для этого использовался спектрометр, подключённый к ноутбуку, а в качестве источника света применялась настольная лампа.

Сначала была измерена интенсивность света без каких-либо преград -- её значение составило $ I_0 \approx 8250 $ усл. ед. Затем была измерена интенсивность света, прошедшего через стекло с напылением: $ I = 7000 $ усл. ед.

Отсюда был найден коэффициент пропускания плёнки:
\[
T = \frac{I}{I_0} = \frac{7000}{8250} \approx 0,848.
\]

Из справочных данных было определено значению $ T \approx 0,85$ соответствует толщина $ t_0 \approx 1,2\text{ нм}$. 
\textcolor{gray}{Сопоставим теоретическую толщину с практической:}
\[
t_0^T = \frac{m}{4\pi \rho R_0^2 }=\frac{38 \cdot 10^{-3}}{2,7 \cdot 4\pi \cdot 15^2 }=0,5 \cdot 10^{-9} \text{ м}=0,5 \text{ нм}
\]

\subsection*{Выводы}

В ходе лабораторной работы было осуществлено напыление алюминиевой плёнки на стеклянную подложку с использованием резистивного испарителя из вольфрамовой проволоки. После напыления толщина плёнки была оценена оптическим методом по коэффициенту пропускания света и сопоставлена со справочными данными.

Эксперимент подтвердил работоспособность используемого метода для получения тонких металлических плёнок. Однако были выявлены несколько его существенных недостатков:
\begin{enumerate}
    \item[{\bfseries 1.}] Расчёт толщины основан на предположении о точечном источнике и изотропном испарении, однако в действительности капля имеет собственный объём, а неравномерное распределение температуры по капле и влияние формы испарителя не соответствуют идеальной модели.
    \item[{\bfseries 2.}] Не учитывается частичное окисление алюминия остаточными газами, что снижает эффективную массу осаждённого металла и искажает оптические свойства плёнки. 
    \item[{\bfseries 3.}] При перегорании вольфрамовой проволоки процесс напыления прервался, что свидетельствует о том, что метод не всегда является надёжным.
\end{enumerate}

Для улучшения качества эксперимента можно предложить следующие варианты:
\begin{enumerate}
    \item[{\bfseries 1.}] Использование более надёжных испарителей (фольга, тигли).
    \item[{\bfseries 2.}] Создание более низкого давления в установке и более плавный нагрев испарителя.
    \item[{\bfseries 3.}] Прямой контроль толщины во время напыления для точного получения необходимой толщины плёнки.
\end{enumerate}


\end{document}


