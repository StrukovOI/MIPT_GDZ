\documentclass[a4paper,12pt]{article} % тип документа

% Поля страниц
\usepackage[left=2.5cm,right=2.5cm,
    top=2cm,bottom=2cm,bindingoffset=0cm]{geometry}
%%% Работа с русским языком
\usepackage{cmap}                           % поиск в PDF
\usepackage{mathtext} 			 	       % русские буквы в формулах

%Матеша
\usepackage{amsmath,amsfonts,amssymb,amsthm,mathtools} % AMS
\usepackage{icomma} % "Умная" запятая

%\mathtoolsset{showonlyrefs=true} % Показывать номера только у тех формул, на которые есть \eqref{} в тексте.

%% Шрифты
\usepackage{euscript}	 % Шрифт Евклид
\usepackage{mathrsfs} % Красивый матшрифт
    
%Отступ после заголовка    
\usepackage{indentfirst}


% Рисунки
\usepackage{floatrow,graphicx,calc}
\usepackage{wrapfig}

% Создаёем новый разделитель
\DeclareFloatSeparators{mysep}{\hspace{1cm}}

% Ссылки?
\usepackage{hyperref}
\usepackage[rgb]{xcolor}
\hypersetup{				% Гиперссылки
    colorlinks=true,       	% false: ссылки в рамках
	urlcolor=blue          % на URL
}


%  Русский язык
\usepackage[T2A]{fontenc}			% кодировка
\usepackage[utf8]{inputenc}			% кодировка исходного текста
\usepackage[english,russian]{babel}	% локализация и переносы


% Математика
\usepackage{amsmath,amsfonts,amssymb,amsthm,mathtools}


% Что-то 
\usepackage{wasysym}


%Заговолок
% \author{Серебренников Даниил Б02-826}
% \title{Лабораторная работа \No 2.2.3}


\begin{document}


\begin{center}
\footnotesize{ФЕДЕРАЛЬНОЕ ГОСУДАРСТВЕННОЕ АВТОНОМНОЕ УЧРЕЖДЕНИЕ}\\
\footnotesize{МОСКОВСКИЙ ФИЗИКО-ТЕХНИЧЕСКИЙ ИНСТИТУТ\\(НАЦИОНАЛЬНЫЙ 			ИССЛЕДОВАТЕЛЬСКИЙ УНИВЕРСИТЕТ)}\\
\footnotesize{ФИЗТЕХ-ШКОЛА ЭЛЕКТРОНИКИ, ФОТОНИКИ И МОЛЕКУЛЯРНОЙ ФИЗИКИ \\}
\hfill \break
\hfill\break
\hfill\break
\hfill \break
\hfill \break
\hfill \break
\hfill \break
\hfill \break
\hfill \break

\hfill \break
\large{Лабораторная работа \\ по курсу Вакуумная электроника\\\textbf{Методы \\ получения высокого вакуума}}\\
\hfill \break
\hfill \break
\hfill \break
\hfill \break
\hfill \break
\begin{flushright}
    Винокуров Владислав\\
    Рагозина Елизавета\\
	Струков Олег\\
	Группа Б04-404
\end{flushright}
\hfill \break
\hfill \break
\hfill \break

\hfill \break
\hfill \break
\hfill \break
\hfill \break
\end{center}
\hfill \break
\hfill \break
\hfill \break
\hfill \break
\hfill \break
\begin{center}
	Долгопрудный, 2025 г.
\end{center}
\thispagestyle{empty} % выключаем отображение номера для этой страницы


\newpage

\section*{Цель работы}
\begin{enumerate}
    \item Ознакомиться с принципами работы вакуумной техники: пластинчато-роторного насоса, турбомолекулярного насоса, ионизационного, ёмкостного и терморезистивного вакуумметров.
    \item Ознакомиться с методами вакуумных расчётов, найти зависимость величины газового потока в системе от давления.
    \item Определить производительность турбомолекулярного насоса.
    \item Рассчитать объем рабочей камеры.
\end{enumerate}


\section*{Теоретическая часть}

\subsection*{Основные понятия, используемые в работе}

% В вакуумной технике используются следующие ключевые параметры:

\begin{itemize}
\item \textbf{Поток газа} $Q$ — количество газа, проходящего через поперечное сечение трубопровода за единицу времени
\item \textbf{Проводимость} $U$ — способность вакуумной системы пропускать газ:
\begin{equation}
U = \frac{Q}{(P_2 - P_1)}
\label{eq:conductance}
\end{equation}
\item \textbf{Быстрота действия насоса} $S_n$ — объём газа, поступающего в насос в единицу времени при давлении $P_1$:
\begin{equation}
S_n = -\frac{dV_n}{d\tau}\bigg|_{P_1}
\label{eq:pump_speed}
\end{equation}
% \item \textbf{Быстрота откачки объема} $S_0$ — скорость откачки со стороны откачиваемого объекта
\end{itemize}

\subsection*{Основное уравнение вакуумной техники}

Для системы, состоящей из откачиваемого объема, трубопровода и насоса, справедливо соотношение:

\begin{equation}
\frac{1}{S_0} = \frac{1}{U} + \frac{1}{S_n}
\label{eq:main_vacuum}
\end{equation}

где $U$ — проводимость трубопровода, соединяющего насос с откачиваемым объемом.

\subsection*{Режимы течения газа}

\textbf{Вязкостный режим} (ламинарное течение) характерен для высоких давлений. Проводимость круглого трубопровода:

\begin{equation}
U_{\text{тв}} = \frac{\pi r_0^4 (P_2 + P_1)}{16\eta l}
\label{eq:viscous_flow}
\end{equation}

\textbf{Молекулярный режим} возникает при низких давлениях. Проводимость круглого трубопровода:

\begin{equation}
U_{\text{мол}} = \frac{\pi d^3 \langle v \rangle}{12l}
\label{eq:molecular_flow}
\end{equation}

где $\langle v \rangle = \sqrt{\frac{8RT}{\pi\mu}}$ — средняя тепловая скорость молекул.

\subsection*{Проводимость диафрагмы}

Для молекулярного течения через отверстие проводимость определяется формулой:

\begin{equation}
U_{\text{отв}} = \frac{S \langle v \rangle}{4} = 91 \cdot d^2 \quad [\text{л/с}]
\label{eq:orifice_conductance}
\end{equation}

где $d$ выражено в см.

\subsection*{Уравнение откачки}

Процесс откачки описывается дифференциальным уравнением:

\begin{equation}
\frac{dP}{dt} = -\frac{S(P) \cdot P}{V}
\label{eq:pumping_equation}
\end{equation}

При постоянной быстроте действия $S = const$ решение имеет вид:

\begin{equation}
P(t) = P_0 \cdot \exp\left(-\frac{S}{V} \cdot t\right)
\label{eq:exponential_pumping}
\end{equation}

В реальных условиях, когда $S$ зависит от давления, используется линеаризованная форма:

\begin{equation}
\ln P(t) = \ln P_0 - \frac{S}{V} \cdot t
\label{eq:linearized_pumping}
\end{equation}

\subsection*{Методы измерения быстроты действия}

\textbf{Метод постоянного объема} основан на измерении зависимости $P(t)$:

\begin{equation}
S_n = \frac{V}{\tau} \ln \frac{P_0}{P_1}
\label{eq:constant_volume}
\end{equation}

\textbf{Метод постоянного давления} используется для высокопроизводительных насосов:

\begin{equation}
S_n = \frac{Q}{P_n}
\label{eq:constant_pressure}
\end{equation}

где $Q$ — измеренный поток газа, напускаемого в систему.

\subsection*{Критерий режима течения}

Режим течения газа определяется соотношением между длиной свободного пробега молекул $\lambda$ и характерным размером системы $d$:

\begin{itemize}
\item Вязкостный режим: $\lambda \ll d$
\item Молекулярный режим: $\lambda \gg d$
\item Переходный режим: $\lambda \sim d$
\end{itemize}

Длина свободного пробега вычисляется по формуле:

\begin{equation}
\lambda = \frac{kT}{\sqrt{2}\pi d_м^2 P}
\label{eq:mean_free_path}
\end{equation}




\section*{Лабораторная установка}

Лабораторная установка предназначена для ознакомления с основными приборами вакуумной техники: насосами, манометрами, измерителями расхода газа. Схема установки представлена на рисунке 1.
\begin{figure}[h]
    \centering
    \includegraphics[width=\textwidth]{facility.PNG}
    \caption{Схема лабораторной установки}
    \label{fig:vac}
\end{figure}

На схеме обозначены: \\
$B_1$ - вакуумметр ёмкостной\\
$B_2$ - вакуумметр терморезисторный\\
$B_3$ - вакуумметр ионизационный\\
$K_1$ - кран турбомолекулярного насоса\\
$K_3$ - высоковакуумная заслонка\\
$K_4$ - форвакуумная заслонка\\
$K_2, K_7$ - коммутационные краны\\
Д - диафрагма\\
FC - регулятор газового потока (flow controller)\\
ТМН - турбомолекулярный насос\\
ФВН - форвакуумный насос\\


\section*{Ход работы}
\subsection*{Форвакуумный насос}
    После ознакомления с установкой был запущен форвакуумный насос, ёмкостной (В1) и терморезисторный (В2) вакуумметры и регулятор газового потока, изначально установленный на 0 см$^3$/мин.
    Затем была получена зависимость давления от времени при откачке системы форвакуумным насосом (рис. 2) и зависимость давления от величины газового потока (рис. 3)
    
    Получена зависимость производительности насоса $S = Q/B$ от входного давления.
    Численные результаты приведены в таблице 1.

\subsection*{Турбомолекулярный насос}
 Далее был включён турбомолекулярный насос и ионизационный вакуумметр (В3), после чего была получена зависимость давления в высоковакуумной части от времени при откачке и при потоке газа через диафрагму.




Появление скачков на некоторых графиках связано с тем, что ионизационный вакуумметр переключает режим работы, повышая накал нити при уменьшении давления.

\subsubsection*{Параметры установки}
\begin{itemize}
    \item Диаметр диафрагмы: $d = 100$ мкм $= 0.01$ см
    \item Газ: воздух (молярная масса $M = 29$ г/моль)
    \item Температура: $T = 293$ К
\end{itemize}

\subsubsection*{Поток воздуха через диафрагму}
Определим, можно ли считать течение газа через диафрагму молекулярным. Для этого оценим длину свободного пробега молекул:
\begin{center}
$\lambda = \frac{kT}{\sigma P} \approx 1,24$ м,
\end{center}
где $k = 1,38 \cdot 10^{-23}$ Дж/К -- постоянная Больцмана\\
    %   $T \approx 293$ К – комнатная температура\\
       $\sigma = 62,5 \cdot 10^{-20} $ м$^2$ -- среднее эффективное сечение рассеяния для воздуха\\
       $P \approx 3 \cdot 10^{-5}$ Торр -- максимальное давление в высоковакуумной части системы\\

Видно, что $d \ll \lambda$, поэтому течение газа через диафрагму можно считать молекулярным. Следовательно, справедлива формула нахождения молекулярного потока через диафргаму:
\begin{center}
$Q = S\sqrt{\frac{RT}{2\pi\mu}}(P_2-P_3)$
\end{center}
где $P_2, P_3$ - давления на В2 и В3 соответственно\\
$S = \frac{\pi d^2}{4}$ - площадь отверстия в диафрагме\\
$\mu$ - молярная масса воздуха.

Подставив величины, получаем, что 
\begin{center}
$Q \approx 9,1\cdot10^{-4}(P_2-P_3)$ л/с (давление выражено в Торрах)
\end{center}
График зависимости величины потока через диафрагму от времени изображён на рисунке 6.


\subsubsection*{Производительность турбомолекулярного насоса}
Рассмотрим модель потока через турбомолекулярный насос. В установившемся режиме выполняется баланс:
\begin{center}
$P_3 \cdot S(P_3) = Q$
\end{center}

Отсюда определим быстроту действия насоса определим через значение потока:
\begin{center}
$S(P_3) = \frac{Q}{P_3}$
\end{center}


\subsection*{Определение объёма вакуумной системы}

\subsubsection*{Основное уравнение откачки}

Процесс откачки описывается дифференциальным уравнением:
\[
\frac{dP}{dt} = -\frac{S(P) \cdot P}{V}
\]
где:
\begin{itemize}
    \item $P$ -- давление в системе, Торр
    \item $t$ -- время, с
    \item $S(P)$ -- быстрота действия насоса, л/с
    \item $V$ -- объём системы, л
\end{itemize}

\subsubsection*{Решение для постоянной быстроты действия}

При условии $S = const$ на выбранном участке, уравнение имеет решение:
\[
P(t) = P_0 \cdot \exp\left(-\frac{S}{V} \cdot t\right)
\]
где $P_0$ -- начальное давление при $t = 0$.

\subsubsection*{Линеаризация зависимости}

Прологарифмируем уравнение откачки:
\[
\ln P(t) = \ln P_0 - \frac{S}{V} \cdot t
\]

Получили линейную зависимость вида
\[
y = A + Bx
\]

Для расчётов был взят участок, на котором S почти линейно зависела от времени, найдено её среднее значение $S_{\text{ср}} = 33,8$ л/с, а также угловой коэффициент наклона зависимости $\ln P(t) = \ln P_0 - \frac{S}{V} \cdot t$, 
то есть отсюда $S/V \approx 0,00938 ~c^{-1}$.
Таким образом, $V \approx 3600$ л.

При попытке графической аппроксимации было выяснено, что начальную часть кривой лучше всего описывает зависимость с $V \approx 3600$ л. Данный вариант изображён на рис. 8.

\subsubsection*{Условия применимости аппроксимирующей формулы}

\begin{enumerate}
    \item Быстрота действия насоса $S$ постоянна на выбранном временном интервале
    \item Отсутствуют значительные натекания газа в системе
    \item Температура системы постоянна
    \item Однотипный режим течения газа (молекулярный или вязкостный)
\end{enumerate}



\section*{Анализ полученных результатов}

\subsection*{Характеристики форвакуумного насоса}
\begin{itemize}
\item Быстрота действия форвакуумного насоса при атмосферном давлении составляет около 0,75 л/с
\item Зависимость $S(P)$ показывает относительную стабильность производительности насоса
\item Предельное давление в форвакуумной части, достигнутое в результате выполнения работы, составляет 0,016 Торр.
\end{itemize}

\subsection*{Характеристики турбомолекулярного насоса}
\begin{itemize}
\item Быстрота действия турбомолекулярного насоса варьируется от 8 до 36 л/с в рабочем диапазоне давлений
\item Насос работает в диапазоне давлений $10^{-5} - 10^{-3}$ Торр, предельное давление -- $1,7\cdot10^{-6}$ Торр
\item Зависимость $P_3(t)$ при несильно низких давлениях аппроксимируется функцией $P(t) = P_0 \cdot \exp\left(-\frac{S}{V} \cdot t\right)$
\end{itemize}

\subsection*{Анализ течения через диафрагму}
\begin{itemize}
\item Подтверждён молекулярный режим течения ($d \ll \lambda$)
\item Рассчитанная проводимость диафрагмы соответствует теоретическим ожиданиям
\item Поток через диафрагму стабилен в установившемся режиме
\end{itemize}

\subsection*{Оценка точности определения объёма}
\begin{itemize}
\item Рассчитанный объём системы (3600 л) не является правдоподобным
\end{itemize}
% Так как $P_3 \ll P_2$, формула примет вид
% \begin{center}
% $Q \approx \frac{\pi d^2}{4}\sqrt{\frac{RT}{2\pi\mu}}P_2$
% \end{center}


% \subsubsection*{Расчёт проводимости диафрагмы}
% Для молекулярного течения через диафрагму используем формулу из методички:
% \begin{equation}
% U = 91 \cdot d^2 = 91 \cdot (0.01)^2 = 9.1 \times 10^{-3} \, \text{л/с}
% \end{equation}

% \subsubsection*{Расчёт потока через диафрагму}
% Поток через диафрагму рассчитывается по формуле:
% \begin{equation}
% Q = U \cdot (P_2 - P_1) = 9.1 \times 10^{-3} \cdot (P_{\text{B2}} - P_{\text{B3}}) \, \text{Торр·л/с}
% \end{equation}
% где $P_{\text{B2}}$ -- давление в форвакуумной части, $P_{\text{B3}}$ -- давление в высоковакуумной части.

% \subsubsection*{Расчёт быстроты действия турбомолекулярного насоса}
% Быстрота действия насоса определяется как:
% \begin{equation}
% S = \frac{Q}{P_1} = \frac{Q}{P_{\text{B3}}} \, \text{л/с}
% \end{equation}

% \subsubsection*{Аппроксимация кривой откачки}
% Кривая откачки аппроксимируется функцией:
% \begin{equation}
% P(t) = P_{\text{lim}} + (P_0 - P_{\text{lim}}) \cdot \exp\left(-\frac{t}{\tau}\right)
% \end{equation}
% где:
% \begin{itemize}
%     \item $P_0$ -- начальное давление
%     \item $P_{\text{lim}}$ -- предельное давление
%     \item $\tau$ -- характерное время откачки
% \end{itemize}


% Для расчёта параметров использовались полные временные зависимости давлений $P_{\text{B2}}(t)$ и $P_{\text{B3}}(t)$, записанные с частотой 1 Гц. Обработка проводилась программными методами с использованием:

% \begin{enumerate}
%     \item \textbf{Поток через диафрагму:} $Q(t) = U \cdot (P_{\text{B2}}(t) - P_{\text{B3}}(t))$
%     \item \textbf{Мгновенная быстрота действия:} $S(t) = Q(t) / P_{\text{B3}}(t)$
%     \item \textbf{Средняя быстрота действия:} усреднение $S(t)$ на установившемся участке
%     \item \textbf{Аппроксимация кривой откачки:} методом нелинейной регрессии
%     \item \textbf{Объём системы:} $V = S_{\text{ср}} \cdot \tau$
% \end{enumerate}


% \subsection*{Результаты расчётов}

% \begin{table}[h!]
% \centering
% \caption{Параметры турбомолекулярного насоса}
% \begin{tabular}{|l|c|c|}
% \hline
% Параметр & Значение & Единица измерения \\
% \hline
% Диаметр диафрагмы & 100 & мкм \\
% Проводимость диафрагмы & $9.1 \times 10^{-3}$ & л/с \\
% Средняя быстрота действия & \textbf{XXX} & л/с \\
% Предельное давление & \textbf{XXX} & Торр \\
% Характерное время откачки $\tau$ & \textbf{XXX} & с \\
% Рассчитанный объём камеры & \textbf{XXX} & л \\
% \hline
% \end{tabular}
% \end{table}

% \begin{figure}[h!]
% \centering
% \includegraphics[width=0.9\textwidth]{3_turbo_analysis_complete.pdf}
% \caption{Полный анализ работы турбомолекулярного насоса}
% \label{fig:turbo_analysis}
% \end{figure}

% \subsection{Анализ результатов}

% \begin{enumerate}
%     \item \textbf{Проводимость диафрагмы} составляет $9.1 \times 10^{-3}$ л/с, что характерно для малых отверстий в молекулярном режиме течения.
    
%     \item \textbf{Быстрота действия турбомолекулярного насоса} в рабочем диапазоне составляет \textbf{XXX} л/с, что соответствует типичным значениям для насосов данного типа.
    
%     \item \textbf{Предельное давление} турбомолекулярного насоса составило \textbf{XXX} Торр, что демонстрирует его способность создавать высокий вакуум.
    
%     \item \textbf{Аппроксимация кривой откачки} показывает, что процесс описывается экспоненциальной зависимостью с характерным временем $\tau = \textbf{XXX}$ с.
    
%     \item \textbf{Рассчитанный объем камеры} составляет \textbf{XXX} л, что согласуется с геометрическими размерами вакуумной системы.
% \end{enumerate}

% \subsection{Выводы}

% \begin{itemize}
%     \item Турбомолекулярный насос обеспечивает получение высокого вакуума до $10^{-7}$ Торр
%     \item Метод измерения потока через диафрагму позволяет определить быстроту действия насоса в высоковакуумном диапазоне
%     \item Экспоненциальная модель адекватно описывает процесс откачки
%     \item Полученные параметры соответствуют техническим характеристикам используемого оборудования
% \end{itemize}




\newpage
\section*{Выводы}
    В результате выполнения работы
\begin{enumerate}

\item Освоены принципы работы вакуумного оборудования: форвакуумного пластинчато-роторного насоса, турбомолекулярного насоса, вакуумметров различных типов

\item Экспериментально определены характеристики насосов

\item Подтверждена применимость формулы молекулярного течения для расчёта потока через диафрагму

\item Определён теоретический объём вакуумной системы

\item Получено прдельное минимальное давления порядка $10^{-6}$ Торр

\item Методики измерений показали свою работоспособность, однако применимость их в некоторых случаях ограничена.
\end{enumerate}

\section*{Приложение}

    \begin{table}[h!]
\centering
\caption{Зависимость давления и быстроты действия от газового потока}
\label{tab:pressure_vs_flow}
\begin{tabular}{|c|c|c|c|}
\hline
Qset, см$^3$/мин & Q, см$^3$/мин & P, Торр & S, л/с \\
\hline
0 & 0 & 0,094 & 0 \\
5 & 5,03 & 0,216 & 0,295 \\
10 & 10,07 & 0,300 & 0,425 \\
15 & 15,07 & 0,379 & 0,504 \\
20 & 20,07 & 0,454 & 0,560 \\
25 & 25,06 & 0,526 & 0,604 \\
30 & 30,07 & 0,597 & 0,638 \\
35 & 35,07 & 0,669 & 0,664 \\
40 & 40,07 & 0,739 & 0,687 \\
45 & 45,07 & 0,810 & 0,705 \\
50 & 50,10 & 0,879 & 0,722 \\
55 & 55,50 & 0,931 & 0,755 \\
60 & 55,76 & 0,933 & 0,757 \\
55 & 55,40 & 0,931 & 0,754 \\
50 & 50,08 & 0,884 & 0,718 \\
45 & 45,08 & 0,815 & 0,701 \\
40 & 40,06 & 0,746 & 0,680 \\
35 & 35,06 & 0,675 & 0,658 \\
30 & 30,07 & 0,602 & 0,633 \\
25 & 25,06 & 0,531 & 0,598 \\
20 & 20,06 & 0,457 & 0,556 \\
15 & 15,07 & 0,380 & 0,502 \\
10 & 10,07 & 0,298 & 0,428 \\
5 & 5,06 & 0,209 & 0,307 \\
0 & 0 & 0,082 & 0 \\
\hline
\end{tabular}
\end{table}



    
    \begin{figure}[ht!]
        \center{\includegraphics[width=1\textwidth]{2 P(t).pdf}}
        \caption{График зависимости давления в форвакуумной части от времени}
    \end{figure}

    \begin{figure}[ht!]
        \center{\includegraphics[width=1\textwidth]{2 P(Q).pdf}}
        \caption{График зависимости давления в форвакуумной части от газового потока}
    \end{figure}

    \begin{figure}[ht!]
        \center{\includegraphics[width=1\textwidth]{2 S(P).pdf}}
        \caption{График зависимости быстроты действия форвакуумного насоса от давления}
    \end{figure}

    \begin{figure}[ht!]
        \center{\includegraphics[width=1\textwidth]{3 P(t).pdf}}
        \caption{График зависимости давления в высоковакуумной части от времени}
    \end{figure}

    \begin{figure}[ht!]
        \center{\includegraphics[width=1\textwidth]{Q(t).pdf}}
        \caption{График зависимости потока воздуха через диафрагму от времени}
    \end{figure}

    \begin{figure}[ht!]
        \center{\includegraphics[width=1\textwidth]{S(B3).pdf}}
        \caption{График зависимости быстродействия турбомолекулярного насоса от впускного давления}
    \end{figure}

    % \begin{figure}[ht!]
    %     \center{\includegraphics[width=1\textwidth]{B3end.pdf}}
    %     \caption{График зависимости давления в высоковакуумной части от времени при окончательной откачке}
    % \end{figure}

    \begin{figure}[ht!]
        \center{\includegraphics[width=1\textwidth]{app B3.pdf}}
        \caption{Аппроксимация давления в высоковакуумной части}
    \end{figure}


\end{document}