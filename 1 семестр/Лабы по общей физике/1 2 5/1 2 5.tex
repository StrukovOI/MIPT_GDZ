\documentclass[a4paper]{article}
\usepackage[utf8]{inputenc}
\usepackage[russian,english]{babel}
\usepackage[T2A]{fontenc}
\usepackage[left=10mm, top=20mm, right=18mm, bottom=15mm, footskip=10mm]{geometry}
\usepackage{indentfirst}
\usepackage{amsmath,amssymb}
\usepackage[italicdiff]{physics}
\usepackage{graphicx}
\graphicspath{{images/}}
\DeclareGraphicsExtensions{.pdf,.png,.jpg}
\usepackage{wrapfig}

\usepackage{caption}
\captionsetup[figure]{name=Рисунок}
%\captionsetup[table]{name=Таблица}
  
\title{\begin{center}Лабораторная работа №1.2.5\end{center}
Исследование вынужденной регулярной прецессии гироскопа}
\author{Струков О. И. \\
Б04-404}
\date{}


\begin{document}

\maketitle
\newpage
	
	\section*{Введение}
	
	\textbf{Цель работы:} исследовать вынужденную прецессию гироскопа, установить зависимость скорости вынужденной прецессии от величины момента сил, действующий на ось гироскопа и сравнить ее со скоростью, рассчитанной по скорости прецессии.\\
	\textbf{Оборудование:} гироскоп в кардановом подвесе, секундомер, набор грузов, отдельный ротор гироскопа, цилиндр известной массы, крутильный маятник, штангенсциркуль, линейка.
	
	\section*{Теоретические сведения}
	
	В этой работе исследуется зависимость скорости прецессии гироскопа от момента силы, приложенной к его оси. Для этого к оси гироскопа подвешиваются грузы. Скорость прецессии определяется по числу оборотов рычага вокруг вертикальной оси и времни, которое на это ушло, определяемоу секундомером. В процессе измерений рычаг не только поворачивается в результате прецессии гироскопа, но и опускается. Поэтому его в начале опыта следует преподнять на 5-6 градусов.  Опять надо закончить, когда рычаг опустится на такой же угол.\\
	\begin{center}$
		\begin{array}{cc}
			\includegraphics[width=0.40\textwidth]{img1.png}&
			\includegraphics[width=0.40\textwidth]{img2.png}\\
		\end{array}$
	\end{center}
	
	Измерение скорости прецессии гироскопа позволяет вычислить угловую скорость вращения его ротора. Расчет производится по формуле:
	
	\begin{equation}
		\Omega = \frac{mgl}{I_z\omega_0},
	\end{equation}
	
	где $m$ -- масса груза, $l$ -- расстояние от центра карданова подвеса до точки крепления груза на оси гироскопа, $I_z$ -- момент инерции гироскопа по его главной оси вращения. $\omega_0$ -- частота его вращения относительно главной оси, $\Omega$ -- частота прецессии.\\
	Момент инерции ротора относительно оси симметрии $I_0$ измеряется по крутильным колебаниям точной копии ротора, подвешиваемой вдоль оси симметрии на десткой проволоке. Период крутильных колебаний $T_0$ зависит от момента инерции $I_0$ и модуля кручения проволоки $f$:
	
	\begin{equation}
		T_0 = 2\pi\sqrt{\frac{I_0}{f}}.
	\end{equation}
	
	Чтобы исключить модуль кручения проволоки, вместо ротора гироскопа к той же проволоке подвешивают цилиндр правильной формы с известными размерами и массой, для которого легко можно вычислить момент инерции $I_\text{ц}$. Для определения момента инерции ротора гироскопа имеем:
	
	\begin{equation}
		I_0 = I_\text{ц}\frac{T_0^2}{T_\text{ц}^2},
		\label{moment}
	\end{equation}
	Здесь $T_\text{ц}$ -- период крутильных колебаний цилиндра.\\
	\begin{center}
		\includegraphics[width=0.6\textwidth]{img3.png}
	\end{center}
	
	Скорость вращения ротора гироскопа можно определить и не прибегая к исследованию прецессии. У используемых в работе гироскопов статор имеет две обмотки, необходимые для быстрой раскрутки гироскопа. В данной работе одну обмотку искользуют для раскрутки гироскопа, а вторую -- для измерения числа оборотов ротора. Ротор электромотора всегда немного намагничен. Вращаясь, он наводит во второй обмотке переменную ЭДС индукции, частота которой равна частоте врещения ротора. Частоту этой ЭДС можно, в частности, измерить по фигурам Лиссажу, получаемым на экране осциллографа, если на один вход подать исследуемую ЭДС, а на другой -- переменное напряжение с хорошо прокалиброванного генератора. При совпадении частот на экране получаем эллипс.
\newpage
\section*{Ход работы}
\begin{enumerate}
	\item Ось гироскопа была установлена в горизонтальное положение на высоте $h_0 = 11,1$ см от поверхности стола. Известно, что центр оси вращения гироскопа находится на расстоянии $l = 11,9$ см от конца рычага.
	\item Было включено питание гироскопа и выждано необходимое время, чтобы вращение ротора успело стабилизироваться.
	\item Было установлено, что ротор вращается достаточно быстро: при лёгком постукивании по рычагу он не изменял своего положения в пространстве. При более сильном нажатии на рычаг гироскоп начинал медленно вращаться против часовой стрелки.
	\item К рычагу гироскопа был подвешен груз. При этом началась его прецессия: из-за трения в оси рычаг начал медленно опускаться.
	\item Была проделана серия экспериментов, в которой рычаг отклоняли на несколько градусов вверх (до высоты $h_1$), подвешивали к нему груз, а затем с помощью секундомера определяли период и угловую скорость регулярной прецессии $\Omega = \frac{2\pi}{T}$. Для каждого груза был определён момент силы $M = mgl$, создаваемый им. Измерения продолжались до тех пор, пока рычаг не опускался на несколько градусов ниже горизонтальной оси (до высоты $h_2$), сделав целое число оборотов вокруг вертикальной. Также была измерена скорость опускания рычага. Для каждого из семи грузов опыт был повторён два раза. Результаты измерений и вычислений занесены в таблицу. Зависимость $\Omega$ от $M$ изображена на графике.
\begin{center}
	\begin{tabular}{|c|c|c|c|c|c|c|c|c|c|}
	\hline
	$m$, г & $M$, Н$\cdot$м & Время, с & Об. & Период, с & $\Omega, \text{с}^{-1}$ & $h_1$, см & $h_2$, см & $\Omega_f, \text{с}^{-1}\cdot10^{-4}$ & $M_{\text{тр}}$, Н$\cdot$м$\cdot10^{-3}$ \\ \hline
	335 & 0,391 & 269 & 9 & 29,889 & 0,210 $\pm$ 0,004 & 12,5 & 10,0 & 7,825 & 1,456 $\pm$ 0,03 \\ \hline
	335 & 0,391 & 270 & 9 & 30,000 & 0,209 $\pm$ 0,003 & 12,5 & 10,0 & 7,796 & 1,456 $\pm$ 0,03 \\ \hline
	270 & 0,315 & 297 & 8 & 37,125 & 0,169 $\pm$ 0,002 & 12,5 & 9,8 & 7,656 & 1,426 $\pm$ 0,03 \\ \hline
	270 & 0,315 & 298 & 8 & 37,250 & 0,169 $\pm$ 0,002 & 12,5 & 9,8 & 7,630 & 1,426 $\pm$ 0,03 \\ \hline
	215 & 0,251 & 328 & 7 & 46,857 & 0,134 $\pm$ 0,001 & 12,5 & 9,4 & 7,965 & 1,491 $\pm$ 0,03 \\ \hline
	215 & 0,251 & 329 & 7 & 47,000 & 0,134 $\pm$ 0,001 & 12,5 & 9,4 & 7,941 & 1,491 $\pm$ 0,03 \\ \hline
	173 & 0,202 & 292 & 5 & 58,400 & 0,108 $\pm$ 0,001 & 12,5 & 9,8 & 7,787 & 1,462 $\pm$ 0,03 \\ \hline
	173 & 0,202 & 351 & 6 & 58,500 & 0,107 $\pm$ 0,001 & 12,5 & 9,3 & 7,686 & 1,445 $\pm$ 0,03 \\ \hline
	142 & 0,166 & 355 & 5 & 71,000 & 0,088 $\pm$ 0,0006 & 12,5 & 9,2 & 7,839 & 1,468 $\pm$ 0,03 \\ \hline
	142 & 0,166 & 356 & 5 & 71,200 & 0,088 $\pm$ 0,0006 & 12,5 & 9,2 & 7,817 & 1,468 $\pm$ 0,03 \\ \hline
	116 & 0,135 & 347 & 4 & 86,750 & 0,072 $\pm$ 0,0004 & 12,5 & 9,2 & 8,019 & 1,499 $\pm$ 0,03 \\ \hline
	116 & 0,135 & 347 & 4 & 86,750 & 0,072 $\pm$ 0,0004 & 12,5 & 9,3 & 7,774 & 1,453 $\pm$ 0,03 \\ \hline
	77 & 0,090 & 397 & 3 & 132,333 & 0,047 $\pm$ 0,0002 & 12,5 & 8,8 & 7,870 & 1,490 $\pm$ 0,03 \\ \hline
	77 & 0,090 & 395 & 3 & 131,667 & 0,048 $\pm$ 0,0002 & 12,5 & 8,9 & 7,693 & 1,449 $\pm$ 0,03 \\ \hline
	\end{tabular}
\end{center}
Вычислена угловая скорость опускания гироскопа $\Omega_f$ по приведённой ниже формуле, результаты добавлены в таблицу. Определено, что среднее значение составляет $< \Omega_f> = 7,807\cdot10^{-4}\text{с}^{-1}$ 
\[ \Omega_f = \frac{\arcsin((h_1 - h_0) / l) - \arcsin((h_2 - h_0) / l)}{t} \]

	\item Был определён момент инерции ротора гироскопа относительно оси симметрии $I_0$. Для этого ротор, извлечённый из такого же гироскопа, был подвешен к концу вертикально висящей проволоки так, чтобы ось симметрии гироскопа была вертикальна, и был измерен период крутильных колебаний получившегося маятника. Затем ротор был заменён цилиндром, у которого были измерены радиус $r_{\text{ц}}$ и масса $m_{\text{ц}}$ (с учётом погрешностей), и для него был определён период крутильных колебаний. Момент инерции цилиндра $I_{\text{ц}}$ можно найти по формуле $I_{\text{ц}} = \frac{1}{2}m_{\text{ц}}r_{\text{ц}}^2$. Момент инерции ротора был определён с помощью формулы (3).
	\begin{center}
		\begin{tabular}{|c|c|c|c|c|c|c|c|c|}
		\hline
		$m_{\text{ц}}$, г & $r_{\text{ц}}$, мм & $\sigma_{m_{\text{ц}}}$, г & $\sigma_{r_{\text{ц}}}$, мм & $T_{\text{ц}}$, с & $T_0$, с & $\sigma_T$, с & $I_{\text{ц}}$, кг$\cdot$м$^2$$\cdot$10$^{-3}$ & $I_0$, кг$\cdot$м$^2$$\cdot$10$^{-3}$ \\ \hline
		1616,2 & 39 & 0,1 & 0,05 & 4,03 & 3,20 & 0,025 & 1,229 & 0,775 \\ \hline
		\end{tabular}
	\end{center}
	\item Погрешность однократного измерения времени $\sigma_t \approx 0,25$ с (скорость реакции человека). Были оценены погрешности $I_0$ и $\Omega$ с помощью формул ниже. Так как измерение периода прецессии для каждого груза было проведено дважды, и в полученных значениях не было сильных отклонений, случайную погрешность $\Omega$ можно считать примерно равной нулю.
	\[ \sigma_{I_0} = I_0\sqrt{2\left(\frac{\sigma_{T_0}}{T_0}\right)^2+2\left(\frac{\sigma_{T_{\text{ц}}}}{T_{\text{ц}}}\right)^2+\left(\frac{\sigma_{m_{\text{ц}}}}{m_{\text{ц}}}\right)^2+2\left(\frac{\sigma_{r_{\text{ц}}}}{r_{\text{ц}}}\right)^2} \qquad \sigma_{\Omega} = \Omega \cdot \frac{2\sigma_t}{t}\]
Таким образом, $\sigma_{I_0} = 1,1\cdot$10$^{-5}$ кг$\cdot$м$^2$. Данные о $\sigma_{\Omega}$ добавлены в таблицу с её значениями.

\item Частота вращения гироскопа $\omega_0$ была определена с помощью углового коэффициента $k$ построенного графика зависимости $\Omega$ от $M$: $\omega_0 = \frac{1}{kI_0}$. $k$ и его погрешность $\sigma_k$ определены с помощью МНК. Погрешность частоты вращения $\sigma_{\omega_0} = \omega_0\sqrt{\left(\frac{\sigma_k}{k}\right)^2 + \left(\frac{\sigma_{I_0}}{I_0}\right)^2}$.
\begin{center}
	\begin{tabular}{|c|c|c|c|}
	\hline
	$k$, $\frac{1}{\text{Дж$\cdot$с}}$ & $\sigma_k$, $\frac{1}{\text{Дж$\cdot$с}}$ & $\omega_0$, с$^{-1}$ & $\sigma_{\omega_0}$, с$^{-1}$ \\ \hline
	0,538 & 0,002 & 2398 & 35 \\ \hline
	\end{tabular}
\end{center}
\item По скорости опускания рычага во время прецессии был определён момент сил трения $M_{\text{тр}}$ и оценена его погрешность $\sigma_{M_{\text{тр}}}$. Данные добавлены в таблицу выше.
	\[ M_{\text{тр}} = I_0 \omega_0 \Omega_f \qquad \sigma_{M_{\text{тр}}} = M_{\text{тр}}\sqrt{\left(\frac{\sigma_{I_0}}{I_0}\right)^2 + \left(\frac{\sigma_{\omega_0}}{\omega_0}\right)^2 + \left(\frac{\sigma_{\Omega_f}}{\Omega_f}\right)^2}\]
\item Гироскоп был выключен, и с помощью осциллографа измерялось время, за которое он замедлялся до значения частоты вращения $\nu$, равного 180 с$^{-1}$ с шагом в 20 с$^{-1}$. Результаты измерений внесены в таблицу.
\begin{center}
	\begin{tabular}{|c|c|c|c|c|c|c|c|c|c|c|c|}
	\hline
	$t$, с & 4,56 & 55,3 & 110 & 166 & 227 & 290 & 356 & 432 & 513 & 600 & 694 \\ \hline
	$\nu$, с$^{-1}$ & 380 & 360 & 340 & 320 & 300 & 280 & 260 & 240 & 220 & 200 & 180 \\ \hline
	$\frac{t}{I_0}, \frac{\text{с}}{\text{кг$\cdot$м}^2}$ & 5884 & 71355 & 141935 & 214194 & 292903 & 374194 & 459355 & 557419 & 661935 & 774193 & 895484 \\ \hline
	$\ln(\frac{\nu_0}{\nu(t)})$ & 0,0043 & 0,0584 & 0,1156 & 0,1762 & 0,2407 & 0,3100 & 0,3838 & 0,4639 & 0,5509 & 0,6462 & 0,7515 \\ \hline
	\end{tabular}
\end{center}
Момент замедляющих сил трения может быть выражен следующим образом:
\[ M_{\text{тр}} = -k\omega_0 \implies -M_{\text{тр}} = I_0\frac{d\omega}{dt} \implies -\frac{k}{I_0}dt = \frac{d\omega}{\omega} \implies \frac{kt}{I_0} = \ln(\frac{\omega_0}{\omega(t)}) = \ln(\frac{2\pi\nu_0}{2\pi\nu(t)}) = \ln(\frac{\nu_0}{\nu(t)})\]
Из пункта 8 известно, что циклическая частота вращения гироскопа $\omega_0 \approx 2398$ с$^{-1}$, тогда обычная частота его вращения $\nu_0 = \frac{\omega_0}{2\pi} \approx 381,65$ с$^{-1}$.

Построен график зависимости $\ln(\frac{\nu_0}{\nu(t)})$ от $\frac{t}{I_0}$. С помощью МНК определены коэффициент $k$ и его погрешность $\sigma_k$. По формуле выше вычислен модуль момента замедляющих сил $M_{\text{тр}}$ и оценена его погрешность $\sigma_{M_{\text{тр}}}$ по формуле $\sigma_{M_{\text{тр}}} = M_{\text{тр}}\sqrt{\left(\frac{\sigma_k}{k}\right)^2 + \left(\frac{\sigma_{\omega_0}}{\omega_0}\right)^2}$. Результаты занесены в таблицу ниже.
\begin{center}
	\begin{tabular}{|c|c|c|c|}
	\hline
	$k$, $\frac{\text{кг$\cdot$м}^2}{\text{с}}\cdot10^{-7}$ & $\sigma_k$, $\frac{\text{кг$\cdot$м}^2}{\text{с}}$ & $M_{\text{тр}}$, Н$\cdot$м$\cdot10^{-3}$& $\sigma_{M_{\text{тр}}}$, Н$\cdot$м$\cdot10^{-3}$ \\ \hline
	8,39 & 0,03 & 2,01 & 0,03 \\ \hline
	\end{tabular}
\end{center}
\end{enumerate}
\section*{Вывод}
В результате работы были рассчитаны угловые скорости вращения и регулярной прецессии (в зависимости от действующих сил) гироскопа, определены момент инерции ротора гироскопа и моменты различных сил трения двумя разными спсобами - по скорости опускания рычага гироскопа и по скорости его замедления при выключении. Были оценены погрешности полученных значений. 

\newpage
\center{\textbf{График зависимости $\Omega(M)$}}
\begin{figure}[h]
	\center{\includegraphics[scale=0.8]{График.pdf}}
\end{figure}
\center{\textbf{График зависимости $\ln(\frac{\nu_0}{\nu(t)})$ от $\frac{t}{I_0}$}}
\begin{figure}[h]
	\center{\includegraphics[scale=0.8]{График 2.pdf}}
\end{figure}


\end{document}