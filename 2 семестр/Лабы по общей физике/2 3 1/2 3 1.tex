\documentclass[a4paper, 12pt]{article}
\usepackage[a4paper,top=1.5cm, bottom=1.5cm, left=1cm, right=1cm]{geometry}
\usepackage{cmap}					% поиск в PDF
\usepackage{mathtext} 				% русские буквы в формулах
\usepackage[T2A]{fontenc}			% кодировка
\usepackage[utf8]{inputenc}			% кодировка исходного текста
\usepackage[english,russian]{babel}	% локализация и переносы

\usepackage{amsmath}
\usepackage{indentfirst}
\usepackage{longtable}
\usepackage{graphicx}
\usepackage{array}

\usepackage{wrapfig}
\usepackage{siunitx} % Required for alignment
\usepackage{subfigure}
\usepackage{multirow}
\usepackage{rotating}
\usepackage{caption}
\usepackage{subcaption}


\graphicspath{{img/}}


\title{\begin{center}Лабораторная работа №2.3.1\end{center}
Получение и измерение вакуума}
\author{Струков Олег \\ Б04-404}
\date{}

\begin{document}
    \pagenumbering{gobble}
    \maketitle
    \newpage
    \pagenumbering{arabic}

    \textbf{Цель работы:} 1) измерение объёмов форвакуумной и высоковакуумной частей установки; 2) определение скорости откачки системы в стационарном режиме, а также по ухудшению и по улучшению вакуума.

    \textbf{Оборудование:} вакуумная установка с манометрами: масляным, термопарным ПМТ-2 и ионизационным ПМИ-2, вакуумметры Мерадат ВИТ-16 и Мерадат ВИТ-19.

    \section*{Экспериментальная установка}

    \begin{figure}[h]
        \center{\includegraphics[width=0.8\textwidth]{ustanovka}}
        \caption{Схема экспериментальной установки.}
        \label{ris:ustanovka}
    \end{figure}

    \paragraph{}
    Установка изготовлена из стекла и состоит из форвакуумного баллона (ФБ), высоковакуумного диффузионного насоса (ВН), высоковакуумного баллона (ВБ), масляного (М) и ионизационного (И) манометров, термопарных манометров ($М_1$ и $М_2$), форвакуумного насоса (ФН) и соединительных кранов (Рис. \ref{ris:ustanovka}). Кроме того, в состав установки входят: вариатор (автотрансформатор с регулируемым выходным напряжением), или реостат и амперметр для регулирования тока нагревателя диффузионного насоса.

    \paragraph{Маслянный манометр:}
    Представляет собой U-образную трубку, до половины наполненную вязким маслом, обладающим весьма низким давлением насыщенных паров. Так как плотность масла мала, $\rho = 0,885 ~г/см^3$, то при помощи манометра можно измерить только небольшие разности давлений (до нескольких торр). Во время откачки и заполнения установки атмосферным воздухом кран $К_4$ соединяющий оба колена манометра, должен быть открыт во избежание выброса масла и загрязнения установки. Кран $К_4$ закрывается только при измерении давления U-образным манометром.


    \newpage

    \paragraph{Термопарный манометр:}
    Чувствительным элементом манометра является термопара, заключенная в стеклянный баллон. Устройство термопары пояснено на (Рис. \ref{ris:termoparni_monometr}). По нити накала НН пропускается ток постоянной величины. Термопара ТТ присоединяется к милливольтметру, показания которого определяются температурой нити накала и зависят от отдачи тепла в окружающее пространство. Потери тепла определяются теплопроводностью нити и термопары, теплопроводностью газа, переносом тепла конвективными потоками газа внутри лампы и теплоизлучением нити (инфракрасное тепловое излучение). В обычном режиме лампы основную роль играет теплопроводность газа. При давлениях >1 торр теплопроводность газа, а вместе с ней и ЭДС термопары практически не зависят от давления газа, и прибор не работает. При улучшении вакуума средний свободный пробег молекул становится сравнимым с диаметром нити, теплоотвод падает и температура спая возрастает. При вакууме $~10^{-3}$ торр теплоотвод, осуществляемый газом, становится сравнимым с другими видами потерь тепла и температура нити становится практически постоянной. Градуировочная кривая термопарного манометра приведена на (Рис. \ref{ris:termopara_graduirovka}).

    \begin{figure}[h]
    \centering
    \begin{minipage}{0.3\textwidth}
        \centering
        \includegraphics[width=0.9\textwidth]{termoparni_monometr}
        \caption{Схема термопарного манометра.}
        \label{ris:termoparni_monometr}
    \end{minipage}\hfill
    \begin{minipage}{0.7\textwidth}
        \centering
        \includegraphics[width=0.9\textwidth]{termopara_graduirovka}
        \caption{Градуировочная кривая термопарного манометра.}
        \label{ris:termopara_graduirovka}
    \end{minipage}
    \end{figure}

    \newpage

    \paragraph{Ионизационный манометр:}
    Схема ионизационного манометра изображена на (Рис. \ref{ris:ionizacionni_monometr}). Он представляет собой трехэлектродную лампу. Электроны испускаются накалённым катодом и увлекаются электрическим полем к аноду, имеющему вид спирали. Проскакивая за её витки, электроны замедляются полем коллектора и возвращаются к катоду, а от него вновь увлекаются к аноду. Прежде чем осесть на аноде, они успевают много раз пересечь пространство между катодом и коллектором. На своем пути электроны ионизуют молекулы газа. Ионы, образовавшиеся между анодом и коллектором, притягиваются полем коллектора и определяют его ток. Ионный ток в цепи коллектора пропорционален плотности газа и поэтому может служить мерой давления. Вероятность ионизации зависит от рода газа, заполняющего лампу (а значит, и откачиваемый объём). Калибровка манометра верна, если остаточным газом является воздух. Накалённый катод ионизационного манометра перегорает, если давление в системе превышает $10^{-3}$ торр. Поэтому включать ионизационный манометр можно, только убедившись по термопарному манометру, что давление в системе не превышает $10^{-3}$ торр. При измерении нить ионизационного манометра сильно греется. При этом она сама, окружающие её электроды и стенки стеклянного баллона могут десорбировать поглощенные ранее газы. Выделяющиеся газы изменяют давление в лампе и приводят к неверным показаниям. Поэтому перед измерениями ионизационный манометр прогревается (обезгаживается) в течение 10–15 мин. Для прогрева пропускается ток через спиральный анод лампы.

    \vspace{1cm}

    \begin{figure}[h]
        \center{\includegraphics[width=0.5\textwidth]{ionizacionni_monometr}}
        \caption{Схема ионизационной лампы.}
        \label{ris:ionizacionni_monometr}
    \end{figure}

    \newpage

    \paragraph{Диффузионный насос:}
    Откачивающее действие диффузионного насоса основано на диффузии (внедрении) молекул разреженного воздуха в струю паров масла. Попавшие в струю молекулы газа увлекаются ею и уже не возвращаются назад. Устройство одной ступени масляного диффузионного насоса схематически показано на (Рис. \ref{ris:diffuzionni_nasos}) (в лабораторной установке используется несколько откачивающих ступеней). Масло, налитое в сосуд А, подогревается электрической печкой. Пары масла поднимаются по трубе Б и вырываются из сопла В. Струя паров увлекает молекулы газа,которые поступают из откачиваемого сосуда через трубку ВВ. Дальше смесь попадает в вертикальную трубу Г. Здесь масло осаждается на стенках трубы и маслосборников и стекает вниз, а оставшийся газ через трубу ФВ откачивается форвакуумным насосом. Диффузионный насос работает наиболее эффективно при давлении, когда длина свободного пробега молекул воздуха примерно равна ширине кольцевого зазора между соплом В и стенками трубы ВВ. В этом случае пары масла увлекают молекулы воздуха из всего сечения зазора. Давление насыщенных паров масла при рабочей температуре, создаваемой обогревателем сосуда А, много больше $5\cdot10^{-2}$ торр. Именно поэтому пары масла создают плотную струю, которая и увлекает с собой молекулы газа. Если диффузионный насос включить при давлении, сравнимом с давлением насыщенного пара масла, то последнее никакой струи не создаст и масло будет просто окисляться и угорать.

    Диффузионный насос, используемый в нашей установке, имеет две ступени и соответственно два сопла. Одно сопло вертикальное (первая ступень), второе сопло горизонтальное (вторая ступень). За второй ступенью имеется ещё одна печь, но пар из этой печи поступает не в сопло, а по тонкой трубке подводится ближе к печке первой ступени. Эта печь осуществляет фракционирование масла. Легколетучие фракции масла, испаряясь, поступают в первую ступень, обогащая её легколетучей фракцией масла. По этой причине плотность струи первой ступени выше и эта ступень начинает откачивать при более высоком давлении в форвакуумной части установки. Вторая ступень обогащается малолетучими фракциями. Плотность струи второй ступени меньше, но меньше и давление насыщенных паров масла в этой ступени. Соответственно в откачиваемый объём поступает меньше паров масла и его удается откачать до более высокого вакуума, чем если бы мы работали только с одной ступенью.


    \begin{figure}[h]
    \centering
    \begin{minipage}{0.35\textwidth}
        \centering
        \includegraphics[width=1\textwidth]{diffuzionni_nasos}
        \caption{Схема работы диффузионного насоса.}
        \label{ris:diffuzionni_nasos}
    \end{minipage}\hfill
    % \begin{minipage}{0.65\textwidth}
    %     \centering
    %     \includegraphics[width=0.9\textwidth]{Установка.jpg}
    %     \caption{Диффузионный насос используемый в работе.}
    %     \label{ris:nasos_irl}
    % \end{minipage}
    \end{figure}


    \section*{Теоретическая часть}

    \paragraph{Процесс откачки:}
    Производительность насоса определяется скоростью откачки $W$ (л/с): W — это объём газа, удаляемого из сосуда при данном давлении за единицу времени. Скорость откачки форвакуумного насоса равна ёмкости воздухозаборной камеры, умноженной на число оборотов в секунду.

    Обозначим через $Q_д$ количество газа, десорбирующегося с поверхности откачиваемого объёма в единицу времени, через $Q_и$ -- количество газа, проникающего в единицу времени в этот объём извне -- через течи. Будем считать, что насос обладает скоростью откачки $W$ и в то же время сам является источником газа; пусть $Q_н$ — поток газа, поступающего из насоса назад в откачиваемую систему. $Q=Q_д + Q_и + Q_н$ измеряем в единицах (моль/с). Получаем формулу
    \begin{equation*}
        -\frac{VdP}{RT}=\left(\frac{PW}{RT} - Q\right)dt
    \end{equation*}
    При предельном давлении $dP=0$ и поэтому получаем
    \begin{equation*}
        Q=\frac{P_{пр}W}{RT}
    \end{equation*}
    Подставляя получаем
    \begin{equation*}
        -VdP=W(P-P_{пр})dt
    \end{equation*}
    Интегрируем полученное уравнение и получаем
    \begin{equation}
        P-P_{пр}=(P_0 - P_{пр})\exp\left(-\frac{W}{V}t\right)
    \end{equation}
    $P_{пр}$ относительно $P_0$ можно пренебречь:
    \begin{equation}
        P - P_{пр}=P_0\exp\left(-\frac{W}{V}t\right) \label{W}
    \end{equation}
    Как видим, величина $\tau=V/W$ показывает характерное время откачки системы.

    Теперь попробуем понять чем обусловлена скорость откачки. Очевидно, скорость $W$ зависит от скорости откачки насоса $W_н$, но она так же зависит от трубопровода соединяющего насос к откачиваемой части, т.к. если трубопровод не сможет обеспечить достаточное количество газа к входу насоса то, производительность упадёт.

    \begin{figure}[h]
        \center{\includegraphics[width=0.7\textwidth]{nasos_sketch}}
        \caption{Схема насоса с трубопроводом.}
        \label{ris:nasos_sketch}
    \end{figure}
    Попробуем описать систему математически. Пусть у нас есть насос со скоростью откачки $W_н$ и трубопровод с пропускной способностью $C$. Давление в откачиваемом объёме -- $P_1$. Исследовав схему \ref{ris:nasos_sketch} получаем

    \begin{equation*}
        C(P_1 - P_2)=W_нP_2 \Rightarrow P_2=\frac{CP_1}{C+W_н} \Rightarrow WP_1=W_нP_2=\frac{CW_н}{C+W_н}P_1
    \end{equation*}
    Как видим, для результирующей скорости $W$ верно соотношение

    \begin{equation*}
        \frac{1}{W} = \frac{1}{W_н} + \frac{1}{C}
    \end{equation*}
    Обобщая это выражение для последовательно соединенных труб получаем

    \begin{equation}
        \frac{1}{W} = \frac{1}{W_н} + \frac{1}{C_1} + \frac{1}{C_2} + ...
        \label{resulting_speed}
    \end{equation}
    Заметим только что данные формулировки верны при молекулярном режиме течения, когда вязкое трение не имеет большого вклада в движение газа.

    \paragraph{Течение газа через трубу:} Для количества газа, протекающего через трубу в условиях высокого вакуума или, как говорят, в кнудсеновском режиме, справедлива формула

    \begin{equation}
        \frac{d(PV)}{dt} = \frac{4}{3}r^3 \sqrt{\frac{2\pi RT}{\mu}} \frac{P_2 - P_1}{L}
        \label{prop_spos_truba}
    \end{equation}
    где $r$ и $L$ соответственно радиус и длина трубы. Если пренебречь давлением $P_1$ у конца, обращенного к насосу, получаем формулу для пропускной способности трубы
    \begin{equation}
    C_{тр} = \frac{dV}{dt} = \frac{4}{3}\frac{r^3}{L}\sqrt{\frac{2\pi RT}{\mu}} \label{C_trubki}
    \end{equation}
    Для пропускной способности отверстия (например в кранах) имеем формулу

    \begin{equation}
    C_{отв}=S\frac{\bar v}{4}
    \end{equation}




\section*{{Ход работы}}
\subsection*{Определение объёмов форвакуумного и высоковакуумного баллонов}
\begin{enumerate}
    \item Все краны были открыты, выждано 2 минуты, пока воздух заполнял всю установку.
    \item В капилляре между кранами $К_5$ и $К_6$ было заперто примерно $V_{\text{кап}}$ = 50 см$^3$ воздуха при атмосферном давлении $p_a$ = 99,09 кПа.
    \item С помощью форвакуумного насоса была начата откачка установки, и через несколько минут внутри было получено давление $1,1\cdot10^{-2}$ мм рт. ст.
    \item Были закрыты краны $К_2$, $К_3$ и $К_4$, а кран $К_5$ открыт, благодаря чему воздух из капилляра заполнил форвакуумную часть установки. 
    \item Полученное в ней давление было измерено с помощью масляного манометра: верхнее значение уровня масла в коленах манометра составило $h_1 = 34,7 $ см масл. ст., нижнее -- $h_2 = 9,7$ см масл. ст. Разность уровней $\Delta h_{\text{фв}} = 25,0$ см масл. ст.
    \item Зная плотность используемого масла -- $\rho = 0,885~г/см^3$, и поскольку процесс можно считать изотермическим (т.е. $PV = const$), можно определить объём форвакуумной части установки: $V_{\text{фв}} = \dfrac{p_\text{a}V_{\text{кап}}}{\rho g \Delta h_{\text{фв}}} - V_{\text{кап}}$. Таким образом, $V_{\text{фв}} = 2285,0 ~ см^3 $.
    \item Был открыт кран $К_3$ и аналогичным образом определены объёмы всей установки $V_{\text{полн}}$ и её высоковакуумной части $V_{\text{вв}}$: $h_3 = 30,6 $ см масл. ст., $h_4 = 14,5$ см масл. ст., $\Delta h_{\text{полн}} = 16,1$ см масл. ст., $V_{\text{полн}} = \dfrac{p_\text{a}V_{\text{кап}}}{\rho g \Delta h_{\text{полн}}}$, $V_{\text{вв}} = V_{\text{полн}} - V_{\text{кап}} - V_{\text{фв}}$. Получены значения $V_{\text{полн}} = 3548,2 ~ см^3$ и $V_{\text{вв}} = 1263,1 ~ см^3 $.
    \item По окончании измерений был открыт кран $К_4$ для уравновешивания масла в манометре.
\end{enumerate}


\subsection*{Получение высокого вакуума и измерение скорости откачки}
\begin{enumerate}
    \item Для продолжения откачки установки для выполнения второй части работы был открыт кран $К_2$. При этом в установке не осталось запертых объёмов.
    \item После того, как давление упало ниже $1\cdot10^{-2}$ мм рт. ст., был закрыт кран $К_6$ и начата откачка высоковакуумного баллона с помощью диффузионного насоса. Для этого с помощью источника питания был начат его нагрев.
    \item Через некоторое время масло в диффузионном насосе закипело, и давление в итоге составило $5\cdot10^{-4}$ мм рт. ст.
    \item Затем был запущен ионизационный манометр ПМИ-2 вакуумметра ВИТ-19 и проведена "Дегазация". После этого было достигнуто предельное давление $p_{\text{пр}} = 1,2 \cdot10^{-4}$ мм рт. ст.
    \item Для нахождения скорости откачки по ухудшению и улучшению вакуума сначала закрывался кран $К_3$ (тем самым отключалась откачка высоковакуумного баллона) и на видео фиксировалось ухудшение вакуума во времени до предела около $7,6\cdot10^{-4}$ мм рт. ст. Затем кран $К_3$ открывался и фиксировалось улучшение вакуума во времени. Данные измерения были проведены дважды, полученные данные изображены на графиках (расположены в конце отчёта).
 

    
    \item Логарифмируя формулу (2), получаем $\ln(P - P_{\text{пр}}) = \ln P_0 - \dfrac{W}{V_{\text{вв}}}t$. Таким образом, из зависимости $\ln(P - P_{\text{пр}})$ от t был найден коэффициент наклона прямой $k = -\dfrac{W}{V_{\text{вв}}}$:
    \begin{equation*}
        k_1 = -0,246~c^{-1}, \qquad k_2 = -0,200~c^{-1}, \qquad k_{\text{ср}} = \dfrac{k_1 + k_2}{2} = -0,223~c^{-1}
    \end{equation*}
    \item Исходя из полученных данных, можно найти скорость откачки: $W = -k_{\text{ср}}V_{\text{}} = 282 ~ см^3/с$
    \item Из уравнения ($Q_{\text{и}}$ -- поток газа, поступающий извне, $Q_{\text{д}}$ -- количество десорбирующегося газа)

            \begin{align*}
                V_{вв} dP &= (Q_д + Q_и)dt
            \end{align*}

            Следует зависимость (f -- коэффициент наклона прямой графика зависимости $P(t)$ при ухудшении вакуума):

            \begin{equation*}
                Q_д + Q_и = f V_{вв}
            \end{equation*}

            \begin{align*}
                f_1 &= 1,156 \cdot 10^{-5}~ мм ~рт. ~ст./с\\
                f_2 &= 1,143 \cdot 10^{-5} ~мм ~рт. ~ст./с\\
                f_{\text{ср}} = \frac{f_1 + f_2}{2} &= 1,150 \cdot 10^{-5} ~мм ~рт. ~ст./с
            \end{align*}

            Учитывая, что $P_{пр}W = Q_д + Q_и + Q_н$, получаем

            \begin{equation*}
                Q_н = P_{пр} W - f_{\text{ср}} V_{вв} = 1,93 \cdot 10^{-2}~мм~ рт.~ ст. \cdotсм^3/с\\
            \end{equation*}

    \item По формуле (5) была оценена пропускная способность трубки от высоковакуумного баллона до насоса: L = 10,8 см, r = 0,4 см, T = 296 К, следовательно, $C_{\text{тр}} = 567,3 ~ см^3/с$, что примерно в два раза больше скорости откачки.
    \item Далее был открыт кран К$_5$, что создало таким образом искусственную течь (высоковакуумная часть установки была соединена через капилляр с форвакуумной). Вакуум стал ухудшаться, и через 5 минут достиг установившегося значения $P_{\text{уст}} = 2,7 \cdot 10^{-4}$ мм рт. ст. При этом со стороны форвакуумной части давление составляло $P_{\text{фв}} = 8,5\cdot10^{-3}$ мм рт. ст.
    \item Обозначим как $Q_1$ сумму всех натеканий, кроме натекания через искуственную течь. Тогда при закрытом капилляре выполняется $P_{пред}W = Q_1$, а при открытом -- $P_{уст}W = Q_1 + \dfrac{(PV)_{кап}}{dt}$. Отсюда получаем
$$W = \frac{C_{тр} P_{фв}}{P_{уст} - P_{пред}} = 327,0~ см^3/с$$

\section*{Вывод}
В результате выполнения работы был получен высокий вакуум (Р = 1,2$\cdot10^{-4}$ мм рт. ст.), получены зависимости давления от времени при ухудшении и улучшении вакуума, а также исследована одна из главных особенностей экспериментальной установки -- скорость откачки воздуха. При расчёте через улучшение вакуума $W_1 = 282~см^3/с$, а при расчёте после создания искусственной течи $W_2 = 327~см ^3/с$. Полученные результаты достаточно близки и выглядят реалистично.




\end{enumerate}


    \begin{figure}[h]
        \center{\includegraphics[width=1\textwidth]{1 1}}
        \caption*{Ухудшение вакуума (первый опыт)}
    \end{figure}

    \begin{figure}[h]
        \center{\includegraphics[width=0.9\textwidth]{2 1}}
        \caption*{Ухудшение вакуума (второй опыт)}
    \end{figure}

    \begin{figure}[h]
        \center{\includegraphics[width=0.9\textwidth]{1 2 0}}
        \caption*{Улучшение вакуума (первый опыт)}
    \end{figure}

    \begin{figure}[h]
        \center{\includegraphics[width=1\textwidth]{2 2 0}}
        \caption*{Улучшение вакуума (второй опыт)}
    \end{figure}







\end{document}
