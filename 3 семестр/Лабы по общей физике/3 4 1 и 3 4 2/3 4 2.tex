\documentclass[a4paper,12pt]{article} % добавить leqno в [] для нумерации слева
\usepackage[a4paper,top=1.3cm,bottom=2cm,left=1.5cm,right=1.5cm,marginparwidth=0.75cm]{geometry}
%%% Работа с русским языком
\usepackage{cmap}					% поиск в PDF
\usepackage[warn]{mathtext} 		% русские буквы в фомулах
\usepackage[T2A]{fontenc}			% кодировка
\usepackage[utf8]{inputenc}			% кодировка исходного текста
\usepackage[english,russian]{babel}	% локализация и переносы
\usepackage{physics}
\usepackage{multirow}
\usepackage{floatflt}
%%% Нормальное размещение таблиц (писать [H] в окружении таблицы)
\usepackage{float}
\restylefloat{table}
% \usepackage{mathtext}
\usepackage{amssymb}
% \usepackage{amsmath}
% \usepackage[russian]{babel}
\usepackage{indentfirst}
% \usepackage[pdftex]{graphicx}
\usepackage{siunitx}

\newcommand{\rref}[1]{(\ref{#1})}
\newcommand{\Equip}[3]{{\bf #1:} $\Delta = \pm #2$ \si{#3}

}
\newcommand{\equip}[1]{{\bf #1}

}
\usepackage{placeins}
\usepackage{graphicx}

\usepackage{wrapfig}
\usepackage{tabularx}

\usepackage{hyperref}
\usepackage[rgb]{xcolor}
\hypersetup{
	colorlinks=true,urlcolor=blue
}

%%% Дополнительная работа с математикой
\usepackage{amsmath,amsfonts,amssymb,amsthm,mathtools} % AMS
\usepackage{icomma} % "Умная" запятая: $0,2$ --- число, $0, 2$ --- перечисление

%% Номера формул
%\mathtoolsset{showonlyrefs=true} % Показывать номера только у тех формул, на которые есть \eqref{} в тексте.

%% Шрифты
\usepackage{euscript}	 % Шрифт Евклид
\usepackage{mathrsfs} % Красивый матшрифт
\usepackage{pgfplots}
\pgfplotsset{compat=1.9}

%% Свои команды
\DeclareMathOperator{\sgn}{\mathop{sgn}}

%% Перенос знаков в формулах (по Львовскому)
\newcommand*{\hm}[1]{#1\nobreak\discretionary{}
	{\hbox{$\mathsurround=0pt #1$}}{}}

\title{\begin{center}Лабораторная работа №3.4.2\end{center}
Закон Кюри-Вейсса}
\author{Струков О. И. \\ Б04-404}
\date{}

\begin{document}

    \pagenumbering{gobble}
    \maketitle
    \newpage
    \pagenumbering{arabic}
\section*{Теоретические сведения}

В данной работе проводится исследование зависимости магнитной восприимчивости гадолиния, который является ферромагнетиком, от температуры. Исследование приведено для температур от 14 до 40 \si{\degreeCelsius}. На основании этой зависимости вычисляется точка Кюри гадолиния.

Одной из основных макроскопических характеристик веществ, которая используется для описания их магнитных свойств, является вектор намагниченности $\mathbf{M}$ — суммарный магнитный момент единичного
объёма вещества. В ряде веществ между намагниченностью $\mathbf{M}$ и напряжённостью магнитного поля $\mathbf{H}$ имеет место линейная зависимость:
где скалярная величина $\chi$ — магнитная восприимчивость единичного
объёма вещества. Вещества с отрицательной магнитной восприимчивостью $(\chi < 0)$ называют диамагнетиками, а вещества с $(\chi >0)$ принадлежат к классу парамагнетиков.

Кроме диа- и парамагнетиков ($\chi \le 10^{-3}$) существуют также ферромагнетики, для которых $\chi \ge 10^{4}$. Причём зависимость $\mathbf{M(H)}$ в таких веществах нелинейна. 

Магнитные и другие физические свойства ферромагнетиков зависят от температуры. Намагниченность насыщения $M_s$ (равная максимальной намагниченности при данной температуре) имеет максимум при $T=0$ и монотонно убывает до $0$ при $T=\Theta$ -- температуре Кюри. Поведение ферромагнетика при больших температурах описывается законом Кюри-Вейсса:
\begin{equation}
	\chi = \frac{C}{T-\Theta_p},
	\label{a}
\end{equation}
где $\Theta_p$ -- парамагнитная температура Кюри.
\section*{Экспериментальная установка}

\begin{figure}[h]
    \centering
    \includegraphics[width=15cm]{Screenshot_11.png}
    \caption{Схема экспериментальной установки}
    \label{fig:vac}
\end{figure}

Схема установки изображена на рис. 1. Исследуемый ферромагнитный образец (гадолиний) расположен внутри пустотелой катушки самоиндукции, которая служит индуктивностью колебательного контура, входящего в состав LC-автогенератора. Катушка с образцом помещена в стеклянный сосуд, залитый трансформаторным маслом. Температура образца регулируется с помощью термостата.  \par
При изменении температуры по закону Кюри-Вейсса изменяется магнитная восприимчивость образца в катушке и, следовательно, изменяется самоиндуктивность этой катушки. При этом изменяется период колебаний автогенератора. Поэтому получаем, что 
\begin{equation}
    \frac{1}{\chi} \thicksim  (T - \Theta_p) \thicksim  \frac{1}{(\tau^2 - \tau_0^2)},
\end{equation}
где $\tau$ и $\tau_0$ - период колебаний в цепи с сердечником в катушке и без него соответственно. Измерения проводятся в интервале температур от 14 $^{\circ}$С до 40 $^{\circ}$С

\section*{Расчётные формулы}
Магнитная восприимчивость определяется по формуле:
\begin{equation}
	(L-L_0)\sim \chi,
\end{equation}
где $L$ -- самоиндукция катушки с образцом, а $L_0$ -- без образца. Тогда из 
\begin{align}
	\tau = 2 \pi \sqrt{L C}, \\
	\tau_0 = 2 \pi \sqrt{L_0 C}
\end{align}
следует:
\begin{align}
	(L-L_0)\sim (\tau^2 - \tau_0^2), \\
	\chi \sim (\tau^2 - \tau_0^2).
	\label{b}
\end{align}

% Из \rref{a} и \rref{b}, окончательно: 
% \begin{equation}
% 	\frac{1}{\chi}\sim \frac{1}{\tau^2 - \tau^2_0}
% \end{equation}
% \section{Оборудование и инструментальные погрешности}

%{\bf : } $\Delta = \pm  $ ед
% Оборудование, используемое в работе, представлено на рис. \ref{fig:set}. Исследуемый образец находится внутри индуктивности в колбе с трансформаторным маслом, температура которого поддерживается термостатом. Катушка является частью LC-контура, частота которого фиксируется частотомером.

% \Equip{Цифровой вольтметр}{1*10^{-5}}{\volt}
% \Equip{Частотомер}{1*10^{-5}}{\volt}
% \Equip{Термостат}{0.1}{\degreeCelsius}
% {\bf LC-контур: }$\tau_0 = 9,045 $ мкс


\section*{Ход работы}
\begin{enumerate}
	\item Поскольку максимальная разность температуры образца и рабочей жидкости составляет $\Delta T = 0,5~ ^oC$, было определено, что максимальная допустимая ЭДС термопары составляет 0,02 мВ.
	\item После включения термостата и измерительных приборов в сеть была получена зависимость периода колебаний LC-генератора от температуры образца, определяемой по ртутному термометру, вставленному в термостат. 
	Измерения проводились от 10,5 до 41 $^oC$ с интервалом в $2~^oC$. Период колебаний без образца $\tau_0 = 9,045 $ мкс.
	Результаты представлены в таблице, зависимости $(\tau^2-\tau_0^2) = f(T)$ и $1/(\tau^2-\tau_0^2) = f(T)$ изображены на графике.
\item Линейную часть второй зависимости можно аппроксимировать прямой у = 0,01749х - 5,122. Она пересекает ось Ох в точке 292,9 К, то есть с учётом погрешности (приборной и среднеквадратичного отклонения) парамагнитная точка Кюри $\Theta_p = (292,9 \pm 13,4)$ К $ = (19,9 \pm 13,4)~^oC$. 

% \begin{table}[!ht]
%     \centering
%     \begin{tabular}{|c|c|c|c|}
%     \hline
%         T, K & $\tau$, мкс & $\tau^2-\tau_0^2, \text{мкс}^2$ & $\dfrac{1}{\tau^2-\tau_0^2} \text{мкс}^{-2}$ \\ \hline
%         $(283,5\pm0,1)$ & $(10,889\pm0,001)$ & 36,758 & 0,0272 \\ \hline
%         288 & 10,809 & 35,022 & 0,0286 \\ \hline
%         290 & 10,730 & 33,321 & 0,0300 \\ \hline
%         292 & 10,595 & 30,442 & 0,0328 \\ \hline
%         294 & 10,345 & 25,207 & 0,0397 \\ \hline
%         296 & 9,993 & 18,048 & 0,0554 \\ \hline
%         298 & 9,670 & 11,697 & 0,0855 \\ \hline
%         300 & 9,460 & 7,680 & 0,1302 \\ \hline
%         302 & 9,360 & 5,798 & 0,1725 \\ \hline
%         304 & 9,310 & 4,864 & 0,2056 \\ \hline
%         306 & 9,280 & 4,306 & 0,2322 \\ \hline
%         308 & 9,253 & 3,806 & 0,2627 \\ \hline
%         310 & 9,230 & 3,381 & 0,2958 \\ \hline
%         312 & 9,214 & 3,086 & 0,3241 \\ \hline
%         314 & 9,202 & 2,865 & 0,3491 \\ \hline
%     \end{tabular}
% \end{table}


\begin{table}[!ht]
    \centering
    \begin{tabular}{|c|c|c|c|}
    \hline
        T, K & $\tau$, мкс & $\tau^2-\tau_0^2$, мкс$^2$ & $\dfrac{1}{\tau^2-\tau_0^2}$, мкс$^{-2}$ \\ \hline
        $283,5\pm0,1$ & $10,889\pm0,001$ & $36,75\pm0,02$ & $0,0272\pm0,0002$ \\ \hline
        $288,0\pm0,1$ & $10,809\pm0,001$ & $35,02\pm0,02$ & $0,0286\pm0,0002$ \\ \hline
        $290,0\pm0,1$ & $10,730\pm0,001$ & $33,32\pm0,02$ & $0,0300\pm0,0002$ \\ \hline
        $292,0\pm0,1$ & $10,595\pm0,001$ & $30,44\pm0,02$ & $0,0328\pm0,0002$ \\ \hline
        $294,0\pm0,1$ & $10,345\pm0,001$ & $25,20\pm0,02$ & $0,0397\pm0,0003$ \\ \hline
        $296,0\pm0,1$ & $9,993\pm0,001$ & $18,04\pm0,02$ & $0,0554\pm0,0006$ \\ \hline
        $298,0\pm0,1$ & $9,670\pm0,001$ & $11,69\pm0,02$ & $0,0855\pm0,002$ \\ \hline
        $300,0\pm0,1$ & $9,460\pm0,001$ & $7,68\pm0,02$ & $0,1302\pm0,003$ \\ \hline
        $302,0\pm0,1$ & $9,360\pm0,001$ & $5,79\pm0,02$ & $0,1725\pm0,006$ \\ \hline
        $304,0\pm0,1$ & $9,310\pm0,001$ & $4,86\pm0,02$ & $0,2056\pm0,008$ \\ \hline
        $306,0\pm0,1$ & $9,280\pm0,001$ & $4,30\pm0,02$ & $0,2322\pm0,01$ \\ \hline
        $308,0\pm0,1$ & $9,253\pm0,001$ & $3,80\pm0,02$ & $0,2627\pm0,01$ \\ \hline
        $310,0\pm0,1$ & $9,230\pm0,001$ & $3,38\pm0,02$ & $0,2958\pm0,02$ \\ \hline
        $312,0\pm0,1$ & $9,214\pm0,001$ & $3,08\pm0,02$ & $0,3241\pm0,02$ \\ \hline
        $314,0\pm0,1$ & $9,202\pm0,001$ & $2,86\pm0,02$ & $0,3491\pm0,02$ \\ \hline
    \end{tabular}
\end{table}
\FloatBarrier
    \begin{figure}[ht!]
        \center{\includegraphics[width=1\textwidth]{t2.pdf}}
        % \caption*{}
    \end{figure}

    \begin{figure}[ht!]
        \center{\includegraphics[width=1\textwidth]{1т.pdf}}
        % \caption*{}
    \end{figure}
\FloatBarrier


\end{enumerate}







\newpage


\newpage
\section*{Вывод}

В ходе работы была экспериментально определена парамагнитная точка Кюри гадолиния на основе исследования температурной зависимости магнитной восприимчивости. 

Экспериментально установлена зависимость периода колебаний LC-генератора от температуры в интервале от 283,5 K до 314 K. 
Получена зависимость $1/(\tau^2 - \tau_0^2)$ от температуры, методом линейной аппроксимации парамагнитного участка зависимости $1/(\tau^2 - \tau_0^2) = f(T)$ определена точка Кюри гадолиния $\Theta_p = (292,9 \pm 13,4)$ K, что очень хорошо согласуется с табличным значением 293,2 K.

На графике зависимости $(\tau^2 - \tau_0^2) = f(T)$ наблюдается характерный излом в районе точки Кюри, что соответствует фазовому переходу второго рода из ферромагнитного в парамагнитное состояние.

























% \section{Результаты измерений и обработка данных}

% Напряжение на вольтметре при $\Delta T
% = 0.5$ \si{\degreeCelsius} равно $20$ \si{\micro \volt}.

% Исходные результаты измерений в табл. 1.

% \begin{table}[]
% 	\centering
% 	\label{tab}
% 	\begin{tabular}{|l|l|l|l|}
% 		\hline
% 		\textbf{Температура термостата, \si{\degreeCelsius}} & \textbf{Напряжение, \si{\volt}} & \textbf{Период, \si{\micro \second}} & \textbf{Температура жидкости, \si{\degreeCelsius}} \\ \hline
% 		14.1                                                 & -15                                          & 7.92                                 & 13.7                                               \\ \hline
% 		16.1                                                 & -15                                          & 7.855                                & 15.7                                               \\ \hline
% 		18.1                                                 & -15                                          & 7.744                                & 17.7                                               \\ \hline
% 		20.1                                                 & -15                                          & 7.56                                 & 19.7                                               \\ \hline
% 		22.1                                                 & -13                                          & 7.343                                & 21.8                                               \\ \hline
% 		24.1                                                 & -12                                          & 7.194                                & 23.8                                               \\ \hline
% 		26.1                                                 & -13                                          & 7.123                                & 25.8                                               \\ \hline
% 		28.1                                                 & -15                                          & 7.085                                & 27.7                                               \\ \hline
% 		30.1                                                 & -17                                          & 7.058                                & 29.7                                               \\ \hline
% 		32.1                                                 & -18                                          & 7.04                                 & 31.7                                               \\ \hline
% 		34.1                                                 & -18                                          & 7.026                                & 33.7                                               \\ \hline
% 		36.1                                                 & -19                                          & 7.016                                & 35.6                                               \\ \hline
% 		38.1                                                 & -19                                          & 7.008                                & 37.6                                               \\ \hline
% 		40.1                                                 & -18                                          & 7.002                                & 39.7                                               \\ \hline
% 	\end{tabular}
% 	\caption{Исходные данные}
% \end{table}



% \subsection{Оценка погрешностей}

% \section{Вывод}

% \begin{thebibliography}{9}
% \bibitem{Siv} Сивухин Д. В. \emph{Общий курс физики. Том 2 Термодинамика и молекулярная физика}, 2003
% \end{thebibliography}
\end{document}