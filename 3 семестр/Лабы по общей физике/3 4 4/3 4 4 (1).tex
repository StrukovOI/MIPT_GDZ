\documentclass[a4paper, 12pt]{article}

% Libraries (packets)
\usepackage[T2A]{fontenc}
\usepackage[utf8]{inputenc}
\usepackage[english, russian]{babel}
\usepackage{caption}
\usepackage{listings}
\usepackage{amsmath, amsfonts, amssymb, amsthm, mathtools}
\usepackage[warn]{mathtext}
\usepackage[mathscr]{eucal}
\usepackage{wasysym}
\usepackage{graphicx}
\usepackage{indentfirst}
\usepackage{float}
\usepackage{wrapfig}
\usepackage{fancyhdr}
\usepackage{lscape}
\usepackage{xcolor}
\usepackage[normalem]{ulem}
\usepackage{titlesec}
\usepackage{hyperref}
\usepackage[top=1.5cm, bottom=1.5cm, left=1.5cm, right=1.5cm]{geometry} % Маленькие поля

% Убираем нумерацию разделов и подразделов
\titleformat{\section}{\normalfont\Large\bfseries}{}{0em}{}
\titleformat{\subsection}{\normalfont\large\bfseries}{}{0em}{}
\titleformat{\subsubsection}{\normalfont\normalsize\bfseries}{}{0em}{}

\begin{document}

\title{\begin{center}Лабораторная работа № 3.4.4\end{center}
Петля гистерезиса (статический метод)}
\author{Струков О. И. \\ Б04-404}
\date{}

\thispagestyle{empty}
\pagenumbering{gobble}
\maketitle
\newpage
\pagenumbering{arabic}

\textbf{Цель работы:} исследование кривых намагничивания ферромагнетиков с помощью баллистического гальванометра.

\textbf{В работе используются:} генератор токов намагничивания (ГТН), тороид, соленоид, баллистический гальванометр с осветителем и шкалой, мультиметр-амперметр, лабораторный автотрансформатор (ЛАТР), разделительный трансформатор, ключи, переключатели.

                    
\section{Теоретическая часть}
	
% Ферромагнетики -- вещества, которые при определенной температуре обладают самопроизвольной намагниченностью $M$ в отсутствие внешнего магнитного поля. В ферромагнетиках образуются отдельные намагниченные области – домены (от $10^{-2}$ до $10^{-6}$ см$^3$), магнитные моменты в которых ориентируются параллельно.

Зависимость вектора магнитной индукции ${B}$ ферромагнетика от вектора напряжённости магнитного поля $H$ нелинейна. В системе СИ эта связь имеет вид
\begin{equation} 
    \label{eq:BM}
    {B} = \mu_0({H} + {M})
\end{equation}

При этом намагниченность зависит не только от состояния вещества, а также от его предудущих состояний, то есть зависимость ${B}({H})$ не является функцией состояния. График этой зависимости изображён на рис. 1.

\begin{figure}[h!]
    \center{\includegraphics[width = 9 cm]{1.png}}
    \caption{Петля гистерезиса ферромагнетика}
    \label{ris:hysteresis}
\end{figure}

Измерение зависимости $B(H)$: На тороидальный сердечник (рис. 2), изготовленный из исследуемого образца, равномерно намотана намагничивающая обмотка с числом витков $N$, а поверх неё — измерительная обмотка с числом витков $N'$. При скачкообразном изменении тока в намагничивающей обмотке в измерительной
обмотке возникает ЭДС индукции. Ток, вызванный этой ЭДС, регистрируется гальванометром $\text{Г}$, работающим в баллистическом (импульсном) режиме: его отклонение пропорционально полному заряду $\Delta q$, протекшему через него.

\begin{figure}[h!]
    \centering
    \includegraphics[width = 7 cm]{2.png}
    \caption{Схема для измерения индукционного тока}
    \label{ris:ust1}
\end{figure}

Напряжённость поля $H$ в сердечнике пропорциональна току $I$ в первичной (намагничивающей) обмотке, а изменение магнитной индукции $\Delta B$ — заряду $\Delta q$, протекшему через вторичную (измерительную) обмотку. Таким образом, измеряя токи $\Delta I$ и суммируя отклонения $\Delta q$ гальванометра $\text{Г}$, можно получить зависимость $B(H)$ для материала сердечника.

Напряжённость магнитного поля в тороиде равна 

\begin{equation}
    \label{eq:H}
    H = \frac{N}{\pi D} I
\end{equation}

Гальванометр измеряет протёкший через него заряд баллистическим методом, то есть

\begin{equation}
    \Delta x = \frac{\Delta q}{b}
\end{equation}

Учитывая, что заряд возникает из-за тока электромагнитной индукции, получим

\begin{equation}
    \Delta x = \frac{S_t N'}{b R_t} \Delta B 
\end{equation}

Здесь $R_t$ -- полное сопротивление измерительной цепи тороида, $S_{t}$ -- площадь поперечного сечения тороида.

Баллистическую постоянную $b$ можно определить с помощью следующей схемы (рис. 3): вместо тороида возьмём соленоид, и, воспользовавшись той же формулой, получим

\begin{equation}
    \Delta x = \frac{S_c N_c'}{bR_c} \Delta B_c = \frac{\mu_0 S_c N_c' N_c}{b R_c l_c} \Delta I_c
\end{equation}

\begin{figure}[h]
    \centering
    \includegraphics[width = 6 cm]{3.png}
    \caption{Схема для калибровки
    гальванометра}
    \label{kalibr}
\end{figure}

Таким образом, можно исключить калибровочную постоянную $b$, учитывая, что в исследуемой схеме будет подобраны сопротивления схем с тороидом и соленоидом были равны.

\begin{equation}
    \label{eq:B}
    \Delta B = \mu_0 \left(\frac{d_c}{d_t}\right)^2 \frac{N_{c}'}{N'} \frac{N_{c}}{l_c} \Delta I_c \frac{\Delta x}{\Delta x_c}
\end{equation}

Схема для исследования петли гистерезиса представлена на рис. 4.

\begin{figure}[h]
    \centering
    \includegraphics[width = 15 cm]{4.png}
    \caption{Схема установки для исследования петли гистерезиса}
    \label{}
\end{figure}

\begin{figure}[h]
    \centering
    \includegraphics[width = 15 cm]{5.png}
    \caption{Схема установки с соленоидом вместо тороида}
    \label{}
\end{figure}

% Здесь $R_m$ -- сопротивление нагрузки для контроля амплитуды отклонения луча (работа гальванометра реализована через лампу, чтобы можно было исследовать отклонение луча, зависящего от пройденного заряда).

% Для калибровочной постоянной нужно собрать такую же схему, где вместо тороида следует поместить соленоид.

Чтобы получить начальную кривую намагничивания, необходимо размагнитить сердечник путём подключения к цепи переменного тока и постепенного уменьшения его апмлитуды.

\begin{figure}[ht!]
    \centering
    \includegraphics[width = 10 cm]{6.png}
    \caption{Схема установки для размагничивания образца}
    \label{}
\end{figure}

\newpage
\newpage

\section{Ход работы}
\begin{enumerate}
    \item Для подготовки к работе была собрана схема с тороидом, проверена её работоспособность, установлено начальное значение $R_M = 300$ Ом, превыщающее сопротивление соленоида, включён осветитель гальванометра. Установлено, что при любых изменениях силы тока зайчик не выходит за пределы шкалы.
    \begin{table}[!ht]
    \centering
    \caption*{Характеристики исследуемого образца}
    \begin{tabular}{|c|c|c|c|c|}
    \hline
        Материал & $N_{T0}$ & $N_{T1}$ & D, м & $d_t$, м \\ \hline
        Железо & 1750 & 300 & 0,1 & 0,01 \\ \hline
    \end{tabular}
    \end{table}

    \begin{table}[ht]
    \centering
    \caption*{Характеристики соленоида}
    \begin{tabular}{|c|c|c|c|c|}
    \hline
        $R_C$, Ом & $l_C$, м & $d_C$, м & $N_{C0}$ & $N_{C1}$ \\ \hline
        60 & 0,8 & 0,07 & 825 & 435 \\ \hline
    \end{tabular}
\end{table}


    \item Было сделано четыре серии измерений зависимости отклонения зайчика при изменении силы тока. По формулам (2) и (6) найдены значения $\Delta B$ и $ H$, исследуемых в эксперименте. Часть результатов представлена в таблицах:
\begin{table}[ht]
    \centering
    \caption*{Понижение силы тока}
    \begin{tabular}{|c|c|c|c|}
    \hline
        I, мA & $\Delta x$, мм & H, А/м & $\Delta B$, Тл \\ \hline
        830,9 & 0,0 & -145,360 & 0,000 \\ \hline
        325,0 & $(56\pm0,5)$ & -145,282 & $(0,242\pm0,002)$ \\ \hline
        127,4 & $(55\pm0,5)$ & -141,042 & $(0,238\pm0,002)$ \\ \hline
        102,2 & $(34\pm0,5)$ & -139,919 & $(0,147\pm0,002)$ \\ \hline
        84,0 & $(25\pm0,5)$ & -137,317 & $(0,108\pm0,002)$ \\ \hline
        66,8 & $(14\pm0,5)$ & -136,004 & $(0,061\pm0,002)$ \\ \hline
        56,0 & $(7\pm0,5)$ & -113,841 & $(0,030\pm0,002)$ \\ \hline
        44,5 & $(6\pm0,5)$ & -95,891 & $(0,026\pm0,002)$ \\ \hline
        39,1 & $(5\pm0,5)$ & -78,810 & $(0,022\pm0,002)$ \\ \hline
        28,1 & $(6\pm0,5)$ & -42,534 & $(0,026\pm0,002)$ \\ \hline
        15,4 & $(7\pm0,5)$ & -34,787 & $(0,030\pm0,002)$ \\ \hline
        0,0 & $(15\pm0,5)$ & 0,000 & $(0,065\pm0,002)$ \\ \hline
    \end{tabular}
\end{table}

\begin{table}[!ht]
    \centering
    \caption*{Повышение силы тока}
    \begin{tabular}{|c|c|c|c|}
    \hline
        I, мA & $\Delta x$, мм & H, А/м & $\Delta B$, Тл \\ \hline
        0,0 & 0,0 & 0,000 & 0,0000 \\ \hline
        15,4 & $(-3\pm0,5)$ & 34,787 & $(-0,012\pm0,002)$ \\ \hline
        28,1 & $(-3\pm0,5)$ & 42,533 & $(-0,013\pm0,002)$ \\ \hline
        39,1 & $(-3\pm0,5)$ & 78,809 & $(-0,013\pm0,002)$ \\ \hline
        44,5 & $(-2\pm0,5)$ & 95,835 & $(-0,008\pm0,002)$ \\ \hline
        55,9 & $(-4\pm0,5)$ & 113,395 & $(-0,017\pm0,002)$ \\ \hline
        66,7 & $(-3\pm0,5)$ & 136,558 & $(-0,013\pm0,002)$ \\ \hline
        83,9 & $(-14\pm0,5)$ & 137,648 & $(-0,060\pm0,002)$ \\ \hline
        105,1 & $(-32\pm0,5)$ & 139,139 & $(-0,138\pm0,002)$ \\ \hline
        129,3 & $(-39\pm0,5)$ & 141,318 & $(-0,169\pm0,002)$ \\ \hline
        336,7 & $(-62\pm0,5)$ & 145,610 & $(-0,268\pm0,002)$ \\ \hline
        859,6 & $(-59\pm0,5)$ & 144,688 & $(-0,255\pm0,002)$ \\ \hline
    \end{tabular}
\end{table}

\item Образец размагнитили, подключив к цепи переменного тока и постепенно уменьшив его аплитуду до нуля. Затем аналогичным способом получили зависимость $\Delta B(H)$ для начальной кривой намагничивания. Результаты представлены в таблице:

\begin{table}[!ht]
    \centering
    \caption*{Начальная кривая намагничивания}
    \begin{tabular}{|c|c|c|c|}
    \hline
        I, мA & $\Delta x$, мм & H, А/м & $\Delta B$, Тл \\ \hline
        0,0 & 0,0 & 0,000 & 0,0000 \\ \hline
        15,3 & $(8\pm0,5)$ & 1,230 & $(0,035\pm0,002)$ \\ \hline
        28,1 & $(17\pm0,5)$ & 2,534 & $(0,074\pm0,002)$ \\ \hline
        39,1 & $(44\pm0,5)$ & 3,810 & $(0,190\pm0,002)$ \\ \hline
        44,5 & $(22\pm0,5)$ & 4,891 & $(0,095\pm0,002)$ \\ \hline
        55,9 & $(45\pm0,5)$ & 5,396 & $(0,195\pm0,002)$ \\ \hline
        65,7 & $(27\pm0,5)$ & 6,558 & $(0,117\pm0,002)$ \\ \hline
        81,9 & $(61\pm0,5)$ & 9,649 & $(0,264\pm0,002)$ \\ \hline
        103,2 & $(71\pm0,5)$ & 15,696 & $(0,307\pm0,002)$ \\ \hline
        131,4 & $(60\pm0,5)$ & 24,875 & $(0,260\pm0,002)$ \\ \hline
        332,9 & $(79\pm0,5)$ & 53,725 & $(0,342\pm0,002)$ \\ \hline
        861,6 & $(71\pm0,5)$ & 146,688 & $(0,307\pm0,002)$ \\ \hline
    \end{tabular}
\end{table}

\item Для построения петли гистерезиса были просуммированы все скачки $\Delta B$ и получена зависимость $B(H)$. В результате была собрана полная петля, и аналогичным образом на график была добавлена Начальная кривая намагничивания:
\begin{figure}[h]
    \centering
    \includegraphics[width = 17 cm]{Петля лучше.pdf}
    \caption{График петли гистерезиса}
    \label{}
\end{figure}

\item Из графика определена коэрцитивная сила железа $H_C = 144 \pm 3$ А/м, а такие индукция насыщения $B_S = 1,84 \pm 0,04$ Тл.
\item По наклону графика определено значение максимальной дифференциальной магнитной проницаемости $\mu = \frac{1}{\mu_0} \frac{dB}{dH} = 465 \pm 5$. 





\end{enumerate}

\newpage
\section*{Вывод}
В результате выполнения работы были получены предельная петля гистерезиса и начальная кривая намагничивания. Затем из графика зависимости $B(H)$ были определены коэрцитивная сила, индукция насыщения и максимальная дифференциальная магнитная проницаемость. Результаты в сравнении с табличными:

\begin{table}[!ht]
    \centering
    \begin{tabular}{|c|c|c|}
    \hline
         & Эксперимент & Справочное значение  \\ \hline
        $H_C$, А/м & $144 \pm 3$ & 70 \\ \hline
        $B_S$, Тл & $ 1,84 \pm 0,04$ & 2,16  \\ \hline
        $\mu$ & $465 \pm 5$ & 6000 \\ \hline
    \end{tabular}
\end{table}

Как видно, с табличными значениями только $B_S$ сошлось по порядку величины. Вероятнее всего, расхождение связано с неидеальностью установки, возможностью существования примесей в исследуемом образце и ограниченностью максимально возможной силы тока -- возможно, мы не дошли до участка, где зависимость становится линейной, и можно более точно определить максимальную дифференциальную магнитную проницаемость
































% \newpage
% Для начала соберём нужны данные для дальнейшего исследования. $N_t = 1750$, $N' = 300$, $R_0 = 5,6$ Ом (сопротивление гальванометра).

% Собрав схему и настроив гальванометр, обойдём всю кривую гистерезиса, исследуя, какое значение сопротивления $R_m$ нужно подобрать, чтобы отклонение луча не заходило за пределы измерительной шкалы. Получим $R_m = 220$ Ом.

% \subsection{Предельная петля гистерезиса}

% Начнём измерять предельную петлю гистерезиса с максимального тока (тока насыщения), равного $I_0 = 1,7 \pm 0,02$ А. 

% Результаты измерений зависимости $\Delta B(I)$ находятся в приложении к отчёту лабораторной работы.

% \subsection{Калибровка гальванометра}  

% Подсоединив соленоид вместо тороида найдём значения $\Delta I_c = 1,71 \pm 0,02$ А и $x_c = 13,5 \pm 0,2$ см. Здесь учтена погрешность измерения отклонения луча, в которую входит погрешность определения нулевой точки луча и точки измерения.

% Характеристика соленоида: $N_c' = 500$, $N_c = 940$, $R_c = 46$ Ом, $l_c = 0,8$ м, $d_c= 0,07$ м.

% \subsection{Начальная кривая намагничивания} 

% Теперь с помощью генератора переменного тока, уменьшая амплитуду тока но нуля, вернём тороид в начальное состояние $(0,0)$ в зависимости $H(B)$, иначе говоря, размагнитим.

% Теперь проведём такие же измерения, как и для предельной петли, дойдя до тока насыщения. 

% \subsection{Обработка данных}  

% Для начала, учитывая зависимость напряжённости магнитного поля и силы тока намагничивающей обмотки, переведём все значени сил тока в напряжённость (с учётом знаков напряжённости):

% \begin{equation}
%     H = \frac{N_t}{\pi D} I
% \end{equation}

% Средний диаметр тороида равен $D = 0,1$ м, откуда находим $H$.

% Теперь, используя формулу ниже, переведём значения $\Delta x$ в значения $\Delta B$ как для предельной петли, так и для начальной кривой.

% \begin{equation}
%     \label{eq:B2}
%     \Delta B = \mu_0 \left(\frac{d_c}{d_t}\right)^2 \frac{N_{c}'}{N'} \frac{N_{c}}{l_c} \Delta I_c \frac{\Delta x}{\Delta x_c}
% \end{equation}

% Здесь $d_t = 0,01$ м, остальные значения уже известны.

% Просуммировав изменения индукции для начальной кривой на каждом шаге, получим максимальное значение индукции $B_{max} = 1,35 \pm 0,04$, через которое с помощью значений $\Delta B$ можно найти все остальные значения индукции.

% Построим по всем точкам петлю гистерезиса (см рис. 5).

% \begin{figure}[h!]
%     \centering
%     \includegraphics[width = 14 cm]{gist.png}
%     \caption{График петли гистерезиса}
%     \label{gist}
% \end{figure}

% Найдём из графика оценку коэрцитивной силы стали (исследуемого материала) $H_c = 1540 \pm 20 \; A/\text{м}$ (точность высока, т.к. высока точность измерений сил тока). 

% Также найдём индукцию насыщения $B_s = 1,37 \pm 0,05$ Тл.

% По наклону графика определим значение максимальной дифференциальной магнитной проницаемости $\mu = \frac{1}{\mu_0} \frac{dB}{dH} = 533 \pm 21$. 

% Заметим, что значения по порядку сходятся со справочными $\mu = 1100$ (если материал -- чистая мягкая сталь) и $H_c = 1540 \pm 20 \; A/\text{м}$ (если сталь углеродистая).

% \section{Заключение}

% Для стали действительно справедлива ферромагнитная теория, так как зависимость $B(H)$ имеет вид петли гистерезиса. Значения коэрцитивной силы, индукции насыщения и дифференциальной магнитной проницаемости зависят полностью от сплава стали, которай является исследуемым в этом эксперименте материалом. Для полученных значений определить сплав стали не удалось из-за расхождений в одном из показателей, но по порядку величин значения сходятся.

\end{document}