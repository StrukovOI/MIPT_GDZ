\documentclass[a4paper]{article}
\usepackage[utf8]{inputenc}
\usepackage[russian,english]{babel}
\usepackage[T2A]{fontenc}
\usepackage[left=10mm, top=20mm, right=18mm, bottom=15mm, footskip=10mm]{geometry}
\usepackage{indentfirst}
\usepackage{amsmath,amssymb}
\usepackage[italicdiff]{physics}
\usepackage{graphicx}
\graphicspath{{images/}}
\DeclareGraphicsExtensions{.pdf,.png,.jpg}
\usepackage{wrapfig}

\usepackage{caption}
\captionsetup[figure]{name=Рисунок}
%\captionsetup[table]{name=Таблица}
  
\title{\begin{center}Лабораторная работа №1.2.1\end{center}
Определение скорости полёта пули при помощи баллистического маятника}
\author{Струков О. И. \\
Б04-404}
\date{}


\begin{document}

\maketitle
\newpage
\begin{center}
\end{center}

\section*{Аннотация}
    \par \textbf{Цель работы:} Определить скорость полёта пули применяя законы сохранения и использую баллистические маятники.\\
    \par \noindent \textbf{Оборудование:} Духовое ружьё на штативе, осветитель, оптическая система для измерения отклонений маятника, измерительная линейка, пули и весы для их взвешивания, баллистические маятники.


\subsection*{Экспериментальная установка}
При попадании пули в цилиндр любая его точка движется по окружности известного радиуса, поэтому его смещение с помощью собирающей линзы можно перевести в линейное отклонение на линейке.\\
	\begin{figure}[!h]
		\begin{center}
			\includegraphics[scale = 0.7]{ustan1}
			\caption{Схема установки для измерения скорости полета пули}
		\end{center}
	\end{figure}

\subsection*{Метод баллистического маятнкиа, совершающего поступательное движение}
При контакте пули с цилиндром можно записать ЗСИ:
	\begin{equation}
		 mu = (M+m)V
	\end{equation}
	где $m$ -- масса пули, $u$ -- скорость пули перед ударом, $V$-скорость цилиндра вместе с пулей после удара.
	\begin{equation}
		u=\frac{M+m}{m}V \approx \frac{M}{m}V \;\;\;\;\; V^2=2gh \;\;\;\;\; h = L(1-cos \varphi ) = 2L^2 sin \frac{\varphi^2}{2} \;\;\;\;\;\;\; \varphi \approx \frac{\Delta x}{L} 
	\end{equation}
        где $\varphi$ - угол отклонения маятника от вертикали, $\Delta x$ - отклонение маятника\\
	Тогда скорость пули можно выразить как
	\begin{equation} \label{vel1}
	 u=\frac{M}{m} \sqrt{\frac{g}{L}} \Delta x
	\end{equation}

	\begin{figure}[!h]
		\begin{center}
			\includegraphics[scale = 0.7]{ustan12}
			\caption{Cхема установки для измерения скорости полета пули}
		\end{center}
	\end{figure}
\newpage
\subsection*{Метод крутильного баллистического маятника}
	\begin{figure}[!h]
		\begin{center}
			\includegraphics[scale = 0.7]{ustan2}
			\caption{Cхема установки для измерения скорости полета пули с крутильным баллистическим маятником}
		\end{center}
	\end{figure}
Считая удар неупругим, можно записать уравнение
	$$mur=I \Omega$$
	$r-$расстояние от линии полёта пули до оси вращения, $I$ -- момент инерции относительно этой оси, $\Omega$ -- угловая скорость маятника сразу после удара.
	
	Можно пренебречь затуханием колебаний и потерями энергии и записать ЗСЭ:
	$$ k \frac{\varphi^2}{2} = I \frac{\Omega^2}{2} $$
	\noindent где $k$ -- модуль кручения проволоки, $\varphi$ -- максимальный угол поворота маятника, тогда:
	\begin{equation} \label{vel2}
		 u = \varphi \frac{\sqrt{kI}}{mr} 
	\end{equation}
\begin{equation}
		\label{phi}
		\varphi \approx \frac{x}{d}
	\end{equation}
	
	где $x$ -- смещение изображения нити осветителя на шкале, которое легко можно измерить.
	
	Периоды колебаний маятника с грузами и без можно выразить как
	$$T_1= 2 \pi \sqrt{\frac{I - 2MR^2}{k}} \qquad T_2 = 2 \pi \sqrt{\frac{I}{k}}$$
	Тогда $\sqrt{kI}$ можно найти как:
	\begin{equation}
		\sqrt{kI} = \frac{4 \pi M R^2 T_2}{T_2^2 - T_1^2}
		\label{kl}
	\end{equation}
	$R$ -- расстояние от оси вращения до центров грузиков, $M$ - масса грузиков.
\newpage
\section*{Ход работы}
\subsection*{Метод баллистического маятника, совершающего поступательное движение}
\begin{enumerate}
\item Было проведено ознакомление с устройством баллистического маятника, прослушан инструктаж безопасности.
\item На аналитических весах были определены массы пяти используемых в данном эксперменте пулек. Значения внесены в таблицу. Абсолютной погрешностью каждого измерения является $\sigma_m = 0,005$ г.
\begin{center}
	\begin{tabular}{|c|c|c|c|c|c|}
	\hline
	Номер пули & 1 & 2 & 3 & 4 & 5 \\ \hline
	Масса, г & 0,514 & 0,506 & 0,509 & 0,511 & 0,513 \\ \hline
	\end{tabular}
\end{center}
\item С помощью длинной линейки было измерено расстояние $L$ для каждой из четырёх проволок. Среднее значение всех измерений составило $L = 220,2$ см, а погрешность измерения - $\sigma_L = 0,1$ см. Масса маятника была известна: $M = (2905 \pm 5)$ г.
\item Оптическая система, предназначенная для измерения поворота маятника была настроена корректно, на стене было получено чёткое изображение шкалы для определения амплитуды колебаний, маятник покоился.
\item Был произведён холостой выстрел в сторону мишени, в результате чего было установлено, что маятник практически не реагирует на удар воздушной струи из ружья.
\item Затухание колебаний являлось малым, так как через десять колебаний их амплитуда уменьшилась меньше, чем на половину.
\item Было произведено пять выстрелов и для каждого из них по формуле 3 была определена скорости полёта пули. Результаты занесены в таблицу.
\begin{center}
	\begin{tabular}{|c|c|c|c|c|c|}
	\hline
	Номер пули & 1 & 2 & 3 & 4 & 5 \\ \hline
	Амплитуда отклонения $\Delta x$, мм & 11,25 & 11,25 & 11,0 & 11,0 & 11,25 \\ \hline
	Погрешность амплитуды $\sigma_{\Delta x}$, мм & 0,25 & 0,25 & 0,25 & 0,25 & 0,25 \\ \hline
	Скорость пули $u$, м/с & 134,2 & 136,4 & 132,6 & 132,0 & 134,5 \\ \hline
	Погрешность скорости $\sigma_u$, м/с & 3,3 & 3,4 & 3,3 & 3,3 & 3,3 \\ \hline
	\end{tabular}
\end{center}
\item Оценена погрешность определения скорости пули в каждом выстреле по формуле ниже. Результаты добавлены в таблицу.
\[ \sigma_u = u\sqrt{\left(\frac{\sigma_M}{M}\right)^2 + \left(\frac{\sigma_m}{m}\right)^2 + \frac{1}{4}\left(\frac{\sigma_L}{L}\right)^2 + \left(\frac{\sigma_{\Delta x}}{\Delta x}\right)^2} \]
\item Среднее значение скорости полёта пули составило $u_{\text{ср}} = 134,0$ м/с. Разброс отдельных результатов от него варьируется от 0,2 м/с до 2,4 м/с, что находится в рамках погрешности. Поскольку при получении значений ошибки замечены не были и сильные отклонения отсутствовали, можно прийти к выводу, что данный разброс возникает в результате существования погрешностей измерений и различной скоростью полёта пуль, зависящей от условий при выстреле (небольшие различия в форме и массе пуль, потоки воздуха в кабинете).
\end{enumerate}



\subsection*{Метод крутильного баллистического маятника}
\begin{enumerate}
\item  Было проведено ознакомление с конструкцией, прослушан инструктаж безопасности.
\item На аналитических весах были определены массы пяти используемых в данном эксперменте пулек. Значения внесены в таблицу. Абсолютной погрешностью каждого измерения является $\sigma_m = 0,005$ г.
\begin{center}
	\begin{tabular}{|c|c|c|c|c|c|}
	\hline
	Номер пули & 1 & 2 & 3 & 4 & 5 \\ \hline
	Масса, г & 0,504 & 0,503 & 0,510 & 0,508 & 0,514 \\ \hline
	\end{tabular}
\end{center}
\item С помощью линейки, рулетки и штангенциркуля были измерены расстояния $r$, $R$ и $d$. Значения и их абсолютные погрешности внесены в таблицу.
\begin{center}
	\begin{tabular}{|c|c|c|c|c|c|}
	\hline
	$r$, см & $\sigma_r$, см & $R$, см & $\sigma_R$, см & $d$, см & $\sigma_d$, см \\ \hline
	22,4 & 0,1 & 31,3 & 0,1 & 135,7 & 0,1 \\ \hline
	\end{tabular}
\end{center}
\item Оптическая система, предназначенная для измерения поворота маятника была настроена корректно, на линейке чётко была видна отметка точка, создаваемая лазером, колебания маятника отсутствовали.
\item Преподаватель подтвердил факт того, что маятник практически не реагирует на воздушную струю из ружья при совершении холостых выстрелов.
\item Затухание колебаний являлось малым, так как через десять колебаний их амплитуда уменьшилась меньше, чем на половину.
\item При измерении десяти полных крутильных колебаний были определены их периоды $T_1$ (без грузов) и $T_2$ (с грузами). Измерена масса каждого из грузов $M$. Была найдена величина $\sqrt{kI}$ по формуле 6 и оценена её погрешность $\sigma_{\sqrt{kI}}$ по формуле ниже. Результаты и их абсолютные погрешности указаны в таблице.
\[ \sigma_{\sqrt{kI}} = \sqrt{kI}\sqrt{\left(\frac{\sigma_M}{M}\right)^2 + 4\left(\frac{\sigma_R}{R}\right)^2 + 4\left(\frac{\sigma_{T_1}}{T_1}\right)^2 + 9\left(\frac{\sigma_{T_2}}{T_2}\right)^2} \]
\begin{center}
	\begin{tabular}{|c|c|c|c|c|c|c|}
	\hline
	$M$, г & $\sigma_M$, г & $T_1$, с & $T_2$, с & $\sigma_T$, с & $\sqrt{kI}$, $\text{кг}\cdot\text{м}^2/\text{с}$ & $\sigma_{\sqrt{kI}}$, $\text{кг}\cdot\text{м}^2/\text{с}$ \\ \hline
	713,6 & 0,1 & 5,05 & 6,62 & 0,05 & 0,3174 & 0,004 \\ \hline
	\end{tabular}
\end{center}
\item Всего было произведено пять выстрелов и определена скорость полёта пули при каждом из них по формуле 4. Результаты представлены в таблице.
\begin{center}
	\begin{tabular}{|c|c|c|c|c|c|}
	\hline
	Номер пули & $x$, см & $\sigma_x$, см & $u$, м/с & $\sigma_u$, м/с \\ \hline
	1 & 4,6 & 0,1 & 95,3 & 2,4 \\ \hline
	2 & 4,4 & 0,1 & 91,3 & 2,3 \\ \hline
	3 & 4,7 & 0,1 & 96,2 & 2,4 \\ \hline
	4 & 4,5 & 0,1 & 92,5 & 2,3 \\ \hline
	5 & 4,7 & 0,1 & 95,5 & 2,4 \\ \hline
	\end{tabular}
\end{center}
\item Оценена погрешность определения скорости пули при каждом выстреле. Результаты добавлены в таблицу выше.
\[ \sigma_u = u\sqrt{\left(\frac{\sigma_{\sqrt{kI}}}{\sqrt{kI}}\right)^2 + \left(\frac{\sigma_m}{m}\right)^2 + \left(\frac{\sigma_r}{r}\right)^2 + \left(\frac{\sigma_x}{x}\right)^2 + \left(\frac{\sigma_d}{d}\right)^2}\]
\item Среднее значение скорости полёта пули составило $u_{\text{ср}} = 94,2$ м/с. Разброс отдельных результатов от него варьируется от 1,1 м/с до 2,9 м/с, что выходит за пределы погрешности. Данный разброс возникает по обеим причинам, указанным в задании: из-за погрешностей измерений и различной скоростью полёта пуль, зависящей от условий при выстреле (небольшие различия в форме и массе пуль, потоки воздуха в кабинете).
\end{enumerate}




\end{document}