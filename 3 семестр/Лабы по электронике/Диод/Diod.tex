\documentclass[a4paper]{article}
\usepackage[utf8]{inputenc}
\usepackage[russian,english]{babel}
\usepackage[T2A]{fontenc}
\usepackage{multirow}
\usepackage[left=10mm, top=20mm, right=18mm, bottom=15mm, footskip=10mm]{geometry}
\usepackage{indentfirst}
\usepackage{amsmath,amssymb}
\usepackage{booktabs}
\usepackage[italicdiff]{physics}
\usepackage{graphicx}
\graphicspath{{images/}}
\DeclareGraphicsExtensions{.pdf,.png,.jpg}
\usepackage{wrapfig}
\usepackage{subcaption}
\usepackage{caption}
\usepackage{array}
\usepackage{siunitx}
\usepackage{geometry}
\usepackage{float}
\captionsetup[figure]{name=Рисунок}
\captionsetup[table]{name=Таблица}
  
\begin{document}

\include{titlepage}
\begin{titlepage}
\thispagestyle{empty}

\begin{center}
\small
Министерство науки и высшего образования Российской Федерации\\[2mm]
\textbf{ФЕДЕРАЛЬНОЕ ГОСУДАРСТВЕННОЕ АВТОНОМНОЕ
ОБРАЗОВАТЕЛЬНОЕ УЧРЕЖДЕНИЕ ВЫСШЕГО ОБРАЗОВАНИЯ}\\[1mm]
«МОСКОВСКИЙ ФИЗИКО-ТЕХНИЧЕСКИЙ ИНСТИТУТ\\
(НАЦИОНАЛЬНЫЙ ИССЛЕДОВАТЕЛЬСКИЙ УНИВЕРСИТЕТ)»\\
(МФТИ, Физтех)
\end{center}

\vspace{18mm}

\begin{center}
\large \textbf{КАФЕДРА ЭЛЕКТРОНИКИ}\\[10mm]
\Large \textbf{ОТЧЁТ}\\[2mm]
\large \textbf{ПО ЛАБОРАТОРНОЙ РАБОТЕ}\\[14mm]
\Large \textbf{ТЕРМОЭЛЕКТРОННЫЙ ДИОД}
\end{center}

\vfill

% Блоки с подписями слева, линии и пояснения справа
\noindent
\begin{tabular}{@{}p{0.44\linewidth}p{0.52\linewidth}@{}}
\textit{Работу выполнили:} & Винокуров Владислав,\\%[8mm]
& Рагозина Елизавета,\\
& Струков Олег,\\
& Топольский Михаил, группа Б04-404 \\[8mm]

\textit{Работу принял, оценка:} & \\[3mm]
 \\
\end{tabular}


\vspace{12mm}

\begin{center}
\small Долгопрудный, \the\year{}~г.
\end{center}

\end{titlepage}
\newpage


\textbf{Цели работы:} практическое изучение явления термоэлектронной эмиссии и
процессов токопрохождения в вакууме, изготовление вакуумного
диода и исследование некоторых его характеристик. \par
\textbf{Оборудование:} самодельный диод из никилевой пластины и вольфрамовой нити, насос для откачки воздуха,гермитичная камера для проведения эксперимента (крепление диода), источник питания,
амперметр, вольтметр, соединительные провода.

\section*{Теоретические сведения}

Термоэлектронной эмиссией (ТЭЭ) называется испускание
электронов поверхностью нагретых проводящих тел. \par
Явление термоэлектронной эмиссии и токопрохождение в вакууме
являются основными процессами, определяющими работу многих
электровакуумных приборов.

\begin{wrapfigure}[11]{r}{5.0cm}\vspace{-6mm}
  \centering
  \includegraphics[width=5cm]{p1.PNG}
  \caption{Конструкция простейшего термоэлектронного диода}
  \label{Theor}
\end{wrapfigure}

Простейший прибор для наблюдения ТЭЭ -- термоэлектронный диод, помещенный в объем с низким давлением остаточных газов, состоит из двух металлических электродов (рис. 1). Электрод, эмитирующий электроны, называется \emph{катодом} или \emph{эмиттером}. Получающий электроны электрод обычно
называется \emph{анодом} или \emph{коллектором}. При термоэмиссии катод нагрет до высокой температуры $T$, а анод имеет существенно более низкую температуру.
Прикладывая разность потенциалов между катодом и анодом $V_A$ и измеряя ток $I_A$, протекающий между электродами, получим вольтамперную характеристику (ВАХ) диода, т.е. зависимость $I_A = f(V_A)$.

\subsection*{Формула Ричардсона-Дешмана}

Для прямоугольного потенциального барьера Ричардсон и Дэшман (1928) рассчитали максимальную плотность тока (тока насыщения) термоэлектронной эмиссии, которую может обеспечить при температуре $T$ термокатод с работой выхода электронов $\varphi$ (формула Ричардсона--Дэшмана):

\begin{equation}
j = A_0 (1 - \bar{r}) T^2 \exp \left( -\frac{\varphi}{kT} \right),
\end{equation}

где

\begin{equation}
A_0 = \frac{4\pi l_0 m k^2}{h^3} = 120.14 \, \text{A/см}^2\text{K}^2
\end{equation}

-- термоэмиссионная постоянная Зоммерфельда; $T$ -- температура катода по абсолютной шкале Кельвина; $\bar{r}$ -- коэффициент отражения электронов на границе металл-вакуум (обычно не превышает 0,07 и при оценочных расчётах им можно пренебречь); $\varphi$ -- 
работа выхода электронов из катода; $k = 1.38 \cdot 10^{-23} \, \text{Дж/К}$ -- постоянная Больцмана. Сила тока ТЭЭ определяется выражением $I = jS_k$, где $S_k$ -- площадь эмитирующей поверхности катода.

При больших напряжениях проявляется эффект Шоттки -- понижение потенциального барьера на границе металл-вакуум при приложении вытягивающего электроны электрического поля напряженностью $E$. Снижение работы выхода электронов, вызванное искажением барьера, определяется формулой

\begin{equation}
\Delta \varphi = e \sqrt{eE} \, (\text{в CГСЭ}) = 3,8 \cdot 10^{-4} \sqrt{E} [\text{эВ}],
\label{eq:3}
\end{equation}

где напряженность $E$ измеряется в В/см.

В формуле Ричардсона-Дэшмана учёт эффекта Шоттки приводит лишь к умножению тока насыщения на $\exp(\Delta \varphi / kT)$. С учётом (\ref{eq:3}) выражение (1) для плотности тока насыщения принимает вид

\begin{equation}
j = A_0 (1 - \bar{r}) T^2 \exp\left(-\frac{\varphi - e \sqrt{eE}}{kT}\right).
\label{eq:4}
\end{equation}

Для плоскопараллельной геометрии диода и максвелловского распределения по скоростям электронов, вылетевших из катода, зависимость анодного тока определяется формулой

\begin{equation}
I_A = I_0 \exp\left(-\frac{eV_m}{kT}\right),
\label{eq:5}
\end{equation}

\subsection*{Параметры термоэлектронных катодов}

Применяемые на практике термоэлектронные эмиттеры характеризуются эффективностью. Эффективность катода $H$ определяется отношением плотности тока насыщения к той удельной мощности накала $p_H$, которую необходимо подводить к единице поверхности катода для поддержания стационарных условий работы. Другими словами, эффективность -- это электронный ток, получаемый на единицу мощности накала $P_H$:

\begin{equation}
H = I / P_H = j / p_H.
\label{eq:14}
\end{equation}

Подставляя формулу Ричардсона--Дэшмана в (16) и пренебрегая степенной зависимостью $H$ от $T$, получим

\begin{equation}
H \approx C_1 \exp(-\varphi/kT).
\label{eq:15}
\end{equation}

Срок службы катода $\tau$ определяется в основном скоростью испарения рабочего вещества катода:

\begin{equation}
\tau = C_2 \exp(-q/kT),
\label{eq:16}
\end{equation}

где $q$ -- теплота испарения.

С повышением температуры эмиссионная способность и эффективность термокатода экспоненциально растут, а срок службы экспоненциально уменьшается. Обычно термокатоды имеют эффективность 5--100 мА/Вт, а срок службы -- от 5 до 100 тысяч часов. Критерием пригодности вещества для применения в качестве термокатода является условие $\varphi / q < 0,5$.

\subsection*{Статические характеристики и параметры диода}

Крутизна диода в данной точке его характеристики определяется тангенсом наклона касательной к кривой $I_A = f(V_A)$ и приближенно равна отношению приращения тока анода к приращению на аноде:

Дифференцируем уравнение тока $I_A$ (формулу закона ``трёх вторых'') по $V_A$, получим аналитическое выражение крутизны в заданной точке:

\begin{equation}
S = \frac{dI_A}{dV_A} = \frac{3}{2} gV_A^{\frac{1}{2}}.
\end{equation}

В случае цилиндрической геометрии, как следует из (13), крутизна характеристики данного типа лампы определяется отношением $S_A / r_A^2$, т.е. она тем больше, чем длиннее катод и чем меньше расстояние катод-анод.

Внутреннее сопротивление -- величина, обратная крутизне. От этого сопротивления необходимо отличать сопротивление лампы постоянному току, которое определяется по закону Ома и равно:

\begin{equation}
R_0 = \frac{V_A}{I_A} = \frac{V_A}{gV_A^{3/2}} = \frac{1}{gV_A^{1/2}}.
\label{eq:18}
\end{equation}

Первеанс определяется как отношение тока к анодному напряжению в степени трех вторых (12). первеанс не зависит от анодного напряжения и наиболее полно характеризует электронный поток.

Обратное напряжение $V_{\text{обр}}$ характеризует максимальное допустимое напряжение между анодом и катодом, которое выдерживает лампа без пробоя при подаче на анод отрицательного напряжения.

Допустимая мощность. Электроны, падающие на анод, передают свою кинетическую энергию. Температура анода вследствие этого повышается. Энергия, сообщенная аноду электронами на единицу времени, определяется выражением $P_A = I_A V_A$. Кроме того, на анод поступает часть тепла, излучаемого катодом. Полную мощность, выделяемую на аноде, следует подсчитывать по формуле

\begin{equation}
P_A = I_A V_A + \alpha P_H, \quad \text{где } \alpha = 0,2 - 1,0.
\label{eq:19}
\end{equation}

\newpage

\section*{Схема и описание экспериментальной установки}

Схема экспериментальной установки выглядит следующим образом:

\begin{figure}[h!]
	\centering
	\includegraphics[width=8cm]{p2.PNG}
	\caption{Электрическая схема для измерений характеристик диода}
	% \label{fig:Holl2}
\end{figure}

где, 1 -- исследуемый диод, 2 -- регулируемый источник тока накала, 3 -- вольтметр, 4 -- регулируемый источник напряжения, 5 -- амперметр.

% \section*{Порядок выполнения работы}

\subsection*{Изготовление диода}

\begin{wrapfigure}[11]{l}{5.0cm}\vspace{-6mm}
  \centering
  \includegraphics[width=4cm]{p3.PNG}
  \caption{Вид диода в сборке}
  % \label{Theor}
\end{wrapfigure}

Сначала свернули никелевую пластину в цилиндр, затем при помощи сварки скрепили края цилиндра вместе.

Затем сложили никелевую проволоку по нужной геометрии и припаяли сбоку цилиндра (отвод анода).

Следующим шагом припаяли анодную часть к креплению, с помощью которого диод будет крепиться в установке.

Далее была изготовлена катодная часть с помощью никелевой проволоки и вольфрамовой нити и припаяна к основной конструкции.

После этого была проверена работоспособность диода и начаты измерения.


\clearpage

\subsection*{Проведение измерений}
\subsubsection*{Параметры диода}
Анод: материал никель, диаметр 10 мм, высота 35 мм, толщина пластины 0,15 мм.

Катод: материал вольфрам, диаметр проволоки 0,15 мм, длина 45 мм, $\phi = 4,5$ эВ, $\alpha = 2,29\cdot10^{-3}\text{ к}^{-1}$, $\rho = 6,4 \cdot 10^{-6}$.

Во время измерений давление в камере было $3,3\cdot10^{-5}$ торр.


\begin{table}[h!]
\centering
\setlength{\tabcolsep}{3.5pt}

\begin{tabular}{|c|c|c|c|c|c|c|c|c|c|c|c|c|c|c|c|}
    \hline
        I, A & 0,1 & 0,2 & 0,3 & 0,4 & 0,5 & 0,6 & 0,7 & 0,8 & 0,9 & 1 & 1,1 & 1,2 & 1,3 & 1,4 & 1,5 \\ \hline
        U, B & 0,022 & 0,044 & 0,0693 & 0,0975 & 0,130 & 0,170 & 0,225 & 0,29 & 0,381 & 0,47 & 0,656 & 0,760 & 0,911 & 0,974 & 1,101 \\ \hline \hline
        I, A & 1,6 & 1,7 & 1,8 & 1,9 & 2 & 2,1 & 2,2 & 2,3 & 2,4 & 2,5 & 2,6 & 2,7 & 2,8 & 2,9& \\ \hline
        U, B & 1,294 & 1,389 & 1,60 & 2,09 & 2,26 & 2,50 & 2,80 & 3,03 & 3,24 & 3,5 & 3,78 & 4,14 & 4,44 & 4,66& \\ \hline
      \end{tabular}
\caption{Прогрев катода}
\end{table}

% \begin{table}[h!]
% \centering
% \setlength{\tabcolsep}{2.5pt}
% \begin{tabular}{c|cccccccccccccccccccc}
% \toprule
% \( I_\text{знак} \), А  & 1,1 & 1,2 & 1,3 & 1,4 & 1,5 & 1,6 & 1,7 & 1,8 & 1,9 & 2 & 2,1 & 2,2 & 2,3 & 2,4 & 2,5 & 2,6 & 2,7 & 2,8 & 2,9 \\
% \midrule
% \( U_\text{знак} \), В & 0,729 & 0,906 & 1,041 & 1,209 & 1,452 & 1,578 & 1,852 & 1,991 & 2,27 & 2,48 & 2,74 & 2,97 & 3,14 & 3,41 & 3,73 & 4,12 & 4,34 & 4,66 & 5,09 \\
% \bottomrule
% \end{tabular}
% \end{table}

% Давление в камере после прогрева $1\cdot10^{-4}$ Торр \par






\subsubsection*{Измерение вольт-амперных характеристик диода}

\begin{table}[h!]
\centering
\setlength{\tabcolsep}{3.5pt}
\begin{tabular}{|c|c|c|c|c|c|c|c|c|c|c|c|c|c|c|c|c|c|c|c|}
    \hline
        U, B & 1 & 2 & 3 & 4 & 5 & 10 & 20 & 30 & 40 & 50 & 60 & 70 & 80 & 90 & 100 & 110 & 120 & 130 & 140 \\ \hline
        I, мкA & 0,76 & 1,12 & 1,383 & 1,55 & 1,94 & 2,8 & 5,05 & 6,45 & 6,83 & 7,00 & 7,18 & 7,36 & 7,52 & 7,7 & 7,84 & 8,04 & 8,23 & 8,36 & 8,53 \\ \hline
      \end{tabular}
\caption{\( I_{\text{нак}} = 2{,}9\,\text{A},\quad U_{\text{нак}} = 5{,}0\,\text{B} \)}

\end{table}


\begin{table}[h!]
\centering
\setlength{\tabcolsep}{3.5pt}
\begin{tabular}{|c|c|c|c|c|c|c|c|c|c|c|c|c|c|c|c|c|c|c|c|}
    \hline
        U, B & 1 & 2 & 3 & 4 & 5 & 10 & 20 & 30 & 40 & 50 & 60 & 70 & 80 & 90 & 100 & 110 & 120 & 130 & 140 \\ \hline
        I, мкA & 0,65 & 0,783 & 0,98 & 1,01 & 1,25 & 2,06 & 5,04 & 6,2 & 6,54 & 6,75 & 6,9 & 7,13 & 7,28 & 7,5 & 7,7 & 7,78 & 7,85 & 7,96 & 8,05 \\ \hline
\end{tabular}
\caption{\( I_{\text{нак}} = 2{,}8\,\text{A},\quad U_{\text{нак}} = 4{,}9\,\text{B} \)}

\end{table}

\begin{table}[h!]
\centering
\setlength{\tabcolsep}{3.5pt}
\begin{tabular}{|c|c|c|c|c|c|c|c|c|c|c|c|c|c|c|c|c|c|c|c|}
    \hline
        U, B & 1 & 2 & 3 & 4 & 5 & 10 & 20 & 30 & 40 & 50 & 60 & 70 & 80 & 90 & 100 & 110 & 120 & 130 & 140 \\ \hline
        I, мкA & 0,45 & 0,505 & 0,612 & 0,601 & 0,825 & 1,67 & 2,73 & 2,85 & 2,92 & 3,05 & 3,1 & 3,17 & 3,21 & 3,24 & 3,27 & 3,3 & 3,34 & 3,37 & 3,41 \\ \hline
\end{tabular}
\caption{\( I_{\text{нак}} = 2{,}7\,\text{A},\quad U_{\text{нак}} = 4{,}6\,\text{B} \)}

\end{table}

\begin{table}[h!]
\centering
\setlength{\tabcolsep}{3.5pt}
\begin{tabular}{|c|c|c|c|c|c|c|c|c|c|c|c|c|c|c|c|c|c|c|c|}
    \hline
        U, B & 1 & 2 & 3 & 4 & 5 & 10 & 20 & 30 & 40 & 50 & 60 & 70 & 80 & 90 & 100 & 110 & 120 & 130 & 140 \\ \hline
        I, мкA & 0,16 & 0,27 & 0,34 & 0,43 & 0,59 & 0,91 & 0,96 & 0,995 & 1,02 & 1,04 & 1,053 & 1,07 & 1,08 & 1,09 & 1,102 & 1,120 & 1,136 & 1,145 & 1,155 \\ \hline
\end{tabular}
\caption{\( I_{\text{нак}} = 2{,}6\,\text{A},\quad U_{\text{нак}} = 4{,}2\,\text{B} \)}

\end{table}

\begin{table}[h!]
\centering
\setlength{\tabcolsep}{2.5pt}
\begin{tabular}{|c|c|c|c|c|c|c|c|c|c|c|c|c|c|c|c|c|c|c|c|}
    \hline
        U, B & 1 & 2 & 3 & 4 & 5 & 10 & 20 & 30 & 40 & 50 & 60 & 70 & 80 & 90 & 100 & 110 & 120 & 130 & 140 \\ \hline
        I, мкA & 0,147 & 0,188 & 0,26 & 0,367 & 0,417 & 0,478 & 0,498 & 0,51 & 0,52 & 0,53 & 0,54 & 0,547 & 0,555 & 0,562 & 0,567 & 0,572 & 0,58 & 0,584 & 0,589 \\ \hline
\end{tabular}
\caption{\( I_{\text{нак}} = 2{,}5\,\text{A},\quad U_{\text{нак}} = 4{,}0\,\text{B} \)}

\end{table}

\begin{table}[h!]
\centering
\setlength{\tabcolsep}{2.0pt}
\begin{tabular}{|c|c|c|c|c|c|c|c|c|c|c|c|c|c|c|c|c|c|c|c|}
    \hline
        U, B & 1 & 2 & 3 & 4 & 5 & 10 & 20 & 30 & 40 & 50 & 60 & 70 & 80 & 90 & 100 & 110 & 120 & 130 & 140 \\ \hline
        I, мкA & 0,079 & 0,108 & 0,144 & 0,151 & 0,153 & 0,162 & 0,167 & 0,171 & 0,174 & 0,176 & 0,178 & 0,181 & 0,182 & 0,184 & 0,186 & 0,188 & 0,189 & 0,191 & 0,192 \\ \hline
\end{tabular}
\caption{\( I_{\text{нак}} = 2{,}4\,\text{A},\quad U_{\text{нак}} = 3{,}7\,\text{B} \)}

\end{table}


\newpage
\subsection*{Обработка результатов}

Построены графики зависимостей тока накала от напряжения накала и сопротивления катода от
приложенной мощности:

\begin{figure}[h!]
	\centering
	\includegraphics[width=18cm]{I(U).pdf}
	\caption{График зависимости тока накала от напряжения накала.}
\end{figure}


\begin{figure}[h!]
	\centering
	\includegraphics[width=18cm]{R(P).pdf}
	\caption{График зависимости сопротивления катода от
приложенной мощности.}
\end{figure}
\newpage


Построены графики зависимости температуры катода от тока
накала, определённой различными методами:
\begin{enumerate}
    \item \textbf{Первый график:}
    \[
    T = T_0 + \frac{\left( \dfrac{R_T}{R_0} - 1 \right)}{\alpha},
    \]
    % где 
    % \[
    % T_0 = 273\,\text{К}, \quad
    % R_0 = 0,145\,\text{Ом}, \quad
    % \alpha = 9,29\cdot 10^{-3}\,\text{К}^{-1}, \quad
    % R_T = \frac{U_{\text{нак}}}{I_{\text{нак}}}.
    % \]
    
    По экспериментальным данным тока и напряжения накала были рассчитаны сопротивления катода при различных температурах, что позволило определить температурную зависимость.

    \item \textbf{Второй график:}
    \[
    T = \left( \frac{P_{\text{накала}}}{\varepsilon \sigma S_{\text{изл}}} + T_0^4 \right)^{1/4},
    \]
    % где 
    % \[
    % P_{\text{накала}} = U_{\text{нак}} \cdot I_{\text{нак}}, \quad
    % \varepsilon = 0.25, \quad
    % \sigma = 5,67\cdot 10^{-12}\,\frac{\text{Вт}}{\text{см}^2\cdot\text{К}^4}, \quad
    % S_{\text{изл}} = 0,1884\,\text{см}^2, \quad
    % T_0 = 300\,\text{К}.
    % \]
    
    Для цилиндрического катода площадь излучающей поверхности рассчитывалась по формуле $S_{\text{изл}} = \pi d l$, где $d = 0,15\,\text{мм}$ -- диаметр вольфрамовой проволоки, $l = 45\,\text{мм}$ -- длина активной части катода.

    \item \textbf{Третий график:}
    Используется формула Ричардсона-Дешмана для плотности насыщенного тока термоэлектронной эмиссии:
    \[
    j_{\text{нас}} = A_0 (1 - \tau) T^2 \exp\!\left( -\frac{e\varphi}{k T} \right),
    \]
    % где 
    % \[
    % \varphi = 4,5\,\text{эВ}, \quad
    % A_0 = 120,4\,\frac{\text{А}}{\text{см}^2\cdot\text{К}^2}, \quad
    % j_{\text{нас}} = \frac{I_{\text{нас}}}{S_{\text{к}}}, \quad
    % \tau = 0.07, \quad
    % k = 1,38\cdot 10^{-23}\,\frac{\text{Дж}}{\text{К}}, \quad
    % e = 1,6\cdot 10^{-19}\,\text{Кл}.
    % \]
    
    Значения тока насыщения $I_{\text{нас}}$ определялись по экспериментальным ВАХ диода при различных токах накала (значения токов насыщения соответствовали анодному току при анодном напряжении $U_A = 140\,\text{В}$).
\end{enumerate}

% \begin{table}[H]
% \centering
% \label{tab:current_g}
% \begin{tabular}{|c|c|}\hline
% $I_{\text{нак}}$ [A] & $g$ [мкА/В$^{3/2}$] \\
% \hline
% 2,4 & 5,65 \\\hline
% 2,5 & 9,62 \\\hline
% 2,6 & 32,83 \\\hline
% 2,7 & 53,57 \\\hline
% 2,8 & 59,48 \\\hline
% 2,9 & 64,45 \\\hline
% \end{tabular}
% \caption{Зависимость первеанса $g$ от тока накала $I_{\text{нак}}$}

% \end{table}

Теоретическое значение первеанса для цилиндрического диода рассчитывалось по формуле:
\[
g_{\text{теор}} = \frac{4\pi\varepsilon_0}{9}\sqrt{\frac{2e}{m}}l \cdot f\left(\frac{r_A}{r_K}\right),
\]
% где $\varepsilon_0 = \SI{8,85e-12}{\farad\per\meter}$ -- электрическая постоянная, $l = \SI{35}{\milli\meter}$ -- длина катода, $r_A = \SI{5}{\milli\meter}$ -- радиус анода, $r_K = \SI{0.075}{\milli\meter}$ -- радиус катода, $f(r_A/r_K) \approx 1$ -- функция геометрии системы.

Экспериментальное значение первеанса определялось из начального участка ВАХ диода в координатах $\lg(I_A)$ от $\lg(U_A)$ для режима пространственного заряда. Угловой коэффициент прямой должен быть близок к 3/2, а первеанс рассчитывался по формуле $g = I_A/U_A^{3/2}$.

На основе уравнения Чайлда-Ленгмюра был вычислен удельный заряд электрона:
\[
\frac{e}{m} = \left(\frac{9g}{4\pi\varepsilon_0 l}\right)^2 \cdot \frac{1}{18}.
\]

КПД диода для каждого значения тока накала рассчитывался по формуле:
\[
\eta = \frac{I_A \cdot U_A}{I_{\text{нак}} \cdot U_{\text{нак}}} \cdot 100\%.
\]



\begin{figure}[H]
	\centering
	\includegraphics[width=16.5cm]{T(I).pdf}
	\caption{Графики зависимости температуры катода от тока
накала через изменения
сопротивления, на основании расчётов с
использованием уравнения энергетического
баланса и на основании расчётов с
использованием уравнения Ричардсона–Дэшмана.}
\end{figure}


\subsection*{Определение первеанса диода и удельного заряда электрона}

График зависимости анодного тока от анодного напряжения в координатах $\lg(I_A)$ от $\lg(U_A)$ для различных значений тока накала представлен на рисунке 7. Для начальных участков ВАХ была проведена линейная аппроксимация, соответствующая закону $I_A = g \cdot U_A^{3/2}$.



% \subsection*{Теоретическое значение первеанса}

% Для цилиндрического диода теоретическое значение первеанса рассчитывается по формуле:
% \[
% g_{\text{теор}} = \frac{4\pi\varepsilon_0}{9}\sqrt{\frac{2e}{m}}l \cdot f\left(\frac{r_A}{r_K}\right),
% \]
% где $\varepsilon_0 = 8,85 \cdot 10^{-12}$ Ф/м -- электрическая постоянная, $e/m = 1,76 \cdot 10^{11}$ Кл/кг -- табличное значение удельного заряда электрона, $l = 0,045$ м -- длина катода, $r_A = 5$ мм -- радиус анода, $r_K = 0,075$ мм -- радиус катода, $f(r_A/r_K) \approx 1$ -- функция геометрии системы.

% Подставляя значения, получаем:
% \[
% g_{\text{теор}} = 330~\text{мкА/В}^{3/2}
% \]

\subsection*{Расчёт удельного заряда электрона}

Удельный заряд электрона рассчитывался из уравнения Чайлда-Ленгмюра:
\[
\frac{e}{m} = \left(\frac{9g}{4\pi\varepsilon_0 l}\right)^2 \cdot \frac{2}{3}
\]

Среднее значение удельного заряда электрона, полученное в ходе эксперимента:
\[
\left(\frac{e}{m}\right)_{\text{эксп}} = 0,393 \cdot 10^{11}~\text{Кл/кг}
\]

Табличное значение удельного заряда электрона:
\[
\left(\frac{e}{m}\right)_{\text{табл}} = 1,760 \cdot 10^{11}~\text{Кл/кг}
\]



Среднее значение углового коэффициента аппроксимирующей прямой: 0,540 (теоретическое значение равно 1,5), отклонение составляет 0,960.
\begin{table}[h!]
\centering
\begin{tabular}{|c|c|c|c|c|}
\hline
$I_{\text{нак}}$ [А] & $g_{\text{теор}}$ [мкА/В$^{3/2}$] & $g_{\text{эксп}}$ [мкА/В$^{3/2}$] & $e/m$ [$\cdot 10^{11}$ Кл/кг] & КПД [\%] \\
\hline
2,4 & 5,65 & 30,43 & 0,020 & 0,0003 \\\hline
2,5 & 9,62 & 60,30 & 0,078 & 0,0008 \\\hline
2,6 & 32,83 & 76,03 & 0,125 & 0,0015 \\\hline
2,7 & 53,57 & 158,00 & 0,538 & 0,0038 \\\hline
2,8 & 59,48 & 236,40 & 1,205 & 0,0082 \\\hline
2,9 & 64,45 & 313,10 & 2,115 & 0,0097 \\\hline
\end{tabular}
\caption{Зависимость теоретического и экспериментального первеанса $g$ от тока накала $I_{\text{нак}}$, а также рассчитанные значения удельного заряда электрона $e/m$ и КПД}
\label{tab:combined_g_values}
\end{table}
\subsection*{Анализ результатов}

1. С увеличением тока накала первеанс диода $g$ возрастает, что соответствует увеличению эмиссии электронов с катода и увеличению крутизны ВАХ.

2. Экспериментальные значения первеанса значительно меньше теоретического расчёта. Наибольшее приближение к теоретическому значению наблюдается при максимальном токе накала 2,8 А, где $g_{\text{эксп}}/g_{\text{теор}} = 0,716$.

3. Большое отклонение экспериментального значения $e/m$ от табличного (78\%) связано с рядом факторов:
\begin{itemize}
    \item Наличие объёмного заряда в межэлектродном промежутке
    \item Неидеальная геометрия диода (нарушение соосности анода и катода)
    \item Неоднородность температуры по длине катода
    \item Влияние контактной разности потенциалов между анодом и катодом
    \item Наличие остаточных газов в вакуумной камере (давление $3,3 \cdot 10^{-5}$ торр)
    % \item Ограниченный начальный участок линейного режима ВАХ для аппроксимации
\end{itemize}

4. Значения КПД диода крайне малы, что объясняется значительными потерями энергии на нагрев катода и низкой эффективностью преобразования тепловой энергии в электрическую. С ростом тока накала КПД увеличивается, но остаётся на уровне тысячных долей процента.

\begin{figure}[H]
	\centering
	\includegraphics[width=16.5cm]{lg.pdf}
	\caption{График зависимости анодного тока от анодного
напряжения при различных значениях тока накала в
координатах $\lg(I_A)$ от $\lg(V_A)$.}
\end{figure}

















\subsection*{График зависимости анодного тока от тока накала при различных анодных напряжениях}

На рисунке 8 представлен график зависимости анодного тока от тока накала для пяти различных значений анодного напряжения в логарифмическом масштабе.

Анализ графика показывает нелинейную зависимость анодного тока от тока накала для всех значений анодного напряжения. При небольших токах накала (2,4--2,6 А) наблюдается рост анодного тока с увеличением тока накала, 
что соответствует переходу диода из режима, ограниченного пространственным зарядом, в режим насыщения. При токах накала выше 2,7 А зависимость становится более пологой, что указывает на приближение к режиму насыщения термоэлектронной эмиссии.

Увеличение анодного напряжения приводит к смещению кривой зависимости $\lg(I_A)$ от $I_{\text{нак}}$ вверх,
что соответствует увеличению анодного тока при тех же значениях тока накала. При малых анодных напряжениях (10 В, 30 В) зависимость более крутая, что указывает на сильное влияние пространственного заряда.
При больших анодных напряжениях (90 В, 110 В, 140 В) зависимости становятся более пологими и близкими друг к другу, что указывает на приближение к режиму насыщения, где ток определяется в основном температурой катода и слабо зависит от анодного напряжения.

Максимальное значение анодного тока ($I_A = 8,53$ мкА) достигается при максимальном токе накала (2,9 А) и максимальном анодном напряжении (140 В). При уменьшении анодного напряжения до 10 В анодный ток при том же токе накала составляет всего 2,8 мкА, что в 3 раза меньше. Это качественно согласуется с теоретическими представлениями о работе термоэлектронного диода и подтверждает наличие двух основных режимов работы: режима, ограниченного пространственным зарядом, и режима насыщения.



\begin{figure}[h!]
	\centering
	\includegraphics[width=18cm]{last.pdf}
	\caption{График зависимости анодного тока от тока накала}
\end{figure}



\newpage


\section*{Вывод}
В ходе выполнения лабораторной работы был изготовлен термоэлектронный диод и проведены исследования его характеристик. Экспериментально подтверждены основные закономерности термоэлектронной эмиссии и токопрохождения в вакууме.

Во-первых, была исследована зависимость температуры катода от тока накала тремя методами: по изменению сопротивления, по уравнению энергетического баланса и по уравнению Ричардсона-Дэшмана. Полученные температурные зависимости качественно совпадают, что подтверждает правильность теоретических представлений о процессах нагрева вольфрамового катода.

Во-вторых, была экспериментально определена зависимость анодного тока от анодного напряжения в логарифмических координатах и рассчитан первеанс диода. Среднее экспериментальное значение первеанса составило $g = 112,2$ мкА/В$^{3/2}$ при теоретическом $g_{\text{теор}} = 330$ мкА/В$^{3/2}$. Расчет удельного заряда электрона по формуле Чайлда-Ленгмюра дал значение $(e/m)_{\text{эксп}} = 0,393 \cdot 10^{11}$ Кл/кг при табличном $(e/m)_{\text{табл}} = 1,76 \cdot 10^{11}$ Кл/кг. Относительная погрешность составила 78\%, что объясняется неидеальностью конструкции диода, наличием объёмного заряда и другими экспериментальными факторами.

В-третьих, был исследован режим насыщения диода и определены значения КПД при различных токах накала. КПД оказался крайне мал (до 0,0082\%), что характерно для термоэлектронных приборов с вольфрамовым катодом из-за больших потерь энергии на нагрев.

В-четвертых, построена зависимость анодного тока от тока накала при различных анодных напряжениях. Анализ графиков подтвердил существование двух режимов работы диода: режима, ограниченного пространственным зарядом (при малых токах накала и напряжениях), и режима насыщения (при больших токах накала).

Таким образом, в ходе работы были получены практические навыки изготовления вакуумных приборов, измерения их характеристик и анализа полученных результатов с точки зрения физических законов термоэлектронной эмиссии и движения электронов в вакууме.







% \section*{К рисунку 8 и 9}
% \subsection*{Экспериментальные значения $g$}
% \begin{tabular}{c|c}
% $I_{\text{нак}}$ [A] & $g$ [мкА/В$^{3/2}$] \\
% \hline
% 2.4 & 5,65 \\
% 2.5 & 9,62 \\
% 2.6 & 32,83 \\
% 2.7 & 53,57 \\
% 2.8 & 59,58 \\
% 2.9 & 64,45 \\
% \end{tabular}

% \vspace{1em}


% \subsection*{Теоретическое значение $g$}
% \[
% g_{\text{теор}} = 104.72~\text{мкА/В}^{3/2}
% \]


% \vspace{1em}


% \subsection*{Сравнение экспериментальных и теоретических значений}
% \begin{tabular}{c|c|c}
% $I_{\text{нак}}$ [A] & $g_{\text{эксп}}$ [мкА/В$^{3/2}$] & Отношение $g_{\text{эксп}} / g_{\text{теор}}$ \\
% \hline
% 2.4 & 5,65 & 0.165 \\
% 2.5 & 9,62 & 0.174 \\
% 2.6 & 32,83 & 0.347 \\
% 2.7 & 53,58 & 0.622 \\
% 2.8 & 59,48 & 0.643 \\
% 2.9 & 64,45 & 0.691 \\
% \end{tabular}

% \vspace{1em}


% \subsection*{Анализ зависимости $g$ от тока накала}
% С увеличением тока накала $I_{\text{нак}}$:

% Увеличивается эмиссия электронов с катода\par
% Увеличивается коэффициент $g$\par
% Это соответствует увеличению крутизны ВАХ\par

% \section*{К рисунку 9}

% \section*{ВЫЧИСЛЕНИЕ ОТНОШЕНИЯ $e/m$}

% \begin{align*}
% d &= \SI{1.50e-4}{\meter} \\
% S &= \SI{1.88e-5}{\meter\squared} \\
% g &= \SI{1.01e-4}{\ampere\per\volt\tothe{3/2}} \\
% \varepsilon_0 &= \SI{8.85e-12}{\farad\per\meter}
% \end{align*}

% \subsection*{Результат:}
% \begin{align*}
% e/m &= \SI{4.73e10}{\text{КЛ}\per\kilogram} \\
% e/m &= \SI{4.73e10}{\text{КЛ}\per\kilogram} 
% \end{align*}

% \subsection*{Сравнение с табличным значением:}
% \begin{align*}
% \text{Табличное значение: } & \SI{1.76e11}{\text{КЛ}\per\kilogram} \\
% \text{Отклонение: } & 73.1\%
% \end{align*}


% \section*{К рисунку 10}
% \section*{МАКСИМАЛЬНЫЕ ЗНАЧЕНИЯ КПД ДЛЯ КАЖДОГО ТОКА НАКАЛА}

% \begin{table}[h]
% \centering
% \begin{tabular}{ccccccc}
% \toprule
% $I_{\text{нак}}$ [A] & $U_{\text{нак}}$ [B] & $P_{\text{нак}}$ [Br] & $U_a$ [B] & $I_a$ [мкА] & $P_a$ [мкВт] & КПД [\%] \\
% \midrule
% 2.4 & 3.41 & 8.184 & 30 & 55 & 1650.0 & 0.020 \\
% 2.5 & 3.73 & 9.325 & 30 & 92 & 2760.0 & 0.030 \\
% 2.6 & 4.12 & 10.712 & 30 & 378 & 11340.0 & 0.106 \\
% 2.6 & 4.12 & 10.712 & 30 & 378 & 11340.0 & 0.106 \\
% 2.7 & 4.34 & 11.718 & 130 & 1860 & 241800.0 & 2.022 \\
% 2.8 & 4.66 & 13.048 & 140 & 6210 & 869400.0 & 6.247 \\
% 2.9 & 5.09 & 14.761 & 130 & 15000 & 1950000.0 & 11.669 \\
% \bottomrule
% \end{tabular}
% \end{table}

% \section*{АНАЛИЗ РЕЗУЛЬТАТОВ}

% \begin{enumerate}
% \item КПД вакуумного диода очень низкий (менее 0,1\%)
% \item С увеличением тока накала максимальный КПД увеличивается
% \item Основная потребляемая мощность -- мощность накала
% \item Вакуумные диоды неэффективны как преобразователи энергии
% \item Основное применение вакуумных диодов -- выпрямление и детектирование, а не энергопреобразование
% \end{enumerate}


% \section*{К рисунку 11}
% \section*{ЗАВИСИМОСТЬ АНОДНОГО ТОКА ОТ ТОКА НАКАЛА}

% \begin{table}[h]
% \centering
% \small
% \begin{tabular}{c|cccccc}
% \toprule
% \multirow{2}{*}{$U_a$ [B]} & \multicolumn{6}{c}{$I_a$ [мкА] (lg($I_a$))} \\
% \cmidrule{2-7}
%  & 2.4 A & 2.5 A & 2.6 A & 2.7 A & 2.8 A & 2.9 A \\
% \midrule
% 1   & 17 (1.23)  & 23 (1.36)  & 43 (1.63)  & 70 (1.85)  & 75 (1.88)  & 90 (1.95)  \\
% 5   & 48 (1.68)  & 83 (1.92)  & 334 (2.52) & 595 (2.77) & 660 (2.82) & 720 (2.86) \\
% 10  & 51 (1.71)  & 86 (1.93)  & 350 (2.54) & 1273 (3.10) & 1970 (3.29) & 2150 (3.33) \\
% 20  & 53 (1.72)  & 90 (1.95)  & 367 (2.56) & 1356 (3.13) & 3010 (3.59) & 6170 (3.79) \\
% 30  & 55 (1.74)  & 92 (1.96)  & 378 (2.58) & 1402 (3.15) & - & 9010 (3.95) \\
% 50  & - & - & - & 1617 (3.21) & 4740 (3.68) & 14400 (4.16) \\
% 100 & - & - & - & 1780 (3.25) & 5360 (3.73) & 14900 (4.17) \\
% \bottomrule
% \end{tabular}
% \end{table}

% \section*{АНАЛИЗ РЕЗУЛЬТАТОВ}

% \begin{enumerate}
% \item С увеличением тока накала анодный ток увеличивается
% \item Зависимость $\lg(I_a)$ от $I_a$ близка к линейной
% \item Чем выше анодное напряжение, тем больше анодный ток при том же токе накала
% \item При малых $U_a$ (1-5 В) зависимость более слабая
% \item При больших $U_a$ (20-100 В) наблюдается более сильная зависимость
% \item Это объясняется увеличением термоэлектронной эмиссии с ростом температуры катода
% \end{enumerate}

% \section*{Вывод}

% Изучены явления термоэлектронной эмиссии и процессов токопрохождения в вакууме, изготовлен вакуумный диод 
% и иследованы некоторые его характеристики. Выяснена зависимость и построен график тока накала от напряжения (зависимость напоминает график корня при небольших напряжениях и токах). Также выяснено, как
% зависит напряжение катода от приложенной к нему мощности, и по имеющимся данным построен график. Была выяснена
% зависимость температуры накала от тока накала с помощью трёх методов: метод определения по сопротивлению показал наибольшее отклонение от остальных двух. Возможно, это связано с неточностью определения геометрических параметров диода, а также с не совсем корректно рабочим источником питания и напряжения.
% Далее было посчитано теоретическое и экспериментальное значение первеанса диода g, была выяснена его зависимость: чем больше ток накакла, тем меньше отклонение от теоретического значения. Возможно, если бы мы сделали еще пару значений тока накала, то удалось бы минимизировать эту разницу.
% Было вычислено отношение e/m: оно отличилось от табличного более чем в 3 раза, это связано с несовершенством снятых нами характеристик.
% Были определены значения КПД для кажого тока накала и зависимость анодного тока накала. Их результаты описаны выше. 

\end{document}