\documentclass[a4paper,12pt]{article} % тип документа

% Поля страниц
\usepackage[left=2.5cm,right=2.5cm,
    top=2cm,bottom=2cm,bindingoffset=0cm]{geometry}
    
%Пакет дял таблиц   
\usepackage{multirow} 
    
%Отступ после заголовка    
\usepackage{indentfirst}


% Рисунки
\usepackage{floatrow,graphicx,calc}
\usepackage{wrapfig}

% Создаёем новый разделитель
\DeclareFloatSeparators{mysep}{\hspace{1cm}}

% Ссылки?
\usepackage{hyperref}
\usepackage[rgb]{xcolor}
\hypersetup{				% Гиперссылки
    colorlinks=true,       	% false: ссылки в рамках
	urlcolor=blue          % на URL
}


%  Русский язык
\usepackage[T2A]{fontenc}			% кодировка
\usepackage[utf8]{inputenc}			% кодировка исходного текста
\usepackage[english,russian]{babel}	% локализация и переносы


% Математика
\usepackage{amsmath,amsfonts,amssymb,amsthm,mathtools, mathrsfs}


% Что-то 
\usepackage{wasysym}


\begin{document}
\title{\begin{center}Лабораторная работа № 3.3.4\end{center}
Эффект Холла в полупроводниках}
\author{Струков О. И. \\ Б04-404}
\date{}

\thispagestyle{empty}
\pagenumbering{gobble}
\maketitle
\newpage
\pagenumbering{arabic}
\textbf{Цель работы:} измерение подвижности и концентрации носителей заряда в полупроводниках.

\textbf{В работе используются:} электромагнит с источником питания, миллиамперметр, милливебметр, реостат, цифровой вольтметр, источник питания, образец полупроводника.


\section*{Теоретическая часть}
	Эффект Холла - явление возникновения поперечной разности потенциалов при помещении проводника с постоянным током в магнитное поле.


% Суть эффекта Холла заключается в возникновении поперечной разности потенциалов в проводнике с током, помещённом в магнитное поле. 

Например, рассмотрим однородную металлическую пластину, по которой течёт ток $I$ вдоль оси $x$ (рис. 1). При помещении пластины в магнитное поле, направленное по оси $y$, в результате данного эффекта между гранями А и Б возникает разность потенциалов.

    \begin{figure}[H]
		\center{\includegraphics[scale=0.7]{Holl1.PNG}}
		\caption{Образец с током в магнитном поле}
		\label{Рис. 1}
	\end{figure}

% \subsection*{Механизм эффекта}
На электрон, движущийся со средней скоростью $\langle \vec{v} \rangle$ в электромагнитном поле, действует сила Лоренца:
\[
\vec{F}_{л} = -e\vec{E} - e \langle \vec{v} \rangle \times \vec{B},
\]
где $e$ — заряд электрона, $\vec{E}$ — напряжённость электрического поля, $\vec{B}$ — индукция магнитного поля.

В проекции на ось $z$:
\[
F_{B} = e | \langle {v_{x}} \rangle | B.
\]
Под действием этой силы электроны отклоняются к грани Б, создавая на ней избыточный отрицательный заряд. На грани А накапливаются нескомпенсированные положительные заряды, что приводит к возникновению электрического поля $E_{z}$, направленного от А к Б. В установившемся режиме:
\[
F_{E} = eE_{z} = F_{B} \quad \Rightarrow \quad E_{z} = | \langle {v_{x}} \rangle | B.
\]
Разность потенциалов между гранями:
\[
U_{\text{AБ}} = E_{z}l = | \langle {v_{x}} \rangle | Bl.
\]

% \subsection*{Связь с током}
Силу тока в данном случае можно выразить как
\[
I = ne| \langle {v_{x}} \rangle |la,
\]
где $n$ — концентрация носителей, $a$ — толщина пластины. 

Тогда ЭДС Холла:
\[
\mathscr{E}_{X} = U_{\text{AБ}} = \frac{IB}{nea} = R_{X}\frac{IB}{a},
\]
где $R_{X} = \dfrac{1}{ne}$ — постоянная Холла.

\subsection*{Случай полупроводников}
В полупроводниках носителями заряда кроме электронов могут быть дырки, поэтому в этом случае постоянная Холла выражается следующим образом:
\[
R_{X} = \frac{nb^{2}_{e} - pb^{2}_{p}}{e(nb_{e} + pb_{p})^{2}},
\]
где $n$, $p$ — концентрации электронов и дырок, $b_{e}$, $b_{p}$ — их подвижности.

% \begin{center}
%     \includegraphics[width=0.5\linewidth]{Holl1.png}
%     \textit{Рис. 1. Образец с током в магнитном поле}
% \end{center}
	% \begin{itemize}
	% 	\item
	% 		ЭДС Холла:
	% 		\begin{equation}
	% 			\label{E}
	% 			\mathscr{E_\text{х}} = U_{34} - U_0;
	% 		\end{equation}
	% 	\item
			% Постоянная Холла:
			% \begin{equation}
			% 	\label{R}
			% 	R_\text{х} = -\frac{\mathscr{E_\text{х}}}{B} \cdot \frac{a}{I};
			% \end{equation}
	% 	\item
	% 		Концентрация носителей тока в образце:
	% 		\begin{equation}
	% 			\label{n}
	% 			n = \frac{1}{R_\text{х} e}
	% 		\end{equation}
	% 	\item
	% 		Удельная проводимость материала образца:
	% 		\begin{equation}
	% 			\label{sigma}
	% 			\sigma = \frac{I L_{35}}{U_{35}al}
	% 		\end{equation}
	% 	\item
	% 		Подвижность носителей тока:
	% 		\begin{equation}
	% 			\label{b}
	% 			b = \frac{\sigma}{en}
	% 		\end{equation}
		
	% \end{itemize}
\section*{Экспериментальная установка}
	Электрическая установка для измерения ЭДС Холла представлена на рисунке 2.
	\begin{figure}[H]
		\center{\includegraphics[scale=0.9]{ustanovka.pdf}}
		\caption{Схема установки для исследования эффекта Холла в полупроводниках.}
		\label{Рис. 2}
	\end{figure}

В зазоре электромагнита создаётся постоянное магнитное поле, регулируемое источником питания (рис. 1а). Направление тока в обмотках изменяется через разъём $K_1$.

Изучаемый образец (рис. 1б) подключается к батарее. Ток через него регулируется реостатом $R$ и измеряется миллиамперметром $A_2$.

При помещении образца в магнитное поле между контактами 3 и 4 возникает разность потенциалов $U_{34}$, измеряемая цифровым вольтметром.

\subsection*{Возможная погрешность}
Из-за неточности подпайки контакты 3 и 4 могут не лежать на эквипотенциали, что приводит к дополнительному омическому падению напряжения. Для исключения этой погрешности применяются два метода:

1. Измерение при противоположных направлениях магнитного поля:
\[
\mathscr{E}_{X} = \frac{U_{34}(+B) - U_{34}(-B)}{2}
\]

2. Измерение напряжения $U_0$ между контактами 3 и 4 в отсутствие поля:
\[
\mathscr{E}_{X} = U_{34} - U_0
\]

\subsection*{Результат измерений}
По знаку $\mathscr{E}_{X}$ определяется тип проводимости (электронный или дырочный). Проводимость материала рассчитывается по формуле:
\[
\sigma = \frac{I \cdot L_{35}}{U_{35} \cdot a \cdot l}
\]
где $I$ — ток через образец, $U_{35}$ — напряжение между контактами 3 и 5 в отсутствие поля, $L_{35}$ — расстояние между контактами, $a$ и $l$ — геометрические параметры образца.


%\section{Модель эксперимента}
\newpage
\section*{Ход работы}
	\floatsetup[table]{}	
	\begin{table}[H]
		\caption{Параметры установки и исследуемого образца.}
		% \label{table:const}
		\begin{tabular}{|c|c|c|c|}
			\hline
			\begin{tabular}[c]{@{}c@{}}Расстояние между\\ контактами 3 и 5\\ $L_{35}$, мм\end{tabular} & \begin{tabular}[c]{@{}c@{}}Толщина образца\\ $a$, мм\end{tabular} & \begin{tabular}[c]{@{}c@{}}Ширина образца\\ $l$, мм\end{tabular} & \begin{tabular}[c]{@{}c@{}}Постоянная катушки\\ $SN$, см$^2\cdot\,$вит.\end{tabular} \\ \hline
				15                                                                                         & 2                                                               & 8                                                                &                                                                 75                   \\ \hline
			\end{tabular}
	\end{table}
\begin{enumerate}
    \item С помощью милливебеметра была исследована зависимость индукции В магнитного поля в зазоре электромагнита от силы тока через обмотку электромагнита, изображённая на графике ниже. 
    Можно предположить, что в выбранном диапазоне значений I график можно аппроксимировать прямой $B = kI + B_0$. С помощью МНК были определены значения коэффициентов: 
    \[k = (0,917\pm0,020) ~\text{Тл/А}, \qquad B_0 = (0,023 \pm 0,013) ~\text{Тл}\]
    \begin{figure}[H]
		\center{\includegraphics[scale=0.9]{B(I).pdf}}
		\caption{Зависимость $B = f(I)$.}
		\label{Рис. 3}
	\end{figure}

    \item Для восьми значений силы тока, протекающего через образец, от 0,3 до 1 мА была измерена зависимость ЭДС Холла от индукции магнитного поля. Графики полученных зависимостей изображены на рисунке 4:
        \begin{figure}[H]
		\center{\includegraphics[scale=0.6]{E(B).pdf}}
		\caption{Зависимость $\mathscr{E}_{\text{Х}} = f(B)$.}
		\label{Рис. 4}
	\end{figure}
	\item Для каждой из полученных зависимостей с помощью МНК был определён коэффициент наклона K(I):
	\begin{table}[!ht]
    \centering
    \begin{tabular}{|c|c|}
    \hline
		I, мА & K(I), мВ/Тл \\ \hline
        (0,30 $\pm$ 0,01) & (3,39 $\pm$ 0,3) \\ \hline
        (0,40 $\pm$ 0,01) & (4,48 $\pm$ 0,3)  \\ \hline
        (0,50 $\pm$ 0,01) & (5,53 $\pm$ 0,3)  \\ \hline
        (0,60 $\pm$ 0,01) & (6,69 $\pm$ 0,4)  \\ \hline
        (0,70 $\pm$ 0,01) & (7,63 $\pm$ 0,4)  \\ \hline
        (0,80 $\pm$ 0,01) & (8,65 $\pm$ 0,4)  \\ \hline
        (0,90 $\pm$ 0,01) & (9,53 $\pm$ 0,5)  \\ \hline
        (1,00 $\pm$ 0,01) & (10,50 $\pm$ 0,5)  \\ \hline
    \end{tabular}
\end{table}
\item По полученным точкам проведена наилучшая прямая, получен график зависимости $K = f(I)$
    \begin{figure}[H]
		\center{\includegraphics[scale=0.9]{K(I).pdf}}
		\caption{Зависимость $K = f(I)$.}
		\label{Рис. 4}
	\end{figure}
\item С помощью МНК найден угловой коэффициент полученной прямой $\kappa = (10,2 \pm 0,6)~\dfrac{\text{В}}{\text{А$\cdot$Тл}}$.
\[R_X = \kappa a\]
Таким образом, $R_X = (0,020 \pm 0,001)~\dfrac{\text{В$\cdot$м}}{\text{А$\cdot$Тл}}$
\item Отсюда можно найти концентрацию носителей тока в образце: $n = (3,1\pm 0,2) \cdot 10^{20}$ м$^{-3}$.
\item Определено напряжение между контактами 3 и 5 в отсутствие поля: $U_{35} = (0,172 \pm 0,001)$ В. По формуле из введения найдена проводимость материала: $\sigma = (5,45 \pm 0,04)~\dfrac{1}{\text{Ом}\cdot\text{м}}$. 
\item Используя полученные значения концентрации n и проводимости $\sigma$ с помощью формулы $b = \dfrac{\sigma}{ne}$ вычислена подвижность b носителей тока в общепринятых внесистемных единицах: $b = (1050 \pm 60)~\dfrac{\text{см}^2}{\text{В}\cdot\text{с}}$




\end{enumerate}

\section*{Вывод}
Итоги работы сведены в конечную таблицу:

\begin{tabular}{|c|c|c|c|c|}
	\hline 
	$ R_X, $ & \multirow{2}{*}{Знак носителей} & $ n, $ &$  \sigma, $ & $ b, $ \\ 

	м$^3/$Кл &  & $ 10^{20}, $ м$^{-3} $ &  Ом$^{-1} \cdot$ м$^{-1}  $&  см$^2/ $В $\cdot$ с  \\ 
	\hline 
$ 	0,020 \pm 0,001  $ & $ +  $ & $ 3,1\pm 0,2  $ & $ 5,45 \pm 0,04 $ & $1050 \pm 60 $ \\ 
	\hline 
\end{tabular} 

В результате выполнения работы были определены постоянная Холла образца, концентрация носителей в нём, удельная проводимость и подвижность носителей. Сами носители имеют положительный заряд, поэтому их тип является дырочным.
Поскольку нам не было известно, из какого материала изготовлен образец, полученные данные нельзя сравнить с какими-либо конкретными табличными, однако в общем случае видно, что все они, за исключением почтоянной Холла по порядку совпадают с имеющимися данными в справочнике.



















% \newpage
% 	%В таблице~\ref{table:const} приведены константы, используемые в лабораторной работе. 	
% 	\floatsetup[table]{capposition=top}	
% 	\begin{table}[H]
% 		\caption{Параметры установки и исследуемого образца.}
% 		\label{table:const}
% 		\begin{tabular}{|c|c|c|c|}
% 			\hline
% 			\begin{tabular}[c]{@{}c@{}}Расстояние между\\ контактами 3 и 5\\ $L_{35}$, мм\end{tabular} & \begin{tabular}[c]{@{}c@{}}Толщина образца\\ $a$, мм\end{tabular} & \begin{tabular}[c]{@{}c@{}}Ширина образца\\ $l$, мм\end{tabular} & \begin{tabular}[c]{@{}c@{}}Постоянная катушки\\ $SN$, см$^2\cdot\,$вит.\end{tabular} \\ \hline
% 				6                                                                                          & 2,2                                                               & 7                                                                &                                                                 72                   \\ \hline
% 			\end{tabular}
% 	\end{table}
		
		
% 	%В таблице~\ref{table:error} приведены значения и случайные ошибки измерения величин, определяемых в ходе эксперимента.
% 	\floatsetup[table]{capposition=top}	
% 	\begin{table}[H]
% 		\caption{Некоторые измеряемые величины и их погрешность.}
% 		\label{table:error}
% 		\begin{tabular}{|c|c|c|c|c|}
% 			\hline
% 			& $\Phi$, мВб & $I_M$, А & $U_{34}$, мкВ & $I$, мА \\ \hline
% 			Величина          & 1,0         & 1,00     & 50          & 0,5     \\ \hline
% 			Погрешность       & 0,1         & 0,01     & 1           & 0,005   \\ \hline
% 			$\varepsilon$, \% & 1           & 1        & 2           & 1       \\ \hline
% 		\end{tabular}
% 	\end{table}


% 	\floatsetup[table]{capposition=top}	
% 	\begin{table}[H]
% 		\caption{Калибровка электромагнита.}
% 		\label{table:calibration}
% 		\begin{tabular}{|c|c|c|c|c|c|c|c|c|c|c|}
% 			\hline
% 			№           & 1    & 2    & 3    & 4    & 5    & 6    & 7    & 9    & 9    & 10   \\ \hline
% 			$I_M$, A    & 0,61 & 0,71 & 0,83 & 0,91 & 1,06 & 1,11 & 1,15 & 1,22 & 1,38 & 1,45 \\ \hline
% 			$\Phi$, мВб & 4,2  & 4,7  & 5,4  & 5,8  & 6,6  & 6,7  & 6,9  & 7,1  & 7,6  & 7,7  \\ \hline
% 			$B$, мТл    & 583  & 653  & 750  & 806  & 917  & 931  & 958  & 986  & 1056 & 1069 \\ \hline
% 		\end{tabular}
% 	\end{table}


% 	\floatsetup[table]{capposition=top}	
% 	\begin{table}[H]
% 		\caption{Зависимость $U_{34}$ от $I_M$ при фиксированном $I$.}
% 		%\footnote{Измерения проводились при калиброванных значениях магнитной индукции катушки}
% 		\label{table:U_34}
% 		\begin{tabular}{|c|c|c|c|c|c|c|c|c|c|c|}
% 			\hline
% 			$I$, мА    & 0,24 & 0,26 & 0,28 & 0,30 & 0,35 & 0,45 & 0,65 & 0,85 & 1,00 & 1,00 \\ \hline
% 			$U_0$, мкВ & -49  & -56  & -61  & -65  & -77  & -100 & -140 & -183 & -220 & -220 \\ \hline
% 			№          & \multicolumn{10}{c|}{$U_{34}$, мкВ}                                 \\ \hline
% 			1          & -2   & -2   & -5   & -4   & -5   & -7   & -5   & -5   & -13  & -450 \\ \hline
% 			2          & 4    & 4    & 4    & 3    & 4    & 4    & 10   & 15   & 10   & -480 \\ \hline
% 			3          & 11   & 10   & 10   & 11   & 13   & 17   & 28   & 35   & 35   & -508 \\ \hline
% 			4          & 16   & 15   & 17   & 17   & 21   & 26   & 41   & 55   & 58   & -533 \\ \hline
% 			5          & 23   & 24   & 25   & 25   & 30   & 38   & 58   & 76   & 85   & -560 \\ \hline
% 			6          & 25   & 25   & 27   & 28   & 33   & 44   & 65   & 84   & 94   & -570 \\ \hline
% 			7          & 27   & 27   & 29   & 30   & 35   & 46   & 70   & 90   & 100  & -577 \\ \hline
% 			8          & 28   & 29   & 30   & 33   & 39   & 50   & 74   & 97   & 110  & -587 \\ \hline
% 			9          & 32   & 32   & 35   & 37   & 44   & 56   & 84   & 108  & 125  & -603 \\ \hline
% 			10         & 34   & 35   & 38   & 40   & 47   & 60   & 89   & 115  & 132  & -613 \\ \hline
% 		\end{tabular}
% 	\end{table}

% Дополнительно при силе тока в 1 мА, протекающем через образец, измерим $U_{35} = -2,531$ мВ.
		
% \newpage
% \section {Обработка результатов}
% 	Для калибровки электромагнита необходимо экстраполировать график зависимости $B = f(I)$ (рис.~\ref{ris:B=f(I)}). Не трудно убедиться, что с большой точностью зависимость является линейной в данном диапазоне токов. С меньшей достоверностью зависимость можно описать многочленом третей степени, на который хорошо ложатся экспериментальные точки. Однако в связи прецизионностью источника питания нам достаточно знать конечный набор значений магнитного поля $B$ и проводить измерения $U_{34}$ только на них.
% 	% \begin{figure}[H]
% 	% 	\center{\includegraphics[scale=0.4]{B=f(I).pdf}}
% 	% 	\caption{Зависимость $B = f(I)$.}
% 	% 	\label{ris:B=f(I)}
% 	% \end{figure}


% 	Построим серию прямых $\mathscr{E}_\text{х} =\mathscr{E}_\text{х} (B)$ (рис.~\ref{ris:Ex}). Отметим, что $\sigma_{\mathscr{E}_\text{х}} = 2 \sigma_{U_{34}} = 2$~мкВ, а $\sigma_B = \sigma_\Phi / SN = 14$ мТл.
% 	% \begin{figure}[H]
% 	% 	\center{\includegraphics[scale=0.5]{Ex.pdf}}
% 	% 	\caption{Серия зависимостей $\mathscr{E}_\text{х}$ от $B$ при различных $I$.}
% 	% 	\label{ris:Ex}
% 	% \end{figure}
% \newpage
% 		\begin{table}[H]
% 		\caption{$k = \Delta \mathscr{E}_\text{х} / \Delta B $.}
% 		\label{table:k}
% 		\begin{tabular}{|c|c|c|c|c|c|c|c|c|c|c|}
% 			\hline
% 			$I$, мА            & 0,24 & 0,26 & 0,28 & 0,3,0 & 0,35 & 0,45 & 0,65 & 0,85 & 1,00 & 1,00 \\ \hline
% 			$k$, мкВ/Тл        & 72,7 & 74,9 & 84   & 88,6  & 104  & 136  & 191  & 244  & 296  & -324 \\ \hline
% 			$\sigma_k$, мкВ/Тл & 1,6  & 1,7  & 2    & 1,7   & 2    & 3    & 4    & 6    & 5    & 7    \\ \hline
% 		\end{tabular}
% 	\end{table}


% 	По полученным данным построим график зависимости $k$ от $I$ и проанализируем его.
% 	% \begin{figure}[H]
% 	% 	\center{\includegraphics[scale=0.5]{k=k(I).pdf}}
% 	% 	\caption{Зависимость $k$ от $I$.}
% 	% 	\label{ris:k=k(I)}
% 	% \end{figure}


% 	Методом наименьших квадратов определяем, что $k/I = (295\,\pm\,3)$ $\frac{\text{мВт}}{\text{Тл}\cdot\text{А}}$, откуда согласно формуле~(\ref{R}) $R_\text{х} = (649\,\pm\,7)$ см$^3$/Кл.
	
% 	Рассчитаем концентрацию носителей тока в образце по формуле~(\ref{n}): $n = (962\,\pm~\,1)~\cdot~10^{19}\,\text{м}^3$, удельную проводимость по формуле (\ref{sigma}): $\sigma = (153,9\,\pm\, 0,8)$ (Ом$\cdot$м)$^{-1}$.
	 
% 	Вычислим подвижность носителей тока в материале образца по формуле~(\ref{b}):
% 	 \[
% 	\boxed{b = (1000\,\pm\, 10) \frac{\text{см}^2}{\text{В}\cdot\text{с}}}
% 	\]
% \section {Обсуждение результатов}
% 	В ходе данной лабораторной работы мы исследовали эффект Холла в полупроводнике, а именно в Германии. Нам удалось определить постоянную Холла, которая в данных диапазонах токов и значений магнитной индукции магнитного поля оказалась постоянной и равной $R_\text{х} = 649$ см$^3$/Кл с относительной ошибкой 1\%. Так же вычислили концентрацию носителей тока в образце при том предположении, что количество носителей одного типа намного больше другого типа: $n = 962\, \cdot \, 10^{19}\,\text{м}^3$. Зная направление тока в проводнике, полярность вольтметра, направление тока в катушках, можно определить тип проводимости. В нашей работе тип проводимости в Германии оказался дырочным.
	
% 	Более того, мы вычислили подвижность дырок в исследуемом Германии: $b = 1000\, \frac{\text{см}^2}{\text{В}\cdot\text{с}}$ с точностью в 1\%. Но наш результат отличается от табличного для носителей в области собственной проводимости $b_0 = 1800\, \frac{\text{см}^2}{\text{В}\cdot\text{с}}$ (при температуре $T$ = 293 К), по чему можно сделать вывод, что наш образец является не чистым, а с примесями. Хотелось бы отметить, что дополнительная ошибка измерений может быть связана с сильной зависимостью концентрации основных носителей токов от температуры. Действительно, для отрыва электрона от атома полупроводника и превращения его в электрон проводимости необходимо сообщить ему некоторое колличество энергии. Естественно, что такая энергия поставляется тепловыми колебаниями атомов решетки. В нашей работе температура температура образца была как минимум комнатной ($T = 298$ К) и как максимум могла повыситься вследствие протекающего через образец постоянного тока.
	
	
% \section{Выводы}
% 	\begin{enumerate}
% 		\item 
% 			Вычислили постоянную Холла: $R_\text{х} = (649\,\pm\,7)$ см$^3$/Кл;
% 		\item 
% 			Определили концентрацию носителей тока в образце: $n = (962\,\pm~\,1)~\cdot~10^{19}\,\text{м}^3$;
% 		\item
% 			Рассчитали удельную проводимость: $\sigma = (153,9\,\pm\, 0,8)$ (Ом$\cdot$м)$^{-1}$;
% 		\item 	
% 			Германий является легированным образцом с подвижностью $b = (1000\,\pm\, 10) \frac{\text{см}^2}{\text{В}\cdot\text{с}}$;
% 		\item
% 			Тип носителей в исследуемом материале - дырочный.
% 	\end{enumerate}








\end{document}