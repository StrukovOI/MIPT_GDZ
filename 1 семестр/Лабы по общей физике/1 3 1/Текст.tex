\documentclass[a4paper, 10pt]{article}%тип документа

%отступы
\usepackage[left=2cm,right=2cm,top=2cm,bottom=3cm,bindingoffset=0cm]{geometry}

%Русский язык
\usepackage[T2A]{fontenc} %кодировка
\usepackage[utf8]{inputenc} %кодировка исходного кода
\usepackage[english,russian]{babel} %локализация и переносы

%Вставка картинок
\usepackage{graphicx}
\graphicspath{{pictures/}}
\DeclareGraphicsExtensions{.pdf,.png,.jpg}

%Графики
\usepackage{pgfplots}
\pgfplotsset{compat=1.9}

%Математика
\usepackage{amsmath, amsfonts, amssymb, amsthm, mathtools}

%Заголовок
\date{}
\author{Струков Олег \\
Б04-404}
\title{Работа 1.3.1 \\
Определение модуля Юнга на основе исследования деформаций растяжения и изгиба}
\begin{document}
\maketitle
\newpage
\textbf{Цель работы:} экспериментально получить зависимость между напряжением и деформацией (закон Гука) для двух простейших напряженных состояний упругих тел: одноосного растяжения и чистого изгиба; по результатам измерений вычислить модуль Юнга. \\
\textbf{В работе используется:} прибор лермантова, проволока из исследуемого материала, зрительная трубка со шкалой, набор грузов, микрометр, рулетка; во второй части - стойка для изгибания балки, индикатор для измерения величины прогиба, набор исследуемых стержней, грузы, линейка, штангенциркуль.
\center{\textbf{Определение модуля Юнга по измерениям растяжения проволоки (рис.1)}}
\begin{figure}[h]
\center{\includegraphics{131_1.jpg}}
\end{figure}


\begin{enumerate}

\item Диаметр проволоки известен:
$d = (0,51 \pm 0,01) \text{мм}$.

Находим площадь поперечного сечения проволоки:
\[S =\dfrac{ \pi (\overline{d})^2}{4} = 0,204 \text{ мм}^2\]
\[\sigma_S = S\sqrt{2\left( \dfrac{\sigma_d}{d}\right) ^2} = 0,006 \text{ мм}^2\]
\[S = (0,204\pm0,006) \text{ мм}^2\]
\item Измеряем длину проволоки $l_0 = (174 \pm 1)  \text{ см}$

\item Направляем зрительную трубу на зеркальце так, чтобы чётко было видно шкалу, тогда свет от неё будет падать примерно перпендикулярно шкале на зеркало, поэтому удлинение проволоки можно выразить так:
\[\Delta l =\dfrac{nr}{2h}\]
\[ \sigma_{\Delta l} = \Delta l\sqrt{\left( \dfrac{\sigma_{n}}{n}\right)^2 + \left(\dfrac{\sigma_r}{r}\right)^2+\left(\dfrac{\sigma_h}{h}\right)^2} \]
где $r = (20 \pm 0,1)\text{ см}$ - длина рычага, разница показаний шкалы - $n$, расстояние от шкалы до проволоки - $h = (141\pm1)\text{ см}$.

Значения всех найденных величин и погрешностей погрешностей указаны в таблице 1

\item Учитывая, что $\sigma_{\text{предел}} = 900 \text{ Н}/\text{мм}^2$ получаем, что предельный вес, который можно повесить, чтобы не выйти за пределы $P_{\text{предел}} = 0,3 \sigma_{\text{предел}} S \approx 55,08 H$. 
\item Снимем зависимость удлинения проволоки от массы грузов при увеличении и уменьшении нагрузки 2 раза (таблица 2). 
\item По полученным результатам строим график зависимости  $m(\Delta l)$.
В недеформированном состоянии проволока, как правило, изогнута, и при малых нагрузках её "удлинение" определяется не растяжением, а выпрямлением, поэтому начальный участок графика при обработке следует исключить. Вычислим коэффициент наклона получившийся прямой для каждой серии измерений, начиная с четвёртой точки, с помощью МНК (таблица 3), после чего найдём его среднее значение и погрешность. 
\[k=\dfrac{\langle m\Delta l\rangle-\langle m\rangle \langle \Delta l\rangle}{\langle \Delta l^2\rangle - \langle \Delta l\rangle^2}\]
\[\sigma_k = k\sqrt{\left( \dfrac{\sigma_{\Delta m}}{\Delta m} \right)^2 + \left( \dfrac{\sigma_{\Delta l}}{\Delta l} \right)^2 }\]

$\sigma_k = 0,53\text{ кг/м}$

$k = (15,09 \pm 0,53) \text{ кг/м}$

Затем с помощью данного коэффициента определим модуль Юнга и его погрешность по формулам
\[E = k\dfrac{lg}{S}\]
\[\sigma_E = E\sqrt{\left( \dfrac{\sigma_{k}}{k} \right)^2 + \left( \dfrac{\sigma_{S}}{S} \right)^2 + \left( \dfrac{\sigma_{l_0}}{l_0} \right)^2 }\]

$\sigma_E = 5,8\text{ ГПа}$

$E = (126,2 \pm 5,8) \text{ ГПа}$

\item Погрешность модуля Юнга вычислим по формуле
\[\sigma_E = E\sqrt{\left( \dfrac{\sigma_{k}}{k} \right)^2 + \left( \dfrac{\sigma_{S}}{S} \right)^2 + \left( \dfrac{\sigma_{l_0}}{l_0} \right)^2 }\]


\item Сравнивая полученное значение модуля Юнга с табличными, определим металл, из которого изготовлена проволока. Модулем Юнга, равным 110-125 ГПа обладает медь, поэтому можно предположить, что используемая в эксперименте проволока была изготовлена из меди.
\end{enumerate}


\newpage
\begin{table}
\centering
\caption{Используемые величины и их погрешности}
\begin{tabular}{|c|c|c|}
\hline
Величина & Значение & Погрешность \\ \hline

Диаметр проволоки & $d = 0,51\text{ мм}$ & $\sigma_d = 0,01\text{ мм}$ \\ \hline

Площадь поперечного сечения проволоки & $S = 0,204\text{ мм}^2$ & $\sigma_S = 0,006\text{ мм}^2$ \\ \hline

Длина проволоки & $l_0 = 174\text{ см}$ & $\sigma_{l_0} = 1\text{ см}$ \\ \hline

Длина рычага & $r = 20\text{ см}$ & $\sigma_r = 0,1\text{ см}$ \\ \hline

Масса груза & $\Delta m$ & $\sigma_{\Delta m} = 0,1\text{ г}$ \\ \hline

Разница показаний шкалы & $\Delta n$ & $\sigma_{\Delta n} = 1\text{ мм}$ \\ \hline

Расстояние от шкалы до проволоки & $h = 141\text{ см}$ & $\sigma_h = 1\text{ см}$ \\ \hline

Удлинение проволоки & $\Delta l$ & $\sigma_{\Delta l} = 0,112 \text{ см}$ \\ \hline

Средний коэффициент зависимости  $m(\Delta l)$ & $k = 15,09 \text{ кг/м}$ & $\sigma_k = 0,53 \text{ кг/м}$ \\ \hline

Модуль Юнга & $E = 126,2 \text{ ГПа}$ & $\sigma_E = 5,8\text{ ГПа}$ \\ \hline

\end{tabular}
\end{table}
\begin{table}
\centering
\caption{Зависимость показаний шкалы от нагрузки}
\begin{tabular}{|c|c|c|c|c|c|c|c|c|}
\hline
№ измерения & $\Delta m$, гр & $P$, Н & $n_1 \downarrow$, см & $n_1\uparrow$, см & $n_2 \downarrow$, см & $n_2 \uparrow$, см & $\overline{\Delta n}$, см & $\overline{\Delta l}$, см \\ \hline

1 &0     & 0     & 19,7 & 20,4 & 20,4 & 20,3 & 0      & 0     \\ \hline

2 &244,8 & 2,488 & 22,1 & 22,7 & 22,7 & 22,5 & 2,3    & 2,3   \\ \hline

3 &245,2 & 4,9   & 24   & 24,8 & 24,8 & 24,7 & 2,075  & 4,375  \\ \hline

4 &245,2 & 7,352 & 25,7 & 26,7 & 26,6 & 26,6 & 1,825  & 6,2    \\ \hline

5 &245,3 & 9,805 & 27,8 & 28,5 & 28,4 & 28,5 & 1,9    & 8,1    \\ \hline

6 &244,9 & 12,254 & 29,6 & 30,15 & 30 & 30,2 & 1,6875  & 9,7875 \\ \hline

7 &245,7 & 14,711 & 31,1 & 31,7 & 31,7 & 31,7 & 1,6125 & 11,4   \\ \hline

8 &245,5 & 17,166 & 33   & 33,4 & 33,2 & 33,3 & 1,625  & 13,025 \\ \hline

9 &245,8 & 19,624 & 34,6 & 34,9 & 34,9 & 34,9 & 1,6    & 14,625 \\ \hline

10 &245,9 & 22,083 & 36,4 & 36,4 & 36,4 & 36,4 & 1,575 & 16,2   \\ \hline

\end{tabular}
\end{table}

\begin{table}[h!]
\centering
\caption{Коэффициент зависимости  $m(\Delta l)$}
\begin{tabular}{|c|c|c|c|c|}
\hline
Номер серии измерений & 1 & 2 & 3 & 4   \\ \hline

Значение $k$, кг/м & 14,94 & 15,21 & 15,07 & 15,16 \\ \hline

Погрешность $\sigma_k$, кг/м & 1,63 & 0,17 & 0,13 & 0,21 \\ \hline

\end{tabular}
\end{table}

\newpage
\center{\textbf{График зависимости $m(\Delta l)$}}
\begin{figure}[h]
\center{\includegraphics[scale=0.5]{2.pdf}}
\end{figure}

\maketitle
\newpage


\center\Large{{\textbf{Работа 1.3.2 \\
Определение модуля кручения при помощи крутильных колебаний}}}

\Large{\textbf{Теоретическая часть}}

\begin{flushleft}
\large
При закручивании цилиндрических стержней круглого сечения распределение деформаций и напряжений одинаково по длине стержня только вдали от мест, где прикладываются закручивающие моменты.
Для этих областей можно считать, что каждое поперечное сечение поворачивается поворачивается как жествкое,
то есть частички материала не сходят с радиальных линий, на которых они были в начале, и все
эти линии поворачиваются на один и тот же угол. Такое напряженное состояние назвается чистым кручением.

При такой деформации любая прямая линия, проведенная до закручивания цилиндра почастицам материала и параллельная оси симметрии,
при закручивании превращается в спираль(винтовую линию). 

В системе можно возбудить крутильные колебания. Вращение стержня с закрепленными
на нем грузиками вокрунг вертикальной оси проиходит под действием упругого момента $M$.
С учетом выражения для момента $M$ получим, что это вращение описывается уравнением колебаний:
\begin{equation}
    I\frac{d^2 \varphi }{d t^2} + f \varphi =0
\end{equation}
Следовательно период кoлебаний системы связан с расстоянием $r$ от оси вращения до грузов и
моментом инерции стержня $I_0$ следующим образом:
\begin{equation}
    \omega^2 = \frac{f}{I}
\end{equation}

\begin{equation}
    T = 2\pi\sqrt{\frac{I}{f}}
\end{equation}
Эти зависимости были получены для незатухающих колебаний. Поэтому для их применения необходимо убедиться, что в рассматриваемой системе диссипативными силами можно пренебречь. Для этого стоит убедиться, что период колебаний не зависит от начальной амплитуды и что амплитуда уменшьется не более чем в 2 раза после около 10 колебаний.\\
Применяя Теорему Гюйгенса-Штейнера
\begin{equation}
    T^2 = (2\pi)^2\frac{I}{f} = (2\pi)^2\frac{I_{0}}{f} + (2\pi)^2\frac{(m_{1}+m_{2})r^2}{f},
\end{equation}
где $I_{0} = \frac{1}{4}mr^2 + \frac{1}{12}ml^2 + \frac{1}{4}ml^2 = \frac{1}{4}mr^2 + \frac{1}{3}ml^2$
\end{flushleft}

\Large{\textbf{Экспериментальная установка}}
\begin{flushleft}
\large
Экспериментальная установка, используемая в этой части работы,
изображена на рис. 1 и состоит из длинной вертикально висящей проволоки П, к нижнему концу которой прикреплен горизонтальный металлический стержень С с двумя симметрично расположенными грузами
Г. Их положение на стержне можно фиксировать. Верхний конец проволоки зажать в цангу и при помощи специального приспособления может
вместе с цангой поворачиваться вокруг вертикальной оси. Таким способом в системе можно возбуждать крутильные колебания. Вращение
стержня С с закрепленными на нем грузами Г вокруг вертикальной
оси происходит под действием упругого момента, возникающего в проволоке.


Характеристики установки с учётом погрешностей приведены в таблице 4.

\begin{figure}[!h]
    \begin{center}
        \includegraphics[scale=0.7]{ystanovka2.png}
        \begin{center}
        \caption{Схема установки}
        \end{center}
        \label{graphic1b}
    \end{center}
\end{figure}
\end{flushleft}


\Large{\textbf{Ход работы}}
\large
\begin{enumerate}

\item Установим диапазон амплитуд, в котором пременимы результаты, полученные для незатухающих колебаний. Для этого укрепим грузы на некотором расстоянии от проволоки и возбудим в системе крутильные колебания. Измерив время двадцати периодов колебаний, находим период $T_1 = 2,3785 \text{ с}$. Уменьшив амплитуду тем же способом найдём соответствующий период $T_2 = 2,3765\text{ с}$. Поскольку $T_1 \approx T_2$, для проведения измерений можно выбрать любую амплитуду не больше первой.
\item Экспериментальным путём установлено, что после двадцати периодов колебаний амплитуда уменьшается меньше, чем в два раза.
\item Установим грузы на стержне на одинаковом расстоянии $l$ от оси системы (проволоки) до центра масс каждого груза и измерим период колебаний $T$. Проведём измерения для семи различных значений $l$. Результаты измерений указаны в таблице 5.

Построим график зависимости $T^2(l^2)$.
\item Период крутильных колебаний можно выразить с помощью следующей формулы:
\[T^2 = 4\pi^2\dfrac{I_0}{f} + 4\pi^2\dfrac{2m}{f}l^2,\]
где $I_0$ - момент инерции стержня с грузами относительно оси вращения, $l$ - расстояние от оси системы (проволоки) до центра масс каждого груза, $f$ - модуль кручения, связанный с модулем сдвига $G$ следующей формулой:
\[f = \dfrac{\pi R^4 G}{2l},\]
где $R$ - радиус проволоки.

Таким образом, для нахождения модуля сдвига $G$ сначала необходимо найти угол наклона $k$ графика зависимости $T^2(l^2)$ с помощью МНК, а затем вычислить модуль кручения $f$
\[k=\frac{\langle xy\rangle-\langle x\rangle \langle y\rangle}{\langle x^2\rangle - \langle x\rangle^2}\]
\[f= \dfrac{8\pi^2m}{k}\]

После чего найдём модуль сдвига $G$ по формуле
\[G = \dfrac{2l_0f}{\pi R^4}\]

\item Перейдём к оценке погрешностей. Погрешности измеренных величин указаны в таблице 4, погрешность измерения периода $\sigma_T = 0,03 \text{ с}$. Далее оценим погрешности рассчитанных величин:
\[\sigma_{k} = \frac{1}{\sqrt{n}}\sqrt{\frac{\langle y^2 \rangle - \langle y \rangle ^2}{\langle x^2 \rangle - \langle x \rangle ^2} - k^2}, n = 7\]\\

\[\sigma_f = f\sqrt{\left( \dfrac{\sigma_{m}}{m} \right)^2 + \left( \dfrac{\sigma_{k}}{k} \right)^2 }\]

\[\sigma_G = G\sqrt{\left( \dfrac{\sigma_{l_0}}{l_0} \right)^2 + \left( \dfrac{\sigma_{f}}{f} \right)^2 + \left( \dfrac{4\sigma_{R}}{R} \right)^2}\]

\end{enumerate}

\begin{table}[h!]
\begin{center}
\caption{Используемые величины и их погрешности}
\begin{tabular}{|c|c|c|}
\hline
Величина & Значение & Погрешность \\ \hline
Масса грузов $m$, гр     & 373,0 & 0,1 \\ \hline
Диаметр проволоки $d$, мм & 1,98 & 0,02 \\ \hline
Длина проволоки $l_0$, мм   & 1750 & 2    \\ \hline
Диаметр центрального цилиндра $d_0$, мм & 48 & 0,1 \\ \hline
Длина одного груза $l_1$, мм & 37 & 0,1 \\ \hline
Коэффициент $k$, $\dfrac{\text{с}^2}{\text{м}^2}$ & 801,2021 & 7,625 \\ \hline
Модуль кручения $f$, $\dfrac{\text{кг$\cdot$м}^2}{\text{с}^2}$ & $3,676\cdot10^{-2}$ & $0,035\cdot10^{-2}$ \\ \hline
Модуль сдвига $G$, $\dfrac{\text{Н}}{\text{м}^2}$ & $4,26\cdot10^{10}$ & $0,18\cdot10^{10}$ \\ \hline
\end{tabular}
\end{center}
\end{table}

\begin{table}[h!]
\begin{center}
\caption{Экспериментальные данные}
\begin{tabular}{|c|c|c|}
\hline
Номер измерения &$l$, см & Период, $T$, с \\ \hline
1&5,15  & 1,9235 \\ \hline
2&6,15  & 2,14535 \\ \hline
3&7,15  & 2,3785 \\ \hline
4&8,15  & 2,62575  \\ \hline
5&9,15  & 2,876 \\ \hline
6&10,15 & 3,137 \\ \hline
7&11,15 & 3,397 \\ \hline
\end{tabular}
\end{center}
\end{table}

\newpage
\center{\textbf{График зависимости $T^2(l^2)$}}
\begin{figure}[h]
\center{\includegraphics[scale=1]{T2L2.pdf}}
\end{figure}

\end{document}
