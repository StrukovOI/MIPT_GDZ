\documentclass[a4paper,11pt]{extarticle} % тип документа
\usepackage[left=1.6cm,right=1.6cm]{geometry}
%%%Библиотеки
    %\usepackage[warn]{mathtext}	
    \usepackage[T2A]{fontenc}   %Кодировка
\usepackage[utf8]{inputenc} %Кодировка исходного текста
    \usepackage[english, russian]{babel} %Локализация и переносы
    \usepackage{caption}
    \usepackage{listings}
    \usepackage{amsmath, amsfonts, amssymb, amsthm, mathtools}
    \usepackage[warn]{mathtext}
    \usepackage[mathscr]{eucal}
    \usepackage{wasysym}
    \usepackage{graphicx} %Вставка картинок правильная
    \usepackage{indentfirst}
    \usepackage{float}    %Плавающие картинки
    \usepackage{wrapfig}  %Обтекание фигур (таблиц, картинок и прочего)
    \usepackage{fancyhdr} %Загрузим пакет
    \usepackage{lscape}
    \usepackage{xcolor}
    \usepackage[normalem]{ulem}
    
    \usepackage{titlesec}
    \titlelabel{\thetitle.\quad}
\usepackage{tikz,pgfplots} % рисунки


%%%Конец библиотек


%Заголовок

\begin{document}
\title{\begin{center}Лабораторная работа № 3.2.6\end{center}
Изучение гальванометра}
\author{Струков О. И. \\ Б04-404}
\date{}

\thispagestyle{empty}
\pagenumbering{gobble}
\maketitle
\newpage
\pagenumbering{arabic}



\textbf{Цель работы:} изучение работы высокочувствительного зеркального гальванометра магнитоэлектрической системы в режимах измерения постоянного тока и электрического заряда.

\textbf{В работе используются:} зеркальный гальванометр с осветителем и
	шкалой, источник постоянного напряжения, делитель напряжения, магазин сопротивлений, эталонный конденсатор, вольтметр, переключатель, ключи, линейка.

\section*{Теоретическая часть}

\text {Баллистический гальванометр} -- электроизмерительный прибор магнитоэлектрической системы, отличающийся высокой чувствительностью к току и сравнительно большим периодом свободных колебаний. Главной частью высокочувствительного гальванометра магнитоэлектричес
койсистемы является подвешенная на вертикальной нити рамка, помещённая в полепостоянного магнита (рис. 1). \par

    \begin{figure}[ht]
	\center{\includegraphics[scale=0.9]{Схема1.png}}
    \caption{Рамка с током в магнитном поле}
	\end{figure}
На помещённую в магнитное поле обтекаемую током рамку гальванометра действуют момент закрученной нити, момент магнитных сил и тормозящий момент (зависит от сил сопротивления воздуха и от вихревых токов). Учитывая все эти моменты, уравнение движения рамки принимает вид
\begin{center}
    $\ddot \varphi + 2 \gamma \dot \varphi + \omega_0^2\varphi = KI $,
\end{center}
где $\gamma$ -- коэффициент затухания подвижной системы гальванометра, $\omega_0$ -- собственная частота колебаний рамки, I - сила тока, K - коэффициент пропорциональности.


Динамическая постоянная гальванометра определяется при пропускании через рамку постоянного тока:
\begin{center}
    $C_I = \frac{I}{\varphi} = \frac{D}{BSN}$,
\end{center}
где $B$ - индукция магнитного поля в рамке, $S$ - площадь одного витка рамки, $D$ - модуль кручения нити. \par
При пропускании коротких импульсов тока через баллистический гальванометр начальная скорость движения рамки пропорциональна электрическому заряду, прошедшему через рамку за всё время импульса. Отношение баллистических постоянных в критическом и свободном режимах равно $e$.

1) Определение динамической постоянной:

\begin{figure}[h]
    \centering
    \includegraphics[width=10cm]{Scheme1.png}
    \caption{Схема установки для работы гальванометра в стационарном режиме}
    \label{fig:vac}
\end{figure}

Постоянное напряжение $U = 1,5$ В снимается с блока питания и измеряется вольтметром $V$. Ключ $K_3$ позволяет менять величину тока через гальванометр Г, делитель напряжения - менять величину тока в широких пределах. Ключ $K_2$ служит для включения гальванометра, кнопка $K_1$ -- для его успокоения. Магазин сопротивлений $R$ позволяет менять режим работы гальванометра от колебательного до апериодического. \par
При малых $R_1$ сила тока, протекающего через гальванометр, может быть вычислена по формуле 
\begin{equation}
    I = U_0 \frac{R_1}{R_2} \frac{1}{R + R_0}.
\end{equation}
Динамическую постоянную вычисляется по формуле 
\begin{equation}
    C_I = \frac{2aI}{x},
\end{equation}
где $a$ - расстояние от шкалы до зеркальца.

2) Определение критического сопротивления гальванометра:

Выполняется с помощью той же цепи, что и на рис. 1. При больших $R$ движение рамки имеет колебательный характер, с уменьшением $R$ затухание увеличивается, и колебательный режим переходит в апериодический. \par
Найдём логарифмический декремент затухания колебаний рамки  $\Theta$.
\begin{equation}
    \Theta = ln\frac{x_n}{x_{n+1}} = \gamma T = \frac{2\pi \gamma}{\sqrt{\omega_0^2 - \gamma^2}} = \frac{2\pi R_3}{\sqrt{(R_0 + R)^2 - R_3^2}}
\end{equation}

где введено обозначение:
$$R_3 = \frac{(BSN)^2}{2\sqrt{JD}} = R_0 + R_\text{кр}$$

Тогда при $R = R_\text{кр}$  выполняется: $\Theta \rightarrow \infty$.

Преобразуя (3) получим

\begin{equation}
    \frac{1}{\theta^2} = \frac{(R+R_0)^2}{4\pi^2 R_3^2} - \frac{1}{4\pi^2}
\end{equation}


 Из (3) можно получить уравнение прямой в координатах $X = (R_0 + R)^2$ и $Y = 1/\Theta^2$:

$$\dfrac{1}{\theta^2} = \dfrac{(R_0 + R)^2}{4 \pi^2 R_2^3} - \dfrac{1}{4 \pi^2}$$

Тогда
\begin{equation}
    R_\text{кр} = \frac{1}{2\pi}\sqrt{\frac{\Delta X}{\Delta Y}} - R_0
\end{equation}

3) Определение баллистической постоянной и критического сопротивления гальванометра, работающего в баллистическом режиме:

Для изучения работы гальванометра в режиме измерения заряда используется схема, представленная на рис. 2.

\begin{figure}[h]
    \centering
    \includegraphics[width=10cm]{Scheme2.png}
    \caption{Схема установки для определения баллистической постоянной}
    \label{fig:vac}
\end{figure}

При нормальном положении кнопки $K_0$ конденсатор $C$ заряжается до напряжения
\begin{center}
    $U_c = \frac{R_1}{R_2}U_0$
\end{center}
Заряд конденсатора равен
\begin{center}
    $q = \frac{R_1}{R_2}U_0 C$
\end{center}
При нажатии на ключ $K_0$ конденсатор отключается от источника постоянного напряжения и подключается к гальванометру. К моменту замыкания ключа $K_4$ весь заряд успевает пройти через гальванометр, рамка получает начальную скорость. Баллистическая постоянная гальванометра определяется при критическом сопротивлении
\begin{equation}
    C Q_{cr} = \frac{q}{\varphi_{max kr}} = 2a\frac{R_1}{R_2} \frac{U_0 C}{l_{max kr}}
\end{equation}

Следует помнить, что наблюдать
колебания рамки при полном отсутствии затухания,
конечно, невозможно, т. к. даже при разомкнутой
внешней цепи (бесконечное сопротивление) остаётся трение в подвеске и трение рамки о воздух. Величину максимального отклонения гальванометра без затухания
$\phi_0$ можно, однако, рассчитать, если при разомкнутой цепи измерены максимальное отклонение рамки
$\phi_1$ и логарифмический декремент затухания $\theta_0$: $\phi_0 = \phi_1 \cdot e^{0,25 \theta_0}$ или $l_0 = l_1 \cdot e^{0,25 \theta_0}$.


























\newpage
\section*{Ход работы}

\subsection*{1. Определение динамической постоянной $C_I$}

Для определения динамической постоянной гальванометра измерялась зависимость отклонения зайчика $x$ от силы тока $I$, протекающего через рамку. Ток рассчитывался по формуле:
\[
I = U_0 \frac{R_1}{R_2} \frac{1}{R + R_0},
\]
где $U_0$ -- напряжение источника, $R_1/R_2$ -- коэффициент делителя, $R$ -- сопротивление магазина, $R_0$ -- сопротивление рамки гальванометра.

Параметры установки:
\begin{table}[H]
\centering
\begin{tabular}{|c|c|}
\hline
$U_0$ & 1,35 В \\ \hline
$R_0$ & 610 Ом \\ \hline
$R_2$ & 10 кОм \\ \hline
Делитель $ \frac{R_1}{R_2}$ & $\frac{1}{2000}$ \\ \hline
$2a$ & 2,8 м \\ \hline
\end{tabular}
\end{table}

По полученным данным был построен график зависимости $I = f(x)$.

\begin{figure}[H]
\centering
\begin{tikzpicture}
\begin{axis}[
    width=0.8\textwidth,
    height=0.5\textwidth,
    xlabel={$x$, мм},
    ylabel={$I$, нА},
    grid=both,
    xmin=70, xmax=250,
    ymin=70, ymax=250,
    legend pos=north west,
    error bars/x dir=both,
    error bars/x fixed=2.0,
    error bars/y dir=none,
    tick label style={/pgf/number format/.cd, fixed, fixed zerofill, precision=1}
]

\addplot+[only marks, mark=*, mark size=2pt, color=blue] coordinates {
    (244, 240.2135231) (221, 217.0418006) (200, 197.9472141)
    (189, 186.9806094) (174, 172.6342711) (162, 160.3325416)
    (151, 149.6674058) (138, 137.4745418) (130.5, 129.5585413)
    (109, 108.6956522) (94, 93.61997226) (73.5, 73.28990228)
};

\addplot[domain=70:250, samples=2, color=red, thick] {1.0115*x - 3.4455};

\legend{Экспериментальные точки, Линейная аппроксимация};

\end{axis}
\end{tikzpicture}
\caption{Зависимость силы тока $I$ через гальванометр от отклонения зайчика $x$}
\label{fig:I_x_graph}
\end{figure}

Как видно из графика, экспериментальные точки хорошо ложатся на прямую линию, что свидетельствует о линейности шкалы гальванометра в исследуемом диапазоне.

% С помощью МНК получено уравнение линейной зависимости:
% \[
% I = (1,0115 \pm 0,0112) \cdot x - (3,45 \pm 1,92) \quad [\text{нА}],
% \]
% где $x$ выражено в мм.

Угловой коэффициент прямой:
\[
k = \frac{I}{x} = 1,0115 \pm 0,0112 \quad \text{нА/мм}.
\]

Динамическая постоянная гальванометра рассчитывается по формуле:
\[
C_I = 2a \cdot \frac{I}{x} = 2a \cdot k,
\]
где $2a$ -- удвоенное расстояние от зеркальца гальванометра до шкалы.

% С учётом значения $2a = 2,8$ м получаем:
% \[
% C_I = 2,8322 \quad \text{нА·м/мм}.
% \]

% Погрешность определения $C_I$:
% \[
% \sigma_{C_I} = 2a \cdot \sigma_k = 2.8 \cdot 0.0112 = 0.0314 \quad \text{нА·м/мм}.
% \]

Таким образом,
\[
C_I = (2,83 \pm 0,03) \ \text{нА·м/мм}.
\]

Чувствительность гальванометра к току:
\[
S_I = \frac{1}{C_I} \approx (0,353 \pm 0,004) \ \text{мм/м/нА}.
\]

% \textbf{Вывод по пункту 1:} 
% В результате измерений определена динамическая постоянная гальванометра $C_I = (2.83 \pm 0.03)$ нА·м/мм и чувствительность к току $S_I = (0.353 \pm 0.004)$ мм/м/нА. Линейность графика $I(x)$ подтверждает пропорциональность угла отклонения рамки силе тока.


















\subsection*{2. Определение критического сопротивления $R_\text{кр}$ подбором}

Был измерен период свободных колебаний $T_0 \approx 3,41$ с.


Подобрано наибольшее сопротивление $R$, при котором зайчик не переходит за нулевое деление шкалы после размыкания ключа $K_3$:
\[
R_{\text{кр}} \approx 6190 \text{ Ом}.
\]







\subsection*{3. Определение критического сопротивления $R_\text{кр}$ по зависимости логарифмического декремента}

Для разных $R$ были измерены отклонения $x_n$ и $x_{n+1}$, вычислены $\Theta = \ln(x_n / x_{n+1})$ и построен график $1/\Theta^2 = f\big((R + R_0)^2\big)$.

% \begin{table}[H]
% \centering
% \begin{tabular}{|c|c|c|c|c|c|c|}
% \hline
% $R, \text{ кОм}$ & $x_n, \text{см}$ & $x_{n+1}, \text{см}$ & $\Theta$ & $1/\Theta^2$ & $\sigma_{1/\Theta^2}$ & $(R+R_0)^2, \text{кОм}^2$ \\ \hline
% 15,8 & 8,2 & 1,0 & 2,10 & 0,23 & 0,02 & 260,1 \\ \hline
% 16,8 & 8,3 & 1,1 & 2,02 & 0,25 & 0,02 & 294,5 \\ \hline
% 17,9 & 8,3 & 1,2 & 1,93 & 0,27 & 0,02 & 331,2 \\ \hline
% 18,9 & 8,2 & 1,4 & 1,77 & 0,32 & 0,03 & 370,0 \\ \hline
% 20,0 & 8,0 & 1,5 & 1,67 & 0,36 & 0,03 & 415,2 \\ \hline
% 21,1 & 7,8 & 1,5 & 1,65 & 0,37 & 0,03 & 462,8 \\ \hline
% 22,2 & 7,8 & 1,6 & 1,58 & 0,40 & 0,03 & 512,7 \\ \hline
% 23,3 & 7,8 & 1,7 & 1,52 & 0,43 & 0,04 & 564,9 \\ \hline
% 26,5 & 7,4 & 1,8 & 1,41 & 0,50 & 0,04 & 742,0 \\ \hline
% 28,7 & 7,3 & 2,1 & 1,25 & 0,64 & 0,05 & 866,8 \\ \hline
% 30,8 & 7,0 & 2,1 & 1,20 & 0,69 & 0,05 & 998,6 \\ \hline
% 33,0 & 6,8 & 2,3 & 1,08 & 0,86 & 0,07 & 1137,4 \\ \hline
% 36,2 & 6,5 & 2,3 & 1,03 & 0,94 & 0,07 & 1391,3 \\ \hline
% 39,5 & 6,1 & 2,3 & 0,97 & 1,06 & 0,09 & 1664,1 \\ \hline
% 42,8 & 5,8 & 2,4 & 0,88 & 1,29 & 0,11 & 1955,9 \\ \hline
% 46,1 & 5,6 & 2,4 & 0,84 & 1,42 & 0,13 & 2266,6 \\ \hline
% 49,4 & 5,2 & 2,4 & 0,77 & 1,69 & 0,17 & 2596,3 \\ \hline
% 52,7 & 5,0 & 2,4 & 0,73 & 1,88 & 0,20 & 2944,0 \\ \hline
% \end{tabular}
% \caption{Зависимость декремента от сопротивления}
% \end{table}

\begin{figure}[H]
\centering
\begin{tikzpicture}
\begin{axis}[
    width=0.8\textwidth,
    height=0.5\textwidth,
    xlabel={$(R_0 + R)^2$, кОм$^2$},
    ylabel={$1/\Theta^2$},
    grid=both,
    xmin=200, xmax=3000,
    ymin=0, ymax=2.0,
    legend pos=north west,
    error bars/y dir=both,
    error bars/y explicit,
    % Добавляем настройки для меток на оси X
    xtick={0,500,1000,1500,2000,2500,3000},
    % xticklabel style={
    %     rotate=45,  % поворот меток на 45 градусов
    %     anchor=north east,  % точка привязки для повернутых меток
    %     font=\small  % уменьшаем размер шрифта
    % },
    % Уменьшаем плотность меток на оси Y
    ytick={0,0.5,1.0,1.5,2.0},
    tick label style={/pgf/number format/.cd, fixed, fixed zerofill, precision=2}
]

\addplot+[only marks, mark=*, mark size=2pt, color=blue] coordinates {
    (260.1, 0.23) 
    (380.2, 0.36) (512.7, 0.38) 
    (742.0, 0.56) (866.8, 0.64) (998.6, 0.69)
    (1391.3, 0.97) (1664.1, 1.06) (1955.9, 1.29) (2266.6, 1.42)
    (2596.3, 1.69) (2944.0, 1.88)
};

\addplot[domain=200:3000, samples=2, color=red, thick] {0.000609*x + 0.086641};

\legend{Экспериментальные точки, Линейная аппроксимация};

\end{axis}
\end{tikzpicture}
\caption{Зависимость $1/\Theta^2$ от $(R_0 + R)^2$}
\label{fig:theta_graph}
\end{figure}
По графику определён угловой коэффициент линейной зависимости:
\[
k = (6,09 \pm 0,09) \cdot 10^{-4}\text{ кОм$^{-2}$}
\]

Согласно формуле (4), угловой коэффициент графика равен:
\[
k = \frac{1}{4\pi^2 R_3^2},
\]
где $R_3 = R_0 + R_\text{кр}$.

Отсюда находим:
\[
R_3 = \frac{1}{2\pi\sqrt{k}}  \approx 6450 \ \text{Ом}.
\]

% Теперь вычисляем критическое сопротивление:
% \[
% R_\text{кр} = R_3 - R_0 = 6.45 - 0.61 = 5.84 \ \text{кОм}.
% \]

% Оценим погрешность $R_\text{кр}$. Сначала найдём погрешность $R_3$:
% \[
% \frac{\sigma_{R_3}}{R_3} = \frac{1}{2} \frac{\sigma_k}{k} = \frac{1}{2} \cdot \frac{0.000009}{0.000609} \approx 0.0074.
% \]
% \[
% \sigma_{R_3} = 6.45 \cdot 0.0074 \approx 0.05 \ \text{кОм}.
% \]

% Погрешность $R_0$ принимаем $\sigma_{R_0} = 0.01$ кОм (класс точности магазина сопротивлений). Тогда:
% \[
% \sigma_{R_\text{кр}} = \sqrt{\sigma_{R_3}^2 + \sigma_{R_0}^2} = \sqrt{0.05^2 + 0.01^2} \approx 0.05 \ \text{кОм}.
% \]

Таким образом,
\[
R_\text{кр} = (5840 \pm 50) \ \text{Ом}.
\]

% \textbf{Вывод по пункту 3:} 
% По зависимости логарифмического декремента затухания от сопротивления получено значение критического сопротивления $R_\text{кр} = (5840 \pm 50)$ кОм. График демонстрирует хорошую линейность, что подтверждает теоретическую зависимость $1/\Theta^2$ от $(R_0 + R)^2$.























\subsection*{4. Определение баллистической постоянной и $R_\text{кр}$ в баллистическом режиме}

Была собрана схема номер 2. Зафиксировано $C = 2,0$ мкФ, делитель напряжения $R_1/R_2 = 1/20$. Измерены отклонения в режиме свободных колебаний: $l_0 = 196$ мм.
\begin{figure}[H]
\centering
\begin{tikzpicture}
\begin{axis}[
    width=0.8\textwidth,
    height=0.5\textwidth,
    xlabel={$R_0 + R$, кОм},
    ylabel={$l_{\max}$, мм},
    grid=both,
    xmin=0, xmax=12,
    ymin=30, ymax=130,
    legend pos=south east,
    error bars/y dir=both,
    error bars/y fixed=2.0,
    error bars/x dir=none,
    tick label style={/pgf/number format/.cd, fixed, fixed zerofill, precision=1}
]

% Экспериментальные точки
\addplot+[only marks, mark=*, mark size=2pt, color=blue] coordinates {
    (1.21, 37)    % 600+610=1210 Ом = 1.21 кОм
    (2.61, 58)    % 2000+610=2610 Ом = 2.61 кОм
    (5.61, 88)    % 5000+610=5610 Ом = 5.61 кОм
    (6.61, 97)    % 6000+610=6610 Ом = 6.61 кОм
    (7.61, 102)   % 7000+610=7610 Ом = 7.61 кОм
    (8.61, 108)   % 8000+610=8610 Ом = 8.61 кОм
    (9.61, 114)   % 9000+610=9610 Ом = 9.61 кОм
    (10.61, 119)  % 10000+610=10610 Ом = 10.61 кОм
};

\legend{Экспериментальные точки};

\end{axis}
\end{tikzpicture}
\caption{Зависимость максимального отклонения $l_{\max}$ от полного сопротивления $R_0 + R$}
\label{fig:lmax_graph_correct}
\end{figure}
Поправку на затухание при разомкнутой цепи:
\[
l_0' = l_0 \cdot e^{0,25\theta_0} = 196 \cdot e^{0,0275} \approx 201\ \text{мм}.
\]

Максимальное отклонение в критическом режиме:
\[
l_{\max\text{кр}} = \frac{l_0'}{e} = \frac{201}{2,718} \approx 74\ \text{мм}.
\]

По графику (рис. 6) найдено значение $(R_0 + R)^{-1}$, соответствующее $l_{\max} = 74$ мм:
\[
(R_0 + R)^{-1} \approx 0,00017\ \text{Ом}^{-1} = 0,17 \cdot 10^6\ \text{Ом}^{-1}.
\]

Отсюда полное сопротивление цепи:
\[
R_0 + R \approx \frac{1}{0,00017} \approx 5880\ \text{Ом}.
\]

Учитывая, что $R_0 = 610$ Ом, получаем:
\[
R \approx 5880 - 610 = 5270\ \text{Ом}
\]

С учётом погрешности:
\[
R_\text{кр} = (5300 \pm 300)\ \text{Ом}.
\]

Рассчитаем баллистическую постоянную при $R_\text{кр}$. Заряд конденсатора:
\[
q = \frac{R_1}{R_2} \cdot C \cdot U_0 = 1,35 \cdot 10^{-7}\ \text{Кл}.
\]

Баллистическая постоянная:
\[
C_{Q_\text{кр}} = \frac{q}{U_{\max\text{кр}}} = 2a \cdot \frac{q}{l_{\max\text{кр}}} \approx 5,11 \cdot 10^{-6}\ \text{Кл·м/мм}.
\]

% Или в более удобных единицах:
% \[
% C_{Q_\text{кр}} = 51,1\ \text{м·нК/мм}.
% \]

% Оценим погрешность. Основные вклады:
% - Погрешность $2a$: $\sigma_{2a} = 0,1$ м
% - Погрешность $l_{\max\text{кр}}$: $\sigma_{l} = 2$ мм
% - Погрешность $q$: $\sigma_q \approx 0,05 \cdot 10^{-7}$ Кл

% Относительная погрешность:
% \[
% \frac{\sigma_{C_{Q_\text{кр}}}}{C_{Q_\text{кр}}} = \sqrt{\left(\frac{0,1}{2,8}\right)^2 + \left(\frac{2}{74}\right)^2 + \left(\frac{0,05}{1,35}\right)^2} \approx \sqrt{0,0013 + 0,0007 + 0,0014} \approx 0,06.
% \]

% Абсолютная погрешность:
% \[
% \sigma_{C_{Q_\text{кр}}} = 51,1 \cdot 0,06 \approx 3,1\ \text{м·нК/мм}.
% \]

С учётом погрешности:
\[
C_{Q_\text{кр}} = (51 \pm 3)\ \text{м·нК/мм}.
\]

Время релаксации:
\[
t = R_0 C = 610 \cdot 2,0 \cdot 10^{-6} = 1,22 \cdot 10^{-3}\ \text{с}.
\]

Период свободных колебаний $T_0 \approx 3,35$ с, следовательно $t \ll T_0$ — условие баллистического режима выполняется.


















\section*{Выводы}

В работе определены параметры зеркального гальванометра тремя методами. Результаты представлены в таблице:

\begin{table}[H]
\centering
\begin{tabular}{|c|c|c|c|c|}
\hline
 \multicolumn{3}{|c|}{$R_\text{кр}, \text{ Ом}$} & $C_I, \dfrac{\text{нА·м}}{\text{мм}}$ & $C_{Q_\text{кр}}, \dfrac{\text{м·нК}}{\text{мм}}$ \\
\cline{1-3}
 Подбор & Метод $1/\Theta^2$ & Балл. метод & & \\ \hline
 $6190$ & $5840 \pm 50$ & $5300 \pm 300$ & $2,83 \pm 0,03$ & $51 \pm 3$ \\ \hline
\end{tabular}
\end{table}



 Значения критического сопротивления, полученные тремя методами, находятся в диапазоне от 5300 до 6190 Ом. 
   Наиболее точным является метод зависимости $1/\Theta^2$ от $(R_0 + R)^2$, дающий $R_\text{кр} = (5840 \pm 50)$ Ом.

 Метод подбора даёт наиболее высокое значение ($6190$ Ом), что может отличаться от истинного из-за неточности визуального определения положения зайчика.

 Баллистический метод даёт промежуточное значение $(5,3 \pm 0,3)$ кОм. Погрешность здесь выше из-за 
   необходимости определения $l_{\max}$ по графику.

Все измерения подтверждают теоретические зависимости для магнитоэлектрического гальванометра. 
   Расхождения в значениях $R_\text{кр}$ могут быть 
   объяснены особенностями экспериментальной установки и методики измерений.

\end{document}