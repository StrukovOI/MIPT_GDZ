\documentclass[a4paper, 12pt]{article}
\usepackage[a4paper,top=1.5cm, bottom=1.5cm, left=1cm, right=1cm]{geometry}
\usepackage{cmap}					% поиск в PDF
\usepackage{mathtext} 				% русские буквы в формулах
\usepackage[T2A]{fontenc}			% кодировка
\usepackage[utf8]{inputenc}			% кодировка исходного текста
\usepackage[english,russian]{babel}	% локализация и переносы

\usepackage{amsmath}
\usepackage{indentfirst}
\usepackage{longtable}
\usepackage{graphicx}
\usepackage{array}

\usepackage{wrapfig}
\usepackage{siunitx} % Required for alignment
\usepackage{subfigure}
\usepackage{multirow}
\usepackage{rotating}
\usepackage{caption}

\graphicspath{{.}}


\title{\begin{center}Лабораторная работа № 2.1.4\end{center}
Определение теплоёмкости твёрдых тел}
\author{Струков О. И. \\ Б04-404}
\date{}

\begin{document}
    \pagenumbering{gobble}
    \maketitle
    \newpage
    \pagenumbering{arabic}

    \textbf{Цель работы:} 1. прямое измерение кривых нагревания $T_{heat}(t)$ и охлаждения $T_{cool}(T)$ пустого калориметра и системы «калориметр + твёрдое тело»; 2. определение коэффициента теплоотдачи стенок калориметра; 3. определение теплоёмкости пустого калориметра и удельной теплоёмкости твёрдого тела

	\textbf{Оборудование:} калориметр с нагревателем и термометром сопротивления; универсальный вольтметр В7-78/3 в режиме омметра ($\sigma_{T} = 0.05~K$), измеритель температуры - термопара K-типа совместно с универсальным вольтметром В7-78/2 ($\sigma_{T_{комн}} = 0.1~K$), источник питания GPS-72303, универсальные вольтметры В7-78/3 (в режиме амперметра) ($\sigma_I = 0.01~A$) и KEITHLEY (в режиме вольтметра) ($\sigma_U = 0.1~В$) для измерения мощности нагревателя, компьютерная программа АКИП для сопряжения персонального компьютера и универсальных вольтметров В7-78/2 и В7-78/3 ($\sigma_t = 0.01~с$).

    \section*{Теоретическая часть}

        В данной работе теплоёмкость определяется по формуле
        \begin{equation}
            C = \frac{\Delta Q}{\Delta T},
            \label{eq:dQdT}
        \end{equation}
        где $\Delta Q$ -- количество тепла, подведенного к телу, и $\Delta T$ -- изменение температуры тела, произошедшее в результате подвода тепла.

        Температура внутри калориметра измеряется термометром сопротивления. В реальных условиях $\Delta Q \neq P \Delta t$, так как часть энергии уходит из калориметра благодаря теплопроводности его стенок. В результате количество тепла $\Delta Q = C \Delta T$, подведённое к системе "тело + калориметр" будет меньше на величину тепловых потерь:

        \begin{equation}
            C \Delta T = P \Delta t - \lambda(T - T_к) \Delta T
            \label{eq:C_Delta_T}
        \end{equation}
        где $\lambda$ - коэффициент теплоотдачи стенок калориметра, $T$ - температура тела и калориметра, $T_к$ - комнатная температура.

        Уравнение (\ref{eq:C_Delta_T}) является основной расчетной формулой работы. В дифференциальной форме для процессов нагревания и охлаждения ($P = 0$) соответственно оно имеет следующий вид:

        \begin{align}
            C dT &= P dt - \lambda \left[ T_{heat}(t) - T_к(t) \right] dt \label{eq:CdT} \\
            C dT &= - \lambda \left[ T_{cool}(t) - T_к(t) \right] dt \label{eq:CdT2}
        \end{align}
        где $P$ ‒ мощность нагревателя, $\lambda$ ‒ коэффициент теплоотдачи стенок калориметра, $t$ – время, измеряемое от момента включения нагревателя, $T_{heat}(t)$ ‒ температура тела в момент времени $t$ на кривой нагревания, $T_{cool}(t)$ ‒ температура тела в момент времени $t$ на кривой охлаждения, $T_к(t)$ ‒ температура окружающего калориметр воздуха (комнатная) в момент времени $t$, $dt$ ‒ время, в течение которого температура тела изменилась на $dT$

        \subsection*{Экспериментальная установка}

            Установка состоит из калориметра с пенопластовой изоляцией, помещенного в ящик из многослойной клееной фанеры. Внутренние стенки калориметра выполненным из материала с высокой теплопроводностью. Надёжность теплового контакта между телом и стенками обеспечивается их формой: они имеют вид усеченных конусов и плотно прилегают друг к другу.

            \begin{figure}[ht]
                \centering
                \includegraphics[width=0.8\textwidth]{calorimeter.png}
                \caption{Схема устройства калориметра}
                \label{fig:calorimeter}
            \end{figure}

            Экспериментально измеряемые данные:

                1. $R_{heat}(t)$ ‒ кривая зависимости термометра сопротивления от времени при нагревании калориметра с телом при P=const.

                2. $R_{cool}(t)$ ‒ кривая зависимости термометра сопротивления от времени при охлаждении калориметра с телом при $P = 0$ (нагреватель выключен!).

                3. $T_к(t)$ ‒ кривая зависимости комнатной температуры от времени

        \subsection*{Методика эксперимента}

            Температура измеряется термометром сопротивления. Сопротивление проводника изменяется с температурой по закону

            \begin{equation}
                R_{T} = R_{273}(1 + \alpha (T - 273)),
                \label{RT}
            \end{equation}
            где $R_{T}$ -- сопротивление термометра про $T  ^{\circ}C$, $R_{0}$ -- его сопротивление при $0  ^{\circ}C$, $\alpha$ -- температурный коэффициент сопротивления.

            Выразим сопротивление $R_{273}$ через измеренное значение $R_{к}$ -- сопротивление термометра при комнатной температуре. Согласно (\ref{RT}), имеем

            \begin{equation}
                R_{273} = \frac{R_{к}}{1 + \alpha (T_к - 273)},
                \label{R273}
            \end{equation}

            Подставляя (\ref{R273}) в (\ref{RT}), найдём:

            \begin{equation}
                T(R_T) = 273 + \frac{R_T}{\alpha R_к} \left[ 1 + \alpha (T_к - 273) \right] - \frac{1}{\alpha}
                \label{T_R_T}
            \end{equation}

            Формула (\ref{T_R_T}) позволяет легко пересчитать кривые $R_{heat}(t)$, $R_{cool}(t)$ в кривые $T_{heat}(t)$, $T_{cool}(t)$. Входящий в формулу температурный коэффициент сопротивления меди равен $\alpha = 4.28*10^{-3} град^{-1}$.

            Из уравнения (\ref{eq:CdT2}) при $T_к(t) = T_к = const$:

            \begin{equation}
                C dT_{cool} = - \lambda \left[ T_{cool} - T_к \right] dt
                \label{C_dT_cool}
            \end{equation}

            Это дифференциальное уравнение с разделяющимися переменными $T_{cool}$ и $t$:

            \begin{equation}
                \frac{C dT_{cool}}{- \lambda \left[ T_{cool} - T_{к} \right]} = dt
                \label{eq:diff}
            \end{equation}

            После интегрирования в пределах от $t=0$ ($T_{cool} = T$) до произвольного момента времени $t$:

            \begin{equation}
                \frac{-C}{\lambda} \ln \frac{T_{cool} - T_{к}}{T - T_к} = t
                \label{eq:diff_integral}
            \end{equation}

            Отсюда находим явную зависимость от времени:

            \begin{equation}
                T_{cool}(t) = (T - T_к) e^{\frac{- \lambda}{C} t} + T_к
                \label{T_cool_t}
            \end{equation}

            Уравнение (\ref{T_cool_t}) легко спрямляется в координатах ($\ln \frac{T_{cool} - T_к}{T - T_к}$, $t$). Тангенс угла наклона данной прямой позволяет определить отношение искомых величин $\frac{\lambda}{C}$.

            Из уравнения (\ref{eq:CdT}) при $T_к(t) = T_к = const$:

            \begin{equation}
                C dT_{heat} = P dt - \lambda \left[ T_{heat} - T_к \right] dt
                \label{C_dT_heat}
            \end{equation}

            Это дифференциальное уравнение с разделяющимися переменными $T_{heat}$ и $t$:

            \begin{equation}
                \frac{C dT_{heat}}{P - \lambda \left[ T_{heat} - T_{к} \right]} = dt
                \label{eq:diff2}
            \end{equation}

            После интегрирования в пределах от $t = 0$ ($T_{heat} = T_к$) до произвольного момента времени $t$:

            \begin{equation}
                \frac{-C}{\lambda} \ln \frac{P - \lambda (T_{heat} - T_{к})}{P} = t
                \label{eq:diff_integral2}
            \end{equation}

            Отсюда находим явную зависимость от времени:

            \begin{equation}
                T_{heat}(t) = \frac{P}{\lambda} (1 - e^{\frac{-\lambda}{C} t}) + T_к
                \label{T_heat_t}
            \end{equation}

            Уравнение (\ref{T_heat_t}) позволяет по найденному ранее из кривой охлаждения отношению $\frac{\lambda}{C}$ определить $\lambda$, а зная $\lambda$ и $\frac{\lambda}{C}$ легко найти искомую теплоёмкость $С$.

            Метод измерений величин $С$ и $\lambda$ рассмотренный выше, даёт хорошие результаты при стабильной комнатной температуре во время проведения эксперимента и является по своей сути интегральным. $С$ и $\lambda$ определяются из уравнений (\ref{T_cool_t}) и (\ref{T_heat_t}), которые следуют из уравнений (\ref{eq:CdT}) и (\ref{eq:CdT2}) после их интегрирования. При существенных колебаниях комнатной температуры ($\sim 2-3~^0C$) интегральные уравнения (\ref{T_cool_t}) и (\ref{T_heat_t}) могут привести к достаточно большой погрешности в определении величин $С$ и $\lambda$. В этом случае следует использовать дифференциальные методы, основанные на измерении величин $\left( \frac{dT}{dt} \right)_{heat}$ и $\left( \frac{dT}{dt} \right)_{cool}$ в окрестностях каких-либо «удобных» точек. К таким «удобным» точкам относится точка на кривой нагревания, при которой температура калориметра совпадает с комнатной. Действительно, дифференцируя уравнение (\ref{eq:CdT}) по времени при $T_{heat}(t) = T_к(t)$, получим простую и удобную формулу для определения теплоёмкости $С$:

            \begin{equation}
                C = \frac{P}{(dT_{heat} / dt)_{T = T_к}}
                \label{eq:C}
            \end{equation}

            Она даёт хорошие результаты, если её применение никак не связано с моментом включения нагревателя. Причина проста: сразу после включения нагревателя в калориметре происходят переходные процессы формирования тепловых потоков, которые не описываются уравнением (\ref{eq:CdT}) и соответственно уравнением (\ref{eq:C}). Чтобы обойти данную трудность, перед включением нагревателя необходимо охладить калориметр до температуры на $\sim 2-5~^oC$ ниже комнатной. В этом случае при подходе к точке $T_{heat}(t) = T_к(t)$ все переходные процессы уже закончатся и уравнение (\ref{eq:C}) будет корректным.

            Другими «удобными» точками для определения $С$ и $\lambda$ являются точки при одной и той же температуре $T$ на кривых нагревания $T_{heat}(t)$ и охлаждения $T_{cool}(t)$ соответственно. Действительно продифференцируем уравнения (\ref{eq:CdT}) и (\ref{eq:CdT2}) по времени:

            \begin{align}
                C \left( \frac{dT}{dt} \right)_{heat} &= P - \lambda \left[ T_{heat}(t) - T_к(t) \right] \label{eq:C_diff_heat}\\
                C \left( \frac{dT}{dt} \right)_{cool} &= - \lambda \left[ T_{cool}(t) - T_к(t) \right] \label{eq:C_diff_cool}
            \end{align}

            Определим $A = \left( \frac{dT}{dt} \right)_{heat}$ и $B = \left( \frac{dT}{dt} \right)_{cool}$ при одной и той же температуре $T$ на кривых $_{heat}(t)$ и $T_{cool}(t)$ соответственно. Тогда с учетом введённых обозначений, решая систему уравнений (\ref{eq:C_diff_heat}) и (\ref{eq:C_diff_cool}), получим следующие выражения для $С$ и $\lambda$:

            \begin{align}
                \lambda &= \frac{P}{(T - T_{к2})(1 - \frac{A}{B}) + T_{к2} - T_{к1}} \label{eq:lambda}\\
                C &= \frac{P}{A - B + A \frac{T_{к1} - T_{к2}}{T - T_{к1}}} \label{eq:C2}
            \end{align}
            где $T_{к1}$ и $T_{к2}$ ‒ комнатная температура в моменты времени $t = t_1$ и $t = t_2$, когда $T_{heat}(t_1) = T_{cool}(t_2) = T$.

            В случае равенства комнатных температур, когда $T_{к1} = T_{к2} = T_к$ формулы (\ref{eq:lambda}) и (\ref{eq:C2}) упрощаются

            \begin{align}
                \lambda &= \frac{P}{(T - T_к)(1 - \frac{A}{B})} \label{eq:lambda_fin}\\
                C &= \frac{P}{A - B} \label{eq:C_fin}
            \end{align}

            \begin{figure}[ht]
                \centering
                \includegraphics[width=0.6\textwidth]{diff_method.png}
            \end{figure}

            Следует иметь в виду, что определение величины $B$ на кривой охлаждения $T_{cool}(t)$ необходимо производить на участках кривой достаточно далеких от момента выключения нагревателя, после того как в калориметре закончатся переходные процессы «переполюсовки» тепловых потоков. Корректный интервал времени для определения В можно определить экспериментально из кривой $T_{cool}(t)$, спрямляя её в координатах ($\ln \frac{T_{cool} - T_к}{T - T_к}$, $t$) , после чего исключить из рассмотрения начальный нелинейный участок:

            \begin{figure}[ht]
                \centering
                \includegraphics[width=0.6\textwidth]{not_lin_fragment.png}
            \end{figure}
    \newpage
    \section*{Ход работы}
    \begin{enumerate}
        \item Было проведено ознакомление с установкой и изучение порядка действий в ходе эксперимента. В таблице приведены основные параметры установки и исследуемых предметов: сила тока $I$, напряжение $U$ и мощность нагревательного элемента ($P=UI$), массы медного и алюминиевого стержней соответственно:
        \begin{center}
            \begin{tabular}{|c|c|c|c|c|}
            \hline 
            $I$, А & $U$, В & $P$, Ватт& $m_{\text{м}}$, г & $m_{\text{ал}}$, г \\ \hline
            0,225 & 27,09 & 6,10 & 575,8 $\pm$ 0,5 & 286,9 $\pm$ 0,5 \\ \hline
            \end{tabular}
        \end{center}
        \item Полученные данные о зависимости сопротивления терморезистора от времени были переведены в температуру по формуле $T(R_T) = 14,58\cdot R_T + 39,36$. На основе их и данных, полученных с термопары, были построены кривые зависимости температур от времени на одном графике.
            \begin{figure}[ht]
                \center{\includegraphics[scale=0.7]{Graph.pdf}}
                \caption{Графики зависимости $T_{\text{ком}}$ и $T_{\text{кал}}$ (К) от времени (с)}
            \end{figure}
        \item Построенный график сопоставлен с временными отметками, соответствующими последовательным этапам эксперимента:
        \begin{center}
            \begin{tabular}{|c|l|}
            \hline 
            Время от запуска, с & Этап измерений\\ \hline
            0 & Начало измерения\\ \hline
            100 -- 300 & Охлаждение пустого калориметра латтунным стержнем\\ \hline
            670 -- 2190 & Нагрев пустого калориметра\\ \hline
            2230 -- 2720 & Свободное остывание пустого калориметра\\ \hline
            2760 -- 3450 & Охлаждение калориметра металлическими стержнями\\ \hline
            3630 -- 5360 & Нагрев калориметра с медным стержнем внутри\\ \hline
            5400 -- 6430 & Свободное остывание калориметра с медным стержнем внутри\\ \hline
            6480 -- 7030 & Охлаждение калориметра металлическими стержнями\\ \hline
            7050 -- 8720 & Нагрев калориметра с алюминиевым стержнем внутри\\ \hline
            8800 -- 9600 & Свободное остывание калориметра с алюминиевым стержнем внутри\\ \hline

            \end{tabular}
        \end{center}
        \item Для нахождения $\dfrac{\lambda}{C}$ был исследован участок кривой $T_{cool}$ в координатах ($t, \ln \dfrac{T_{cool}-T_K}{T-T_K}$), где $T_{cool}$ -- начальная температура в процессе остывания, $T_K$ -- средняя комнатная температура, $T$ -- температура калориметра, зависящая от времени. С помощью МНК был найден угловой коэффициент наклона прямой $k_1$, то есть из формулы (10) следует, что $\dfrac{\lambda}{C} = -k_1 = 3,645 \cdot 10^{-4} \text{ с}^{-1}$.
        \begin{figure}[ht]
            \center{\includegraphics[scale=0.7]{Empty cool 2.pdf}}
            \caption{График зависимости $\ln \dfrac{T_{cool}-T_K}{T-T_K}$ от времени для пустого калориметра}
        \end{figure}
        \item Коэффициент тепловых потерь калориметра $\lambda$ можно найти зная угловой коэффициент наклона графика $k_2$ зависимости $T_{heat}(P(1-e^{\frac{-\lambda}{c}t}))$: $\lambda = \dfrac{1}{k_2}$. Таким образом, $k_2=4,709\ \frac{\text{К}\cdot\text{с}}{\text{Дж}}$, $\lambda=0,212\ \frac{\text{Дж}}{\text{К}\cdot\text{с}}$. Отсюда $C = \dfrac{\lambda}{\frac{\lambda}{C}} = 582,64\ \frac{\text{Дж}}{\text{К}}$ -- теплоёмкость калориметра.
        \item Построен график зависимости $\ln \dfrac{T_{cool}-T_K}{T-T_K}$ от времени для калориметра с медным стержнем внутри. Можно заметить, что первые примерно 100 точек отклоняются от линейной зависимости, что вызвано перенаправлением потоков тепла в калориметре, поэтому они не были учтены в расчётах. Аналогично п. 4 из полученной зависимости определено $\dfrac{\lambda_M}{C+C_{\text{М}}} = 2,278\cdot10^{-4}\text{ c}^{-1}$, где $C_{\text{М}}$ -- теплоёмкость исследумого медного образца.
        \begin{figure}[ht]
            \center{\includegraphics[scale=0.7]{Cuprum cool.pdf}}
            \caption{График зависимости $\ln \dfrac{T_{cool}-T_K}{T-T_K}$ от времени для калориметра с медным образцом}
        \end{figure}
        \item Так же, как и в случае с пустым калориметром, из углового коэффициента наклона графика зависимости $T_{heat}(P(1-e^{\frac{-\lambda}{c}t}))$ для нагрева медного образца был найден $\lambda_M = 0,188\ \frac{\text{Дж}}{\text{К}\cdot\text{с}}$. Отсюда $C_M = \dfrac{\lambda_M}{\frac{\lambda_M}{C+C_M}} = 242,59\ \frac{\text{Дж}}{\text{К}}$, где $C_M$ -- теплоёмкость исследумого медного образца.
        \item Для алюминиевого образца также был построен график зависимости $\ln \dfrac{T_{cool}-T_K}{T-T_K}$ от времени и определены $\dfrac{\lambda_{\text{Ал}}}{C+C_{\text{Ал}}} = 1,777\cdot10^{-4}\ \text{ c}^{-1}$, $\lambda_{\text{Ал}} = 0,151\ \frac{\text{Дж}}{\text{К}\cdot\text{с}}$ и $C_{\text{Ал}} = 265,31\ \frac{\text{Дж}}{\text{К}}$, где $C_{\text{Ал}}$ -- теплоёмкость исследумого алюминиевого образца.
        \begin{figure}[ht]
            \center{\includegraphics[scale=0.7]{Aluminium cool.pdf}}
            \caption{График зависимости $\ln \dfrac{T_{cool}-T_K}{T-T_K}$ от времени для калориметра с алюминиевым образцом}
        \end{figure}
        \newpage
        \item Из найденных теплоёмкостей металлических стержней, зная их массу, можно найти удельную теплоёмкость металлов: $c = \dfrac{C}{m}$. Таким образом, для меди и алюминия получаем соответственно: $c_M = 421,3\ \dfrac{\text{Дж}}{\text{кг}\cdot\text{К}}$, $c_\text{Ал} = 924,7\ \dfrac{\text{Дж}}{\text{кг}\cdot\text{К}}$
    
    
    \end{enumerate}
    \subsection*{Нахождение теплоёмкостей дифференциальным методом}
    
    Использую для нахождения теплоёмкостей формулы (16) и (20), для этого в первом случае найду производные $\left(\dfrac{d\, T_{heat}}{d\, t}\right)_{T=T_K}$ в окрестностях точек совпадения температур в комнате и калориметре при их различии не более чем на 0,1 - 0,15 К, а во втором -- $\left(\dfrac{d\,T}{d\,t}\right)_{heat}$ и $\left(\dfrac{d\,T}{d\,t}\right)_{cool}$ при одинаковых температурах. 
    Результаты представлены в таблице:
    \begin{table}[h!]
    \begin{center}
        \begin{tabular}{|c|c|c|c|c|c|}
        \hline 
         & $С_{\text{Калор.}}$, $\frac{\text{Дж}}{\text{К}}$ & $С_{\text{М}}$, $\frac{\text{Дж}}{\text{К}}$ & $С_{\text{Ал}}$, $\frac{\text{Дж}}{\text{К}}$ & $c_{\text{М}}$, $\frac{\text{Дж}}{\text{кг}\cdot\text{К}}$  & $c_{\text{Ал}}$, $\frac{\text{Дж}}{\text{кг}\cdot\text{К}}$ \\[1ex] \hline
        Формула (16) & 631,08 & 229,48 & 245,24 & 398,54 & 854,79 \\ \hline
        Формула (20) & 662,67 & 273,13 & 284,16 & 474,35 & 990,45 \\ \hline
        \end{tabular}
    \end{center}
    \end{table}
    \section*{Вывод}
    В результате выполнения работы двумя различными способами были найдены теплоёмкости исследуемых образцов и с их помощью определены удельные теплоёмкости меди и алюминия. Табличные значения составляют $c_M^{\text{табл}} = 381\ \frac{\text{Дж}}{\text{кг}\cdot\text{К}}$ и $c_{\text{Ал}}^{\text{табл}} = 902,5\ \frac{\text{Дж}}{\text{кг}\cdot\text{К}}$ соответственно.
    Можно заметить, что более точные результаты дал первый способ нахождения, но и с его помощью не удаётся найти достаточно близкие к табличным значения, так как для получения точных результатов необходимо, чтобы температура окружающей среды была постоянной, чего добиться в условиях аудитории, наполненной людьми и нагреваемой самой исследуемой установкой, достаточно сложно.

    






\end{document}