\documentclass[a4paper, 12pt]{article}
\usepackage[a4paper,top=1.5cm, bottom=1.5cm, left=1cm, right=1cm]{geometry}
\usepackage{cmap}					% поиск в PDF
\usepackage{mathtext} 				% русские буквы в формулах
\usepackage[T2A]{fontenc}			% кодировка
\usepackage[utf8]{inputenc}			% кодировка исходного текста
\usepackage[english,russian]{babel}	% локализация и переносы

\usepackage{amsmath}
\usepackage{indentfirst}
\usepackage{longtable}
\usepackage{graphicx}
\usepackage{array}

\usepackage{wrapfig}
\usepackage{siunitx} % Required for alignment
\usepackage{subfigure}
\usepackage{multirow}
\usepackage{rotating}
\usepackage{caption}

\graphicspath{{pictures/}}


\title{\begin{center}Лабораторная работа №2.2.6\end{center}
Определение энергии активации по температурной зависимости вязкости жидкости}
\author{Струков Олег Игоревич\\ Б04-404}
\date{}

\begin{document}
    \pagenumbering{gobble}
    \maketitle
    \newpage
    \pagenumbering{arabic}


    \textbf{Цель работы:} измерение скорости падения шариков при разной температуре жидкости; вычисление вязкости жидкости по закону Стокса и расчёт энергии активации.

    \textbf{Оборудование:} стеклянный цилиндр с исследуемой жидкостью (глицерин); термостат ($\sigma_T = 0,1~К$); телефон (для съёмки видео); микроскоп; мелкие стеклянные и стальные шарики диаметром 1-2 мм.

    \section*{Теоретическая часть}
    \subsection*{Энергия активации}
    Для того чтобы перейти в новое состояние, молекула жидкости должна преодолеть участки с большой потенциальной энергией, превышающей среднюю тепловую энергию молекул. Для этого тепловая энергия молекул должна — вследствие флуктуации — увеличиться на некоторую величину $W$ , называемую энергией активации. Температурная зависимость вязкости жидкости при достаточно грубых предположениях можно описать формулой
    \begin{equation} \label{activation_energy:1}
        \eta = A e^{W/kT}
    \end{equation}

    Из формулы (\ref{activation_energy:1}) следует, что существует линейная зависимость между величинами $\ln\eta$ и $1/T$, и энергию активации можно найти по формуле

    \begin{equation} \label{activation_energy:2}
        W=k\frac{d(\ln\eta) }{d(1/T)}
    \end{equation}

    \subsection*{Измерение вязкости}
    По формуле Стокса, если шарик радиусом $r$ и со скоростью $v$ движется в среде с вязкостью $\eta$, и при этом не наблюдается турбулентных явлении, тормозящую силу можно найти по формуле (\ref{stokes})

    \begin{equation}\label{stokes}
        F = 6\pi\eta \frac{d}{2}v
    \end{equation}


    Для измерения вязкости жидкости рассмотрим свободное падение шарика в жидкости. При медленных скоростях на шарик действуют силы Архимеда и Стокса, выражения для которых мы знаем. Отсюда находим выражения для установившейся скорости шарика и вязкости жидкости

    \begin{align}
        v_{уст}&=\frac{2}{9}g\frac{d^2}{4}\frac{\rho - \rho_ж}{\eta}\label{v_ust}\\
        \eta&=\frac{2}{9}g\frac{d^2}{4}\frac{\rho - \rho_ж}{v_{уст}}\label{eta}
    \end{align}

    Как видим, измерив установившуюся скорость шарика и параметры системы можно получить вязкость по формуле (\ref{eta}).

    Плотность глицерина в зависимости от температуры определяется из графика:
    \begin{figure}[ht]
        \center{\includegraphics[scale=0.3]{plot}}
        
    \end{figure}
    \subsection*{Экспериментальная установка}
    Для измерений используется стеклянный цилиндрический сосуд B, наполненный исследуемой жидкостью (глицерин). Диаметр сосуда $\approx 3$ см, длина $\approx 25$ см. На стенках сосуда нанесены две метки на некотором расстоянии друг от друга. Верхняя метка должна располагаться ниже уровня жидкости с таким расчетом, чтобы скорость шарика к моменту прохождения этой метки успевала установиться. Измеряя расстояние между метками, b время падения определяют установившуюся скорость шарика $v_{уст}$. Сам сосуд B помещен в рубашку D, омываемую водой из термостата. При работающем термостате температура воды в рубашке D, а потому и температура жидкости 12 равна температуре воды в термостате.
    Схема прибора (в разрезе) показана на рис.~\ref{ustanovka}.
    \begin{figure}[ht]
        \center{\includegraphics[scale=0.5]{ustanovka}}
        \caption{Установка для определения коэффициента вязкости жидкости.}
        \label{ustanovka}
    \end{figure}


Число Рейнольдса можно найти следущим образом:
\begin{equation}\label{RE}
    Re = \frac{\rho v D}{\eta}
\end{equation}
Для исследования применимости формулы Стокса можно вычислить путь S и время релаксации $\tau$, необходимые для достижения установившейся скорости. 
\begin{equation}\label{relax}
    \tau = \frac{V\rho}{6\pi\eta r} = \frac{2}{9}\frac{r^2\rho}{\eta}, \qquad S = v_{\text{уст}}\tau\left(\frac{\tau}{t} - 1 + e^{-t/\tau}\right)
\end{equation}
Отсюда следует, что $S\gg v_{\text{уст}}\tau$ при $t\gg\tau$.
    \section*{Ход работы}
\begin{enumerate}
    \item Были выбраны по десять стеклянных и стальных шариков, измерены их средние диаметры. Приборная погрешность измерения равна удвоенной цене деления микроскопа: $\sigma_d = 0,1$ мм. Результаты занесены в таблицу:
    \begin{center}
        \begin{tabular}{|c|c|c|}
        \hline 
        Номер & Диаметр стеклянного шарика, мм & Диаметр стального шарика, мм\\ \hline
        1 & 2,10 & 0,80\\ \hline
        2 & 2,20 & 0,85\\ \hline
        3 & 2,00 & 0,85\\ \hline
        4 & 2,15 & 0,90\\ \hline
        5 & 2,00 & 0,90\\ \hline
        6 & 2,10 & 0,80\\ \hline
        7 & 2,10 & 0,70\\ \hline
        8 & 2,20 & 0,65\\ \hline
        9 & 2,10 & 0,75\\ \hline
        10 & 2,15 & 0,85\\ \hline
        \end{tabular}
    \end{center}
    \item Для вычисления вязкости глицерина $\eta$ необходимо было измерить установившиеся скорости падения шариков в нём и его температуру. В таблицах приведены результаты измерений и вычислений, время указано для прохождения расстояния $l = 0,206$ м. Плотность стекла $\rho_{\text{стекло}} = 2500$ кг/м$^3$, стали $\rho_{\text{сталь}} = 7800$ кг/м$^3$
    \begin{table}[h!]
    \begin{center}
        \caption*{Данные для стеклянных шариков}
        \begin{tabular}{|c|c|c|c|c|c|c|}

        \hline 
        Номер & $T, \,^\circ\text{C}$ & $\rho$, кг/м$^3$ & d, мм & t, с & $v_{\text{уст}}$, м/с$\cdot10^{-3}$ & $\eta$, Па$\cdot$с\\ \hline
        1 & 22,90 & 1258,0 & 2,10 & 60,72 & 3,39 $\pm$ 0,02 & 0,88 $\pm$ 0,04 \\ \hline
        2 & 22,90 & 1258,0 & 2,20 & 61,86 & 3,33 $\pm$ 0,02 & 0,98 $\pm$ 0,04 \\ \hline
        3 & 30,88 & 1254,5 & 2,00 & 36,79 & 5,60 $\pm$ 0,03 & 0,48 $\pm$ 0,02 \\ \hline
        4 & 30,88 & 1254,5 & 2,15 & 34,73 & 5,93 $\pm$ 0,03 & 0,52 $\pm$ 0,02 \\ \hline
        5 & 37,45 & 1252,5 & 2,00 & 25,34 & 8,13 $\pm$ 0,04 & 0,33 $\pm$ 0,02 \\ \hline
        6 & 37,45 & 1252,5 & 2,10 & 24,11 & 8,54 $\pm$ 0,04 & 0,35 $\pm$ 0,02 \\ \hline
        7 & 43,10 & 1250,5 & 2,20 & 15,99 & 12,88 $\pm$ 0,06 & 0,26 $\pm$ 0,01 \\ \hline
        8 & 43,10 & 1250,5 & 2,20 & 14,61 & 14,10 $\pm$ 0,07 & 0,23 $\pm$ 0,01 \\ \hline
        9 & 50,30 & 1247,0 & 2,10 & 8,94 & 23,04 $\pm$ 0,12 & 0,13 $\pm$ 0,01 \\ \hline
        10 & 50,30 & 1247,0 & 2,15 & 8,91 & 23,12 $\pm$ 0,12 & 0,14 $\pm$ 0,01 \\ \hline

        \end{tabular}
    \end{center}
    \end{table}

    \begin{table}[h!]
    \begin{center}        
        \caption*{Данные для стальных шариков}

        \begin{tabular}{|c|c|c|c|c|c|c|}
        \hline 
        Номер & $T, \,^\circ\text{C}$ & $\rho$, кг/м$^3$ & d, мм & t, с & $v_{\text{уст}}$, м/с$\cdot10^{-3}$ & $\eta$, Па$\cdot$с\\ \hline
        1 & 22,90 & 1258,0 & 0,80 & 69,24 & 2,98 $\pm$ 0,01 & 0,77 $\pm$ 0,09 \\ \hline
        2 & 22,90 & 1258,0 & 0,85 & 67,83 & 3,04 $\pm$ 0,01 & 0,85 $\pm$ 0,10 \\ \hline
        3 & 30,88 & 1254,5 & 0,85 & 38,90 & 5,30 $\pm$ 0,03 & 0,49 $\pm$ 0,06 \\ \hline
        4 & 30,88 & 1254,5 & 0,90 & 38,06 & 5,41 $\pm$ 0,03 & 0,53 $\pm$ 0,06 \\ \hline
        5 & 37,45 & 1252,5 & 0,90 & 25,88 & 7,96 $\pm$ 0,04 & 0,36 $\pm$ 0,04 \\ \hline
        6 & 37,45 & 1252,5 & 0,80 & 30,86 & 6,68 $\pm$ 0,03 & 0,34 $\pm$ 0,04 \\ \hline
        7 & 43,10 & 1250,5 & 0,70 & 29,94 & 6,88 $\pm$ 0,03 & 0,25 $\pm$ 0,04 \\ \hline
        8 & 43,10 & 1250,5 & 0,65 & 33,32 & 6,18 $\pm$ 0,03 & 0,24 $\pm$ 0,04 \\ \hline
        9 & 50,30 & 1247,0 & 0,75 & 14,63 & 14,08 $\pm$ 0,07 & 0,14 $\pm$ 0,02 \\ \hline
        10 & 50,30 & 1247,0 & 0,85 & 12,8 & 16,09 $\pm$ 0,08 & 0,16 $\pm$ 0,02 \\ \hline
        \end{tabular}
    \end{center}
    \end{table}
    Погрешность измерения времени $\sigma_t = 0,02$ с, погрешность пройденного расстояния $\sigma_l = 0,001$ м, погрешности скорости $\sigma_{v_{\text{уст}}}$ и вязкости $\sigma_{\eta}$ были определены по следующим формулам:
    \[ \sigma_{v_{\text{уст}}} = v_{\text{уст}}\sqrt{\left(\frac{\sigma_l}{l}\right)^2+\left(\frac{\sigma_t}{t}\right)^2}, \qquad \sigma_{\eta} = \eta\sqrt{4\left(\frac{\sigma_d}{d}\right)^2+\left(\frac{\sigma_{v_{\text{уст}}}}{v_{\text{уст}}}\right)^2}\]
    
    \item Для каждого из опытов было вычислено значение числа Рейнольдса Re по формуле (6), при этом диаметр сосуда составил $D = (2,8\pm0,1)$ см, затем были определены время $\tau$ и путь S релаксации по формулам (7). Результаты представлены в таблице:
    \begin{center}
        \begin{tabular}{|c|c|c|c|c|c|c|}
        \hline 
        & \multicolumn{3}{c|}{Стекло} & \multicolumn{3}{c|}{Сталь} \\
        \cline{2-7}
        \raisebox{1.5ex}[0cm][0cm]{Номер}
        & Re & $\tau$, с$\cdot10^{-3}$ & S, м$\cdot10^{-6}$ & Re & $\tau$, с$\cdot10^{-3}$ & S, м$\cdot10^{-6}$ \\ \hline
        1 & 0,14 $\pm$ 0,01 & 2,78 $\pm$ 0,24 & 9,44 $\pm$ 0,81 & 0,14 $\pm$ 0,02 & 1,45 $\pm$ 0,21 & 4,30 $\pm$ 0,62 \\ \hline
        2 & 0,12 $\pm$ 0,01 & 2,73 $\pm$ 0,23 & 9,10 $\pm$ 0,77 & 0,13 $\pm$ 0,01 & 1,48 $\pm$ 0,20 & 4,48 $\pm$ 0,62 \\ \hline
        3 & 0,41 $\pm$ 0,02 & 4,58 $\pm$ 0,40 & 25,67 $\pm$ 2,24 & 0,38 $\pm$ 0,05 & 2,57 $\pm$ 0,35 & 13,63 $\pm$ 1,88 \\ \hline
        4 & 0,39 $\pm$ 0,02 & 4,85 $\pm$ 0,41 & 28,80 $\pm$ 2,46 & 0,36 $\pm$ 0,04 & 2,63 $\pm$ 0,35 & 14,24 $\pm$ 1,88\\ \hline
        5 & 0,85 $\pm$ 0,04 & 6,64 $\pm$ 0,58 & 54,02 $\pm$ 4,72 & 0,77 $\pm$ 0,09 & 3,87 $\pm$ 0,51 & 30,78 $\pm$ 4,07\\ \hline
        6 & 0,85 $\pm$ 0,04 & 6,98 $\pm$ 0,60 & 59,67 $\pm$ 5,14 & 0,68 $\pm$ 0,09 & 3,24 $\pm$ 0,47 & 21,65 $\pm$ 3,12\\ \hline
        7 & 1,76 $\pm$ 0,08 & 10,51 $\pm$ 0,89 & 135,49 $\pm$ 11,51 & 0,95 $\pm$ 0,14 & 3,34 $\pm$ 0,53 & 22,99 $\pm$ 3,68\\ \hline
        8 & 2,11 $\pm$ 0,10 & 11,50 $\pm$ 0,98 & 162,32 $\pm$ 13,79 & 0,89 $\pm$ 0,14 & 3,00 $\pm$ 0,51 & 18,56 $\pm$ 3,15\\ \hline
        9 & 6,16 $\pm$ 0,30 & 18,75 $\pm$ 1,61 & 432,86 $\pm$ 37,30 & 3,45 $\pm$ 0,46 & 6,83 $\pm$ 1,03 & 69,27 $\pm$ 14,58\\ \hline
        10 & 5,91$\pm$ 0,28 & 18,81 $\pm$ 1,61 & 435,79 $\pm$ 37,29 & 3,5 $\pm$ 0,41 & 7,8 $\pm$ 1,08 & 125,78 $\pm$ 17,34\\ \hline

        \end{tabular}
    \end{center}
    Погрешности нахождения Re, $\tau$ и S были определены следующим образом:
    \[ \sigma_{Re} = Re\sqrt{\left(\frac{\sigma_{v_{\text{уст}}}}{v_{\text{уст}}}\right)^2+\left(\frac{\sigma_{\eta}}{\eta}\right)^2}, \qquad \sigma_{\tau} = \tau\sqrt{4\left(\frac{\sigma_d}{d}\right)^2+\left(\frac{\sigma_{\eta}}{\eta}\right)^2}, \qquad \sigma_{S} = S\sqrt{\left(\frac{\sigma_{v_{\text{уст}}}}{v_{\text{уст}}}\right)^2+\left(\frac{\sigma_{\tau}}{\tau}\right)^2}\]
    Поскольку полученные значения Re < 10, формула Стокса применима во всех рассматриваемых случаях.
    \item Построены графики зависимости $\ln \eta$ от $1/T$.
    \begin{figure}[ht]
        \center{\includegraphics[scale=0.8]{Glas.pdf}}
        \caption*{График зависимости $\ln \eta$ от $1/T$ для стеклянных шариков.}
        \label{Glas}
    \end{figure}
    \begin{figure}[ht]
        \center{\includegraphics[scale=0.8]{Stahl.pdf}}
        \caption*{График зависимости $\ln \eta$ от $1/T$ для стальных шариков.}
        \label{Stahl}
    \end{figure}
    \newpage
    \item По угловым коэффициентам наклона прямых на графиках были найдены энергии активации в обоих случаях (формула 2):
    \[ k_{стекло} = 6585 \text{ К},\qquad \sigma_{k_{стекло}} =207 \text{ К},\qquad k_{сталь} = 5785\text{ К}, \qquad \sigma_{k_{сталь}} =172\text{ К}\]

    \[W_{Стекло} = (9,1 \pm 0,3)\cdot10^{-20} \text{Дж}, \qquad W_{Сталь} = (8,0\pm2)\cdot{10^{-20}} \text{Дж}\]
\end{enumerate}
\section*{Вывод}
В работе была исследована вязкость глицерина при разных температурах с помощью падающих в нём тел шарообразной формы, определены число Рейнольдса, время и путь релаксации для нескольких значений температуры, установлена применимость формулы Стокса в исследуемых условиях и в конечном итоге найдена его энергия активации, близкая к табличному значению $W = 8,5\cdot10^{-20}$ Дж.

\end{document}