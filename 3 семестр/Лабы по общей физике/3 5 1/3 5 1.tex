\documentclass{article}
\usepackage{graphicx} % Required for inserting images

\usepackage{geometry}
\usepackage{amsmath, amsfonts, amssymb, amsthm} % стандартный набор AMS-пакетов для математ. текстов
\usepackage{mathtext}
\usepackage[utf8]{inputenc} % кодировка utf8
\usepackage[russian]{babel} % русский язык
\usepackage[pdftex,dvipsnames]{xcolor} % работа с цветами
% \usepackage[pdftex]{graphicx} % графика (картинки)
% \usepackage{tikz,pgfplots} % рисунки
\usepackage{indentfirst}
% \usepackage[labelfont=bf,labelsep=endash,skip=3pt]{caption} % подпись картинок
\usepackage{fancyhdr,pageslts} % настройка колонтитулов
\usepackage{enumitem} % работа со списками
\usepackage{floatrow,multicol,multirow,longtable,hhline} % работа с таблицами
\usepackage{float,wrapfig} % плавающие объекты
\usepackage{tcolorbox} % рамка вокруг текста
%\usepackage[calc]{datetime2} % дата
\usepackage{bm} % жирное начертание в формулах
\usepackage{physics} % физический пакет
\DeclareMathAlphabet\mathbfcal{OMS}{cmsy}{b}{n}
\usepackage{pgfornament} % красивые рюшечки и вензеля
\usepackage{mdframed}
\usepackage{derivative}
\usepackage{mathrsfs} %EDS
\usepackage{soul} % strikethorugh
%\usepackage{boondox-cal}

% ----------------------------------------
% Настройка шрифта

% Просто закооментируйте следующую строчку, если не работает. Будет другой шрифт, правда :(
% \usepackage{pscyr}

% ----------------------------------------
% Стилевые настройки

\usepackage{boldline} % жирная линия после таблиц (чтобы не было ошибок, этот пакет должен подключаться именно тут!)
\floatsetup[table]{style=Plaintop,floatrowsep=qquad} % настройка оформления таблиц
\setlist[enumerate,itemize]{leftmargin=5mm,itemindent=10mm,itemsep=0mm,
listparindent=0em,labelsep=2mm,topsep=2mm,labelwidth=4mm} % настройки списков

\setlength{\columnsep}{0.5cm} % расстояние между колонками
\setlength{\parskip}{1pt} % расстояние до текста от колонтитула

%\usepackage{titlesec} % управление оформлением section
%\renewcommand{\thesection}{\Roman{section}}
%\titleformat{\section}[block]{\bfseries\large}{\thesection.}{5pt}{}

% ----------------------------------------
% Настройки полей
\geometry{
	left=10mm,
	top=10mm,
	right=10mm,
	bottom=15mm,
	marginparsep=0mm,
	marginparwidth=0mm,
	headheight=0pt,
	headsep=0pt,
	footskip=20pt}

% ----------------------------------------
% Настройки колонтитулов и нумерации страниц
\pagenumbering{arabic}



\newcounter{ntask}
\setcounter{ntask}{0}


\newcommand{\arsh}{\mathrm{arsh} \,\,}
\newcommand{\arch}{\mathrm{arch} \,\,}
\newcommand{\arth}{\mathrm{arth} \,\,}
\newcommand{\arcth}{\mathrm{arcth} \,\,}
\renewcommand{\Re}{\operatorname{Re} \,}
\newcommand{\EDS}{\mathscr{E}}
\newcommand{\diffract}[1]{\frac{\mathrm{d}#1}{\mathrm{d}t}}

\begin{document}
\title{\begin{center}Лабораторная работа № 3.5.1\end{center}
Изучение плазмы газового разряда в неоне}
\author{Струков О. И. \\ Б04-404}
\date{}

\thispagestyle{empty}
\pagenumbering{gobble}
\maketitle
\newpage
\pagenumbering{arabic}

 
\textbf{Цель работы:} Изучение вольт-амперной характеристики тлеющего разряда, изучение свойств плазмы методом зондовых характеристик.

\textbf{В работе используются:} Стеклянная газоразрядная трубка, наполненная изотопом неона, высоковольтный источник питания (ВИП), источник питания постоянного тока, делитель напряжения, резистор, потенциометр, амперметры, вольтметры, переключатели.



\section*{Теоретическая часть}
\subsection*{Плазма}
Плазма газового разряда — это частично ионизированный газ, образующийся при прохождении электрического тока через газовую среду.
В ионизированном газе поле ионов экранируется электронами. Для поля $\mathbf{E}$ и плотности заряда $\rho$:
$$
\text{div}~\mathbf{E} = 4\pi \rho,
$$
а с учётом сферической симметрии:
\begin{equation}
\dfrac{d^2 \varphi}{dr^2}+\dfrac{2}{r}\dfrac{d\varphi}{dr}=-4\pi \rho.
\end{equation}
Плотности заряда электронов и ионов:
\begin{equation}
\begin{array}{c}
\rho_e = -ne \cdot \exp\left(\dfrac{e\varphi}{kT_e}\right),\\
\rho_i = ne.
\end{array}
\end{equation}
В приближении $\dfrac{e\varphi}{kT_e} \ll 1$ получим
\begin{equation}
\varphi = \dfrac{Ze}{r}e^{-r/r_D},
\end{equation}
где $r_D = \sqrt{\dfrac{kT_e}{4\pi n e^2}}$ -- \textit{радиус Дебая}. Число ионов в сфере такого радиуса:
\begin{equation}
N_D = \dfrac{4}{3}\pi r_D^3 n.
\end{equation}

Смещение электронов на $x$ создаёт поле, вызывающее колебания:
$$
\dfrac{d^2x}{dt^2}=-\dfrac{n e^2}{m}x.
$$ 
Отсюда \textit{плазменная частота}:
\begin{equation}
\omega_p = \sqrt{\dfrac{4\pi ne^2}{m}}.
\end{equation}

\subsection*{Одиночный зонд}
При внесении в плазму зонда на него поступают токи:
\begin{equation}
\begin{array}{c}
I_{e0} = \dfrac{n \langle v_e \rangle}{4}eS,\\
I_{i0} = \dfrac{n \langle v_i \rangle}{4}eS.
\end{array}
\end{equation}
Скорости электронов много больше, поэтому $I_{i0} \ll I_{e0}$. Зонд заряжается до плавающего потенциала $-U_f$.

\begin{wrapfigure}{r}{5.5cm}
    \centering
    \includegraphics[scale=0.5]{3.png}
\end{wrapfigure}
Электронный ток:
$$
I_e = I_0 \exp\left( -\dfrac{eU_f}{kT_e} \right).
$$
При подаче на зонд потенциала $U_\text{з}$ измеряют ток $I_\text{з}$. Максимальный ток $I_{e\text{н}}$ -- электронный ток насыщения, минимальный $I_{i\text{н}}$ -- ионный. По формуле Бома:
\begin{equation}
I_{i\text{н}} = 0,4 neS \sqrt{\dfrac{2kT_e}{m_i}}.
\end{equation}

\subsection*{Двойной зонд}
Двойной зонд -- два одинаковых зонда с разностью потенциалов меньше $U_f$. При малых токах ионные токи компенсируются. Для потенциалов:
$$
U_1 = -U_f + \Delta U_1, \quad U_2 = -U_f + \Delta U_2.
$$
Токи через зонды:
\begin{equation}
I_1 = I_{i\text{н}}\left(1 - \exp\left( \dfrac{e\Delta U_1}{kT_e} \right)\right),
\quad
I_2 = I_{i\text{н}}\left(1 - \exp\left( \dfrac{e\Delta U_2}{kT_e} \right)\right).
\end{equation}
При $I_1 = -I_2 = I$ получаем:
\begin{equation}
U = \dfrac{kT_e}{e}\ln\dfrac{1 - I/I_{i\text{н}}}{1 + I/I_{i\text{н}}}, 
\quad
I = I_{i\text{н}} \text{th}\dfrac{eU}{2kT_e}.
\end{equation}
Реальная зависимость:
\begin{equation}
I = I_{i\text{н}} \text{th}\dfrac{eU}{2kT_e} + AU.
\end{equation}

\section*{Экспериментальная установка}
\begin{center}
\includegraphics[scale=0.6]{1.png}
\end{center}
Газоразрядная трубка с холодным катодом наполнена неоном при давлении 2 мм рт. ст. Катод и анод подключаются через балластный резистор $R_\text{б} \approx 450$ кОм к ВИП (до 5 кВ). Ток разряда измеряется миллиамперметром $A_1$, напряжение на трубке -- вольтметром $V_1$ через делитель с коэффициентом 10.% С учётом сопротивления вольтметра $R_V = 10~\mathrm{M\Omega}$, показания занижены в $\alpha = 12.3$ раза.

При подключении к аноду-II разряд возникает в области с двойным зондом. Зонды из молибденовой проволоки ($d=0,2$ мм, $l=5,2$ мм) подключены к источнику питания через потенциометр. Напряжение измеряется вольтметром $V_2$, ток -- мультиметром $A_2$.



\section*{Ход работы}
\subsection*{Вольт-амперная характеристика разряда}
\begin{enumerate}
	\item После ознакомления с теорией и установкой было определено напряжение зажигания разряда при плавном увеличении напряжения ВИП, которое составило $U_{\text{заж}} \approx 150$ В.
	\item После этого с помощью вольтметра 1 и амперметра 1 была снята ВАХ разряда, которая представлена на графике:
	\begin{figure}[H]
    \centering
    \includegraphics[width=1\textwidth]{ВАХ.PDF}
    \caption{Вольт-амперная характеристика разряда с погрешностями}
\end{figure}
	\item По минимальному углу наклона участка кривой, соответствующего поднормальному тлеющему газовому разряду, определено максимальное дифференциальное сопротивление разряда $R_{\text{}} = (2,1 \pm 0,3)$ кОм.

\end{enumerate}
	\newpage

\subsection*{Зондовые характеристики}
\begin{enumerate}
	\item С помощью мультиметров $A_2$ и $V_2$ были сняты ВАХ двойного зонда при фиксированных силах тока разряда 5,167 мА, 3,048 мА и 1,672 мА.
	По полученым зависимостям были построены графики зависимости $I(U)$, а также определены токи насыщения $I_{\text{iн}}$ из пересечений асимптоты к току насыщения с осью $U = 0$.
	\begin{figure}[H]
    \centering
    \includegraphics[width=0.8\textwidth]{5.PDF}
    \caption{ВАХ зонда при токе разряда $I_{\text{р}} = 5,167$ мА}
\end{figure}

	\begin{figure}[H]
    \centering
    \includegraphics[width=0.8\textwidth]{3.PDF}
    \caption{ВАХ зонда при токе разряда $I_{\text{р}} = 3,048$ мА}
\end{figure}

	\begin{figure}[H]
    \centering
    \includegraphics[width=0.8\textwidth]{1,5.PDF}
    \caption{ВАХ зонда при токе разряда $I_{\text{р}} = 1,672$ мА}
\end{figure}

\item Для каждой зависимости были дополнительно определены $\dfrac{dI}{dU} \Big|_{U=0}$ -- наклоны характеристик в начале координат.
По полученым значениям с помощью следующей формулы была определена температура электронов в каждом случае:

\[ kT_e = \dfrac{1}{2} \dfrac{eI_{\text{iн}}}{ \frac{dI}{dU} \Big|_{U=0}} \]

\begin{table}[H]
\centering
\begin{tabular}{|c|c|c|c|c|}
\hline
$I_p$, мА & $I_{\text{iн}}$, мкА & $\frac{dI}{dU} |_{U=0}$, мкА/В & $T_e$, $10^3$ К & $kT_e$, эВ \\ \hline
5,167   & $15,9\pm 0,7$ & $5,15\pm 0,03$ & $17,90\pm 0,79$ & $1,54\pm 0,07$ \\ \hline
3,048   & $10,2\pm 0,3$ & $2,85\pm 0,02$ & $20,75\pm 0,53$ & $1,79\pm 0,05$ \\ \hline
1,672   & $6,1\pm 0,1$  & $1,58\pm 0,01$ & $22,38\pm 0,49$ & $1,93\pm0,04 $ \\ \hline
\end{tabular}
\end{table}

\item Из точек пересечения горизонталей, проведённых из точек пересечения асимптот к току насыщения с осью $U = 0$, с кривой зависимости были определены значения $\Delta U$, с помощью которых были определены энергии и температуры электронов:
\[ kT_e = \dfrac{\Delta U}{2} \quad (\text{эВ})\]

\begin{table}[H]
\centering
\begin{tabular}{|c|c|c|c|}
\hline
$I_p$, мА & $\Delta U$, В & $kT_e$, эВ & $T_e$, $10^3$ К \\ \hline
5,167   & $4,95\pm 0,21$ & $2,48\pm 0,11$ & $28,78\pm 1,28$ \\ \hline
3,048   & $5,54\pm 0,16$ & $2,77\pm 0,08$ & $32,14\pm 0,93$ \\ \hline
1,672   & $5,66\pm 0,09$  & $2,83\pm 0,05$ & $32,84\pm 0,58$ \\ \hline
\end{tabular}
\end{table}

\item Все три зондовые характеристики были изображены на одном графике:
	\begin{figure}[H]
    \centering
    \includegraphics[width=0.8\textwidth]{ВАХ общая.PDF}
    \caption{ВАХ двойного зонда для трёх рассматриваемых токов разряда}
\end{figure}

\item По формуле Бома была рассчитана концентрация электронов $n_e$, исходя из предположения, что она равна концентрации ионов $n_i$:
\[ I_{\text{iн}} = 0{,}4 n_e e S \sqrt{\dfrac{2kT_e}{m_i}} \qquad \Rightarrow \qquad n_e = \frac{I_{\text{iн}}}{0{,}4 e S \sqrt{\dfrac{2kT_e}{m_i}}} \]

Здесь $m_i = 22 \cdot 1,66 \cdot 10^{-27}$ кг -- масса иона неона, $S = \pi d l \approx 3,27\cdot10^{-6} $ м$^2$ -- площадь поверхности зонда.

\item Была рассчитана плазменная частота колебаний электронов по формуле
\[ \omega_p = \sqrt{\frac{4\pi n_e e^2}{m_e}} = 5,6 \cdot 10^4 \sqrt{n_e} \ \text{рад/сек} \qquad \text{[СГС]} \] 

\item Рассчитана электронная поляризационная длина $r_{De}$ по формуле
\[r_{De} = \sqrt{\frac{k T_e}{4\pi n_e e^2}} \ \text{см} \qquad \text{[СГС]}\]
% \[r_{De} = \sqrt{\frac{\varepsilon_0 k T_e}{n_e e^2}} \ \text{м} \qquad \text{[СИ]}\]

\item Рассчитан дебаевский радиус экранирования $r_D$, учитывая, что $T_e >> T_i$, а температура ионов равна комнатной ($T_i \simeq 300$ К).
\[r_D = \sqrt{\frac{k T_i}{4\pi n_e e^2}} \ \text{см} \qquad \text{[СГС]}\]

\item Оценено среднее число ионов в дебаевской сфере
\[N_D = \frac{4}{3} \pi r_D^3 n_i \qquad \text{[СГС]} \]

\item Оценена степень ионизации плазмы $\alpha$ при условии того, что давление в трубке $P \simeq 2$ Торр:
\[ \alpha = \dfrac{n_i}{n}, \qquad n = \dfrac{P}{kT_i}\]

Все результаты представлены в таблице:

\begin{table}[H]
\centering
\begin{tabular}{|c|c|c|c|c|c|c|}
\hline
$I_p$, мА & $n_e$, $10^{15}$ м$^{-3}$ & $\omega_p$, $10^9$ рад/c & $r_{De}$, $10^{-5}$ м & $r_D$, $10^{-6}$ м & $N_D$ & $\alpha$, $10^{-8}$ \\ \hline
5,167 & $16,3\pm 0,8$  & $7,14\pm 0,25$ & $9,2\pm 0,5$ & $9,4\pm 0,5$ & $57\pm 5$ & 3,4\\ \hline
3,048 & $9,89\pm 0,35$ & $5,57\pm 0,14$ & $12,4\pm 0,4$ & $12,0\pm 0,4$ & $78\pm 4$ & 2,0\\ \hline
1,672 & $5,85\pm 0,12$ & $4,28\pm 0,06$ & $16,4\pm 0,3$ & $15,6 \pm 0,3$ & $93\pm 3$ & 1,2\\ \hline
\end{tabular}
\end{table}

\item Построены графики зависимостей электронной температуры и концентрации от тока разряда. На втором графике наблюдается линейная зависимость:
	\begin{figure}[H]
    \centering
    \includegraphics[width=0.8\textwidth]{ТИ.PDF}
    \caption{зависимость электронной температуры от тока разряда}
\end{figure}
	\begin{figure}[H]
    \centering
    \includegraphics[width=0.8\textwidth]{НИ.PDF}
    \caption{зависимость электронной концентрации от тока разряда}
\end{figure}
\end{enumerate}

\section*{Выводы}
В результате выполнения работы были сняты вольтамперные и зондовые характеристики тлеющего разряда, рассчитаны электронные температуры при разных токах разряда, а также другие основные параметры плазмы, оценены их погрешности. Все полученные значения выглядят достоверно.

Поскольку линейные размеры плазмы $>>$ её поляризационной длины, плазму можно считать квазинейтральной.

Полученные числа Дебая $>1$, поэтому плазму можно рассматривать как идеальный газ.














\end{document}