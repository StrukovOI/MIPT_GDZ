\documentclass[a4paper, 12pt]{article}

\usepackage{wrapfig}
\usepackage{graphicx}
\usepackage{mathtext}
\usepackage{amsmath}
\usepackage{siunitx}
\usepackage{multirow}
\usepackage{rotating}
\usepackage{float}

\usepackage[T1,T2A]{fontenc}

\usepackage[russian]{babel}
\usepackage{amsfonts} 
\graphicspath{{pictures/}}
\usepackage[a4paper,left=25mm,right=25mm,top=20mm,bottom=20mm]{geometry} % устанавливает поля документа

\title{\begin{center}Лабораторная работа №1.4.8\end{center}
Измерение модуля Юнга методом акустического резонанса}
\author{Струков О. И. \\
Б04-404}
\date{}

\begin{document}
    \pagenumbering{gobble}
    \maketitle
    \newpage
    \pagenumbering{arabic}
    
    \section{Цель работы}
    \begin{enumerate}
        \item Исследовать явление акустического резонанса в тонком стержне.
        \item Измерить скорость распространения продольных звуковых колебаний в тонких стержнях из различных материалов и различных размеров.
        \item Измерить модули Юнга различных материалов.
    \end{enumerate}
    
    \section{Оборудование}
    \begin{enumerate}
        \item Генератор звуковых частот.
        \item Частотомер.
        \item Осциллограф.
        \item Электромагнитные излучатель и приёмник колебаний.
        \item Набор стержней из различных материалов.
        \item Линейка, штангенциркуль, микрометр.
        \item Весы.
    \end{enumerate}

    \section{Введение}
        Акустическая волна, распространяющаяся в стержне конечной длины $L$,
        испытает отражение от торцов стержня. Если при этом на длине стержня
        укладывается целое число полуволн, то отражённые волны будут складываться в фазе с падающими, что приведёт к резкому усилению амплитуды
        их колебаний и возникновению акустического резонанса в стержне. Измеряя соответствующие резонансные частоты, можно определить скорость
        звуковой волны в стержне и, таким образом, измерить модуль Юнга материала стержня. Акустический метод является одним из наиболее точных
        методов определения упругих характеристик твёрдых тел.
    \section{Уравнение волны в тонком стержне}
    \begin{minipage}{0.6\textwidth}
        Направим ось $X$ вдоль геометрической оси стержня (рис. 1). Разобьём
        исходно недеформированный стержень на тонкие слои толщиной $\Delta x$. При
        продольной деформации среды границы слоёв сместятся в некоторые новые положения. Пусть плоскость среды, находящаяся исходно в точке $x$,
        сместилась к моменту $t$ на расстояние $\xi(x,t)$. Тогда слой, занимавший исходно отрезок $[x; x + \Delta x]$, изменил свой продольный размер на величину $\Delta \xi = \xi(x + \Delta x, t) - \xi(x, t)$.
    \end{minipage}
    \hfill
    \begin{minipage}{0.35\textwidth}
        \includegraphics[scale=0.7]{picture1.png}
    \end{minipage}
        
        \begin{equation}
            \varepsilon = \frac{\partial\xi}{\partial x}
        \end{equation}

        Далее, согласно закону Гука, имеем:
        
        \begin{equation}
            \sigma = \varepsilon E = E \frac{\partial\xi}{\partial x}
        \end{equation}
        
        Здесь напряжение равно $\sigma = \frac{F}{S}$, где 
        
        $F$ — продольная сила, действующая на элементарный участок $\Delta x$, 
        
        $S$ - площадь поперечного сечения стержня. 

        Напряжения, действующие на стенки рассматриваемого элемента в сечениях $x$ и $x + \Delta x$, будут различными. Из-за этого возникнет результирующая возвращающая сила, стремящаяся вернуть элемент стержня в исходное (недеформированное и несмещённое) состояние:
        \begin{equation}
            \Delta F = S \sigma(x + \Delta x) - S \sigma(x) = \frac{\partial\sigma}{\partial x} S\Delta x = \frac{\partial^2\xi}{\partial^2 x} ES\Delta x
        \end{equation}

        Эта сила вызовет ускорение движение элемента стержня массой $\Delta m = S \rho \Delta x$ вдоль оси $X$. Ускорение рассматриваемого элемента — это вторая производная по времени от смещения его границ:
        \begin{equation}
            a = \frac{\partial^2 \xi}{\partial t^2}
        \end{equation}

        Тогда, используя 2-й закон Ньютона:
        \begin{equation}
            \Delta m \cdot a = \Delta F
        \end{equation}

        и соотношения (1) - (4), получим уравнение движения среды:

        \begin{equation}
            S \rho \Delta x \frac{\partial^2 \xi}{\partial t^2} = ES\Delta x \frac{\partial^2\xi}{\partial^2 x}
        \end{equation}

        Cкорость $u$ распространения продольной акустической волны в простейшем случае длинного тонкого стержня определяется соотношением:
        \begin{equation}
            u = \sqrt{\frac{E}{\rho}}
        \end{equation}

        Теперь, используя соотношения (6) - (7), мы можем записать волновое уравнение:
        \begin{equation}
            \frac{\partial^2 \xi}{\partial t^2} = u^2 \frac{\partial^2\xi}{\partial^2 x}
        \end{equation}
        
        Оно имеет универсальный характер и описывает волны самой разной природы: акустические волны в твёрдых телах, жидкостях и газа, волны на струне, электромагнитные волны и т.п. Величина $u$ в уравнении (6) имеет смысл скорости распространения волны.

    \section{Собственные колебания стержня. Стоячие волны}
        В случае гармонического возбуждения колебаний с частотой $f$ продольная волна в тонком стержне может быть представлена в виде суперпозиции двух бегущих навстречу гармонических волн:

        \begin{equation}
            \xi(x, t) = A_1 (\sin{\omega t - kx + \phi_1}) + A_2 (\sin{\omega t - kx + \phi_2}),
        \end{equation}
        где $\omega = 2 \pi f$ — циклическая частота. Коэффициент $k = \frac{2 \pi}{\lambda}$ называют
        волновым числом или пространственной частотой волны.
        
        Пусть концы стержня не закреплены. Тогда напряжения в них должны равняться нулю. Положим координаты торцов равными $x = 0$ и $x = L$. Тогда, используя связь напряжения и деформации (2), запишем граничные условия для свободных (незакреплённых) концов стержня:

        \begin{equation}
            x = 0: \sigma(0) = 0 \rightarrow \frac{\partial\xi}{\partial x} = 0
        \end{equation}
        \begin{equation}
            x = L: \sigma(L) = 0 \rightarrow \frac{\partial\xi}{\partial x} = 0
        \end{equation}

        Нетрудно видеть, что это соотношение будет выполняться при любом $t$, если только у «падающей» и «отражённой» волн одинаковы амплитуды 

        \begin{equation}
            A_1 = A_2
        \end{equation}
        и фазы.
        \begin{equation}
            \phi_1 = \phi_2
        \end{equation}
        
        Далее, перепишем исследуемую функцию (9), используя граничные условия (12) и (13) и формулу суммы синусов:

        \begin{equation}
            \xi(x, t) = 2A \cos(kx)sin(\omega t + \phi)
        \end{equation}

        Колебания вида (14) называют гармоническими стоячими волнами.

        Наконец, воспользуемся вторым граничным условием (9) применительно к функции (12). В результате придём к уравнению $\sin(kL) = 0$, решения которого определяют набор допустимых значений волновых чисел $k$:

        \begin{equation}
            k_n L = \pi n, \quad n = 1, 2, 3,...,
        \end{equation}
        
        или, выражая (13) через длину волны $\lambda = \frac{2 \pi}{k},$, получим

        \begin{equation}
            \lambda_n = \frac{2L}{n}, \quad n \in \mathbb{N}
        \end{equation}
        Таким образом, для возбуждения стоячей волны на длине стержня должно укладываться целое число полуволн.
        
        Допустимые значения частот:
        \begin{equation}
            f_n = \frac{u}{\lambda_n} = n \frac{u}{2L}, \quad n \in \mathbb{N}
        \end{equation}
        называют собственными частотами колебаний стержня длиной $L$. Именно при совпадении внешней частоты с одной из частот $f_n$ в стержне возникает акустический резонанс.
        
        \includegraphics[scale=0.75]{picture2.png}
        Зависимость амплитуды смещения от координаты для собственных колебаний стержня с незакреплёнными концами при $n = 1,2,3$ представлена на рис. 2. 

    \section{Схема}

    \includegraphics[scale=0.75]{picture3.png}
    
    \section{Ход работы}
    \begin{enumerate}

    \item Проведено ознакомление с основными органами управления электронного осциллографа. По техническому описанию к работе проведена предварительная настройка осциллографа и звукового генератора

    \item На подставку 10 между датчиками был помещён медный стержень
    \item Электромагниты размещены напротив торцов стержня так, чтобы торцы стержня совпали с центрами датчиков, а зазор между полюсами электромагнита и торцевой поверхностью стержней составлял 1–3 мм. Плоскость магнитов была расположена строго перпендикулярна оси стержня. Таким образом, $f_1\approx 3100$ Гц.
    \item С помощью медленных перестраиваний звукового генератора вблизи рассчитанной частоты $f_1 \approx 3100$ Гц и наблюдений за амплитудой колебаний на экране осциллографа был найден первый резонанс. При приближении к резонансу амплитуда сигнала с регистрирующего датчика (канал CH2) резко возрастала, а амплитуда опорного сигнала (канал CH1) не менялась. Для увеличения сигнала колебаний стержня датчики были осторожно придвинуты к торцам стержня.
    \item Было определено точное значение первой резонансной частоты $f_1 = 3250,6$ Гц
    \item Были получены резонансы на частотах, соответствующих следующим (кратным) гармоникам. Для этого, плавно перестраивая генератор, добейтесь резонанса вблизи частот $f_n \approx nf_1$, где n = 2,3,…,7. Результаты представлены в таблице 1.
    \item Для определения плотности стержня было проведено несколько измерений размеров и массы различных образцов, изготовленных из того же металла, что и стержень. Результаты представлены в таблице 2. Объём и плотность образцов, а также их погрешность можно определить с помощью следующих формул:
\[V = \frac{\pi d^2 l}{4},      \rho = \frac{m}{V}\]
\[\sigma_V = V\sqrt{\left( \dfrac{\sigma_l}{l} \right)^2 + 4\left( \dfrac{\sigma_{d}}{d} \right)^2 }\]
\[\sigma_{\rho} = \rho\sqrt{\left( \dfrac{\sigma_m}{m} \right)^2 + \left( \dfrac{\sigma_V}{V} \right)^2 }\]
$\sigma_m = 0,001 \text{ г}, \sigma_d = 0,01 \text{ мм}, \sigma_l = 0,1 \text{ мм}$.

    \item Было установлено, что стержень является тонким, поскольку $\frac{R}{\lambda}\ll1$, где R - радиус стержня, $\lambda$ - длина звуковой волны.
    \item Пункты 2-9 были проведены с двумя другими стержнями из стали и дюралюминия с использованием двух разных генераторов колебаний.
    \item Для каждого из исследованных стержней по результатам измерений п. 2–9 были построены графики зависимости частоты $f(n)$ от номера гармоники $n$ (Рис. 1) Каждая зависимость действительно является линейной и проходит через начало координат.
    \item Были построены наилучшие прямые по экспериментальным точкам и определены соответствующие значения скорости звука $u = 2L\frac{f_n}{n}$ с помощью МНК, а также оценена погрешность. Результаты занесены в таблицу 3.
    \item Определены модули Юнга $E$ исследуемых материалов и оценены погрешности результатов по следующим формулам:
    \[E = \rho\cdot(2Lf)^2 = \rho u^2, \sigma_E = E\sqrt{4\left( \dfrac{\sigma_{u}}{u} \right)^2 + \left( \dfrac{\sigma_{\rho}}{\rho} \right)^2 }\]
    Результаты добавлены в таблицу 3.
    \item Для стержня из дюралюминия был проведён дополнительный опыт: перестраивая генератор, я добился возбуждения первой гармоники $f_1$ резонансных колебаний в стержне при «половинной» частоте генератора $f = f_1/2$. На экране осциллографа (в режиме работы «X–Y») удалось получить фигуру Лиссажу, похожую на символ $ \infty $.
    \item Для стержня из дюралюминия вблизи его первого резонанса была измерена АЧХ для определения его добротности как колебательной системы. Результаты измерений занесены в таблицу 4. Значению $A_{max}/\sqrt{2}$ соотвествует $8/\sqrt{2}\approx5,8\text{ делений}$, тогда добротность $Q = f_1/\Delta f = 4253,34/(4257,03 - 4252,28) \approx 895,44$

    
    \begin{table}[h]
        \centering
        \begin{tabular}{|c|c|c|c|}
            \hline
            & Медь & Сталь & Дюралюминий \\ \hline
            n & \multicolumn{3}{c|}{Частота $f_n, \text{ Гц}$} \\ \hline
            1 & 3251 & 4127 & 4253 \\ \hline
            2 & 6483 & 8263 & 8513 \\ \hline
            3 & 9740 & 12391 & 12767 \\ \hline
            4 & 12972 & 16515 & 17021 \\ \hline
            5 & 16512 & 20644 & 21278 \\ \hline
            6 & 19669 & 24758 & 25491 \\ \hline
            7 & 22753 & 28887 & 29746 \\ \hline
        \end{tabular}
        \caption{Резонансы стержней на различных частотах}
    \end{table}

    \begin{table}[h]
        \centering
        \begin{tabular}{|c|c|c|c|c|c|c|c|c|}
            \hline
            \multicolumn{3}{|c|}{Медь}&\multicolumn{3}{c|}{Сталь}&\multicolumn{3}{c|}{Дюралюминий} \\ \hline
            $d$, мм & $l$, мм & $m$, г  & $d$, мм & $l$, мм & $m$, г & $d$, мм & $l$, мм & $m$, г \\ \hline
            11,64 & 29,8 & 29,114 & 11,98 & 29,5 & 26,030 & 11,47 & 30,0 & 8,994 \\ \hline
            11,70 & 30,0 & 29,457 & 11,95 & 29,8 & 26,159 & 12,09 & 30,0 & 9,491 \\ \hline
            11,90 & 30,3 & 30,117 & 11,69 & 31,2 & 28,115 & 11,71 & 30,1 & 9,195 \\ \hline
            11,82 & 39,7 & 39,392 & 11,99 & 39,5 & 34,950 & 11,46 & 30,8 & 9,265 \\ \hline
            12,24 & 40,0 & 40,997 & 11,98 & 39,8 & 35,153 & 11,66 & 40,1 & 12,181 \\ \hline
            11,48 & 40,3 & 38,722 & 11,98 & 40,0 & 35,197 & 12,57 & 41,2 & 13,238 \\ \hline
            11,91 & 40,4 & 40,356 & 12,34 & 40,8 & 36,931 & 11,49 & 42,3 & 12,456 \\ \hline
            11,88 & 41,4 & 41,346 & 12,14 & 41,2 & 37,096 & 11,50 & 42,5 & 12,487 \\ \hline
            \multicolumn{9}{|c|}{Средняя плотность, $\rho_{ср} \pm \sigma_{\rho}$, кг/м$^3$} \\ \hline
            \multicolumn{3}{|c|}{$9032 \pm 29$}&\multicolumn{3}{c|}{$7818 \pm 25$}&\multicolumn{3}{c|}{$2846 \pm 9$} \\ \hline
        \end{tabular}
        \caption{Нахождение плотности стержней}
    \end{table}

    \begin{table}[h]
        \centering
        \begin{tabular}{|c|c|c|c|}
            \hline
            \multicolumn{1}{|l|}{} & Медь  & Сталь & Дюралюминий \\ \hline
            $L, \text{ м}$ & 0,596 & 0,605 & 0,602 \\ \hline
            $\sigma_L, \text{ м}$ & 0,002 & 0,002 & 0,002 \\ \hline
            $u,$ м/c             & 3901  & 4992  & 5115 \\ \hline
            $\sigma_u$, м/c      & 17    & 22    & 24   \\ \hline
            $E$, Па & $137,4\cdot10^{9}$ & $194,8\cdot10^9$ & $74,5\cdot10^9$\\ \hline
            $\sigma_E$, Па & $1,3\cdot10^9$ & $1,8\cdot10^9$ & $0,7\cdot10^9$\\ \hline
        \end{tabular}
        \caption{Длины стержней, скорость звука в них и их модули Юнга}
    \end{table}

    \begin{table}[h]
        \centering
        \begin{tabular}{|c|c|c|}
            \hline
            U, кол-во делений & Частота при понижении, Гц & Частота при повышении, Гц \\ \hline
            8 & 4253,12 & 4256,2 \\ \hline
            7,8 & 4253,06 & 4256,25 \\ \hline
            7,6 & 4253,01 & 4256,3 \\ \hline
            7,4 & 4252,96 & 4256,36 \\ \hline
            7,2 & 4252,87 & 4256,46 \\ \hline
            7 & 4252,81 & 4256,53 \\ \hline
            6,8 & 4252,72 & 4256,6 \\ \hline
            6,6 & 4252,66 & 4256,65 \\ \hline
            6,4 & 4252,54 & 4256,76 \\ \hline
            6,2 & 4252,46 & 4256,8 \\ \hline
            6 & 4252,39 & 4256,9 \\ \hline
            5,8 & 4252,28 & 4257,03 \\ \hline
            5,6 & 4252,14 & 4257,11 \\ \hline
            5,4 & 4252,07 & 4257,21 \\ \hline
            5,2 & 4251,93 & 4257,35 \\ \hline
            5 & 4251,83 & 4257,46 \\ \hline
        \end{tabular}
        \caption{АЧХ дюралюминиевого стержня}
    \end{table}

    \begin{figure}[h]
        \centering
        \includegraphics[scale=1.00]{График.pdf}
        \caption{График зависимости $f_n(n)$}
        \label{fig:enter-label}
    \end{figure}

    \end{enumerate}

\end{document}