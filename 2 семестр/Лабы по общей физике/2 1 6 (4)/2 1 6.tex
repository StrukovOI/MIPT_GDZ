\documentclass[a4paper,12pt]{article} % добавить leqno в [] для нумерации слева
\usepackage[a4paper,top=1.3cm,bottom=2cm,left=1.5cm,right=1.5cm,marginparwidth=0.75cm]{geometry}
%%% Работа с русским языком
\usepackage{cmap}					% поиск в PDF
\usepackage{mathtext} 				% русские буквы в фомулах
\usepackage[T2A]{fontenc}			% кодировка
\usepackage[utf8]{inputenc}			% кодировка исходного текста
\usepackage[english,russian]{babel}	% локализация и переносы
\usepackage{multirow}

\usepackage{graphicx}

\usepackage{wrapfig}
\usepackage{tabularx}

% \usepackage{hyperref}
% \usepackage[rgb]{xcolor}
% \hypersetup{
% colorlinks=true,urlcolor=blue
% }

%%% Дополнительная работа с математикой
\usepackage{amsmath,amsfonts,amssymb,amsthm,mathtools} % AMS
\usepackage{icomma} % "Умная" запятая: $0,2$ --- число, $0, 2$ --- перечисление

%% Номера формул
\mathtoolsset{showonlyrefs=true} % Показывать номера только у тех формул, на которые есть \eqref{} в тексте.

%% Шрифты
\usepackage{euscript}	 % Шрифт Евклид
\usepackage{mathrsfs} % Красивый матшрифт

%% Свои команды
\DeclareMathOperator{\sgn}{\mathop{sgn}}

%% Перенос знаков в формулах (по Львовскому)
\newcommand*{\hm}[1]{#1\nobreak\discretionary{}
{\hbox{$\mathsurround=0pt #1$}}{}}

%% Графики
% \usepackage{tikz}
% \usepackage{pgfplots}
% \pgfplotsset{compat=1.9}

\title{\begin{center}Лабораторная работа №2.1.6\end{center}
Эффект Джоуля-Томсона}
\author{Струков Олег \\ Б04-404}
\date{}
\begin{document}

\pagenumbering{gobble}
\maketitle
\newpage
\pagenumbering{arabic}
% \section*{Аннотация}

% В данной работе исследован эффект Джоуля-Томсона при протекании углекислого газа через малопроницаемую перегородку. Измерено изменение температуры газа вследствие данного эффекта при разных начальных значениях температуры и давления. По результатам измерений найдены коэффициенты в уравнении состояния углекислого газа как газа Ван-дер-Ваальса. Вычислены и проанализированы погрешности измерений.
\textbf{Цель работы:} 1) определить изменения температуры углекислого газа при протека
нии через малопроницаемую перегородку при разных начальных значениях давления
и температуры; 2) вычислить по результатам опытов коэффициенты a и b модели Ван
дер-Ваальса.

\textbf{Оборудование:} трубка с пористой перегородкой; труба Дьюара; термостат; термометры; дифференциальная термопара; микровольтметр; балластный баллон; манометр.

\section*{Теоретическая часть}

\subsection*{Эффект Джоуля-Томсона для газа Ван-дер-Ваальса}

Эффектом Джоуля–Томсона называется изменение температуры газа, медленно протекающего из области высокого в область низкого давления в условиях хорошей тепловой изоляции.
В разреженных газах, которые приближаются по своим свойствам к идеальному газу, при таком течении температура газа не меняется. Эффект Джоуля–Томсона демонстрирует отличие исследуемого газа от идеального.

\begin{figure}[h]
    \centering
    \includegraphics[width = 0.8\textwidth]{schema.png}
    \caption{Принципиальная схема эффекта Джоуля–Томсона}
    \label{fig:schema}
\end{figure}

Рассмотрим стационарное течение одного моля газа в теплоизолированной трубке постоянного сечения на участках до и после пористой перегородки. Для ввода в трубку начального объёма (молярного) газа $V_1$ необходимо совершить работу $A_1 = P_1 V_1$, где $P_1$ - начальное давление. Проходя через сечение на втором участке, газ сам совершает работу $A_2 = P_2 V_2$, где $P_2$ и $V_2$ - давление и молярный объём газа соответственно после прохождения перегородки. Пусть $U_1$, $U_2$, $v_1$ и $v_2$ - внутренняя энергия и скорости газа до и после прохождения перегородки соответственно. Тогда, учитывая отсутствие теплопотерь,

\begin{equation}
    A_1 - A_2 = (U_2 + \frac{\mu {v_2}^2}{2}) - (U_1 + \frac{\mu {v_1}^2}{2}),
\end{equation}

где $\mu$ - молярная масса газа. Перепишем это соотношение через энтальпию $H = U + PV$:

\begin{equation}
    H_1 - H_2 = \frac{1}{2}\mu(v_2^2 - v_1^2).
\end{equation}

Величина, стоящая в правой части уравнения, в данном опыте, достаточно мала: $v_2 \approx 1,4$ м/с, $v_1 > 0,25v_2$, $\mu_{\text{CO}_2} = 0,044$ кг/моль, откуда $\mu(v_2^2 - v_1^2)/2 < 3 \cdot 10^{-2}$ Дж, поэтому можно считать энтальпию в данном процессе постоянной. 

Рассмотрим дифференциальный эффект Джоуля–Томсона, т. е. когда изменения давления и температуры малы. В этом случае коэффициент Джоуля-Томсона $\mu_\text{д-т}$ равен

\begin{equation}
    \mu_\text{д-т} = (\frac{\partial T}{\partial P})_H.
    \label{koeff}
\end{equation}

Одна из моделей реальных газов - модель газа Ван-дер-Ваальса. Для него уравнения состояния имеет вид

\begin{equation}
    (P - \frac{a}{V^2})(V - b) = RT,
    \label{vdv}
\end{equation}

где $a$ и $b$ - коэффициенты Ван-дер-Ваальса, принятые постоянными. Подставляя данное соотношение в выражение \eqref{koeff}, дифференцируя и используя выражение для теплоёмкости $C_P = (\partial H/\partial T)_P$, после преобразований получим

\begin{equation}
    \mu_\text{д-т} = \frac{(2a/RT) - b}{C_P}.
    \label{mainEq}
\end{equation}

\subsection*{Температура инверсии} Из формулы \eqref{mainEq} видно, что эффект Джоуля–Томсона
для не очень плотного газа зависит от соотношения параметров a и b, которые оказывают противоположное влияние на знак эффекта.
Если силы притяжения между
молекулами велики, то основную роль играет член, содержащий a, и газ при расширении охлаждается: $\Delta T < 0$.
В обратном случае, когда доминирует отталкивание, т.е.
слагаемое b, газ нагревается: $\Delta T > 0$. Видно также, что существует температура
инверсии эффекта Джоуля–Томсона

\begin{equation}
    T_{\text{}} = \frac{2a}{Rb},
    \label{inv}
\end{equation}

при прохождении через которую эффект меняет знак. Газ нагревается ($\mu < 0,~
\Delta T > 0$) при $T > T_{\text{инв}}$ и охлаждается ($\mu > 0,~ \Delta T < 0$) при $T < T_{\text{инв}}$.
Для используемого в работе углекислого газа температура инверсии $T_{\text{инв}} \sim 1500$ К (при $P \sim 1$
атм), и при комнатной температуре он будет охлаждаться. Среди всех газов только
у гелия ($T_{\text{инв}} = 46$ К) и водорода ($T_{\text{инв}} = 205$ К) температура инверсии
значительно ниже комнатной, поэтому они при обычных температурах при дросселированиинагреваются.


% Температура газа $T_i$, при которой в дифференциальном эффекте Джоуля-Томсона отсутствует изменение температуры, называется температурой инверсии:

% \begin{equation}
%     T_i = \frac{2a}{Rb}.
% \end{equation}

% При начальных температурах ниже $T_i$ в результате дросселирования газ будет охлаждаться.

% Объединив выражения для температуры инверсии и критической температуры $T_\text{к} = 8a/27Rb$, получим

% \begin{equation}
%     T_i = \frac{27}{4}T_\text{к}.
% \end{equation}

% Зная табличное значение критической температуры углекислого газа и вычислив температуру инверсии, можно судить о применимости модели газа Ван-дер-Ваальса для количественного изучения реальных газов.

\subsection*{Экспериментальная установка}

\begin{figure}
    \centering
    \includegraphics[width = 0.8\textwidth]{setup.PNG}
    \caption{Схема экспериментальной установки}
    \label{fig:setup}
\end{figure}

Использованная в работе экспериментальная установка изображена на рис. \ref{fig:setup}. Основным элементом
установки является трубка 1 с пористой перегородкой 2, через которую пропускается исследуемый газ. Трубка имеет длину 80 мм и сделана из нержавеющей стали, обладающей, как известно, малой теплопроводностью. Диаметр трубки $d = 3$ мм,
толщина стенок 0,2 мм. Пористая перегородка расположена в конце трубки и представляет собой стеклянную пористую пробку со множеством узких и длинных каналов. Пористость и толщина пробки ($l = 5$ мм) подобраны так, чтобы обеспечить оптимальный поток газа при перепаде давлений $ \Delta P = 4$ атм (расход газа составляет около 10 см$^3$/с); при этом в результате эффекта Джоуля–Томсона создается достаточная разность температур.

Углекислый газ под повышенным давлением поступает в трубку через змеевик 5 из балластного баллона 6. Медный змеевик омывается водой и нагревает медленно протекающий через него газ до температуры воды в термостате. Температура воды измеряется термометром T$_\text{в}$, помещенным в термостате. Требуемая температура воды устанавливается и поддерживается во время эксперимента при помощи контактного термометра T$_\text{к}$. Давление газа в трубке измеряется манометром М и регулируется вентилем В (при открывании вентиля В, т. е. при повороте ручки против часовой стрелки, давление $P_1$ повышается). Манометр М измеряет разность между давлением внутри трубки и наружным (атмосферным) давлением. Так как углекислый газ после пористой перегородки выходит в область с атмосферным давлением $P_2$, то этот манометр непосредственно измеряет перепад давления на входе и на выходе трубки. 

Разность температур газа до перегородки и после неё измеряется дифференциальной термопарой медь — константан. Константановая проволока диаметром 0,1 мм соединяет спаи 8 и 9, а медные проволоки (того же диаметра) подсоединены к цифровому вольтметру 7. Отвод тепла через проволоку столь малого сечения пренебрежимо мал. Таблица зависимости разности температур от разности показаний вольтметра изображена на рис. \ref{graph0} Для уменьшения теплоотвода трубка с пористой перегородкой помещена в трубу Дьюара 3, стенки которой посеребрены, для уменьшения теплоотдачи, связанной с излучением. Для уменьшения теплоотдачи за счет конвекции один конец трубы Дьюара уплотнен кольцом 4, а другой закрыт пробкой 10 из пенопласта. Такая пробка практически не создает перепада давлений между внутренней полостью трубы и атмосферой.

\begin{figure}
    \centering
    \includegraphics[width = 0.8\textwidth]{дв дт.png}
    \caption{Чувствительность медно-константановой термопары}
\label{graph0}
\end{figure}

% \section*{Оборудование и инструментальные погрешности}

% \textbf{В работе использовались:} трубка с пористой перегородкой; труба Дьюара; термостат; термометры; дифференциальная термопара; микровольтметр; балластный баллон; манометр.

% \textbf{Инструментальные погрешности:}

% \begin{itemize}
%     \item \textbf{Термометр в термостате:} $\Delta_T = 0,1$ К;
%     \item \textbf{Термопара:} $\Delta_U = 1,0$ мкВ;
%     \item \textbf{Манометр:} $\Delta_P = 0,05$ атм.
% \end{itemize}


\newpage

\section*{Ход работы}
\begin{enumerate}
    \item В начале работы были включены все приборы. При отсутствии потока газа показания вольтметра составляли $[-0,001, 0,001]$ мВ, поэтому в получаемых значениях их можно не учитывать.
    \item Было проведено ознакомление с манометром: его цена деления составляет 0,1 бар, предел измерений -- 6 бар.
    \item Далее был открыт вентиль В настолько, чтобы избыточное давление составило $\Delta P \approx 4$ бар, после этого было выждано около семи минут для завершения переходных процессов, и когда показания вольтметра перестали изменяться, было записано полученное значение.
    \item После этого устанавливалось давление на 0,3 -- 0,5 бар меньше, выжидалось около трёх - пяти минут и снималось полученное значение напряжения.
    \item Данные действия были проведены шестикратно для четырёх различных температур. Разность потенциалов $\Delta U$ на спаях термопары была переведена в разность температур $\Delta T$ на них с помощью значений градуировочных кривых, с помощью МНК были найдены коэффициенты наклона $\Delta U/\Delta T$ прямых изучаемых зависимостей $\Delta T (\Delta P)$, равные коэффициенту Джоуля-Томсона $\mu$. Результаты представлены в таблицах:
    \begin{table}[!ht]
        \centering
        \begin{tabular}{|c|c|c|c|}
        \hline
            & $\Delta P$, бар & $\Delta U$, мВ & $\Delta T$, К \\ \hline
            Т, $^0C$ & 4,1 & 0,131 & 3,291 \\ \hline
            19 & 3,6 & 0,11 & 2,764 \\ \hline
            $\Delta U/\Delta T$, мкВ/К & 3,1 & 0,094 & 2,362 \\ \hline
            39,8 & 2,6 & 0,074 & 1,859 \\ \hline
            $\mu$, К/бар & 2,1 & 0,055 & 1,382 \\ \hline
            0,922 & 1,4 & 0,032 & 0,804 \\ \hline
        \end{tabular}
    \end{table}

    \begin{table}[!ht]
        \centering
        \begin{tabular}{|c|c|c|c|}
        \hline
            & $\Delta P$, бар & $\Delta U$, мВ & $\Delta T$, К \\ \hline
            Т, $^0C$ & 4,1 & 0,13 & 3,194 \\ \hline
            30 & 3,6 & 0,108 & 2,654 \\ \hline
            $\Delta U/\Delta T$, мкВ/К & 3 & 0,087 & 2,138 \\ \hline
            39,8 & 2,5 & 0,071 & 1,744 \\ \hline
            $\mu$, К/бар & 1,9 & 0,053 & 1,302 \\ \hline
            0,834 & 1,5 & 0,04 & 0,983 \\ \hline
        \end{tabular}
    \end{table}

    \begin{table}[!ht]
        \centering
        \begin{tabular}{|c|c|c|c|}
        \hline
            & $\Delta P$, бар & $\Delta U$, мВ & $\Delta T$, К \\ \hline
            Т, $^0C$ & 4,1 & 0,126 & 3,036 \\ \hline
            40 & 3,5 & 0,104 & 2,506 \\ \hline
            $\Delta U/\Delta T$, мкВ/К & 3,05 & 0,086 & 2,072 \\ \hline
            41,5 & 2,5 & 0,07 & 1,687 \\ \hline
            $\mu$, К/бар & 1,9 & 0,053 & 1,277 \\ \hline
            0,782 & 1,6 & 0,044 & 1,060 \\ \hline
        \end{tabular}
    \end{table}

    \begin{table}[!ht]
        \centering
        \begin{tabular}{|c|c|c|c|}
        \hline
            & $\Delta P$, бар & $\Delta U$, мВ & $\Delta T$, К \\ \hline
            Т, $^0C$ & 4,7 & 0,134 & 3,160 \\ \hline
            50 & 4,15 & 0,116 & 2,736 \\ \hline
            $\Delta U/\Delta T$, мкВ/К & 3,55 & 0,096 & 2,264 \\ \hline
            42,4 & 3,15 & 0,084 & 1,981 \\ \hline
            $\mu$, К/бар & 2,45 & 0,065 & 1,533 \\ \hline
            0,709 & 1,85 & 0,048 & 1,132 \\ \hline
        \end{tabular}
    \end{table}
\newpage
    \item Зависимости $\Delta T (\Delta P)$ для каждой исследуемой температуры были изображены на графике. Видно, что зависимость $\Delta T (\Delta P)$ является линейной.
    \begin{figure}[h]
        \centering
        \includegraphics[width = 1\textwidth]{T(P).pdf}
    \end{figure}

    \item Был построен график зависимости $\mu(1/T)$, полученные точки также лежат на одной прямой.
    \begin{figure}[h]
        \centering
        \includegraphics[width = 1\textwidth]{mu.pdf}
    \end{figure}
    Чтобы определить коэффициенты $a$ и $b$ в уравнении состояния газа Ван-дер-Ваальса, необходимо объединить выражения \eqref{mainEq} для двух температур газа в систему:

\begin{equation*}
 \begin{cases}
   \mu_1 = \dfrac{(2a/RT_1) - b}{C_P}; \\
   \mu_2 = \dfrac{(2a/RT_2) - b}{C_P}.
 \end{cases}
\end{equation*}

Отсюда выводятся выражения для искомых коэффициентов:

\begin{equation*}
 \begin{cases}
   a = \dfrac{C_P R}{2} \cdot \dfrac{\mu_1 - \mu_2}{T_1^{-1} - T_2^{-1}}; \\
   b = \dfrac{C_P(\mu_2T_2-\mu_1T_1)}{T_1-T_2}.
 \end{cases}
\end{equation*}
Также по формуле \eqref{inv} для каждой пары коэффициентов a и b была найдена температура инверсии. В таблице ниже представлены полученные результаты.
\newpage
\begin{table}[!ht]
    \centering
    \begin{tabular}{|c|c|c|c|c|}
    \hline
        $t_1, ~^0C$ & $t_2, ~^0C$ & $a$, Па$\cdot$м$^6$/моль$^2$ & $b$, $10^{-4}$ м$^3$/моль & $T_{\text{инв}}$ \\ \hline
        19 & 30 & 1,09 & 5,57 & 471,2 \\ \hline
        19 & 40 & 0,94 & 4,32 & 523,2 \\ \hline
        19 & 50 & 1,00 & 4,81 & 499,5 \\ \hline
        30 & 40 & 0,76 & 2,94 & 621,4 \\ \hline
        30 & 50 & 0,94 & 4,40 & 516,3 \\ \hline
        40 & 50 & 1,14 & 5,85 & 468,3 \\ \hline
    \end{tabular}
\end{table}



\end{enumerate}
Усреднённые результаты: $a = 0,98$ Па$\cdot$м$^6$/моль$^2$, $b = 4,65\cdot10^{-4}$ м$^3$/моль, $T_{\text{инв}} = 516,7 $ К.

Табличные значения для углекислого газа составляют $a = 0,365$ Па$\cdot$м$^6$/моль$^2$, $b = 4,28\cdot10^{-4}$ м$^3$/моль, $T_{\text{инв}} \approx 2000 $ К.

\section*{Вывод}
В результате выполнения работы был исследован эффект Джоуля-Томсона для углекислого газа, получены зависимости $\Delta T (\Delta P)$ для четырёх различных температур, определены коэффициенты Джоуля-Томсона для каждой из них, найдены коэффициенты a и b уравнения и температура инверсии.

Из полученных значений с табличными более-менее точно совпал только коэффициент уравнения b, и коэффициент а совпал только по порядку.
Температура инверсии от табличной отличается примерно в четыре раза, из чего можно сделать вывод, что данный способ изучения параметров газа при температурах, близких к комнатной и его критической температуре (304,15 К для углекислого газа), не даёт точных результатов. 

Однако данный опыт даёт возможность исследовать газ качественно и убедиться в том, что он действительно охлаждается при дросселировании при температурах меньших его температуры инверсии.







\end{document}