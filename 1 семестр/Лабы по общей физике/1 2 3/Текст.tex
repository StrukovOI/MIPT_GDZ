\documentclass[a4paper]{article}
\usepackage[utf8]{inputenc}
\usepackage[russian,english]{babel}
\usepackage[T2A]{fontenc}
\usepackage[left=10mm, top=20mm, right=18mm, bottom=15mm, footskip=10mm]{geometry}
\usepackage{indentfirst}
\usepackage{amsmath,amssymb}
\usepackage[italicdiff]{physics}
\usepackage{graphicx}
\graphicspath{{images/}}
\DeclareGraphicsExtensions{.pdf,.png,.jpg}
\usepackage{wrapfig}

\usepackage{cellspace, graphicx, makecell}
\DeclareGraphicsExtensions{.pdf,.png,.jpg}

\renewcommand\cellspacetoplimit{3pt}
\renewcommand\cellspacebottomlimit{3pt}
\newcommand\rowincludegraphics[2][]{\raisebox{-0.45\height}{\includegraphics[#1]{#2}}}


\usepackage{caption}
\captionsetup[figure]{name=Рисунок}
\captionsetup[table]{name=Таблица}
  
\date{}
\author{Струков Олег \\
Б04-404}
\title{Работа 1.2.3 \\
Определение моментов инерции твёрдых тел с помощью трифилярного подвеса}



\begin{document}

\maketitle

\newpage
\begin{center}
\textbf{\Large Определение моментов инерции твердых тел с помощью трифилярного подвеса}
\end{center}


	\section{Введение}
	\textbf{Цели работы:} измерение момента инерции тел и сравнение результатов с расчетми по теоретиеским формулам; проверка аддитивноски моментов инерции и справедливости формулы Гюйгенса-Штейнера.\\
	\textbf{Оборудование:} трифилярный подвес, секундомер, счетчик числа колебаний, набор тел, момент инерции которых надлежит измерить (диск, стержень, полный цилиндр и другие).
	
	\section{Теоретические сведения}
	
	\begin{wrapfigure}{l}{10cm}
		\includegraphics[width=0.95\linewidth]{foto.jpg}
		\caption{Физический маятник}\label{risunok}
	\end{wrapfigure}
	
	Для наших целей удобно использовать устройство, показанное на Рис. \ref{risunok} и называемое трифилярным подвесом. Оно состоит из укрепленной на некоторой высоте неподвижной платформы $P$ и подвешенной к ней на трех симметрично расположеных нитях $AA'$, $BB'$ и $CC'$, вращающейся платформы $P'$. 
	
	Чтобы не вызывать дополнительных раскачиваний, лучше поворачивать верхнюю платформу, укрепленную на неподвижной оси. После поворота верхняя платформа остается неподвижной в течение всего процесса колебний. После того, как нижняя платформа $P'$ оказывается повернутой на угол $\varphi$ относительно верхней платформы $P$, вощникает момент сил, стремящийся вернуть нижнюю платформу в положение равновесия, при котором относительный поворот платформ отсутствует. В результате платформа совершает крутильные колебания.
	
	
	\par Инерционность при вращении тела относительно оси определяется моментом инерции тела относительно этой оси. Момент инерции твердого тела относительно неподвижной оси вращения вычисляется по формуле:
	
	\begin{equation}
		I = \int r^2 dm
	\end{equation}
	
	Здесь $r$ -- расстояние элемента массы тела $dm$ от оси вращения. Интегрирование проводится по всей массе тела $m$.
	
	Если пренебречь потерями энергии на трение о воздух и крепление нитей, то уравнение сохранения энергии при коебаниях можно записать следующим образом:
	
	\begin{equation}\label{moment}
		\frac{I \dot{\varphi^2}}{2} + mg(z_0-z) = E
	\end{equation}
	
	Здесь $I$ -- момент инерции платформы вместе с исследуемым телом, $m$ -- масса платформы с телом, $\varphi$ -- угол поворота платформы от положения равновесия системы, $z_0$ -- координата по вертикали центра нижней платформы $O'$  при равновесии ($\varphi = 0$), $z$ -- координата той же точки при некотором угле поворота $\varphi$. Превый член в левой части уравнения -- кинетическач энергия вращения, второй член -- потенциальная энергия в поле тяжести, $E$ -- полная энергия системы (платформы с телом).
	
	Воспользуемся системой координат $x, y, z$, связанной с верхней платформой, как показано на Рис. \ref{risunok}. Координаты верхнего конца одной из нитей подвеса точки $C$ в этой системе -- $(r, 0, 0)$. Нижний конец данной нити $C'$, находящийся на нижней платформе, при равновесии имеет координаты $(R, 0, z_0)$, а при повороте платформы на угол $\varphi$ эта точка переходит в $C''$ с координатами $(Rcos\varphi, Rsin\varphi, z)$. расстояние между точками $C$ и $C''$ равно длине нити, поэтому, после некоторых преобразований, получаем: 
	
	\begin{center}
		\[ (R\cos\phi - r)^2 + R^2\sin^2\phi + z^2 = L^2 \]
		
		\[ z^2 = L^2 - R^2 - r^2 + 2Rr\cos\phi \approx z^2_{0} - 2Rr(1 - \cos\phi) \approx z^2_{0} - Rr\phi^2 \]
		
		\[ z = \sqrt{z^2_{0} - Rr\phi^2} \approx z_{0} - \frac{Rr\phi^2}{2z_{0}}\]
	\end{center}

	Подставляя $z$ в уравнение \eqref{moment}, получаем:
	
	\begin{equation}
		\frac{1}{2}I\dot{\varphi^2} + mg \frac{Rr}{2z_0}\varphi^2 = E
	\end{equation}
	
	Дифференцируя по времени и сокращая на $\dot\varphi$, находим уравнение крутильных колебаний системы:
	
	\begin{equation}
		I\ddot\varphi^2 + mg\frac{Rr}{2z_0}\varphi^2 = 0
	\end{equation}
		
	Производная по времени от $E$ равна нулю, так как потерями на трение, как уже было сказано выше, пренебрегаем.
	
	Решение этого уравнения имеет вид:
	
	\begin{equation}
		\varphi = \varphi_0 sin \left(\sqrt{\frac{mgRr}{Iz_0}}t + \theta\right)
	\end{equation}

	Здесь амплитуда $\varphi_0$ и фаза $\theta$ колебаний определяются начальными условиями. Период кртуильных полебаний нашей системы равен:
	
	\begin{equation}
		T = 2\pi \sqrt{\frac{Iz_0}{mgRr}}
	\end{equation}

	Из формулы для периода получаем:
	
	\begin{equation}\label{momin}
		I = \frac{mgRrT^2}{4 \pi^2z_0} = kmT^2
	\end{equation}
	\noindent где $k = \frac{gRr}{4\pi^2z_0}$ -- величина, постоянная для данной установки.
	При возбуждении крутильных колебаний маятникообразных движений платформы не наблюдается -- устройство функционирует нормально.
	
	При выводе формул мы предполагали, что потери энергии, связанные с трением, малы, то есть мало затухание колебаний. Это значит, что теоретические вычисления будут верны, если выполняется условие:
 	\begin{equation}
		\tau \gg T
	\end{equation}

\section{Ход работы}
\large
\begin{enumerate}

\item Без нагрузки нижней платфомы было установлена пригодность установки для измерений, устройство для возбуждения крутильных колебаний функционирует нормально, при этом не возникают нежелательные маятниковообразные движения платформы, счётчик числа колебаний работает исправно.
\item Проверю, достаточно хорошо ли выполняется условие $T \gg \tau$. 
Для этого посмотрю, насколько изменится амплитуда отклонения пустой платформы после тридцати колебаний. При отклонении на $30^\circ$ она уменьшилась примерно на $5^\circ$, при отклонении на $15^\circ$ - менее чем на $3^\circ$.
Таким образом, можно сделать вывод, что рассматриваемое условие выполняется хорошо и потери в системе достаточно малы.
\item Найду рабочий диапазон амплитуд колебаний. Для этого сначала найду период колебаний пустой платформы при отклонении на $30^\circ$, а затем на $15^\circ$. В первом случае период колебаний составил 4,425 с, во втором - 4,432 с. Погрешность измерения периода с помощью таймера $\sigma_T = 0,03 \text{c}$, поэтому можно считать, что в данном диапазоне период не зависит от угла отклонения, и использовать его в качестве рабочего диапазона амплитуд колебаний.
\item Измерю параметры установки, по ним вычислю константу $k$, необходимую для вычисления момента импульса $I$, и её погрешность $\sigma_k$. $z_0$ найду с помощью теоремы Пифагора, зная радиус диска и расстояние от места крепления проволок к потолку до края диска. Результаты укажу в таблице 1.

\[z_0 = \sqrt{a^2 - b^2} \]
\[\sigma_{z_0} = z_0\sqrt{\left( \dfrac{\sigma_{a}}{a} \right)^2 + \left( \dfrac{\sigma_{b}}{b} \right)^2 }\]
\[k = \frac{gRr}{4\pi^2z_0}\]
\[\sigma_k = k\sqrt{\left( \dfrac{\sigma_{R}}{R} \right)^2 + \left( \dfrac{\sigma_{r}}{r} \right)^2 + \left( \dfrac{\sigma_{z_0}}{z_0} \right)^2 }\]

\item Определю момент инерции ненагруженной платформы
\[I_0 = km_0T^2\]
\[\sigma_{I_0} = I_0\sqrt{\left( \dfrac{\sigma_{k}}{k} \right)^2 + \left( \dfrac{\sigma_{m_0}}{m_0} \right)^2 + \left( \dfrac{2\sigma_{T}}{T} \right)^2 }\]

\item Проверю аддитивность моментов инерции. Для этого сначала вычислю моменты инерции для тел № 1 и 2 по отдельности, а затем вместе. Внесу характеристики тел и результаты измерений и вычислений в таблицу 2.
Момент инерции и его погрешность расчитаю по формулам:
		\[I = k(m_0+m)T^2 - I_0\]
		\[\sigma_I = \sigma_{I_0} + \sigma_{I}\]
		
		Как видно, все измеренные моменты инерции $I_i$ не выходят за пределы погрешности $\sigma_{I_i}$.
		
Затем рассчитаю теоретические значения моментов инерции тел и добавлю их в ту же таблицу:
		\[I_1 = \frac{1}{2}m\left(r_1^2+r_2^2\right) = \frac{1}{2}m\left(\left(\frac{D - d}{2}\right)^2 + \left(\frac{D}{2}\right)^2\right)\]
		\[m_1 = m\dfrac{V_1}{V} = m\dfrac{d^2H}{d^2H+D^2h}\]
		\[m_2 = m\dfrac{V_2}{V} = m\dfrac{D^2h}{d^2H+D^2h}\]
		\[I_2 = \frac{1}{8}m_1d^2 + \frac{1}{8}m_2D^2 = \dfrac{1}{8}m\dfrac{d^4H + D^4h}{d^2H+D^2h}\] 
		
		Измерю момент инерции тел 1 и 2 вместе, результаты добавлю в таблицу 2. Как видно из результатов, аддитивность моментов инерции соблюдается, значение лежит в пределе допустимой погрешности:
		\[I_{1+2} = (6,87 \pm 0,42) \cdot 10^{-3} \text{ кг}\cdot\text{м}^2\]
		\[I_1 + I_2 = (6,83 \pm 0,48) \cdot 10^{-3} \text{ кг}\cdot\text{м}^2\]
\item Помещу на платформу цилиндр, разрезанный по диаметру. Постепенно раздвигая половинки цилиндра так, чтобы их общий центр масс всё время оставался на оси вращения платформы, сниму зависимость момента инерции такой системы $I$ от расстояния $x$ каждой из половинок до оси вращения (центра платформы). Результаты внесу в таблицу № 3.
\[I = k(m+m_0)T^2 - I_0\]

Построю график зависимости $I(x^2)$. По нему необходимо определить массу и момент инерции цилиндра. По графику видно, что он представляет собой линейную зависимость $I = ax^2 + b$.
		
		По формуле Гюйгенса-Штейнера:
		
		\[I_x(x) = I + mx^2\]
		
		Найду коэффициенты с помощью МНК:
		
		\[I = b = (1,688 \pm 0,024) \cdot 10^{-3} \text{ кг}\cdot\text{м}^2\]
		\[m = a = (1,558 \pm 0,012) \text{ кг}\]
		
		Как видно из эксперимента, формула Гюйгенса-Штейнера работает, а масса цилиндра, вычисленная с помощью МНК, близка к массе, определённой с помощью весов.

\begin{table}
\centering
\caption{Используемые величины и их погрешности}
\begin{tabular}{|c|c|c|}
\hline
Величина & Значение & Погрешность \\ \hline
Расстояние от центра крепления до края диска & $a = 215,7 \text{ см}$ & $\sigma_a = 0,1 \text{ см}$ \\ \hline
Радиус диска системы & $b = 12,5 \text{ см}$ & $\sigma_b = 0.05 \text{ см}$ \\ \hline
Расстояние от центра крепления до центра диска & $z_0 = 2153,4 \text{ мм}$ & $\sigma_{z_0} = 8,7 \text{ мм}$ \\ \hline
Рассояние от оси до нижнего крепления нити & $R = 115,5 \text{ мм}$ & $\sigma_R = 0,5 \text{ мм}$ \\ \hline
Рассояние от оси до верхнего крепления нити & $r = 30,2 \text{ мм}$ & $\sigma_r = 0,3 \text{ мм}$ \\ \hline
Масса диска & $m_0 = 1026,4 \text{ г}$ & $\sigma_{m_0} = 0,5 \text{ г} $ \\ \hline
Постоянный коэффициент для установки & $k = 4,025\cdot10^{-4} \text{ м$^2$/с$^2$}$ & $\sigma_k = 0,047\cdot10^{-4}\text{ м$^2$/с$^2$}$ \\ \hline
Момент инерции ненагруженной платфомы & $I_0 = 8,07\cdot10^{-3}\text{ кг$\cdot$м$^2$}$ & $\sigma_{I_0 }= 0,14\cdot10^{-3}\text{ кг$\cdot$м$^2$}$ \\ \hline
Суммарная масса половинок цилиндра & $m = 1526,9\text{ г}$ & $\sigma_m = 0,1\text{ г}$ \\ \hline

\end{tabular}
\end{table}


\begin{table}[h]
			\centering
			\caption{Параметры и моменты инерции тел № 1 и 2}
			\begin{tabular}{|c|c|c|c|c|c|c|}
				\hline
				№ & Схема & Параметры & T, c & $I + I_0, 10^{-3} \text{ кг}\cdot\text{м}^2$ & $I, 10^{-3} \text{ кг}\cdot\text{м}^2$ & $I_\text{теор}, 10^{-3} \text{ кг}\cdot\text{м}^2$\\
				\hline
				1 & \rowincludegraphics{body1} &
				\begin{tabular}{c}
					$h = (55,4 \pm 0,1) \text{ мм}$ \\
					$d = (4,4 \pm 0,1) \text{ мм}$ \\
					$D = (159,0 \pm 0,1) \text{ мм}$ \\
					$m = (771,7 \pm 0,1) \text{ г}$ \\
				\end{tabular} & 4,204 & 12,79 & $4,72 \pm 0,22$ & 4,74 \\
				\hline
			
				2 & \rowincludegraphics{body2} &
				\begin{tabular}{c}
					$d = (10,2 \pm 0,1) \text{ мм}$ \\
					$D = (170,6 \pm 0,1) \text{ мм}$ \\
					$h = (3,6 \pm 0,1) \text{ мм}$ \\
					$H = (25,2 \pm 0,1) \text{ мм}$ \\
					$m = (589,5 \pm 0,1) \text{ г}$ \\
				\end{tabular} & 3,964 & 10,22 & $2,15 \pm 0,20$ & 2,09\\
				\hline
				1 + 2 & & $m = (1360,6 \pm 0,1) \text{ г}$ & 3,935 & 14,94 & $6,87 \pm 0,42$ & 6,83 \\
				\hline
			\end{tabular}
		\end{table}

\end{enumerate}

\begin{table}[h]
	\centering
	\caption{Сдвиг половинок цилиндра}
	\begin{tabular}{|c|c|c|c||c|c|c|c|c|}
		\hline
		№ & $x$, мм & $T$, c & $I, 10^{-3} \text{ кг}\cdot\text{м}^2$ & № & $x$, мм & $T$, c & $I, 10^{-3} \text{ кг}\cdot\text{м}^2$ \\
		\hline
		1  & 0  & 3,082	& 1,692	& 10 & 45 & 3,554 & 4,911 \\
		2  & 5  & 3,090	& 1,743	& 11 & 50 & 3,653 & 5,647 \\
		3  & 10 & 3,109	& 1,864	& 12 & 55 & 3,749 & 6,374 \\
		4  & 15 & 3,138	& 2,050	& 13 & 60 & 3,856 & 7,211 \\
		5  & 20 & 3,174	& 2,283	& 14 & 65 & 3,982 & 8,226 \\
		6  & 25 & 3,229	& 2,645	& 15 & 70 & 4,089 & 9,113 \\
		7  & 30 & 3,289	& 3,047	& 16 & 75 & 4,257 & 10,554 \\
		8  & 35 & 3,376	& 3,643	& 17 & 80 & 4,393 & 11,763 \\
		9  & 40 & 3,453	& 4,184	& & & & \\
		\hline
	\end{tabular}
\end{table}

\newpage
\center{\textbf{График зависимости $I(x^2)$}}
\begin{figure}[h]
\center{\includegraphics[scale=1]{График.pdf}}
\end{figure}

\end{document}